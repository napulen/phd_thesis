%%%% Proceedings format for most of ACM conferences (with the exceptions listed below) and all ICPS volumes.
\documentclass[sigconf]{acmart}
%%%% As of March 2017, [siggraph] is no longer used. Please use sigconf (above) for SIGGRAPH conferences.

%%%% Proceedings format for SIGPLAN conferences
% \documentclass[sigplan, anonymous, review]{acmart}

%%%% Proceedings format for SIGCHI conferences
% \documentclass[sigchi, review]{acmart}

%%%% To use the SIGCHI extended abstract template, please visit
% https://www.overleaf.com/read/zzzfqvkmrfzn

\usepackage{booktabs} % For formal tables

%%%%%
\usepackage[colorinlistoftodos, prependcaption, textsize=tiny]{todonotes}
\usepackage{color}
% \usepackage{url}
\urlstyle{sf}
%%%%%

% Copyright
%\setcopyright{none}
\setcopyright{acmcopyright}
%\setcopyright{acmlicensed}
%\setcopyright{rightsretained}
%\setcopyright{usgov}
%\setcopyright{usgovmixed}
%\setcopyright{cagov}
%\setcopyright{cagovmixed}

% DOI
\acmDOI{10.1145/3273024.3273027}

% ISBN
\acmISBN{978-1-4503-6522-2/18/09}

%Conference
\acmConference[DLfM '18]{5th International Conference on Digital Libraries for Musicology}{September 28, 2018}{Paris, France}
\acmYear{2018}
\copyrightyear{2018}

%\acmArticle{4}
\acmPrice{15.00}

% These commands are optional
\acmBooktitle{5th International Conference on Digital Libraries for Musicology (DLfM '18), September 28, 2018, Paris, France}
%\editor{Jennifer B. Sartor}
%\editor{Theo D'Hondt}
%\editor{Wolfgang De Meuter}

\begin{document}
\title{Encoding Matters}
%\titlenote{Produces the permission block, and
%  copyright information}
%\subtitle{Extended Abstract}
%\subtitlenote{The full version of the author's guide is available as
%  \texttt{acmart.pdf} document}

\author{N\'estor N\'apoles}
%\authornote{Dr.~Trovato insisted his name be first.}
\orcid{1234-5678-9012}
\affiliation{%
  \institution{McGill University, CIRMMT}
  %\streetaddress{P.O. Box 1212}
  \city{Montr\'eal}
  \state{QC}
  %\postcode{43017-6221}
}
\email{nestor.napoleslopez@mail.mcgill.ca}

\author{Gabriel Vigliensoni}
%\authornote{Dr.~Trovato insisted his name be first.}
\orcid{1234-5678-9012}
\affiliation{%
  \institution{McGill University, CIRMMT}
  %\streetaddress{P.O. Box 1212}
  \city{Montr\'eal}
  \state{QC}
  %\postcode{43017-6221}
}
\email{gabriel@music.mcgill.ca}

\author{Ichiro Fujinaga}
%\authornote{Dr.~Trovato insisted his name be first.}
\orcid{1234-5678-9012}
\affiliation{%
  \institution{McGill University, CIRMMT}
  %\streetaddress{P.O. Box 1212}
  \city{Montr\'eal}
  \state{QC}
  %\postcode{43017-6221}
}
\email{ichiro.fujinaga@mcgill.ca}

% The default list of authors is too long for headers.
% \renewcommand{\shortauthors}{B. Trovato et al.}


\begin{abstract}
In this paper, we discuss how different encodings in symbolic music files can have consequences for music analysis, where a truthful representation, not only of the musical score, but of the semantics of the music, can change the results of music analysis tools. We introduce a series of examples in which different encodings effectively modify the content of two---apparently equivalent---symbolic music files. These examples have been obtained from comparing three different encodings of a string quartet movement by Ludwig van Beethoven.

We present two scenarios in which encoding discrepancies may be introduced. In the first scenario, they have been introduced during the encoding of the symbolic music file by either the music notation software or the human encoder. The discrepancies introduced in this scenario are typically difficult to notice because they are \emph{visually} identical to an accurate encoding. In the second scenario, the discrepancies have been introduced during the translation of the original file into other symbolic formats. In this scenario, the discrepancies may be related to propagating errors in the original encoding or to an erroneous translation of certain attributes of the musical content. Finally, we discuss the possibility of using the examples provided here for the mitigation of some of these discrepancies in the future.
\end{abstract}


%
% The code below should be generated by the tool at
% http://dl.acm.org/ccs.cfm
% Please copy and paste the code instead of the example below.
%
\begin{CCSXML}
<ccs2012>
<concept>
<concept_id>10002951.10003317.10003371.10003386.10003390</concept_id>
<concept_desc>Information systems~Music retrieval</concept_desc>
<concept_significance>500</concept_significance>
</concept>
<concept>
<concept_id>10011007.10011006.10011050.10010512.10003310</concept_id>
<concept_desc>Software and its engineering~Extensible Markup Language (XML)</concept_desc>
<concept_significance>300</concept_significance>
</concept>
<concept>
<concept_id>10011007.10011074.10011075.10011078</concept_id>
<concept_desc>Software and its engineering~Software design tradeoffs</concept_desc>
<concept_significance>300</concept_significance>
</concept>
<concept>
<concept_id>10011007.10011006.10011072</concept_id>
<concept_desc>Software and its engineering~Software libraries and repositories</concept_desc>
<concept_significance>100</concept_significance>
</concept>
<concept>
<concept_id>10010405.10010469.10010475</concept_id>
<concept_desc>Applied computing~Sound and music computing</concept_desc>
<concept_significance>100</concept_significance>
</concept>
</ccs2012>
\end{CCSXML}

\ccsdesc[500]{Information systems~Music retrieval}
\ccsdesc[300]{Software and its engineering~Extensible Markup Language (XML)}
\ccsdesc[300]{Software and its engineering~Software design tradeoffs}
\ccsdesc[100]{Software and its engineering~Software libraries and repositories}
\ccsdesc[100]{Applied computing~Sound and music computing}

\keywords{Symbolic music, music notation, music encoding, music transcription, music information retrieval, musicxml, mei, humdrum, verovio, humlib, vis, music21}

\maketitle

\input{encodingmatters_body}

\bibliographystyle{ACM-Reference-Format}
\bibliography{sample-bibliography}

\end{document}
