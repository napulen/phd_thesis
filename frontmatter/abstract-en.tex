\chapter*{Abstract}
\addcontentsline{toc}{chapter}{Abstract}
\label{chap:abstract-en}

One of the most common ways to analyze a piece of tonal
music is through Roman numeral analysis. This requires the
inspection of several attributes related to chords and keys.
Chords can be inspected in terms of their properties: root,
quality, inversion, and function. Keys can be inspected in
terms of their temporal scope as modulations or
tonicizations. Each of these attributes (or tasks) of Roman
numeral analysis can be modeled in isolation. However,
recent research has found that analyzing several tonal tasks
simultaneously leads to more robust Music Information
Retrieval models. This has motivated the research of
multitask neural networks for Roman numeral analysis. In
this dissertation, I extend this line of research by
developing a new Roman numeral analysis machine learning
model. The model is based on a new convolutional recurrent
neural network, which is trained with a large dataset of
publicly available Roman numeral analysis annotations. The
model is assisted by a new data-augmentation technique and
multitask learning layout to learn the relevant attributes
of chords and keys. Combining these ideas, the resulting
model seems to achieve an improved performance over rare
chords compared to previous automatic Roman numeral analysis
methods. Among other applications, this model will
facilitate advanced searching in music collections. For
example, searching by chord progressions or by modulation
trajectories.
