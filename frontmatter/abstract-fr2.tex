L'une des façons les plus courantes d'analyser un morceau de
musique tonale est l'analyse par chiffres romains. Cela
nécessite l'inspection de plusieurs attributs liés aux
accords et aux clés. Les accords peuvent être inspectés en
fonction de leurs propriétés : fondamentale, qualité,
inversion et fonction. Les tonalités peuvent être inspectées
en fonction de leur contexte temporel en tant que
modulations ou tonalités. Chacun de ces attributs (ou
tâches) de l'analyse des chiffres romains peut être modélisé
de manière isolée. Cependant, des recherches récentes ont
montré que l'analyse simultanée de plusieurs tâches tonales
conduit à des modèles de recherche d'informations musicales
plus robustes. Cela a motivé l'étude des réseaux neuronaux
multitâches pour l'analyse des chiffres romains. Dans cette
thèse, j'étends cette ligne de recherche en développant un
nouveau modèle d'apprentissage automatique pour l'analyse
des chiffres romains. Le modèle est basé sur un nouveau
réseau neuronal convolutif récurrent, qui est formé sur un
grand ensemble de données d'annotations d'analyse en
chiffres romains disponibles publiquement. Le modèle est
aidé par une nouvelle technique d'augmentation des données
et une configuration d'apprentissage multitâches pour
apprendre les attributs pertinents des accords et des
tonalités. En combinant ces idées, le modèle résultant
semble atteindre des performances améliorées sur les accords
rares par rapport aux méthodes précédentes d'analyse
automatique des chiffres romains. Entre autres applications,
ce modèle facilitera la recherche avancée dans les
collections de musique. Par exemple, la recherche par
progressions d'accords ou par chemins de modulation
