% \makeglossaries
\makenoidxglossaries


%%%%%%%%%%%%%%%%%%%%%%%%
% GLOSSARY
%%%%%%%%%%%%%%%%%%%%%%%%

\newglossaryentry{augmentednet}{name=AugmentedNet,description={The
main neural network model described in this \thesisdiss{}}}

\newglossaryentry{augsix}{name=augmented
sixth,description={A set of special chords in tonal music,
often dubbed the Italian, French, and German augmented sixth
chords}}

\newglossaryentry{backpropagation}{name=backpropagation,description={To
be defined}}

\newglossaryentry{chainrule}{name=chain rule,description={To
be defined}}

\newglossaryentry{chromagram}{name=chromagram,description={To
be defined}}

\newglossaryentry{closed-position}{name=closed-position
form,description={In the context of chord realizations, a
closed-position chord is an arrangement of the chord,
usually in four parts, such that the three upper voices span
a maximum interval of an octave}}

\newglossaryentry{commonpractice}{name=common-practice
period, description={In Western art music, this refers to
the period of music between 1650--1900, which is
characterized by the use of the tonal system}}

\newglossaryentry{cybernetics}{name=cybernetics,description={To
be defined}}

\newglossaryentry{expertsystem}{name=expert
system,description={To be defined}}

\newglossaryentry{frame}{name=frame,description={To be
defined}}

\newglossaryentry{hardcode}{name=hardcode,description={To be
defined}}

\newglossaryentry{humharm}{name=**harm,description={A syntax
for Roman numeral analysis in the family of Humdrum
representations}}

\newglossaryentry{humkern}{name=**kern,description={A syntax
for music notation in the family of Humdrum
representations}}

\newglossaryentry{kernel}{name=kernel,description={To be
defined}}

\newglossaryentry{keyprofile}{name=key profile,
description={A key profile is a pitch-class distribution
used to correlate pitch histograms with a certain key or
scale}}

\newglossaryentry{keyscape}{name=keyscape,description={To be
defined}}

\newglossaryentry{lineoffifths}{name=line of
fifths,description={To be defined}}

\newglossaryentry{markovchain}{name=markov
chain,description={To be defined}}

\newglossaryentry{middlec}{name=middle C,description={To be
defined}}

\newglossaryentry{mirexkey}{name=MIREX key
metric,description={To be defined}}

\newglossaryentry{musicxml}{name=MusicXML, description={A
symbolic music format based on \gls{xml}}}

\newglossaryentry{neapolitan}{name=neapolitan,description={A
special kind of chord in tonal music, analogous to a
$\flat\rn{II}$ Roman numeral}}

\newglossaryentry{nonchord}{name=nonchord
tone,description={To be defined}}

\newglossaryentry{overfit}{name=overfit,
description={Overfitting is a phenomenon where a machine
learning model excessively fits to the training data,
compromising the capability of the model to perform well on
unseen data}}

\newglossaryentry{parameter}{name=parameter,description={To
be defined}}

\newglossaryentry{pianoroll}{name=pianoroll,description={To
be defined}}

\newglossaryentry{pitchclass}{name=pitch
class,description={To be defined}}

\newglossaryentry{pitchspelling}{name=pitch spelling,
description={The name assigned to a given pitch class. For
example, the spelling of pitch class ``8'' could be
$G\sharp$ or $A\flat$}}

\newglossaryentry{realize}{name=realize, description={In the
context of chords, a chord realization is an arrangement of
a chord label into notes. For example, if a chord is
indicated as $\rn{F:V}\rnseven$, one realization consists of
the tuple of notes $(C4, E4, G4, B\flat4)$, whereas a
different realization consists of the notes $(C4, G4, E5,
B\flat5)$}}

\newglossaryentry{romantext}{name=RomanText, description={A
digital standard for Roman numeral analysis}}

\newglossaryentry{realtime}{name=realtime,description={To be
defined}}

\newglossaryentry{salamislice}{name=salami
slice,description={To be defined}}

\newglossaryentry{spiralarray}{name=spiral
array,description={To be defined}}

\newglossaryentry{stream}{name=stream,description={To be
defined}}

\newglossaryentry{vanishinggradients}{name=vanishing
gradients,description={To be defined}}

\newglossaryentry{voice}{name=voice,description={To be
defined}}


%%% Neural network representations

\newglossaryentry{bass19}{name=Bass19, description={An input
representation of the proposed neural network described in
\refsubsec{bassinput} }}

\newglossaryentry{chroma19}{name=Chroma19, description={An
input representation of the proposed neural network
described in \refsubsec{spelledchromainput} }}

\newglossaryentry{duration14}{name=Duration14,
description={An input representation of the proposed neural
network described in \refsubsec{durationinput} }}

\newglossaryentry{bass35}{name=Bass35, description={An
output representation of the proposed neural network
described in \refsubsubsec{bass} }}

\newglossaryentry{tenor35}{name=Tenor35, description={An
output representation of the proposed neural network
described in \refsubsubsec{tenor} }}

\newglossaryentry{alto35}{name=Alto35, description={An
output representation of the proposed neural network
described in \refsubsubsec{alto} }}

\newglossaryentry{soprano35}{name=Soprano35, description={An
output representation of the proposed neural network
described in \refsubsubsec{soprano} }}

\newglossaryentry{pcset121}{name=PitchClassSet121,
description={An output representation of the proposed neural
network described in \refsubsubsec{pitchclassset} }}

\newglossaryentry{rn31}{name=RomanNumeral31, description={An
output representation of the proposed neural network
described in \refsubsubsec{romannumeralnumerator} }}

\newglossaryentry{localkey35}{name=LocalKey35,
description={An output representation of the proposed neural
network described in \refsubsubsec{localkey} }}

\newglossaryentry{tonicization35}{name=Tonicization35,
description={An output representation of the proposed neural
network described in \refsubsubsec{tonicizationkey} }}

\newglossaryentry{harmonicrhythm7}{name=HarmonicRhythm7,
description={An output representation of the proposed neural
network described in \refsubsubsec{harmonicrhythm} }}

%%%%%%%%%%%%%%%%%%%%%%%%
% ACRONYM
%%%%%%%%%%%%%%%%%%%%%%%%



% Computer science acronyms
\newacronym[plural=ANNs, firstplural=Artificial Neural Networks (ANNs)]{ann}{ANN}{Artificial Neural Network}
\newacronym{abc}{ABC}{Annotated Beethoven Corpus}
\newacronym[plural=BLSTMs]{blstm}{BLSTM}{Bidirectional Long Short-Term Memory}
\newacronym{bps}{BPS}{Beethoven Piano Sonatas}
\newacronym{bptt}{BPTT}{Backpropagation Through Time}
\newacronym{cblstm}{CBLSTM}{Convolutional Bidirectional Long Short-Term Memory}
\newacronym[plural=CNNs, firstplural=Convolutional Neural Networks (CNNs)]{cnn}{CNN}{Convolutional Neural Network}
\newacronym[plural=CRNNs, firstplural=Convolutional Recurrent Neural Networks (CRNNs)]{crnn}{CRNN}{Convolutional Recurrent Neural Network}
\newacronym[plural=DBNs, firstplural=Deep Belief Networks (DBNs)]{dbn}{DBN}{Deep Belief Network}
\newacronym{fft}{FFT}{Fast Fourier Transform}
\newacronym{haydnsun}{HaydnSun}{Haydn ``Sun'' String Quartets, Op. 20}
\newacronym[plural=HMMs, firstplural=Hidden Markov Models (HMMs)]{hmm}{HMM}{Hidden Markov Model}
\newacronym{knn}{KNN}{K-Nearest Neighbors}
\newacronym{lisp}{LISP}{LISt Processor}
\newacronym[plural=LSTMs]{lstm}{LSTM}{Long Short-Term Memory}
\newacronym[plural=GRUs, firstplural=Gated Recurrent Units (GRUs)]{gru}{GRU}{Gated Recurrent Unit}
\newacronym{mps}{MPS}{Mozart Piano Sonatas}
\newacronym{mlops}{MLOps}{Machine Learning Operations}
\newacronym[plural=MLPs, firstplural=Multilayer Perceptrons (MLPs)]{mlp}{MLP}{Multilayer Perceptron}
\newacronym{mtl}{MTL}{Multitask learning}
\newacronym{mnist}{MNIST}{Modified National Institute of Standards and Technology}
\newacronym{nade}{NADE}{Neural Autoregressive Density Estimator}
\newacronym[plural=RNNs, firstplural=Recurrent Neural Networks (RNNs)]{rnn}{RNN}{Recurrent Neural Network}
\newacronym{som}{SOM}{Self-Organizing Maps}
\newacronym[plural=SVMs, firstplural=Support Vector Machines (SVMs)]{svm}{SVM}{Support Vector Machine}
\newacronym{tavern}{TAVERN}{Theme and Variation Encodings with Roman Numerals}
\newacronym{wtc}{WTC}{[The] Well-Tempered Clavier}
\newacronym{wir}{WiR}{When in Rome}
\newacronym[plural=XMLs]{xml}{XML}{Extensible Markup Language}

% MIR acronyms
\newacronym{12tet}{12-TET}{Twelve-Tone Equal Temperament}
\newacronym{acr}{ACR}{Automatic Chord Recognition}
\newacronym{cirmmt}{CIRMMT}{Centre for Interdisciplinary Research in Music Media and Technology}
\newacronym{csv}{CSV}{Comma-Separated Values}
\newacronym[plural=DAWs, firstplural=Digital Audio Workstations (DAWs)]{daw}{DAW}{Digital Audio Workstation}
\newacronym{dcml}{DCML}{Digital and Cognitive Musicology Lab}
\newacronym{ddmal}{DDMAL}{Distributed Digital Music Archives and Libraries}
\newacronym{frqsc}{FRQSC}{Fonds de recherche --- Soci\'et\'e et culture}
\newacronym{hlsd}{HLSD}{Hooktheory Lead Sheet Dataset}
\newacronym{gke}{GKE}{Global-Key Estimation}
\newacronym{lke}{LKE}{Local-Key Estimation}
\newacronym[plural=HPCPs, firstplural=Harmonic Pitch Class Profiles (HPCPs)]{hpcp}{HPCP}{Harmonic Pitch Class Profile}
\newacronym{iaml}{IAML}{ International Association of Music Libraries}
\newacronym{mei}{MEI}{Music Encoding Initiative}
\newacronym{midi}{MIDI}{Musical Instrument Digital Interface}
\newacronym{mir}{MIR}{Music Information Retrieval}
\newacronym{mirex}{MIREX}{Music Information Retrieval Evaluation eXchange}
\newacronym{mtg}{MTG}{Music Technology Group}
\newacronym{omr}{OMR}{Optical Music Recognition}
\newacronym{rism}{RISM}{R\'epertoire International des Sources Musicales}
\newacronym[plural=ARNA, firstplural=Automatic Roman Numeral Analysis (ARNA)]{rna}{RNA}{[Automatic] Roman Numeral Analysis}
\newacronym{tei}{TEI}{Text Encoding Initiative}
\newacronym{tsd}{TSD}{Tonic, Subdominant, and Dominant}
\newacronym{simssa}{SIMSSA}{Single Interface for Music Score Search and Analysis}
\newacronym[plural=TPCs, firstplural=Tonal Pitch Classes (TPCs)]{tpc}{TPC}{Tonal Pitch Class}
