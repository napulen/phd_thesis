% \makeglossaries
\makenoidxglossaries

%%%%%%%%%%%%%%%%%%%%%%%%
% GLOSSARY
%%%%%%%%%%%%%%%%%%%%%%%%

\newglossaryentry{humharm}{name=**harm, description={A
syntax for Roman numeral analysis from the family of
\emph{Humdrum} representations}}

\newglossaryentry{humkern}{name=**kern, description={A
syntax for musical notation from the family of
\emph{Humdrum} representations}}

\newglossaryentry{humtext}{name=**text, description={A
syntax for lyrics from the family of \emph{Humdrum}
representations}}

\newglossaryentry{augmentednet}{name=AugmentedNet,description={An
Automatic Roman Numeral Analysis model, originally
introduced in \textcite{napoleslopez2021augmentednet}, and
extended in this \thesisdiss{}. The main model described
throughout the dissertation}}

\newglossaryentry{augsix}{name=augmented
sixth,description={A family of chords in Western tonal music
characterized by an ``augmented sixth'' interval, which
comprises at least three types of chords: the Italian
($\rnIt$), French ($\rnFr$), and German ($\rnGer$) augmented
sixth chords}}

\newglossaryentry{closed-position}{name=closed-position
form,description={In the context of chord realizations, a
closed-position chord is an arrangement of the chord,
usually in four parts, such that the three upper voices span
a maximum interval of an octave}}

\newglossaryentry{commonpractice}{name=common-practice
period, description={In Western art music, this refers to
the period of music between 1650--1900, which is
characterized by the use of the tonal system}}

\newglossaryentry{keyprofile}{name=key profile,
description={A key profile is a pitch-class distribution
used to correlate pitch histograms with a certain key or
scale}}

\newglossaryentry{lineoffifths}{name=line of
fifths,description={To be defined}}

\newglossaryentry{melisma}{name=melisma,description={A model
for Automatic Roman numeral analysis proposed by Temperley,
Sleator, and Sapp. The first end-to-end system for this
task}}

\newglossaryentry{musicxml}{name=MusicXML, description={A
symbolic music format based on \gls{xml}}}

\newglossaryentry{neapolitan}{name=Neapolitan,description={A
special kind of chord in tonal music, analogous to a
$\flat\rn{II}$ Roman numeral}}

\newglossaryentry{nonchord}{name=nonchord
tone,description={To be defined}}

\newglossaryentry{overfit}{name=overfit,
description={Overfitting is a phenomenon where a machine
learning model excessively fits to the training data,
compromising the capability of the model to perform well on
unseen data}}

\newglossaryentry{pitchclass}{name=pitch
class,description={To be defined}}

\newglossaryentry{realize}{name=realize, description={In the
context of chords, a chord realization is an arrangement of
a chord label into notes. For example, if a chord is
indicated as $\rn{F:V}\rnseven$, one realization consists of
the tuple of notes $(C4, E4, G4, B\flat4)$, whereas a
different realization consists of the notes $(C4, G4, E5,
B\flat5)$}}

\newglossaryentry{romantext}{name=RomanText, description={A
digital standard for Roman numeral analysis}}

\newglossaryentry{spine}{name=spine, description={In the
context of the Humdrum format, a spine is a column of data}}

%%% Neural network representations

\newglossaryentry{bass19}{name=LowestNote19,
description={One of the three input representations in the
neural network proposed in this dissertation. It is
described in \refsubsec{lowestsoundingnote} }}

\newglossaryentry{chroma19}{name=Notes19, description={One
of the three input representations in the neural network
proposed in this dissertation. It is described in
\refsubsec{allsoundingnotes} }}

\newglossaryentry{duration14}{name=Onsets14,
description={One of the three input representations in the
neural network proposed in this dissertation. It is
described in \refsubsec{measureandnoteonsets} }}

\newglossaryentry{bass35}{name=Bass35, description={One of
the nine classification tasks in the neural network proposed
in this dissertation. It is described in \refsubsec{Bass35}
}}

\newglossaryentry{tenor35}{name=Tenor35, description={One of
the nine classification tasks in the neural network proposed
in this dissertation. It is described in \refsubsec{Tenor35}
}}

\newglossaryentry{alto35}{name=Alto35, description={One of
the nine classification tasks in the neural network proposed
in this dissertation. It is described in \refsubsec{Alto35}
}}

\newglossaryentry{satb35}{name=SATB35, description={A short
name for the four classification tasks: \gls{soprano35},
\gls{alto35}, \gls{tenor35}, and \gls{bass35} }}

\newglossaryentry{soprano35}{name=Soprano35,
description={One of the nine classification tasks in the
neural network proposed in this dissertation. It is
described in \refsubsec{Soprano35} }}

\newglossaryentry{pcset}{name=pcset, description={A subset
$\elpcset$ of the 12 ``pitch classes'' in the Western
chromatic scale $\elpcset \subset [0, 11]$. In the context
of this dissertation, a pitch-class set always refers to a
set $|\elpcset| = 3$ or $|\elpcset| = 4$ that contains the
pitch classes of a triad or seventh chord. See
\refsec{thevocabularyofpitch-classsets} for further
discussion on pitch-class sets }}

\newglossaryentry{pcset121}{name=PitchClassSet121,
description={An output representation of the proposed neural
network described in \refsubsubsec{PitchClassSet121} }}

\newglossaryentry{rn31}{name=Numerator31, description={An
output representation of the proposed neural network
described in \refsubsubsec{romannumeralnumerator} }}

\newglossaryentry{localkey38}{name=LocalKey38,
description={An output representation of the proposed neural
network described in \refsubsubsec{localkey} }}

\newglossaryentry{tonicization38}{name=Tonicization38,
description={An output representation of the proposed neural
network described in \refsubsubsec{tonicizationkey} }}

\newglossaryentry{harmonicrhythm7}{name=HarmonicRhythm7,
description={An output representation of the proposed neural
network described in \refsubsubsec{harmonicrhythm} }}

%%%%%%%%%%%%%%%%%%%%%%%%
% ACRONYM
%%%%%%%%%%%%%%%%%%%%%%%%

% Computer science acronyms
\newacronym[plural=ANNs, firstplural=Artificial Neural Networks (ANNs)]{ann}{ANN}{Artificial Neural Network}
\newacronym[plural=APIs, firstplural=Application Programming Interfaces (APIs)]{api}{API}{Application Programming Interface}
\newacronym[plural=BLSTMs]{blstm}{BLSTM}{Bidirectional Long Short-Term Memory}
\newacronym{bptt}{BPTT}{Backpropagation Through Time}
\newacronym{cblstm}{CBLSTM}{Convolutional Bidirectional Long Short-Term Memory}
\newacronym[plural=CNNs, firstplural=Convolutional Neural Networks (CNNs)]{cnn}{CNN}{Convolutional Neural Network}
\newacronym[plural=CRNNs, firstplural=Convolutional Recurrent Neural Networks (CRNNs)]{crnn}{CRNN}{Convolutional Recurrent Neural Network}
\newacronym{csv}{CSV}{Comma-Separated Values}
\newacronym[plural=DBNs, firstplural=Deep Belief Networks (DBNs)]{dbn}{DBN}{Deep Belief Network}
\newacronym{fft}{FFT}{Fast Fourier Transform}
\newacronym{gpu}{GPU}{Graphics Processing Unit}
\newacronym{hdf5}{HDF5}{Hierarchical Data Format (v5)}
\newacronym[plural=HMMs, firstplural=Hidden Markov Models (HMMs)]{hmm}{HMM}{Hidden Markov Model}
\newacronym{knn}{KNN}{K-Nearest Neighbours}
\newacronym{lisp}{LISP}{LISt Processor}
\newacronym[plural=LSTMs]{lstm}{LSTM}{Long Short-Term Memory}
\newacronym[plural=GRUs, firstplural=Gated Recurrent Units (GRUs)]{gru}{GRU}{Gated Recurrent Unit}
\newacronym{mlops}{MLOps}{Machine Learning Operations}
\newacronym[plural=MLPs, firstplural=Multilayer Perceptrons (MLPs)]{mlp}{MLP}{Multilayer Perceptron}
\newacronym{mtl}{MTL}{Multitask Learning}
\newacronym{mnist}{MNIST}{Modified National Institute of Standards and Technology}
\newacronym{nade}{NADE}{Neural Autoregressive Density Estimator}
\newacronym{relu}{ReLU}{Rectified Linear Unit}
\newacronym[plural=RNNs, firstplural=Recurrent Neural Networks (RNNs)]{rnn}{RNN}{Recurrent Neural Network}
\newacronym{som}{SOM}{Self-Organizing Maps}
\newacronym[plural=SVMs, firstplural=Support Vector Machines (SVMs)]{svm}{SVM}{Support Vector Machine}
\newacronym{tsv}{TSV}{Tab-Separated Values}
\newacronym[plural=XMLs]{xml}{XML}{Extensible Markup Language}

% MIR acronyms
\newacronym{12tet}{12-TET}{Twelve-Tone Equal Temperament}
\newacronym{abc}{ABC}{Annotated Beethoven Corpus}
\newacronym{acr}{ACR}{Automatic Chord Recognition}
\newacronym{arna}{ARNA}{Automatic Roman Numeral Analysis}
\newacronym{bps}{BPS}{Beethoven Piano Sonatas}
\newacronym{cirmmt}{CIRMMT}{Centre for Interdisciplinary Research in Music Media and Technology}
\newacronym{cts}{CTS}{Computational Tonal Studies}
\newacronym[plural=DAWs, firstplural=Digital Audio Workstations (DAWs)]{daw}{DAW}{Digital Audio Workstation}
\newacronym{dcml}{DCML}{Digital and Cognitive Musicology Lab}
\newacronym{ddmal}{DDMAL}{Distributed Digital Music Archives and Libraries}
\newacronym{frqsc}{FRQSC}{Fonds de recherche --- Soci\'et\'e et culture}
\newacronym{haydnsun}{HaydnSun}{Haydn ``Sun'' String Quartets, Op. 20}
\newacronym{kmt}{KMT}{Key Modulations and Tonicizations}
\newacronym{hlsd}{HLSD}{Hooktheory Lead Sheet Dataset}
\newacronym{ismir}{ISMIR}{International Society for Music Information Retrieval}
\newacronym{gke}{GKE}{Global-Key Estimation}
\newacronym{lke}{LKE}{Local-Key Estimation}
\newacronym[plural=HPCPs, firstplural=Harmonic Pitch Class Profiles (HPCPs)]{hpcp}{HPCP}{Harmonic Pitch Class Profile}
\newacronym{iaml}{IAML}{International Association of Music Libraries}
\newacronym{mei}{MEI}{Music Encoding Initiative}
\newacronym{midi}{MIDI}{Musical Instrument Digital Interface}
\newacronym{mir}{MIR}{Music Information Retrieval}
\newacronym{mirex}{MIREX}{Music Information Retrieval Evaluation eXchange}
\newacronym{mpc}{MPC}{Music Perception and Cognition}
\newacronym{mps}{MPS}{Mozart Piano Sonatas}
\newacronym{mtg}{MTG}{Music Technology Group}
\newacronym{omr}{OMR}{Optical Music Recognition}
\newacronym{rism}{RISM}{R\'epertoire International des Sources Musicales}
\newacronym{rna}{RNA}{Roman Numeral Analysis}
\newacronym{tavern}{TAVERN}{Theme and Variation Encodings with Roman Numerals}
\newacronym{tei}{TEI}{Text Encoding Initiative}
\newacronym{tsd}{TSD}{Tonic, Subdominant, and Dominant}
\newacronym{satb}{SATB}{Soprano, Alto, Tenor, and Bass}
\newacronym{simssa}{SIMSSA}{Single Interface for Music Score Search and Analysis}
\newacronym[plural=TPCs, firstplural=Tonal Pitch Classes (TPCs)]{tpc}{TPC}{Tonal Pitch Class}
\newacronym{wir}{WiR}{When in Rome}
\newacronym{wtc}{WTC}{[The] Well-Tempered Clavier}

% Note symbols
\newacronym[plural=$\musThirtySecond$, firstplural=32\textsuperscript{nd} ($\musThirtySecond$)]{32nd}{$\musThirtySecond$}{Thirty-second note}
\newacronym[plural=$\musSixteenth$, firstplural=16\textsuperscript{th} ($\musSixteenth$)]{16th}{$\musSixteenth$}{Sixteenth note}
\newacronym[plural=$\musEighth$, firstplural=8\textsuperscript{th} ($\musEighth$)]{8thth}{$\musEighth$}{Eighth note}
\newacronym[plural=$\musQuarter$, firstplural=quarter ($\musQuarter$)]{quarter}{$\musQuarter$}{Quarter note}
\newacronym[plural=$\musHalf$, firstplural=half ($\musHalf$)]{half}{$\musHalf$}{Half note}
\newacronym[plural=$\musWhole$, firstplural=whole ($\musWhole$)]{whole}{$\musWhole$}{Whole note}
\newacronym[plural=$\musFlat$, firstplural=flat ($\musFlat$)]{flat}{$\musFlat$}{A ``flat'' accidental}
\newacronym[plural=$\musSharp$, firstplural=sharp ($\musSharp$)]{sharp}{$\musSharp$}{A ``sharp'' accidental}
\newacronym[plural=$\musDoubleFlat$, firstplural=double flat ($\musDoubleFlat$)]{doubleflat}{$\musDoubleFlat$}{A ``double-flat'' accidental}
\newacronym[plural=$\musDoubleSharp$, firstplural=double sharp ($\musDoubleSharp$)]{doublesharp}{$\musDoubleSharp$}{A ``double-sharp'' accidental}
\newacronym[plural=$\meterC$, firstplural=common time ($\meterC$)]{commontime}{$\meterC$}{Common time}


