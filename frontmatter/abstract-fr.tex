\chapter*{R\'esum\'e}
\addcontentsline{toc}{chapter}{R\'esum\'e}
\label{chap:abstract-fr}


L'une des façons les plus courantes d'analyser un morceau de
tonalité la musique passe par l'analyse des chiffres
romains. Cela nécessite la inspection de plusieurs attributs
liés aux accords et aux tonalités. Les accords peuvent être
inspectés en fonction de leurs propriétés : fondamentale,
qualité, inversion et fonction. Les clés peuvent être
inspectées dans termes de leur portée temporelle comme des
modulations ou tonifiants. Chacun de ces attributs (ou
tâches) de Roman l'analyse numérique peut être modélisée
isolément. Cependant, des recherches récentes ont montré que
l'analyse de plusieurs tâches tonales conduit simultanément
à des informations musicales plus robustes Modèles de
récupération. Cela a motivé les recherches de réseaux de
neurones multitâches pour l'analyse des chiffres romains.
Dans cette thèse, je prolonge cet axe de recherche en : (1)
améliorer le processus de conservation des données pour les
ensembles de données existants ; (2) développer une nouvelle
technique d'augmentation de données pour Roman modèles
d'analyse numérique; (3) améliorer la conception des réseaux
de neurones récurrents convolutifs existants ; et (4)
extraire plus de tâches tonales du chiffre romain
annotations. En combinant ces idées, j'ai formé un nouveau
Romain modèle d'apprentissage automatique d'analyse
numérique. Parmi d'autres applications, cela facilitera la
recherche avancée dans recueils de musique. Par exemple, la
recherche par accord progressions ou par trajectoires de
modulation.
