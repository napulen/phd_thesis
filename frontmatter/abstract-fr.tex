\chapter*{R\'esum\'e}
\addcontentsline{toc}{chapter}{R\'esum\'e}
\label{chap:abstract-fr}

Pour analyser un morceau de musique tonale, on en étudie
généralement l'harmonie. Cela consiste en l'inspection de
divers attributs liés aux accords et aux tonalités. Un
accord est caractérisé par sa fondamentale, sa nature, son
renversement et sa fonction. La tonalité est caractérisée
par son étendue temporelle qui la définit comme une
modulation ou une tonicisation. Chacun de ces attributs de
l'harmonie peut être modélisé séparément. Cependant, des
recherches récentes montrent que l'analyse simultanée de
plusieurs tâches tonales conduit à une plus grande
robustesse des modèles d'extraction d'informations
musicales, ce qui incite à l'étude de réseaux neuronaux
multitâches pour l'analyse harmonique. Dans ce mémoire,
j'approfondis cet axe de recherche en développant un nouveau
modèle d'apprentissage automatique pour l'analyse
harmonique. Le modèle est basé sur un nouveau réseau
neuronal convolutif récurrent, entraîné sur un grand
ensemble de données d'annotations d'analyse harmonique
disponibles publiquement. Le modèle est assisté par une
nouvelle technique d'augmentation des données ainsi qu'une
configuration d'apprentissage multitâches inédite pour
l'apprentissage des attributs pertinents des accords et des
tonalités. En combinant ces idées, le modèle résultant
semble obtenir de meilleures performances sur la
caractérisation des accords rares par rapport aux méthodes
précédentes d'analyse harmonique automatique. Entre autres
applications, ce modèle facilitera la fouille, dans les
corpus musicaux et notamment, la recherche par progressions
harmoniques ou par parcours tonal.
