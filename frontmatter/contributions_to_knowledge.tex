\chapter*{Contributions to Original Knowledge}
\addcontentsline{toc}{chapter}{Contributions to Original Knowledge}
\label{chap:contributionstoknowledge}

This \thesisdiss{} is a multidisciplinary research effort at
the intersection of music theory and \gls{mir}.

\paragraph{Chapter 2}

This chapter contributes a historical review of the
\gls{rna} notation throughout multiple harmony books. This
is a contribution to the existing knowledge of this subject
because some of these relationships have not been explored
before. For example, the adoption of a certain \gls{rna}
symbol to denote a specific musical chord.

\paragraph{Chapter 3}

This chapter contributes a survey of \gls{mir} approaches
for \gls{arna}, \gls{gke}, \gls{lke}, and pitch spelling. 

\paragraph{Chapter 4}

This chapter describes the, to the best of my knowledge,
largest publicly available \gls{arna} dataset used for
supervised learning. The dataset is aggregated from publicly
available \gls{rna} annotations, and presents a short survey
of its underlying collections. This chapter also contributes
the description of a new data-augmentation technique
proposed in this dissertation.

\paragraph{Chapter 5}

This chapter describes the main \gls{arna} machine learning
model contributed in this dissertation. The chapter also
introduces an original \gls{mtl} configuration of tonal
tasks, which is distinct to existing \gls{arna} approaches.

\paragraph{Chapter 6}

This chapter contributes the experimental evaluation of the
machine learning model proposed in the dissertation. The
evaluation considers ablation studies to explore the
underlying components of the proposed \gls{crnn} model.
Furthermore, it presents an original experiment illustrating
the effects of two data-augmentation techniques. Lastly, it
compares five \gls{arna} models using the same evaluation
framework. Both the number of methods compared (5) and the
common evaluation framework are a contribution to existing
research.
