\chapter*{R\'esum\'e}
\addcontentsline{toc}{chapter}{R\'esum\'e}
\label{chap:chap0-res}

Cette th\'ese propose un flux de travail automatique pour l'analyse tonale multi-t\^aches de la musique occidentale de pratique courante. 
Dans ce workflow, les annotations de chiffres romains sont d\'ecompos\'ees en plusieurs probl\`emes de classification tonale. 
Ces probl\`emes de classification sont r\'esolus simultan\'ement en utilisant un r\'eseau neuronal profond avec une configuration d'apprentissage multi-t\^aches.

L'analyse des num\'eraux romains est un cadre analytique enseign\'e dans les cours de th\'eorie musicale tonale.
Dans ce cadre, une grande quantit\'e d'informations sur le contexte tonal d'un accord est r\'esum\'ee à l'aide d'une syntaxe compacte. 
Par exemple, une annotation comme \texttt{C:viio65/V} transmet des informations sur une tonalit\'e locale de Do-majeur, un accord de septi\'me diminu\'ee de Fa$\sharp$ en premier renversement, et une tonification de (ou un accord appliqu\'e dans) la tonalit\'e de Sol-majeur.
Ainsi, les annotations en chiffres romains fournissent une description compl\`ete, en particulier, mais pas exclusivement, des changements de tonalit\'e et des accords.