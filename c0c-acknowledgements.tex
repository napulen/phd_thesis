\chapter*{Acknowledgements}
\addcontentsline{toc}{chapter}{Acknowledgements}
\label{chap:chap0-ack}

I will first and foremost thank my PhD advisor, Ichiro Fujinaga.
For the most part of my PhD, Ichiro scheduled weekly meetings with me that were of utmost help. 
Together, we walked through machine-learning prototypes, presentations, code, music theories, and papers. 
Research aside, Ichiro has been a supportive friend since I moved to Montr\'eal. 
He cared about my spouse's and my general health and wellbeing, and for that I am very grateful. 
Thank you, Ich.

Secondly, I would like to thank my family, who have supported me all along my journey of pursuing my dreams: my spouse and best friend, Kinga; my parents, Martha and Eduardo; my siblings, Grecia, Eric, and sister-in-law Fabiola; and my parents in law, Gra\.zyna and Miros\l{}aw.

Studying in Montr\'eal was a very pleasant experience for me, mostly due to the extraordinary people I met.
Particularly, thanks to Gabriel Vigliensoni, Martha Thomae, Matan Gover, Yuval Adler, Alex Daigle, Laurent Feisthauer, Yulia Draginda, Yaolong Ju, Tim de Reuse, Emily Hopkins, Sylvain Margot, Jacob Sanz-Robinson, K\'e Zhang, Claire Arthur, Jorge Calvo Zaragoza, and many others with whom I shared conversations and laughs.

Thanks to McGill professors Julie Cumming, Peter Schubert, William Caplin, Jon Wild, Philippe Depalle, Marcelo Wanderley, and Gary Scavone, from whom I learned different aspects of music and technology that contributed explicitly or implicitly to the work presented in this dissertation.

Thanks to other members of the music information retrieval and computational musicology community, particularly, Mark Gotham, Craig Sapp, Michael Scott Cuthbert, Jacob Tyler Walls, Reinier de Valk, Emilia Parada-Cabaleiro, and others with whom I shared code, papers, and points of view.

Thanks to my co-authors, with whom I discussed important aspects of the research presented here: Gabriel Vigliensoni, Claire Arthur, Laurent Feisthauer, Florence Lev\'e, Mark Gotham, and the Computational Tonal Studies group.

The research presented in this dissertation would not have been possible without the generous support of different institutions. The \emph{Fonds de recherche --- Soci\'et\'e et culture} (FRQSC), from which I obtained a doctoral scholarship from 2019--2022 (dossier number 271479); the \emph{Centre for Interdisciplinary Research in Music Media and Technology} (CIRMMT), from which I obtained several travel awards and a student project award; the SIMSSA project, from which I received a scholarship during the first two years of my PhD (2017--2018); the Schulich School of Music, from which I received additional travel and conference registration awards. Thank you to all these wonderful organizations and their commitment to music research.

A special place for people who have made a mark in my life: Luis Alberto Casillas Santill\'an, Abelardo Guti\'errez, Juan Jos\'e L\'opez Sandoval, Nancy Arana Daniel, Sergio Bola\~nos, Leo Hendrik Reyes, and Juan Rodrigo P\'erez C\'ardenas.

Lastly, my friends, for being always there, especially in the darkest of times: Jhonnatan Razo, Dulce Alcal\'a, and El\'i Lezama. ¡Gracias, amigos!