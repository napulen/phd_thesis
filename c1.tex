\chapter{Introduction}
\label{chap:introduction}

% \begin{quote}
    
% \end{quote}
% \clearpage

% \section{Goal}

The main objective of this dissertation is to propose improvements over the existing methods for automatic tonal analysis, tailored around the \emph{common-practice} period of Western music. 

In recent years, improvement has been observed using multi-task learning (MTL) workflows for tonal analysis.
This research extends over that, proposing additional ways of doing data augmentation, and strengthening the shared representations of MTL approaches with additional tonal tasks, and proposing alternative neural network architectures.

\section{Motivation}

\guide{Why did people invent harmony theories and counterpoint theories.}
I have studied Western tonal music theory in different levels with different instructors throughout my life.
It is unclear to me, \emph{why} is it that Harmony and Counterpoint theories exist and have prevailed to this date.
Is it that these theories were aimed at facilitating the training of new music composers? 
To understand how previous composers thought about music? 
To disentangle the patterns in existing music? 
To understand how music ``works''? 
To simplify the perceptual organization of simultaneous notes into a set of rules that can be more easily understood?
In any case, many theories have emerged over the years.
One of the meta-theories we use nowadays, in Western tonal music, is Roman numeral analysis.
An analytical tool, in principle, Roman numeral analysis is useful as a framework for a theorist or musician to describe the harmonic context of the music.
This is necessary as more complex tonal relationships (e.g., modulations) are introduced in the music composition process.
When moving to the digital domain, we can still think of this framework as an analytical tool that allows us to understand music in a simpler way.
A summary of the harmonic context.
My motivation with this dissertation is to connect the theories that have evolved over the years, with the ruthless and methodological outputs of computer algorithms.
Needless to say, I am not the first to do this, and as the saying goes, I stand in the shoulders of giants to see any further.

\guide{Why do we study these things computationally.}
The reason why I think it is important to study tonality computationally is because music easily evokes human bias.
It is easy to summon the personal bias when listening to a piece of music.
Likewise, when analyzing a piece of music.
I like to think of computer algorithms as unbiased judges of the theory. Emotionless, deterministic.
If the theories generalize well to the musical repertoire, it will show in the output of an algorithm.
If not, it is likely that the output will reveal where the theory fails to explain the musical phenomenon.
In this thesis, it is my intention not only to present my experiments designing a neural network for tonal music analysis, but also the interpretation of the results.
For example, when a state-of-the-art neural network is trained to recognize modulations, what is it that the internal representations of the network (i.e., the hidden layers) are learning?
To my knowledge, this aspect is generally less explored in machine learning literature, where the focus is on improving the evaluation metrics of new algorithms versus old algorithms.
As a theorist, I want to know what is it that the ruthless optimization method is learning when it provides meaningful outputs of tonal music analysis.
The multi-task tonal analysis done here, although it consists of multiple tasks, is focused around two specific problems: chord analysis and key analysis.
These problems have been studied thoroughly in isolation and, sometimes, in conjunction.
I summarize existing methods for chord estimation and key estimation.

% Background

Automatic Chord Recognition (ACR) has been explored thoroughly in the field of Music Information Retrieval (MIR). 
ACR systems typically seek to predict the root and quality of the chords throughout a piece of music via either an audio or a symbolic representation.
A more specific type of chordal analysis, particularly relevant for Western classical music, is functional harmony, or Roman numeral analysis. 
The main difference between ACR and functional harmony is that the latter requires other adjacent tasks to be solved simultaneously, notably including detection and identification of key changes (modulations \cite{feisthauer_estimating_2020, schreiber_local_2020} and tonicizations \cite{napoles_lopez_local_2020}).

% Partly due to the repertoires typically targeted, Roman numeral analysis also involves handling of some additional `special chords' such as Neapolitan and Augmented sixth chords.

\guide{Roman Numeral Analysis.}
The analytical process of functional harmony is commonly described through Roman numeral annotations. 
This annotation system is particularly popular in Western music theory for the analysis of `common-practice' tonal music.
Roman numeral annotations encode a great deal of information about tonality, in a compact syntax.
For instance, an annotation like  \texttt{C:viio65/V} includes an account of the local key (here C), the quality of the chord (diminished seventh), the chord inversion (first), and nature of any tonicization (optional, here true: of the dominant). 

\guide{Problem.}
From a computational perspective, predicting such annotations is challenging given that the model has to predict multiple features correctly and simultaneously. 
In the past, MIR researchers have reconstructed Roman numeral annotations by predicting six sub-tasks: chord quality, chord root, local key, inversion, primary degree, and secondary degree \cite{chen_functional_2018, micchi_not_2020}. 
%
Thus, as a machine learning problem, functional harmony can be expressed as the task of correctly predicting enough features in order to reconstruct the original Roman numeral label. 

% I would cut this duplication of the above: In spite of their compactness, Roman numeral strings are sufficient to provide all of these features, and thus show to be convenient as an annotation system for creating functional harmony datasets.

Recent efforts in this area have seen a great standardization of the notation and conversion routines, \cite{gotham_romantext_2019} which in turn has facilitated the amassing of a relatively large meta-corpus of Roman numeral analyses \cite{gotham_romantext_2019}.
However, despite these developments and the wider resurgence of interest in the field, the performance of functional harmony models for predicting full Roman numeral labels remains relatively low.

\guide{Solution.}
In this paper, we propose a new neural network architecture that improves the prediction of functional harmony and its relevant features. 
Beside the architecture itself, our model benefits from increased data augmentation (beyond key transpositions), and an additional set of output tasks that enhance the effects of multi-task learning demonstrated by other researchers \cite{chen_functional_2018}.

To facilitate the work of other researchers, we release all of our preprocessed datasets, data splits, experiment logs, and the full source code of our network in [ANONYMIZED], under a permissive MIT license.


\section{Research objectives}


\section{Thesis structure}
This thesis is organized in X chapters. 
The next two chapters approach the literature review on the topic.
Chapter \missing{chapter number} explains the literature from the perspective of Music Information Retrieval and Computational Music theory.
Chapter \missing{chapter number} does so from the perspective of music theory.
Chapter \missing{chapter number} addresses what is necessary to mention about \emph{data} throughout this work.
This include the sources of information, statistics about datasets, encoding and data-curation strategies, and any other methods relevant for data preparation.



