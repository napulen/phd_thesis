\chapter{Recurrent Neural Networks (RNNs)}
\label{chap:chap9}

\begin{quote}
    Why is learning in standard Recurrent Neural Networks (RNN) difficult? How did researchers overcome this problem? Describe the differences between standard RNN and Long Short-Term Memory (LSTM). What are the advantages of using LSTM over the standard RNN? Devise and describe the architecture of an LSTM and its input format to detect modulation (i.e., change in local key). What are the hyperparameters for the LSTM and what are the strategies for choosing them? [4-6 pages]
\end{quote}
\clearpage

% Certain phenomena (music, for example) have a strong dependence on time.

% Explaining the meaning of a given input (for example, a note) does not only depend on the input, but on the time where that input occurs.

% In the research related with Artificial Neural Networks, researchers have tried to investigate these type of phenomena using Recurrent Neural Networks (RNN).

% An RNN is trained with sequences of inputs, rather than just inputs.

% They are useful for many different problems.

Recurrent Neural Networks (RNNs) are a special type of ANNs (see Question 2 for more information on ANNs) where the inputs are not considered to be independent but part of a sequence. This distinction allows the network to model time-varying processes and tasks that require sequence-to-sequence models (e.g., language, music, and weather). An additional benefit of RNN archictures is that, unlike other types of deep learning networks such as Convolutional Neural Networks (CNNs), RNNs can process inputs of arbitrary length. This has made CNNs the most robust choice for fixed-size grid-like data (e.g., images) and RNNs a popular choice for tasks where the length of the input is unknown. 

Although modern RNNs have achieved state-of-the-art accuracy in many tasks that involve sequential inputs, this was not easily achieved and required many years of innovations. The most important aspect in which RNNs have seemed to take more time to progress in comparison to other deep learning techniques, is the successful training of the networks, which seemed to be very unstable and remains one of the biggest challenges of research in RNNs.

\subsection{Training RNNs}

Over the years, RNNs have gained a reputation as architectures that are very difficult to train \cite{pascanu_difficulty_2013}, given that many attempts of training such models have failed. For example, in 1994, experiments showed that as the span of dependencies that need to be captured by the RNN increases, the probability of successfully training the network via the Stochastic Gradient Descent (SGD) optimizer rapidly reaches 0 for sequences of 10 or 20 time steps \cite{bengio_learning_1994}. 

The main difficulty of training an RNN is something known as the \emph{vanishing} and \emph{exploding} gradients. When the gradients vanish---something very common---the network is unable to learn long-term dependencies and can only model the dependencies of inputs that were provided a few time steps in the past (rather than modelling the entire sequence, which is usually the goal). When the gradients explode---something less common but more damaging---the convergence of the training is compromised because the large value of the gradient occludes the search of a local minimum by the optimizer algorithm.

One of the reasons why RNNs are more prone to vanishing and exploding gradients is their heavy reliability on \emph{parameter sharing}. Other architectures (e.g. CNNs) also share parameters between different neurons in order to reduce the number of parameters and, hence, the effort of training the network. RNNs, however, make a heavier use of parameter sharing, as their input is usually a fixed-size vector designed to receive the time steps of the input sequence, one at a time. This design decision implies two things: 1) the parameters of the network are shared for every input of the sequence, 2) the parameters need to be updated for every time step of every input sequence.

The second of these design decisions is what mostly contributes to the vanishing and exploding gradients. In other architectures, it is expected that the parameters will be updated once per training example, while in RNNs the parameters are updated several times per training example. 

When the RNN is updating its shared parameters, it is very easy that these parameters grow out of control (explode) or drop to zero really fast (vanishing), similar to how a scalar number would explode or vanish if raised by a very large exponent (e.g., $0.5^{1000}$ or $10^{1000}$).

By making use of modern solutions, however, RNNs have been able to successfully learn long-term dependencies (in the order of hundreds or thousands of time steps \cite{hochreiter_long_1997}), making them practical for many tasks. There have been many proposed solutions \cite{el_hihi_hierarchical_1995, yildiz_re-visiting_2012, jaeger_long_2012}, however, the most successful ones are the \emph{gated RNNs}. Particularly, a type of gated RNN known as the Long Short-Term Memory (LSTM) architecture.

\subsection{Long Short-Term Memory (LSTM)}

The Long Short-Term Memory architecture was introduced in 1997 by Hochreiter and Schmidhuber \cite{hochreiter_long_1997}. Since its introduction, the LSTM has become an increasingly popular architecture, which has been used in many tasks and problems. LSTM architectures extend the basic structure of RNNs and it is relatively safe to assume that any problem that involves an RNN can substitute the RNN with an LSTM, matching or even improving the results.

The main contribution of LSTMs is to try to provide a solution to the issue of vanishing gradients. Given that vanishing gradients are associated with the loss of long-term dependencies, mitigating this issue would enable, in theory, the capability of the network to learn long-term dependencies between the inputs. In order to achieve these long-term dependencies, an LSTM substitutes the regular RNN unit with a more complex one. This new, more complex unit, receives the name of an \emph{LSTM unit}.

The main difference between a regular RNN unit and an LSTM unit is the addition of a cell and three gates: the forget gate, the input gate, and the output gate. In conjunction, the cell and the gates allow the network to control the flow of the information, selecting what information should the network store and what information should it forget. 
The gates of the LSTM unit act as ``regulators'' and each of them is responsible of a different part of the network. The forget gate controls what information of previous time steps in the network should be kept and what should be forgotten. The input gate regulates how much of the new inputs to the network should enter the cell. The output gate controls how much of previous time steps in the network should be used for computing the activation of the output of the network.

Given that RNNs and more specifically LSTMs are useful architectures for modelling sequence-to-sequence processes, they can be useful for musical applications, for example, detecting the changes of key in a symbolic music file (modulation).

% memory cell is that the memory cell has gates that allow the gradient of the unit to be controlled and always norm 1. These gates keep the value of the memory cell under control. 

% Further improvements on LSTMs have extended the elements of the LSTM, for example, adding a forget cell, which allows the memory cell to drop information that is no longer relevant.

% \subsection{Clipping}
% One of the most successful techniques for controlling the vanishing/exploding gradients problem.

% The gradient is computed as usual, whenever that gradient exceeds a threshold value, it is clipped.

% There is a mathematical proof of why this works, developed by Tenescu.

% \subsection{Modern RNNs}
% Together with the improvements on the vanishing/exploding gradients problem, other researchers have also contributed with other ideas.

% For example, during the same year that the LSTM was proposed, another paper proposed a bidirectional RNN that could be used in offline systems (as it requires knowledge of the future, it can’t be used in real-time and other online applications).

% These advances in different fronts have made not only addressed problems related with training but also improved the capabilities and performance of RNNs, many of them are not mutually exclusive with each other, so they can be combined.

% Therefore, it is common to see now architectures that combine these approaches, for example, BLSTM (Bidirectional Long Short-Term Memory) networks.

% These type of models have been successfully trained and provided state-of-the-art in multiple tasks.


\subsection{Designing an LSTM for detecting modulation}

In order to design a system for the detection of ``modulations'', the first step is to define the expectations of such a system in terms of its inputs and outputs. Particularly, given that the term ``modulation'' is very difficult to define, I will consider referring to it as a \emph{local key} finder instead (see Question \ref{chap:chap6} for a further discussion on modulations and local keys).

A local key-finding system produces an output at every time step. The output consists of one of the 24 major and minor keys. That is, the local key for any given time step. The time step units consist of onset events in the symbolic music input (i.e., attacks). The inputs of the system could be the range of valid MIDI note numbers (0-127) in a one-hot encoding scheme, yielding a fixed input vector of dimension 128 for every time step. Nevertheless, given that some local-key-finding models have been successfully implemented without octave information \cite{napoles_lopez_key-finding_2019}, the input vector can be substituted by a 12-dimensional pitch-class vector in a one-hot encoding scheme. This reduces the number of parameters by an order of magnitude and, therefore, is expected to reduce the computational cost of training the model.   

The system consists of a single hidden layer with a recurrent LSTM unit. The gates of the LSTM unit use a sigmoid function as their non-linear activation function and the input to the network uses a \emph{Rectified linear unit} (ReLU) as its activation function.

As a proof of concept and given the scarce data available for training the network, an existing probabilistic model that requires no training \cite{napoles_lopez_key-finding_2019} is used for generating local key annotations in a corpus of 892 MIDI files of classical music. The success criteria of this LSTM system is, therefore, to provide a similar local-key segmentation on unseen MIDI examples as the segmentation provided by the baseline, probabilistic model.

\bibliographystyle{plainnat}
\bibliography{zoterorefs}