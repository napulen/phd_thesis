\chapter*{Abstract}
\addcontentsline{toc}{chapter}{Abstract}
\label{chap:chap0-abs}

This thesis proposes an automatic workflow for multi-task tonal analysis of Western common-practice music. 
In this workflow, Roman numeral annotations are decomposed into multiple tonal classification problems. 
These classification problems are solved simultaneously using a deep neural network with a multi-task learning configuration.

Roman numeral analysis is an analytical framework taught in tonal music theory courses.
In this framework, a great deal of information about the tonal context of a chord is summarized using a compact syntax. 
For instance, an annotation like  \texttt{C:viio65/V} conveys information about a \emph{local key} of C-major, an F$\sharp$ diminished seventh chord in first inversion, and a \emph{tonicization} of (or applied chord in) the key of G-major.
As such, the Roman numeral annotations provide a thorough description of, particularly, but not exclusively, changes of key and chords.