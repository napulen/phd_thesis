\chapter{Artificial Neural Networks (ANNs)}
\label{chap:chap2}

\begin{quote}
    Provide a literature review of Artificial Neural Networks (ANN), in general—--including Recurrent Neural Networks (RNNs), autoencoders, and U-Net—--and their applications in music information retrieval. [4-6 pages]
\end{quote}
\clearpage

Artificial Neural Networks (ANNs) are machine learning algorithms that learn arbitrary functions by automatically training weights (or parameters) that connect the nodes in the neural network architecture. A non-linear activation function is applied to such weights, introducing a non-linear behaviour in the neural network that allows it to learn functions of higher complexity, which a linear model could not possibly learn. The training of the weights is not achieved by programming task-specific rules but instead, by identifying simple characteristics of the training examples and extending them into more complex, more abstract characteristics. The process of decomposing an example into a combination of simpler features is known as \emph{representation learning} and it is one of the main ideas that differentiate ANNs from other classes of machine learning. 

The study of ANNs started around the 1940s and it has been known through different names throughout the years \cite{goodfellow_deep_2016}.

\section{A brief history of ANNs}

The research on ANNs can be traced back to the 1940s, when a bio-inspired \emph{neuron} model was introduced \cite{mcculloch_logical_1943}. This neuron allowed to model very simple functions by manually setting the weights that connected the input into the neuron. This idea was later extended to propose the Perceptron \cite{rosenblatt_perceptron:_1958} and Adaline \cite{widrow_adaptive_1960} models, which were able to automatically derivate such weights from the data. Although these models showed promise, their popularity decreased significantly when it was demonstrated that they could not learn relatively simple functions, like the \emph{XOR} function \cite{minsky_perceptrons:_1972}. Due to their importance, the achievements carried out during this wave of research (1940s-1960s) are acknowledged by the current literature \cite{goodfellow_deep_2016} and usually referred to as the \emph{cybernetics} wave of neural networks research.

Following the wave of cybernetics, a new one started around the 1980s-1990s, colloquially known as \emph{connectionism}. During the work of the \emph{connectionists},\footnote{A term typically used to refer to the scientists of this time period and research field} the research community benefited from the development of the current form of the back-propagation algorithm \cite{rumelhart_learning_1988}. The back-propagation algorithm became (and remains) an elemental process in the training of neural networks, which allows to propagate the error throughout the network by making use of the \emph{chain rule}. Finding the derivatives of each parameter in the network, the values of such parameters can be updated in the ``right direction'' (against the gradient) to improve the classification accuracy through the next batch of training examples. This facilitates the automatic training of large and complicated neural networks, with multiple layers, neurons, and non-linear activation functions. Even though this and other improvements made neural networks a promising area of research, they were still very difficult to train in practice (due to the difficulty of finding a good initialization of the weights) and were typically outperformed by domain-knowledge techniques, losing the interest of many scientists as a consequence.

Finally, a third wave of research started around 2006, when new methods for training neural networks were introduced \cite{hinton_fast_2006}. These new methods not only facilitated the training of neural networks but the training of \textbf{much larger} neural networks. The interest in such larger architectures extended, and in a historical evaluation of the ImageNet dataset in 2012 \cite{krizhevsky_imagenet_2012}, neural networks outperformed the most sophisticated methods of computer vision, setting a prominent gap in performance between neural networks and every other method. This had an enormous implication in the way that neural networks were perceived by the research community and motivated their application into different problems and fields of study. We know this last wave of research as \emph{deep learning}, and it is currently an active and growing wave of research across many fields. Around this umbrella term of \emph{deep learning}, many state-of-the-art machine learning techniques have been developed and continue to be improved.

\section{Deep learning and Music Information Retrieval}
After the growing interest for neural networks in the wave of deep learning research, many new models, architectures, and applications have been proposed and put into practice in recent years.
%achieving good results and generating subfields of research within the umbrella term of deep learning.
Among the most important innovations to the original neural network architectures, we can consider Convolutional Neural Networks (CNNs), AutoEncoders, Recurrent Neural Networks (RNNs), and extensions of CNNs, such as the U-Net.

\subsection{Convolutional Neural Networks (CNNs)}
Convolutional Neural Networks (CNNs) might seem recent, however, they were introduced during the \emph{connectionist} wave of research on neural networks, in 1989 \cite{lecun_generalization_1989, le_cun_handwritten_1989}. The main innovation of CNNs is the idea of \emph{shared parameters}. In a traditional feedforward network (e.g., Multi-Layer Perceptron), every neuron of the network is typically connected to another neuron of the network in the following layer using a \emph{unique} parameter (i.e., used exclusively for connecting those two neurons). Although that gives the network more expressive power and the capability of modelling very complex functions, in practice, it also contributes to a combinatorial explosion of parameters as the neural network grows in number of layers and neurons, which makes it unfeasible (or even impossible) to train it due to the limitations in memory and computing power of modern computers. CNNs, on the other hand, consider re-using the same parameter for connecting different neurons of the network with the neurons of the following layer. By doing this, the number of parameters is reduced, typically, in one or several orders of magnitude compared to a fully-connected, feedforward network.

The idea of sharing parameters is not only important for reducing the effort of training the network, it is also a bio-inspired design motivated by the mechanics of the visual system. It is customary, for example, to refer to the collection of neurons that make use of the same parameter as the \emph{receptive field} of the parameter, a term taken from neurophysiology. 

The shared parameters are modelled through a \emph{kernel} vector. The kernel vector multiplies the inputs of the neural network layer in a way that resembles the mathematical operation of \emph{convolution}, which motivated the use of the term \emph{Convolutional Neural Networks}. After training, it is assumed that each of those kernels will learn a low-level, localized, feature, which is going to be searched across the entire input vector of the network and propagated to deeper (higher-level) kernels of the network.

Given the way that convolutional kernels work, CNNs have become the standard methodology for dealing with fixed-length, grid-like structures (e.g., images), producing state-of-the-art performance in many tasks. For example, they have systematically been the state-of-the-art in the ImageNet challenge since 2012 \cite{krizhevsky_imagenet_2012}, identifying over 1,000 classes of objects in an image.

In Music Information Retrieval (MIR), CNNs have been used for genre recognition \cite{dieleman_audio-based_2011}, chord recognition \cite{humphrey_rethinking_2012}, structural analysis \cite{ullrich_boundary_2014, grill_music_2015}, music tagging \cite{choi_automatic_2016}, instrument recognition \cite{lostanlen_deep_2016}, Optical Music Recognition (OMR) \cite{calvo-zaragoza_end--end_2017, pacha_optical_2018}, beat tracking \cite{gkiokas_convolutional_2017}, source separation \cite{miron_monaural_2017}, syllable segmentation \cite{pons_score-informed_2017}, key detection \cite{korzeniowski_genre-agnostic_2018}, and tempo estimation \cite{schreiber_single-step_2018, schreiber_musical_2019}.

\subsection{AutoEncoders}
AutoEncoders are models designed to copy the input signal into the output, considering a number of restrictions. This process is achieved through two steps, the first corresponds to the \emph{encoding} of the input into a \emph{latent} representation or \emph{code}, and the second step corresponds to the \emph{decoding} of the latent representation (code) into the output.

One of the most useful restrictions imposed to the AutoEncoder architecture is to reduce the size of the latent representation compared to the size of the input. By doing this, the model is forced to learn a restricted amount of data that \emph{characterizes} the input well enough so that the decoder part of the model is able to reconstruct the original signal as much as possible. This can also be thought as a \emph{compression} of the input signal or a reduction of the dimensionality of the input.

Other restrictions imposed to AutoEncoders are the reconstruction of the original signal given a \emph{distorted} or \emph{noisy} input signal. This is usually known as a Denoising AutoEncoder (DAE). AutoEncoders can be thought as unsupervised neural networks given that the training mechanisms are practically identical to the ones used in other neural network architectures. They are also useful in combination with other neural network architectures, as the latent representation of the AutoEncoder can be used as an input feature, for example.

In MIR, AutoEncoders have been used, for example, in Optical Music Recognition (OMR) problems \cite{castellanos_document_2018}.

\subsection{Recurrent Neural Networks (RNNs)}
Recurrent Neural Networks (RNNs) are a type of neural networks designed for dealing with sequential data. Unlike most other neural network architectures, RNNs do not assume that inputs are \emph{independent} from each other and, therefore, they update their parameters considering not only the current input to the network but also the previous inputs processed by the network.

RNNs were introduced after the backpropagation algorithm \cite{rumelhart_learning_1988} was extended into the \emph{Backpropagation Through Time} (BPTT) algorithm, around 1988 \cite{werbos_generalization_1988, werbos_backpropagation_1990}. Nevertheless, the difficulty of training such RNN architectures made them unfeasible in practical applications before the invention of the Long Short-Term Memory (LSTM) architecture \cite{hochreiter_long_1997}. They were popularized during the \emph{deep learning} wave of research and, since then, achieved state-of-the-art performance in many tasks involving sequential data (e.g., speech recognition, natural language processing, music).

Throughout the years, different strategies have been proposed to design RNNs, for example, connecting the output of one time step into the next time step (Jordan RNN), connecting the hidden state of one time step to the next (Elman RNN), and training the network in both directions \cite{schuster_bidirectional_1997}.

It is also not uncommon to see several of these techniques combined in a single architecture, for example, a bidirectional LSTM architecture with convolutional layers (CBLSTM). For a further discussion on RNNs, see Question \ref{chap:chap9}.

In MIR, RNNs and hybrid RNN models with convolutional layers have been quite popular in numerous tasks, for example, onset detection \cite{eyben_universal_2010}, chord recognition \cite{boulanger-lewandowski_audio_2013, sigtia_end--end_2016, sears_evaluating_2018}, voice separation \cite{huang_singing-voice_2014}, music transcription \cite{sigtia_rnn-based_2014}, tempo estimation \cite{bock_accurate_2015}, beat and downbeat tracking \cite{bock_joint_2016, krebs_downbeat_2016}, music generation \cite{liu_predicting_2016, liang_automatic_2017, lim_chord_2017}, music transcription \cite{rigaud_singing_2016, sigtia_end--end_2016, southall_automatic_2016, vogl_recurrent_2016, southall_automatic_2017, vogl_drum_2017, basaran_main_2018}, OMR \cite{calvo-zaragoza_one-step_2017, wel_optical_2017, calvo-zaragoza_camera-primus:_2018}, sequence modelling \cite{ycart_study_2017}, mood detection \cite{delbouys_music_2018}, and instrument recognition \cite{gururani_instrument_2018}.

\subsection{U-Net}
Often, the problems that are expected to be solved by machine learning algorithms require more than identifying the presence of an object in a given input image. This is the case, for example, with image segmentation, where an output class needs to be provided for every pixel in the image.

For such problems, a traditional CNN architecture is not feasible, mainly, because the information of the \emph{location} of the target class is lost through the convolutional layers of the network. More specifically, the \emph{pooling} layers of a CNN facilitate the identification of the strongest features that have been found incrementally across the image, at the expense of losing the information regarding \emph{where} they have been found.

The U-Net is a modification of the traditional CNN architecture, designed to deal with image segmentation problems and making heavy use of data augmentation \cite{ronneberger_u-net:_2015}. It was introduced in 2015 for the segmentation of images in the field of biomedical image processing and, since its introduction, has been applied to numerous tasks in different fields.

The U-Net architecture consists of two stages, the first one corresponds to a series of convolution-nonlinearity-and-pooling layers, analogous to a regular CNN (referred to as the contracting path). The hidden layer that results after several of these convolutional layers is then expanded back to almost the size of the original image (but smaller than the original). The expansion is achieved by replacing the \emph{pooling} layers with up-convolutions. A cropped version of the corresponding hidden layer in the contracting path is copied onto the expanding path, giving the characteristic u-shape of the architecture.

In MIR, models based on the U-Net have been proposed for source separation \cite{jansson_singing_2017, stoller_wave-u-net:_2018} and OMR \cite{hajic_towards_2018}.

\subsection{Other deep learning architectures in MIR}
Throughout the years, the solutions presented in the research field of deep learning have been applied to multiple Music Information Retrieval (MIR) tasks. The most popular ones have already been discussed, however, other applications include the historical Self-Organizing Maps (SOMs) in the early 2000s \cite{kiernan_score-based_2000, harford_automatic_2003}, Deep Belief Networks (DBNs) \cite{hamel_learning_2010, schmidt_learning_2013, chacon_developing_2014, raczynski_multiple_2010, battenberg_analyzing_2012, herwaarden_predicting_2014, zhou_chord_2015}, and deep feedforward networks \cite{cherla_multiple_2014,  liang_content-aware_2015, dawson_key-finding_2018, valk_deep_2018}.

\bibliographystyle{plainnat}
\bibliography{zoterorefs}