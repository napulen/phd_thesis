% Copyright 2022 Néstor Nápoles López

In the work presented at \textcite{napoleslopez2020local},
we discussed the problem of a lack of definition of
modulations and tonicizations, which has implications for
\gls{mir} research. In that work, different harmony
textbooks featured different degrees of consistency in the
use of modulations and tonicizations. Considering that these
represented  easy ``textbook'' examples of modulations, it
is only logical to expect a greater inconsistency in more
complicated corpora, such as the dataset used in this
dissertation. This is a difficult problem, but at the same
time, a fascinating phenomenon of tonal music, the
fluctuation of musical keys, how they are perceived by the
human ear, and the way they are annotated in an analytical
framework like \gls{rna}. This venue of research probably
requires more involvement of the \gls{mpc} and music theory
communities. I am less hopeful that the \gls{mir} will, on
its own, propose general solutions to quantify key changes.
