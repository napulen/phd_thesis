% Copyright 2022 Néstor Nápoles López

The proposed system and others such as the one by
\textcite{micchi2021deep} and \textcite{mcleod2021modular}
operate exclusively in the symbolic domain. One of the
advantages of the representation proposed here for pitch
information is that it is composed of the combination of
pitch classes and generic note letters:

\begin{equation}
    \{0, 1, 2, 3, 4, 5, 6, 7, 8, 9, 10, 11\} \times \{C, D, E, F, G, A, B\}
\end{equation}

The pitch-class part of this representation can be
substituted by a chromagram vector in the audio domain,
enabling different sorts of experiments. For example,
training the network with only pitch classes in the audio,
symbolic, or both domains. In the past, we have been
successful implementing a hybrid model for symbolic and
audio representations in \textcite{napoleslopez2019effects}.
A few preliminary experiments indicate that this would be
possible in \gls{rna} as well, but more formal experiments
are needed.
