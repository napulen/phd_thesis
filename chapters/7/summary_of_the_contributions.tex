% Copyright 2022 Néstor Nápoles López

Throughout the dissertation, various contributions towards
\glspl{rna} have been made in different dimensions of the
problem. \refchap{introductiontoromannumeralanalysis}
presented a historical analysis of the \gls{rna} syntax.
Although several accounts of the history of harmony
exist,\footnote{See, for example,
\textcite{wason1985viennese}, \textcite{grave1988praise},
\textcite{christensen2002tonality},
\textcite{laitz2010graduate}, or
\textcite{sansallovich2013quintas}} most of these do not
focus on the syntax of \gls{rna} annotations. The review
presented in this dissertation focuses solely in the use of
symbols, conventions, and specificities of the \gls{rna}
notation that are relevant for its digitization and
standardization.

\refchap{dataacquisitionandpreparation} documents the
aggregation process of the, to the best of my knowledge,
largest publicly available dataset of \glspl{rna} labels.
The process to overcome the alignment between annotations
and scores is discussed through the use of quantitive
metrics and other resources that facilitate the aggregation
of the data. Furthermore, in \refsubsec{preprocesseddata} of
this chapter, the preprocessed data is made publicly
available for other researchers who may be interested in
using it. The distributed dataset includes the explicit
splits into training, validation, and test portions used for
the experiments conducted in this dissertation. Lastly, in
the context of data augmentation, a new
technique\footnote{First introduced in
\textcite{napoleslopez2021augmentednet}.} based on the
synthesis of training examples was presented in
\refsubsec{synthesisofartificialexamples}. Although less
impactful than the more commonly used transposition, it
presents the advantage that both can be used in combination,
increasing the performance obtained by the network on the
same dataset with both techniques in combination.

\refchap{modeldesign} describes a novel neural network
architecture for \glspl{rna}. The architecture is based on a
previous version presented in
\textcite{napoleslopez2021augmentednet}, however, the
structure of the \gls{mtl} configuration presented here
shows several improvements. Unlike other models, a set of
tasks that have a more balanced distribution of target
classes is proposed for different components. For example,
rather than predicting chord qualities, which are heavily
skewed towards major and dominant seventh examples in the
dataset, chords are modeled by predicting four note
classification tasks.\footnote{Referred throughout the
dissertation as the \gls{satb35} tasks} Each of these tasks
represent a note in a \gls{closed-position} representation
of the annotated chord. This is an equivalent task to
predicting the notes that conform the chord, rather than the
chord, which leads to a more balanced distribution of the
target classes in a supervised learning scenario. Other
contributions of the model include the use of a
\gls{pcset121} task, which summarizes the chord vocabulary
of the system, a \gls{harmonicrhythm7} task based on 7
classes of outputs, and a vocabulary of 31 Roman numeral
numerators, \gls{rn31}. The input representations sent to
the neural network also present contributions. For example,
the \gls{duration14} encoding of note and measure onsets
improved the segmentation of the chords in the ablation
studies shown in \refsec{bestperformingmodelconfiguration}.
The alternative representation of pitch spelling described
in \refsubsubsec{19two-hot-encoding} was also shown to be an
adequate subtitute for the more common representation in
\refsubsubsec{35one-hotencoding}, without a significant loss
of performance.

\refchap{experimentalevaluation} introduced contributions
regarding the evaluation of a \glspl{rna} system. For
example, the performance on rare chords was discussed among
recent approaches in \refsubsec{romannumeralaccuracy}. This
form of evaluation revealed that some of the chords have a
tiny representation in the dataset, however, represent the
chords and tonal situations that most often benefit from the
\gls{rna} notation. In this respect, one of the
contributions is to propose a system that can improve on
such difficult cases, and to indicate to other researchers
that methods for evaluating these rare occurrences of chords
are needed in the future.

Lastly, the source code of the model, experiment logs, and
general documentation of the software is offered as a
contribution to future researchers in the field. These
contributions are discussed in the next sections of this
chapter.
