% Copyright 2022 Néstor Nápoles López

Throughout the dissertation, various contributions towards
\glspl{rna} have been made in different dimensions of the
problem. \refchap{introductiontoromannumeralanalysis} a
historical analysis of the \gls{rna} syntax. Although
several accounts of the history of harmony
exist\footnote{See, for example,
\textcite{christensen2002tonality},
\textcite{grave1988praise}, \textcite{laitz2010graduate},
\textcite{sansallovich2013quintas}, or
\textcite{wason1985viennese}}, most of these do not focus on
the syntax of the annotation. The review presented in this
dissertation focuses solely in the use of symbols,
conventions, and specificities of the \gls{rna} notation
that are relevant for its digitization and standardization.
\refchap{dataacquisitionandpreparation} documents the
aggregation process of the, to the best of my knowledge,
largest publicly available dataset of \glspl{rna} labels.
The process to overcome the alignment between annotations
and scores are discussed through the use of quantitive
metrics and other resources that facilitate the aggregation
of the data. Furthermore, in \refsubsec{preprocesseddata} of
the present chapter, the preprocessed data is made publicly
available for other researchers who may be interested in
using it, including the splits of the training, validation,
and test splits used for the experiments.
