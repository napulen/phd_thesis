% Copyright 2022 Néstor Nápoles López

Throughout the development of \gls{augmentednet}, the
experiments were documented with the \emph{MLflow} Python
library, introduced by \textcite{zaharia2018accelerating}.
Among other things, the software facilitates the recording
of \emph{git}\footnotelink{https://git-scm.com/} commits
(i.e., version of the source code), name identifiers, and
results achieved during an experiment. The logs for each of
the experiments reported in the dissertation are shared in
their original \emph{MLflow} format (i.e., as an
\code{mlruns} folder). One drawback is that the software
requires some experience setting up a local development
environment to visualize the results. Thus, as an additional
convenience, the logs per training epoch of each experiment
have also been uploaded to the
\emph{Tensorboard.dev}
platform,\footnotelink{https://tensorboard.dev/} which is
freely available. All the resources can be found in the
source code repository of the model.
