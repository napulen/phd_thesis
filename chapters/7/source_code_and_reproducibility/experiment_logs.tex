% Copyright 2022 Néstor Nápoles López

Throughout the development of the end-to-end system, the
experiments were documented with the \emph{MLFlow} Python
library, introduced by \textcite{zaharia2018accelerating}.
Among other things, the software facilitates the recording
of \code{git} commits (i.e., version of the source code),
name identifiers, and results achieved during an experiment.
The logs for each of the experiments reported in the
dissertation are shared in their original \emph{MLFlow}
format (i.e., as an \code{mlruns} folder). One drawback is
that the software requires some experience setting up a
local development environment to visualize the results.
Thus, as an additional convenience, the logs per training
epoch of each experiment have also been uploaded to the
\code{Tensorboard.dev}
platform,\footnotelink{https://tensorboard.dev/} which is
freely available at the time of writing.
