% Copyright 2022 Néstor Nápoles López

The aggregated dataset is processed in two sequential steps:
a ``human-friendly'' \gls{tsv} file and a
``machine-friendly'' tensor of sequences. The \gls{tsv} file
is generated by processing \gls{musicxml} and
\gls{romantext} source files with the \emph{music21}
\parencite{cuthbert2010music21} and \emph{pandas}
\parencite{mckinney2011pandas} packages into a ``tabular''
representation. During this process, the objective metrics
are computed and stored in the \gls{tsv} file, which makes
it easier to spot files that may be misaligned or present
bad quality annotations. The tensor of sequences, written as
a binary \emph{numpy} \parencite{oliphant2006guide}
multidimensional array is generated by encoding each
\gls{tsv} file into a numeric representation. The encoding
of these tensors are based on the definition of the input
(\gls{bass19}, \gls{chroma19}, and \gls{duration14}) and
output (\gls{alto35}, \gls{bass35}, \gls{harmonicrhythm7},
\gls{localkey38}, \gls{pcset121}, \gls{rn31},
\gls{soprano35}, \gls{tenor35}, and \gls{tonicization38})
representations.

The generation of the initial \gls{tsv} file is generally
slower\footnote{Nearly 50 minutes to compute on an Intel i7
10750 processor.} and does not change often. The generation
of the tensor sequences is generally faster\footnote{Nearly
3 minutes to compute on an Intel i7 10750 processor,
considering the aggregated dataset (see
\refsec{theaggregateddataset}) without data augmentation.}
and changes frequently for experimentation. Thus, a
preprocessed collection of \gls{tsv} files is distributed
with each of the releases of the system. This can be used to
reproduce the results of the experiments presented here or
to facilitate the exploration of new tonal tasks or
input/output representations.
