% Copyright 2022 Néstor Nápoles López

A comprehensive list of the releases of the system proposed
in this dissertation is presented below. The tagged releases
are ordered chronologically, with a brief description of the
changes they introduce. Some of the tagged releases were
left out for brevity.

\paragraph{First Commit - February 12, 2021}
This is the date of the initial commit of the project, with
the commit identifier \code{f721860}.

\paragraph{0.0.1 - April 12, 2021}
Initial tagged version of the software. Unstable. Closed
source. Introduced the basic workflow for pairing annotation
and scores into a tabular format.

\paragraph{v0.1.0 - May 14, 2021}
This is the commit version accompanying the paper submission
to the \gls{ismir} 2021 conference. The code is not
considered to be in a state of ``production'', but it
produces the final results in the submitted paper. The code
continues to be closed source for double-blind review
purposes.

\paragraph{v1.0.0 - August 5, 2021}
The published version of the \gls{augmentednet} network in
\textcite{napoleslopez2021augmentednet}. This is the version
of the code used for all the revised experiments in the
camera-ready paper. Accompanying this release are the
experiment logs, preprocessed data, and data splits of the
paper. One (and maybe the only) substantial difference
between the architecture submitted in the initial submission
and this one, is the use of \emph{rmsprop} as the optimizer,
instead of \emph{Adam} \parencite{kingma2014adam}. I found
that \emph{rmsprop} was generally better and decided to use
it as the default. Every revised experiment in the
camera-ready paper uses \emph{rmsprop}. As observed before,
the results went slightly above just because of that.

\paragraph{v1.1.0 - February 3, 2022}
This version introduced the \gls{harmonicrhythm7} output
task as a substitute of a previous binary classifier. This
addresses the class imbalance in the previous version of the
task, which leads into a better chord segmentation. The four
\gls{satb35} tasks were also added to the architecture in
this version.

\paragraph{v1.2.0 - February 5, 2022}
Turned the \gls{satb35} tasks into the default ones computed
by the network. These tasks are also used to compute the
chords in the final \gls{rna} outputs of the model.

\paragraph{v1.2.2 - February 5, 2022}
The output \gls{rna} annotations produced by the system are
reconstructed from the \gls{satb35} tasks by default. The
\code{README} file was updated with instructions for
inference and training. A Jupyter notebook
\textcite{kluyver2016jupyter} to run an inference demo An
updated version of the dataset with the new output
representations.


\paragraph{v1.2.3 - February 14, 2022}
Introduced the keydistance algorithm. This algorithm is used
to find the closest key to a given key, which is useful to
force tonicizations when a chord does not exist in the
vocabulary.

\paragraph{v1.2.4 - February 14, 2022}
Extended the vocabulary of PitchClassSets to 121. The
resulting task \gls{pcset121} has a more robust vocabulary
than the previous version based on 94 classes. This means
that \code{v1.2.4} is backwards-incompatible with models
trained on previous versions.

\paragraph{v1.3.0 - February 14, 2022}
Replacing the use of the \code{romanNumeralFromChord()}
function of the \emph{music21}
\parencite{cuthbert2010music21} library with Roman numerals
from the custom chord vocabulary. Using the chord vocabulary
facilitates the standardization of the chord names. The
vocabulary is implemented to standardize both the training
annotations that enter the system, and the output
annotations generated in inference mode.

\paragraph{v1.3.1 - February 14, 2022}
Fixing unit tests to work with the pretrained model of
\code{v1.3.0}. Fine-tuning the chord vocabulary and
improving the implementation of the algorithm to force
tonicizations (see \refsubsec{forcingatonicization}).

\paragraph{v1.3.3 - February 14, 2022}
The transposition algorithm for data augmentation is now
based on tonicizations rather than modulations. Forcing all
$\rn{ii}\rndim{}\rnseven$ annotations to be interpreted as
$\rn{ii}\rnhdim{}\rnseven$. Collapsing other annotations,
such as $\rn{i}^{\rnformat{5}}_{\rnformat{4}}$ into the
closest chord in the vocabulary using cosine similarity.

\paragraph{v1.4.4 - February 21, 2022}
Extending the vocabulary of keys from 35 to 38 clases. The
new vocabulary spans the keys [$\keyBbb$/, $\keyDs$] $\cup$
[$\keygb$, $\keybs$]. Additionally, the translation of Roman
numeral labels before training was improved.

\paragraph{v1.5.0 - March 13, 2022}
Padding the fixed-length inputs with a special token rather
than a value of zero. This makes it possible to distinguish
between ``rest'' and ``padding''. Added an additional option
for texturizing the synthetic examples at each transposition
(as opposed to one texturization per training file).

\paragraph{v1.5.1 - March 26, 2022}
This version removed the \emph{dropout} layers
\parencite{dahl2013improving} that were originally part of
the architecture. It was discovered that these layers were
detrimental for the \gls{harmonicrhythm7} task.

\paragraph{v1.6.0 - June 13, 2022}
This version introduced the new input representation
\gls{duration14}. This input representation encodes the
measure-and note-onset information. That is, it indicates
when a new measure begins and the duration elapsed since the
last measure. Additionally, it indicates when a new onset
begins, and the duration elapsed since the last note onset.
It dramatically improves the performance of the
\gls{harmonicrhythm7} task (around 30\% of accuracy
improvement).

\paragraph{v1.7.0 - June 17, 2022}
The main contribution of this version is that it introduces
the \gls{mps} dataset into the aggregated dataset. The code
that organizes all the individual datasets was restructured
as well.

\paragraph{v1.7.1 - June 20, 2022}
Patching an issue where percussion tracks would introduce
nonsensical pitch information. Patching the \gls{romantext}
output files, which were sometimes invalid. The
\code{README} is pointed to the latest preprocessed dataset
automatically. Removing drum parts from
\code{music21.stream.Stream} objects before processing a
\gls{musicxml} file. Improve the quality of the
\gls{romantext} outputs generated by the system.

\paragraph{v1.8.0 - June 25, 2022}
Texturization patterns can handle durations of a dotted half
(\musHalfDotted) and dotted quarter (\musQuarterDotted)
notes. Renaming the \gls{duration14} task in the code. The
\gls{duration14} was also divided into \code{MeasureOnset7}
and \code{NoteOnset7}, for easier experimentation with note
onsets and measure onsets. The \gls{haydnsun} dataset was
renamed internally, and so were the \gls{tonicization38} and
\gls{localkey38} tasks to correspond with the new vocabulary
of 38 keys. Improving the evaluation of the model by
removing any padding from the input test sequences, which
were considered before.

\paragraph{v1.9.0 - August 4, 2022}
This version introduced the \gls{kmt} dataset into the
aggregated dataset. This is the last dataset to be
integrated into the set of seven publicly available datasets
described in \refchap{dataacquisitionandpreparation}.

\paragraph{v1.9.1 - August 7, 2022}
This is a minor release. It fixed a bug where the
annotations would not generate properly on \gls{musicxml}
files with multiple voices. It also fixed another issue
where $\rnCad$ chords would not be labeled properly although
the predictions were correct. Lastly, it removed the caching
of \gls{musicxml} files from the \emph{music21} Python
library, as this lead to unpredictable behaviour during the
final evaluation of the model.
