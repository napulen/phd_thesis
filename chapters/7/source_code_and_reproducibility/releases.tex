The complete listing of releases of the proposed system is
presented below, ordered from most recent to oldest.

\phdparagraph{v1.8.0 - june 25, 2022}
Texturization patterns can handle durations of a dotted half
(\musHalfDotted) and dotted quarter (\musQuarterDotted)
notes. Renaming the \gls{duration14} task in the code. The
\gls{duration14} was also divided into \code{MeasureOnset7}
and \code{NoteOnset7}, for easier experimentation with note
onsets and measure onsets. The \gls{haydnsun} dataset was
renamed internally, and so were the \gls{tonicization38} and
\gls{localkey38} tasks to correspond with the new vocabulary
of 38 keys. Improving the evaluation of the model by
removing any padding from the input test sequences, which
were considered before.

\phdparagraph{v1.7.1 - june 20, 2022}
Patches an issue where percussion tracks would introduce
nonsensical pitch information. Patches the \gls{romantext}
output files, which were sometimes invalid. Pointing the
\code{README} to the latest preprocessed dataset
automatically. Removing drum parts from
\code{music21.stream.Stream} objects before processing a
\gls{musicxml} file. Improve the quality of the
\gls{romantext} outputs generated by the system.

\phdparagraph{v1.7.0 - june 17, 2022}
The main contribution of this version is that it introduces
the mozart\_piano\_sonatas (mps) dataset.

Reorganizing data collections. Fixing trailing newlines.
Adding mozart\_piano\_sonatas as submodule. Aggregating the
MPS dataset. Release v1.7.0.

\phdparagraph{v1.6.0 - june 13, 2022}
This version introduces a new input representation,
\code{Duration14}.

This input encodes the \code{measure}-and \code{note}-onset
information. That is, it indicates when a new measure begins
and the duration elapsed since the last measure.
Additionally, it indicates when a new onset begins, and the
duration elapsed since the last note onset.

It dramatically improves the performance of the
\code{HarmonicRhythm7} task.


Adding Duration14 input representation. Duration14 fixes.
Updating pretrained model of v1.6.0.

\phdparagraph{v1.5.1 - march 26, 2022}
Same as \code{v1.5.0}. However, the pretrained model was
changed for a version without \code{Dropout} at the end. It
turns out that the \code{HarmonicRhythm} classifier is more
stable this way.

\phdparagraph{v1.5.0 - march 13, 2022}
Improved padding of sequences \#56 Added an additional
option for texturizing the synthetic examples at each
transposition (as opposed to one texturization per training
file) \#58 Improved parsing/resolution of Roman numeral
labels at training and inference

\phdparagraph{v1.4.4 - february 21, 2022}
The main change in \code{v1.4.X} is that the key vocabulary
was extended to 38 keys \code{Bbb/gb, D\#/b\#}.

Some other fixes include:

Better handling of Roman numeral translation before training
More transpositions in data augmentation (byproduct of
extending the key vocabulary)

Known issues:

The key-related tasks are still called \code{LocalKey35} and
 \code{TonicizedKey35}, although they should now be named
 \code{LocalKey38} and \code{TonicizedKey38}. Left with old
 name for compatibility, but changing in future releases of
 \code{v1.4.X}.

\phdparagraph{v1.4.3 - february 20, 2022}
Turning correction (i.e., collapsing vocabulary before
training) back as the default.

\phdparagraph{v1.4.2 - february 20, 2022}
Adding a way to mimic RomanText's parsing.

\phdparagraph{v1.4.1 - february 19, 2022}
A round of changes in the way Roman numerals are processed.
Work in progress.

\phdparagraph{v1.3.6 - february 16, 2022}
Correcting a one-line but crucial fix in \code{inference.py}

\phdparagraph{v1.3.5 - february 15, 2022}
Correcting the Roman numeral reconstruction algorithm.

\phdparagraph{v1.3.4 - february 15, 2022}
Removing obsolete module evaluate.py. Modularizing
duplicated code. Adding pretrained model v1.3.3+.

\phdparagraph{v1.3.3 - february 14, 2022}
Making transposition based on tonicizations rather than
modulations. String-replacing iio7 to half-diminished and
turning I54,i54 into ton….

\phdparagraph{v1.3.2 - february 14, 2022}
ChordQuality11. Updating requirements.txt.

\phdparagraph{v1.3.1 - february 14, 2022}
Fixing unit tests to work with \code{v1.3.0} model
Fine-tuning the chord vocabulary and
\code{forceTonicization} algorithm

\phdparagraph{v1.3.0 - february 14, 2022}
Replacing the use of the \code{romanNumeralFromChord()}
function of the \emph{music21}
\parencite{cuthbert2010music21} library with Roman numerals
from the custom chord vocabulary. Using the chord vocabulary
facilitates the standardization of the chord names. The
vocabulary is implemented to standardize both the training
annotations that enter the system, and the output
annotations generated in inference mode.

\phdparagraph{v1.2.4 - february 14, 2022} 
Extended the vocabulary of PitchClassSets to 121. The
resulting task \gls{pcset121} has a more robust vocabulary
than the previous version based on 94 classes. This means
that \code{v1.2.4} is backwards-incompatible with models
trained on previous versions.


\phdparagraph{v1.2.3 - february 14, 2022}
Introduced the keydistance algorithm. This algorithm is used
to find the closest key to a given key, which is useful to
force tonicizations when a chord does not exist in the
vocabulary.

\phdparagraph{v1.2.2 - february 5, 2022}
The output \gls{rna} annotations produced by the system are
reconstructed from the \gls{satb35} tasks by default. The
\code{README} file was updated with instructions for
inference and training. A Jupyter notebook
\textcite{kluyver2016jupyter} to run an inference demo An
updated version of the dataset with the new output
representations.

\phdparagraph{v1.2.0 - february 5, 2022}
Turned the \gls{satb35} tasks into the default ones computed
by the network. These tasks are also used to compute the
chords in the final \gls{rna} outputs of the model.

\phdparagraph{v1.1.0 - february 3, 2022}
This version introduced the \gls{harmonicrhythm7} output
task as a substitute of a previous binary classifier. This
addresses the class imbalance in the previous version of the
task, which leads into a better chord segmentation. The four
\gls{satb35} tasks were also added to the architecture in
this version. 

\phdparagraph{v1.0.0 - august 5, 2021}
The published version of the \gls{augmentednet} network in
\textcite{napoleslopez2021augmentednet}. This is the version
of the code used for all the revised experiments in the
camera-ready paper. Accompanying this release are the
experiment logs, pre-processed data, and data splits of the
paper. One (and maybe the only) substantial difference
between the architecture submitted in the initial submission
and this one, is the use of \emph{rmsprop} as the optimizer,
instead of \emph{Adam} \parencite{kingma2014adam}. I found
that \emph{rmsprop} was generally better and decided to use
it as the default. Every revised experiment in the
camera-ready paper uses \code{rmsprop}. As observed before,
the results went slightly above just because of that.

\phdparagraph{v0.1.0 - may 14, 2021}
This is the commit version accompanying the paper submission
to the \gls{ismir} 2021 conference. The code is not
considered to be in a state of ``production'', but it
produces the final results in the submitted paper. The code
continues to be closed source for double-blind review
purposes.

\phdparagraph{0.0.1 - april 12, 2021}
Initial tagged version of the software. Unstable. Closed
source. Introduced the basic workflow for pairing annotation
and scores into a tabular format.

\phdparagraph{first commit - february 12, 2021}

This is the date of the initial commit of the project, with
the commit identifier \code{f721860}.
    


