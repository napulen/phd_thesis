The complete listing of releases of the proposed system is
presented below, ordered from most recent to oldest.

\phdparagraph{v1.8.0 - june 25, 2022}
Several minor changes introduced. Texturization patterns can
now handle durations of a \code{dotted half} and
\code{dotted quarter} notes. Renaming \code{Duration14} as
\code{MeasureNoteOnset14}. It was also separated into
\code{MeasureOnset7} and \code{NoteOnset7} \code{haydnop20}
renamed to \code{haydnsun} all throughout
\code{TonicizedKey35} and \code{LocalKey35} renamed to
\code{TonicizedKey38} and \code{LocalKey38} Removed padded
timesteps from evaluation Other minor corrections

Adding Duration14 as a default input representation.
Renaming LocalKey38 and TonicizedKey38. Renaming
MeasureNoteOnset14 and introducing intermediate
representations. Adding more options for experiment names.
Reverse logic of transposition arg. Adding the fix to the
existing texturization patterns for 3/4 and 6/8…. V180.

\phdparagraph{v1.7.1 - june 20, 2022}
Patches an issue where percussion tracks would introduce
nonsensical pitch information. Patches the RomanText output
files, which were sometimes invalid.

Pointing README to latest dataset automatically. Removing
drum parts from music21.stream before processing. Improve
the rntxt output writer. Updating version to 1.7.1.

\phdparagraph{v1.7.0 - june 17, 2022}
The main contribution of this version is that it introduces
the mozart\_piano\_sonatas (mps) dataset.

Reorganizing data collections. Fixing trailing newlines.
Adding mozart\_piano\_sonatas as submodule. Aggregating the
MPS dataset. Release v1.7.0.

\phdparagraph{v1.6.0 - june 13, 2022}
This version introduces a new input representation,
\code{Duration14}.

This input encodes the \code{measure}-and \code{note}-onset
information. That is, it indicates when a new measure begins
and the duration elapsed since the last measure.
Additionally, it indicates when a new onset begins, and the
duration elapsed since the last note onset.

It dramatically improves the performance of the
\code{HarmonicRhythm7} task.


Adding Duration14 input representation. Duration14 fixes.
Updating pretrained model of v1.6.0.

\phdparagraph{v1.5.1 - march 26, 2022}
Same as \code{v1.5.0}. However, the pretrained model was
changed for a version without \code{Dropout} at the end. It
turns out that the \code{HarmonicRhythm} classifier is more
stable this way.

\phdparagraph{v1.5.0 - march 13, 2022}
Improved padding of sequences \#56 Added an additional
option for texturizing the synthetic examples at each
transposition (as opposed to one texturization per training
file) \#58 Improved parsing/resolution of Roman numeral
labels at training and inference

\phdparagraph{v1.4.4 - february 21, 2022}
The main change in \code{v1.4.X} is that the key vocabulary
was extended to 38 keys \code{Bbb/gb, D\#/b\#}.

Some other fixes include:

Better handling of Roman numeral translation before training
More transpositions in data augmentation (byproduct of
extending the key vocabulary)

Known issues:

The key-related tasks are still called \code{LocalKey35} and
 \code{TonicizedKey35}, although they should now be named
 \code{LocalKey38} and \code{TonicizedKey38}. Left with old
 name for compatibility, but changing in future releases of
 \code{v1.4.X}.

\phdparagraph{v1.4.3 - february 20, 2022}
Turning correction (i.e., collapsing vocabulary before
training) back as the default.

\phdparagraph{v1.4.2 - february 20, 2022}
Adding a way to mimic RomanText's parsing.

\phdparagraph{v1.4.1 - february 19, 2022}
A round of changes in the way Roman numerals are processed.
Work in progress.

\phdparagraph{v1.3.6 - february 16, 2022}
Correcting a one-line but crucial fix in \code{inference.py}

\phdparagraph{v1.3.5 - february 15, 2022}
Correcting the Roman numeral reconstruction algorithm.

\phdparagraph{v1.3.4 - february 15, 2022}
Removing obsolete module evaluate.py. Modularizing
duplicated code. Adding pretrained model v1.3.3+.

\phdparagraph{v1.3.3 - february 14, 2022}
Making transposition based on tonicizations rather than
modulations. String-replacing iio7 to half-diminished and
turning I54,i54 into ton….

\phdparagraph{v1.3.2 - february 14, 2022}
ChordQuality11. Updating requirements.txt.

\phdparagraph{v1.3.1 - february 14, 2022}
Fixing unit tests to work with \code{v1.3.0} model
Fine-tuning the chord vocabulary and
\code{forceTonicization} algorithm

\phdparagraph{v1.3.0 - february 10, 2022}

\phdparagraph{v1.2.2 - february 5, 2022}
The paper originally described the **six conventional*tasks,
and the alternative method based on the **75-most common
Roman numerals**.

A Roman numeral can also be reconstructed by predicting the
notes conforming the chord. This has become the main
approach in this version.

Some other additions include:

An updated \code{README} with instructions for inference and
training A notebook to run an inference demo An updated
version of the dataset with the new output representations

\phdparagraph{v1.2.0 - february 5, 2022}
Redefining default output tasks to the eight SATB ones.

\phdparagraph{v1.1.0 - february 3, 2022}
This version introduced the \gls{harmonicrhythm7} output
task as a substitue of a previous binary classifier, which
addresses the class imbalance in chord segmentation. This
representation leads to a better chord segmentation. The
\gls{satb35} tasks were also added to the architecture. 

\phdparagraph{v1.0.0 - august 5, 2021}
The published version of the \gls{augmentednet} network in
\textcite{napoleslopez2021augmentednet}. This is the version
of the code used for all the revised experiments in the
camera-ready paper. Accompanying this release are the
experiment logs, pre-processed data, and data splits of the
paper. One (and maybe the only) substantial difference
between the architecture submitted in the initial submission
and this one, is the use of \code{rmsprop} as the optimizer,
instead of \code{adam} \parencite{kingma2014adam}. I found
that \code{rmsprop} was generally better and decided to use
it as the default. Every revised experiment in the
camera-ready paper uses \code{rmsprop}. As observed before,
the results went slightly above just because of that.

\phdparagraph{v0.1.0 - may 14, 2021}
This is the commit version accompanying the paper submission
to the \gls{ismir} 2021 conference. The code is not
considered to be in a state of ``production'', but it
produces the final results in the submitted paper. The code
continues to be closed source for double-blind review
purposes.

\phdparagraph{0.0.1 - april 12, 2021}
Initial tagged version of the software. Unstable. Closed
source. Introduced the basic workflow for pairing annotation
and scores into a tabular format.

\phdparagraph{first commit - february 12, 2021}

This is the date of the initial commit of the project, with
the commit identifier \code{f721860}.

v1.1.0 - Adding four-note output representation to multitask
    learning - Now the network is capable of predicting 14
    tonal tasks v1.2.0 - Making the four-note output the
    default method for reconstructing Roman numerals - By
    default, 8 tasks will be computed v1.2.1 - Providing
    pre-trained model (14 outputs) v1.2.2 - Updating
    pre-processed dataset that works with v1.2.1 network
    v1.2.3 - Introducing the keydistance metric for forcing
    tonicizations v1.2.4 - Extending the vocabulary of
    PitchClassSets to 121 classes, a more robust vocabulary
    than PitchClassSets94 - This means that v1.2.4 is
    backwards-incompatible with models trained on previous
    versions v1.3.0 - Replacing romanNumeralFromChord with
    custom chord vocabulary - Using this chord vocabulary
    not only at inference, but for standardizing the input
    training data

