% Copyright 2022 Néstor Nápoles López

This dissertation presented an end-to-end system for
\glspl{rna}. \refchap{introduction} introduced the problem
and the motivation to solve it. \gls{rna} is a popular
analytical system for Western tonal music, which allows
analysts to explain complicated concepts (e.g., chromatic
changes of harmony) using a relatively compact syntax. From
an \gls{mir} perspective, it is also an interesting problem
because it encapsulates the tasks of recognizing chords and
keys simultaneously. Thus, it provides a framework to design
more complex tonal music analysis models and unified
datasets to train and evaluate such models.

\refchap{introductiontoromannumeralanalysis} discussed the
\gls{rna} notation from the music theory perspective. In
particular, a timeline of the evolution of the syntax was
presented. This timeline scrutinized the use of symbols and
notations over historical textbooks, and how that probably
led to the notation used today. This study linked tightly
into the topic of digitization of \gls{rna} annotations,
where the conventions and specificities of the notation are
crucial to formalize what each of the symbols means. Several
digital formats and their conventions were discussed at the
end of this chapter.

\refchap{background} introduced a literature review on
various relevant topics, such as music representation, deep
neural networks, music information retrieval, and
computational music theory. The end of chapter introduces
the relevant literature on \glspl{rna} and most recent
approaches tackling the problem.

\refchap{dataacquisitionandpreparation} discussed the
datasets used for training and testing the proposed system.
The datasets are provided by different researchers and come
in various formats. Thus, this chapter also discusses the
challenges in aggregating the datasets, and the process of
preparing the data.

\refchap{modeldesign} discussed the components of the
end-to-end system. The neural network, its inputs and
outputs, as well as the convolutional and recurrent layers.
Additionally, the methods to turn the predictions of the
neural network into \gls{rna} annotations, which are mostly
based on music theory domain knowledge.

\refchap{experimentalevaluation} introduced the experiments
to evaluate the proposed system. First, a number of ablation
studies were proposed to demonstrate the role of different
components of the neural network in the tasks of the
\gls{mtl} layout. The model was later compared and evaluated
against existing approaches for \glspl{rna}. The comparison
considered common representations of the predicted features
across algorithms, as well as the performance on difficult
chords of the \gls{rna} vocabulary. Based on those
experiments, I conclude that the proposed model generally
provides a better performance than previous approaches on
rare chords of the vocabulary, which makes it more useful
for real applications. The last section of this chapter
discusses a real musical example annotated by existing
\glspl{rna} systems, and the implications of those analyses.
