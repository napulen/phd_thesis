% Copyright 2022 Néstor Nápoles López

This dissertation presented an end-to-end system for
\glspl{rna}. In \refchap{introduction}, an introduction to
the problem and its motivation were discussed. Although
several systems for analyzing harmony exist, \gls{rna} has
been a popular method used mostly for Western tonal music of
the common-practice period. It is possible that this is
because this system allows analysts to explain complicated
concepts (e.g., chromatic changes of harmony) with a
relatively simple syntax. From an \gls{mir} perspective, it
is also a very interesting problem because it encapsulates
the task of recognizing chords and keys simultaneously.
Thus, it provides a framework to design more complex tonal
analysis models, and unified datasets to train and evaluate
such models.

\refchap{introductiontoromannumeralanalysis} discussed the
gls{rna} notation from the music theory perspective. In this
chapter, a timeline of the evolution of the syntax was
presented. This timeline is possibly an useful contribution
to music theory, as often researchers spend more time
utilizing the syntax than inspecting its evolution over
time. This study linked tightly into digitization, where the
conventions and specificities of the notation are crucial
for a successful encoding. Several formats were discussed.
