% Copyright 2022 Néstor Nápoles López

This dissertation presented an end-to-end system for
\glspl{rna}. \refchap{introduction} introduced the problem
and the motivation to solve it. \gls{rna} is a popular
analytical system for Western tonal music, which allows
analysts to explain complicated concepts (e.g., chromatic
changes of harmony) using a relatively compact syntax. From
an \gls{mir} perspective, it is also an interesting problem
because it encapsulates the task of recognizing chords and
keys simultaneously. Thus, it provides a framework to design
more complex tonal analysis models, and unified datasets to
train and evaluate such models.

\refchap{introductiontoromannumeralanalysis} discussed the
\gls{rna} notation from the music theory perspective. In
particular, a timeline of the evolution of the syntax was
presented. This timeline is an useful contribution to music
theory, as often researchers spend more time utilizing the
syntax than scrutinizing it. This study linked tightly into
digitization, where the conventions and specificities of the
notation are crucial. Several digital formats were discussed
here.

\refchap{background} introduced a literature review on
various relevant topics, such as music representation, deep
neural networks, music information retrieval, and
computational music theory.

\refchap{dataacquisitionandpreparation} discussed the
datasets used for training and testing the proposed system.
The datasets are provided by different researchers and come
in various formats. Thus, this chapter also discusses the
challenges in aggregating the datasets, and the process of
preparing the data.

\refchap{modeldesign} discussed, mainly, the characteristics
of the proposed neural network model. The different layers
and components of the model, from inputs to outputs, are
described in the chapter. Additionally, the methods to turn
the predictions of the neural network into \gls{rna}
annotations are also described. Those postprocessing methods
are based mostly on music theory domain knowledge.

\refchap{experimentalevaluation} introduced the experiments
to evaluate the proposed system. First, a number of ablation
studies were proposed to demonstrate the role of different
components of the neural network in the tasks of the
\gls{mtl} layout. The model was later compared and evaluated
against existing approaches for \glspl{rna}. The comparison
considered common representations of the predicted features
across algorithms, as well as the performance on difficult
chords of the \gls{rna} vocabulary. Based on those
experiments, I conclude that the proposed model generally
provides a better performance than previous approaches on
rare chords of the vocabulary, which makes it more useful
for real applications. The last section of this chapter
discusses a real musical example annotated by existing
\glspl{rna} systems, and the implications of those analyses.
