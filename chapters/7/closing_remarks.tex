% Copyright 2022 Néstor Nápoles López

This dissertation described the two-year effort of N\'estor
N\'apoles L\'opez tackling \glspl{rna}, with the aid of the
many-year effort of understanding music, music theory, music
encoding, and computer science. The main contribution of the
report is the journey, documented by software revisions,
figures, and experimental results. If this dissertation was
written six months from now, the system would likely look
different. That is, there is much to do, many experiments to
try, and a lot of ideas worth implementing. Hopefully, the
work presented here will be of help to future researchers
tackling this and similar problems by illustrating some of
the findings and insights learned. In a not-too-long future,
I foresee these systems being used widely in music
education. In general, I think of such a system as a ``music
theory'' calculator. They will not replace the role of music
theorists, but they will automate the ``easy'' tasks of
music theory, like labelling clear unambiguous chords in
clear unambiguous contexts. In the same way that a
statistician is freed to let the mind fly with tools that
compute averages, standard deviations, and graphs
automatically, the music theorist will be freed from trivial
analytical tasks, so that they can focus on deeper ones. 
