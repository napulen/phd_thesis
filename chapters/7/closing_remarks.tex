% Copyright 2022 Néstor Nápoles López

In conclusion, this dissertation described a two-year
research effort on \gls{arna}, with the aid of a many-year
effort in music theory, music encoding, and computer science
understanding. The main contribution of the report is the
documented account of experimental results, design
decisions, musical choices, and software development
considerations. If this dissertation was written six months
from now, the model would look different. That is, there is
much to do, many experiments to try, and many ideas worth
implementing. Thus, the ideas introduced in this report
account for the present views on the problem, which will
surely change in the future.  Hopefully, the work presented
here will be of help to future researchers tackling this and
similar problems by illustrating some of the findings and
lessons learned. In a not-too-long future, I foresee these
systems being used by end-user musicians and students. In
general, I think of such a system as a ``music theory''
calculator. They will not replace the role of music
theorists, but they will automate the ``easy'' tasks of
music theory, like labelling clear unambiguous chords in
clear unambiguous tonal contexts. In the same way that a
statistician is freed to let the mind fly with tools that
compute averages, standard deviations, and graphs
automatically for them, I see music theorists released from
trivial analytical tasks, so that they can focus on deeper
quests for musical understanding.
%  When Markov first invented the famous \emph{Markov
% chains}, he computed the probabilities of English letters
% in Shakespeare using pen and paper. Arguably, inventing
% the \emph{Markov chains} is more important than computing
% the features by hand. Similarly, 
