% Copyright 2022 Néstor Nápoles López

This dissertation described the two-year effort of N\'estor
N\'apoles L\'opez tackling \glspl{rna}, with the aid of the
many-year effort of understanding music, music theory, music
encoding, and computer science. The main contribution of the
report is the journey, documented with software revisions,
figures, and experimental results. If this dissertation was
written six months from now, the system would look
different. That is, there is much to do, many experiments to
try, and many an idea worth implementing. Hopefully, the
work presented here will be of help to future researchers
tackling this and similar problems by illustrating some of
the findings and lessons learned. In a not-too-long future,
I foresee these systems being used by end-user musicians and
students. In general, I think of such a system as a ``music
theory'' calculator. They will not replace the role of music
theorists, but they will automate the ``easy'' tasks of
music theory, like labelling clear unambiguous chords in
clear unambiguous tonal contexts. In the same way that a
statistician is freed to let the mind fly with tools that
compute averages, standard deviations, and graphs
automatically for them. When Markov first invented the
famous \emph{Markov chains}, he computed the probabilities
of English letters in Shakespeare using pen and paper.
Arguably, inventing the \emph{Markov chains} is more
important than computing the features by hand. Similarly, I
see  music theorist released from trivial analytical tasks,
so that they can focus on deeper quests for music
understanding, or at least that is my hope for the field. 
