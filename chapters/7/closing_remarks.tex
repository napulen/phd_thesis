% Copyright 2022 Néstor Nápoles López

In conclusion, this dissertation described a two-year
research effort on \gls{arna}, with the aid of a many-year
effort to understand music theory, music encoding, and
computer science principles. The main contribution of the
dissertation is the documented account of experimental
results, machine learning and musical decisions, and
software development considerations. If this dissertation
was written six months from now, the \gls{arna} model would
look different to the one presented here. That is, there is
much to do, many experiments to try, and many ideas worth
implementing. Hopefully, the work presented here will be of
help to future researchers tackling this and similar
problems by illustrating some of the findings and lessons
learned. In a not-too-long future, I foresee these systems
being used by end-user musicians and students. In general, I
think of \gls{arna} systems as a type of ``music theory''
calculators. I do not think they are to replace the role of
thorough---human---musical analysis, but they will automate
some analytical tasks, like labelling clear unambiguous
chords in clear unambiguous tonal contexts. It is common to
see statisticians relying on tools that compute averages,
standard deviations, and graphs automatically for them. I
see music theorists relying on music-analysis tools that
will permit them to pursue deeper quests for musical
understanding.
%  When Markov first invented the famous \emph{Markov
% chains}, he computed the probabilities of English letters
% in Shakespeare using pen and paper. Arguably, inventing
% the \emph{Markov chains} is more important than computing
% the features by hand. Similarly, 
