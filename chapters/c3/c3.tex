\chapter{Data acquisition and preparation}
\label{chap:data}

% \begin{quote}
%     Knowing the name of a note is not trivial.
% \end{quote}
% \clearpage

\section{Available datasets}

\subsection{Annotated Beethoven Corpus (ABC)}
\subsection{Beethoven Piano Sonatas (BPS)}
\subsection{Haydn Op. 20 String Quartets (HaydnOp20)}
\subsection{Mozart Piano Sonatas (MPS)}
\subsection{Theme and Variation Encodings with Roman Numerals (TAVERN)}
\subsection{Schubert Winterreise (SW)}
\subsection{When in Rome (WiR)}
\subsection{The Well-Tempered Clavier, Book I (WTC)}

\section{Data preparation}
\subsection{Aligning a score and an annotation file}
\subsection{Standardizing the notation between datasets}

\section{Data augmentation}
\subsection{Synthesizing examples with voice-leading rules}
\subsection{Artificial texturization of scores}

