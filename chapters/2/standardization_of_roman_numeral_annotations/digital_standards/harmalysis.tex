% Copyright 2022 Néstor Nápoles López

% This is \refsubsubsec{harmalysis}, which introduces the
% harmalysis.

\guide{The harmalysis grammar}
\code{Harmalysis} is a \gls{rna} grammar introduced by
\textcite{napoleslopez2020harmalysis}.

The \code{harmalysis} grammar is based principally on
Huron‘s \gls{humharm} syntax (see
\refsubsubsec{humdrum(**harm)}). The syntax is extended by
borrowing elements from the RomanText format (see
\refsubsubsec{RomanText}), MuseScore's notation for
\gls{rna}\footnote{\href{https://musescore.org/en/handbook/3/chord-symbols\#rna}{https://musescore.org/en/handbook/3/chord-symbols}},
and conventions observed in existing datasets of \gls{rna}.
As  a result,  the harmalysis language is a superset  of
the \gls{humharm} syntax,  which includes additional
features and supports a wider range of customs of harmonic
analysis.

The main advantage of this syntax is that it is formally
defined as a context-free grammar. Thus, the exact
mathematical validation of the syntax is available for
review and discussion, which might not be the case for
representations defined by their textual explanations.
