% Copyright 2022 Néstor Nápoles López

% This is \refsubsubsec{humdrum(**harm)}, which introduces
% the humdrum(**harm).

The \gls{humharm} syntax was proposed by Huron as part of
the Humdrum(\gls{humkern}) format
\parencite{huron1994humdrum}. This syntax includes a
comprehensive notation for triads, seventh chords, special
chords, and descriptive (arbitrary) chords using Roman
numeral notation. The notation is well documented, although
several aspects of the notation are missing from the
explanations.

The original \gls{humharm} documentation is available
online.\footnote{\href{https://www.humdrum.org/Humdrum/representations/harm.rep.html}{https://www.humdrum.org/Humdrum/representations/harm.rep.html}}
The notation summarizes the description of chords based on
four attributes: (1) chord root, (2) chord type, (3)
inversion, and (4) chord alterations. Chord roots are
indicated with Roman numerals. Chord types (or qualities)
are indicated with a case-sensitive notation of the Roman
numeral plus the symbols `$\rn{+}$' and `$\rn{o}$'
indicating augmented and diminished triads, respectively.
All chord inversions are indicated with the letters $a$,
$b$, $c$, and $d$. These letters indicate root position,
first inversion, second inversion, and third inversion,
respectively. The alterations, indicated with `$\rn{-}$' and
`\textbf{\#}', shift the pitch of the Roman numeral scale
degree, most often when indicating non-diatonic chords.
