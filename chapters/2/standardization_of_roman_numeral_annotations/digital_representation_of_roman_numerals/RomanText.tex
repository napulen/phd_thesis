% Copyright 2022 Néstor Nápoles López

% This is \refsubsubsec{romantext}, which introduces the
% romantext.

This notation was first proposed in
\textcite{gotham2019romantext}. An utilitarian aspect of
this notation is that it is implemented as a module inside
the popular \emph{music21} Python library
\parencite{cuthbert2010music21}.\footnote{\href{https://web.mit.edu/music21/doc/moduleReference/moduleRoman.html}{https://web.mit.edu/music21/doc/moduleReference/moduleRoman.html}}


\gls{romantext} is intended as a stand-alone format for
\gls{rna}, meaning it does not require the musical score. As
part of the plain-text Roman numeral annotations, the
location of the annotations (measure and beat) is a
compulsory attribute of the annotation.

Compared to the \gls{humharm} notation, a clear advantage is
that the software implementation facilitates a vast number
of operations with the Roman numeral annotations. For
example:

\begin{itemize}
    \item A \gls{romantext} annotation file can be realized
    into a score with block chords
    \item The Roman numeral annotations can be transformed
    into chord labels, pitch class sets, or other
    representations relatively easily
    \item Easy access is provided to the secondary degrees,
    secondary keys, and other elements implicitly encoded in
    the annotation
\end{itemize}

All the inversions in \gls{romantext} are encoded through
numbers, whereas the **harm syntax defines inversions by
letter exactly to disambiguate these scenarios. Some
disadvantages of the numeric notation for inversions
preferred in RomanText is that it does not allow an
intuitive way to encode inversions for extended 9th, 11th,
and 13th chords. 
