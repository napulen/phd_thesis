% Copyright 2022 Néstor Nápoles López

% This is \refsubsubsec{romantext}, which introduces the
% romantext.

This notation was first proposed in
\textcite{gotham2019romantext}. The syntax is explained in
the conference publication. An utilitarian aspect of this
notation is that it is implemented as a module inside the
popular \code{music21} Python library
\textcite{cuthbert2010music21}.\footnote{\href{https://web.mit.edu/music21/doc/moduleReference/moduleRoman.html}{https://web.mit.edu/music21/doc/moduleReference/moduleRoman.html}}


The RomanText is intended as a stand-alone format for
\gls{rna}, meaning it does not require the score. As part of
the plain-text Roman numeral annotations, the location of
the annotations (measure and beat) is a compulsory attribute
of the annotation.

Compared to the \gls{humharm} notation, a clear advantage is
that the software implementation facilitates a vast number
of operations with the Roman numeral annotations. For
example:

\begin{itemize}
    \item A RomanText annotation file can be realized into a
    score with block chords
    \item The Roman numeral annotations can be transformed
    into Chord labels, pitch class sets, or other
    representations relatively easy
    \item Easy access is provided to the secondary degrees,
    secondary keys, and other elements implicitly encoded in
    the annotation
\end{itemize}

Some disadvantages of the numeric notation for inversions
preferred in RomanText is that it does not allow an
intuitive way to encode inversions for extended 9th, 11th,
and 13th chords. All the inversions are encoded through
numbers, whereas the **harm syntax defines inversions by
letter exactly to disambiguate these scenarios.
