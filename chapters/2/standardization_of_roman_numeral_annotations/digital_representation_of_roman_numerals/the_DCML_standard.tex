% Copyright 2022 Néstor Nápoles López

% This is \refsubsubsec{thedcmlstandard}, which introduces
% the the dcml standard.

This standard for harmonic annotations was developed at the
\gls{dcml} at the \'Ecole Polytechnique F\'ed\'erale de
Lausanne.

This standard was first used in the annotation of the
\gls{abc} dataset introduced by
\textcite{neuwirth2018annotated}. Since then, it has been
revised and documented in a dedicated
repository\footnotelink{https://github.com/DCMLab/standards}
and a reference
manual.\footnotelink{https://dcmlab.github.io/standards/build/html/reference/reference.html}
The introduction section of a tutorial for this annotation
standard summarizes the philosophy behind the
standard:\footnotelink{https://dcmlab.github.io/standards/build/html/tutorial/index.html}

\begin{italicsquote}
    The goal of this tutorial is to provide a systematic and
    condensed way of conveying the annotation philosophy and
    principles (aka ``the guidelines'') behind the
    annotation standard. What has been condensed are more
    than two years of discussions between the music theory
    experts involved in the standard's creation and
    application, i.e. between annotators, reviewers, and
    users. Therefore, this tutorial is not about telling
    anybody ``how harmonic analysis really works'' or ``how
    everyone should be using Roman numerals''. Instead, it
    introduces a set of guidelines for analysts who want to
    use the DCML harmony annotation standard to encode a set
    of musical features in a consistent and machine-readable
    manner so that others can re-use and rely on the encoded
    information.
\end{italicsquote}

In addition to the repository, reference manual, and
tutorial mentioned above, the community behind this standard
has also made available documentation regarding version
control
practices,\footnotelink{https://dcmlab.github.io/standards/build/html/git/git.html}
instructions for corpus creation (described by \gls{dcml} as
the corpus-creation
pipeline),\footnotelink{https://dcmlab.github.io/standards/build/html/pipeline/pipeline.html}
and a page of questions and
examples,\footnotelink{https://dcmlab.github.io/standards/build/html/reference/examples.html}
among other resources.

From the perspective of documentation and standardization,
this is perhaps the most thoroughly described standard for
\gls{rna} in existence today. For example, it offers
specific examples to annotate
voice-leading,\footnotelink{https://dcmlab.github.io/standards/build/html/tutorial/counterpoint.html?highlight=voice+leading}
which is often a problematic type of annotation in digital
representations. The revised version of the standard has
been used in a subsequent iteration of the \gls{abc} dataset
(version 2) as well as the \gls{mps} dataset
\parencite{hentschel2021annotated}. Other collections exist
in a private archive and are expected to be publicly
available one day.
