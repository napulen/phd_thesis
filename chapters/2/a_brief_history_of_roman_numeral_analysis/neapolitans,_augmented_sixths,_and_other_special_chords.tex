% Copyright 2022 Néstor Nápoles López

% This is
% \refsubsec{neapolitans,augmentedsixths,andotherspecialchords},
% which introduces the neapolitans, augmented sixths, and
% other special chords.


The stacks of Arabic numerals in Emery's example
(\reffig{primary_sources/emery1879elements051}) describe
augmented sixth chords (i.e., Italian, French, and German).
This is possibly the first time that the \gls{rna} notation
is extended to describe such chords, beyond the diatonic
harmonies occurring in major and minor scales. Emery also
introduced a special symbol to annotate Neapolitan chords,
$\rn{N.6}$.\footnote{Annotating special chords, such as
augmented sixth chords or Neapolitans is probably one of the
most important characteristics of \gls{rna} to this day, as
these chords are generally beyond the scope of traditional
chord labels. Presumably, this is because identifying these
chords requires understanding of the key, an aspect
neglected by traditional chord labels but natural in
\gls{rna}}} 

Lastly, Emery's notation for the augmented sixth quality is
the $\rn{+}$ symbol. Nowadays, used to denote augmented
triads as well (e.g., $\rn{c:III+}$ in the C harmonic minor
scale). Up to this point, the notation for modulations is to
indicate the changes of key at the level of the Roman
numeral annotations. A full example of the notation can be
seen in another excerpt by Emery, shown in
\reffig{primary_sources/emery1879elements102}.

\phdfigure[Modulations in \textcite{emery1879elements}. Notice that five keys are indicated within eight measures of music, an usually large number of key changes for modern practices. In the modern practice, some of these annotations (e.g., the two D-minor chords) would be expressed as \emph{tonicizations}][1.0]{primary_sources/emery1879elements102}

A notation for augmented triads was introduced by
\textcite{jadassohn1883lehrbuch}, who used a single quote
after the augmented scale degree.

\phdfigure[Augmented triads in \textcite{jadassohn1883lehrbuch}, indicated as $\rn{III'}$][0.8]{primary_sources/jadassohn1883lehrbuch038}

This syntax of the single-quote augmented triad appeared in
at least the treatises by \textcite{broekhoven1889system},
\textcite{buwa1893schule}, and
\textcite{shepard1896harmony}.

\phdfigure[Augmented triads in \textcite{broekhoven1889system}, using a similar notation to the one found in \textcite{jadassohn1883lehrbuch}][0.8]{primary_sources/broekhoven1889system028}
