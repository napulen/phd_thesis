% Copyright 2022 Néstor Nápoles López

% This is
% \refsubsec{neapolitans,augmentedsixths,andotherspecialchords},
% which introduces the neapolitans, augmented sixths, and
% other special chords.


The stacks of Arabic numerals in Emery's example
(\reffig{primary_sources/emery1879elements051}) describe
augmented sixth chords (i.e., Italian, French, and German).
This is possibly the first time that the \gls{rna} notation
is extended to describe chords that lie beyond the diatonic
harmonies occurring in major and minor scales.
\textcite{shepard1896harmony} adopted a similar notation for
augmented sixth chords, as shown in
\reffig{primary_sources/shepard1896harmony137}. Eventually,
these chords would received their own symbols ($\rn{It}$,
$\rn{Fr}$, and $\rn{Ger}$) instead of Arabic numerals,
although that syntax would only be established until the
twentieth century.
\reffig{primary_sources/goldman1965harmony086} shows a more
modern example by \textcite{goldman1965harmony}.

\phdfigure[Augmented sixth chords in \textcite{shepard1896harmony}]{primary_sources/shepard1896harmony137}

\phdfigure[Augmented sixth chords in \textcite{goldman1965harmony} with the symbols $\rn{It.}$ and $\rn{Fr.}$]{primary_sources/goldman1965harmony086}

Emery also introduced a special symbol to annotate
\emph{neapolitan} chords, $\rn{N.6}$.\footnote{Annotating
special chords, such as augmented sixth chords or
\emph{neapolitans} is probably one of the most important
characteristics of \gls{rna} to this day, as these chords
are generally beyond the scope of traditional chord labels.
Presumably, this is because identifying these chords
requires an understanding of the key, an aspect that is
generally neglected in traditional chord labels but
intrinsically available in \gls{rna}.}

Similarly to \textcite{emery1879elements}, Chadwick
expressed Neapolitan chords using a special figure $\rn{N}$.
Chadwick was explicit in expressing a Neapolitan as being in
first inversion, $\rn{N}\rnsix$, as shown in
\reffig{primary_sources/chadwick1897harmony148}. Later,
\textcite{heacox1907lessons} adopted this syntax as well,
notating different chord inversions of the neapolitan using
stacks of Arabic numerals. \textcite{alchin1921applied} also
adopted it.

\phdfigure[Neapolitan in \textcite{chadwick1897harmony}]{primary_sources/chadwick1897harmony148}

\textcite{weber1817versuch} did not introduce examples of
augmented chords. Thus, the Roman numeral notation did not
include a symbol for them. 

\textcite{richter1860lehrbuch} was arguably the first to
introduce the single quote symbol to denote augmented triads
(see \reffig{primary_sources/richter1860lehrbuch34}). This
notation was adopted by several others, including
\textcite{jadassohn1883lehrbuch},
\textcite{broekhoven1889system}, \textcite{buwa1893schule},
and \textcite{shepard1896harmony}. However, this notation is
fairly absent nowadays. Much more common is the use of the
$\rn{III+}$ notation, with a plus ($+$) symbol in place of
the single quote. This notation was maybe popularized by
\textcite{riemann1890katechismus}, as shown in
\reffig{primary_sources/riemann1890katechismus064}.

\phdfigure[Augmented triads in \textcite{riemann1890katechismus}, indicated as $\rn{III+}$][0.8]{primary_sources/riemann1890katechismus064}

% \phdfigure[Augmented triads in
% \textcite{jadassohn1883lehrbuch}, indicated as
% $\rn{III'}$][0.8]{primary_sources/jadassohn1883lehrbuch038}

% \phdfigure[Augmented triads in
% \textcite{broekhoven1889system}, using a similar notation
% to the one found in
% \textcite{jadassohn1883lehrbuch}][0.8]{primary_sources/broekhoven1889system028}

\phdfigure[The notation for augmented triads in \textcite{chadwick1897harmony}, which is similar to the one by \textcite{riemann1890katechismus}]{primary_sources/chadwick1897harmony053}

Chadwick adopted the Riemann notation for augmented triads,
using $\rn{III+}$, as shown in
\reffig{primary_sources/chadwick1897harmony053}.  