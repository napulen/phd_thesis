% Copyright 2022 Néstor Nápoles López

% This is
% \refsubsec{neapolitans,augmentedsixths,andotherspecialchords},
% which introduces the neapolitans, augmented sixths, and
% other special chords.

Emery also introduced a special symbol to denote Neapolitan
chords, $\rn{N.6}$. For the augmented sixth chords (i.e.,
Italian, French, and German), Emery uses a stack of Arabic
numerals. The example shown in
\reffig{primary_sources/emery1879elements051} shows this
notation. Notice that additionally to the stack of Arabic
numerals in the layer of Roman numeral analysis, Emery also
presents a figured bass layer above the staff. Lastly,
Emery's notation for the augmented sixth quality is the
$\rn{+}$ symbol. Nowadays, used to denote augmented triads
as well (e.g., $\rn{c:III+}$ in the C harmonic minor scale).
Up to this point, the notation for modulations is to
indicate the changes of key at the level of the Roman
numeral annotations. A full example of the notation can be
seen in another excerpt by Emery, shown in
\reffig{primary_sources/emery1879elements102}.

\phdfigure[Use of Roman numerals in \textcite{emery1879elements}, with notations for Neapolitan chords and augmented sixth chords][1.0]{primary_sources/emery1879elements051}
\phdfigure[Modulations in \textcite{emery1879elements}. Notice that five keys are indicated within eight measures of music, an usually large number of key changes for modern practices. In the modern practice, some of these annotations (e.g., the two D-minor chords) would be expressed as \emph{tonicizations}][1.0]{primary_sources/emery1879elements102}

A notation for augmented triads was introduced by
\textcite{jadassohn1883lehrbuch}, who used a single quote
after the augmented scale degree.

\phdfigure[Augmented triads in \textcite{jadassohn1883lehrbuch}, indicated as $\rn{III'}$]{primary_sources/jadassohn1883lehrbuch038}

This syntax of the single-quote augmented triad appeared in
at least the treatises by \textcite{broekhoven1889system},
\textcite{buwa1893schule}, and
\textcite{shepard1896harmony}.

\phdfigure[Augmented triads in \textcite{broekhoven1889system}, using a similar notation to the one found in \textcite{jadassohn1883lehrbuch}]{primary_sources/broekhoven1889system028}
