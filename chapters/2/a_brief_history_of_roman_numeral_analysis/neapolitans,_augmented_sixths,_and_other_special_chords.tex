% Copyright 2022 Néstor Nápoles López

% This is
% \refsubsec{neapolitans,augmentedsixths,andotherspecialchords},
% which introduces the neapolitans, augmented sixths, and
% other special chords.


The stacks of Arabic numerals in Emery's example
(\reffig{primary_sources/emery1879elements051}) describe
augmented sixth chords (i.e., Italian, French, and German).
This is possibly the first time that the \gls{rna} notation
is extended to describe such chords, beyond the diatonic
harmonies occurring in major and minor scales. Emery also
introduced a special symbol to annotate \emph{neapolitan}
chords, $\rn{N.6}$.\footnote{Annotating special chords, such
as augmented sixth chords or \emph{neapolitans} is probably
one of the most important characteristics of \gls{rna} to
this day, as these chords are generally beyond the scope of
traditional chord labels. Presumably, this is because
identifying these chords requires an understanding of the
key, an aspect that is generally neglected in traditional
chord labels but intrinsically available in \gls{rna}.}

Similarly to \textcite{emery1879elements}, Chadwick
expressed Neapolitan chords using a special figure $\rn{N}$.
Chadwick was explicit in expressing a Neapolitan as being in
first inversion, $\rn{N}\rnsix$, as shown in
\reffig{primary_sources/chadwick1897harmony148}.

\phdfigure[Neapolitan in \textcite{chadwick1897harmony}]{primary_sources/chadwick1897harmony148}

A notation for augmented triads was introduced by
\textcite{jadassohn1883lehrbuch}, who used a single quote
after the augmented scale degree.

\phdfigure[Augmented triads in \textcite{jadassohn1883lehrbuch}, indicated as $\rn{III'}$][0.8]{primary_sources/jadassohn1883lehrbuch038}

This syntax of the single-quote augmented triad appeared in
at least the treatises by \textcite{broekhoven1889system},
\textcite{buwa1893schule}, and
\textcite{shepard1896harmony}.

\phdfigure[Augmented triads in \textcite{broekhoven1889system}, using a similar notation to the one found in \textcite{jadassohn1883lehrbuch}][0.8]{primary_sources/broekhoven1889system028}

\phdfigure[The notation for augmented triads in \textcite{chadwick1897harmony}, which is similar to the one by \textcite{riemann1890katechismus}]{primary_sources/chadwick1897harmony053}

Chadwick adopted the Riemann notation for augmented triads,
using $\rn{III+}$, as shown in
\reffig{primary_sources/chadwick1897harmony053}.  