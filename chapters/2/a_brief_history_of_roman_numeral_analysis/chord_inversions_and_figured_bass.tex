% Copyright 2022 Néstor Nápoles López

During the earlier \gls{rna} examples of the
19\textsuperscript{th} century, it was common to complement
the notation with figured bass. The figured bass was often
clearly separated from the Roman numerals, for example, by
annotating one below the staff and the other one above, as
seen in Figures
\ref{fig:primary_sources/hamilton1840cathecism044},
\ref{fig:primary_sources/meister1852vollstandige32}, or
\ref{fig:primary_sources/emery1879elements051}.

In \reffig{primary_sources/bussler1878praktische063},
however, a differing example in
\textcite{bussler1878praktische} shows the Roman numerals
and figured bass annotations in the same analytical layer.
One could argue that this syntax is an early version of the
modern chord inversion syntax, where inversions are denoted
by stacks of Arabic numerals next to the Roman numeral. This
practice was further developed in
\textcite{emery1879elements}, shown in
\reffig{primary_sources/emery1879elements051}. In Emery's
example, it is clearer that the stacks of Arabic numerals in
the \gls{rna} layer have a special meaning, as figured bass
annotations also appear above the staff. These stacks of
Arabic numerals are indicators of special chords, which will
be discussed in
\refsubsec{neapolitans,augmentedsixths,andotherspecialchords}.

\phdfigure[A single layer of chord annotations underneath
 the bass staff in \textcite[p.~63]{bussler1878praktische},
 where Roman numerals and figured bass labels are
 intertwined][1.0]{primary_sources/bussler1878praktische063}

\phdfigure[Use of Roman numerals and figured bass in
\textcite[p.~51]{emery1879elements}. In the Roman numeral
layer, stacks of Arabic numerals indicate augmented sixth
chords. Additionally, a notation for \gls{neapolitan} chords
is introduced. See
\refsubsec{neapolitans,augmentedsixths,andotherspecialchords}
for further discussion of \gls{neapolitan} and augmented
sixths chords][1.0]{primary_sources/emery1879elements051}

Additionally to the special chords in
\textcite{emery1879elements}, the so-called \emph{cadential
six-four} chord could have also encouraged the numeric
notation for chord inversions.

\textcite{shepard1896harmony} included figured bass
annotations next to the Roman numeral to indicate certain
chord inversions, notably, the \emph{cadential six-four}
chord, as shown in
\reffig{primary_sources/shepard1896harmony184}. This
notation was consistent even when there was no music
notation involved. In such case, Shepard would write the
Arabic numerals next to the Roman numeral in plain-text
form, as shown in
\reffig{primary_sources/shepard1896harmony117}.

\phdfigure[Numeric inversions in
\textcite[p.~184]{shepard1896harmony}][0.9]{primary_sources/shepard1896harmony184}

\phdfigure[Numeric inversions in plain-text, without
accompanying music notation
\parencite[p.~117]{shepard1896harmony}][0.6]{primary_sources/shepard1896harmony117}

\textcite{chadwick1897harmony} followed a similar practice,
also indicating the numeric inversions in examples without
music notation, as shown in
\reffig{primary_sources/chadwick1897harmony064}.
\textcite{chadwick1897harmony} also provided an explanation
of the numeric inversions, shown in
\reffig{primary_sources/chadwick1897harmony012}. This
notation goes beyond triads, including, for example
$\rn{V}\rnfourthree$, as shown in
\reffig{primary_sources/chadwick1897harmony028}.

Finally, a more compelling example of the role of
\emph{cadential six-four} chords in the notation of chord
inversions can be seen in \textcite{loewengard1908lehrbuch},
where the \emph{cadential six-four} figure
($\rn{I}\rnsixfour$) has the numeric inversion notation, but
the $\rn{ii}\rnsix$ chord in the same line does not (see
\reffig{primary_sources/loewengard1908lehrbuch045}).

\phdfigure[\textcite[p.~38]{chadwick1897harmony} describing
numeric inversions even in examples without music notation
in them][1.0]{primary_sources/chadwick1897harmony038}

\phdfigure[Arabic-numeral inversions in
 \textcite[p.~12]{chadwick1897harmony}][1.0]{primary_sources/chadwick1897harmony012}

\phdfigure[A dominant seventh chord in second inversion in
\textcite[p.~28]{chadwick1897harmony}][1.0]{primary_sources/chadwick1897harmony028}

\phdfigure[Missing first inversion of the $\rn{ii}\rnsix$
chord (measure 6) in
\textcite[p.~45]{loewengard1908lehrbuch}][1.0]{primary_sources/loewengard1908lehrbuch045}

An alternative notation for chord inversions adopted by
several theorists consists of the use of letters. In
\textcite{cutter1902harmonic}, both notations are explained:

\begin{quote}
    ``The inversions of triads and of seventh chords, both
    principal and secondary, will be indicated by the
    customary figurings: $\rnsix$, $\rnsixfour$,
    $\rnsixfive$, $\rnfourthree$, $\rnfourtwo$, attached to
    the respective Roman numerals. Or, the letters a, b, c,
    d, meaning root-form, first, second, and third
    inversions, may be used with these same numerals. Thus:
    $\rn{I}_\rnformat{a}$, $\rn{I}_\rnformat{b}$,
    $\rn{I}_\rnformat{c}$,
    $\rn{ii}\rndim{}^\rnformat{7}_\rnformat{a}$,
    $\rn{ii}\rndim{}^\rnformat{7}_\rnformat{b}$,
    $\rn{ii}\rndim{}^\rnformat{7}_\rnformat{c}$,
    $\rn{IV}^\rnformat{+}_\rnformat{a}$,
    $\rn{iii}^\rnformat{7}_\rnformat{c}$, etc. The
    diminished seventh chord, in its various forms, will be
    marked: $\rn{vii}^\rnformat{o}_\rnformat{7o}$,
    $\rn{vii}\rndim\rnsixfive$,
    $\rn{vii}\rndim\rnfourthree$, $\rn{vii}\rndim\rnfourtwo$
    --- or $\rn{vii}^{\rnformat{o}}_{\rnformat{a}}{}^\rnformat{7o}$,
    $\rn{vii}^{\rnformat{o}}_{\rnformat{b}}{}^\rnformat{7o}$,
    $\rn{vii}^{\rnformat{o}}_{\rnformat{c}}{}^\rnformat{7o}$,
    $\rn{vii}^{\rnformat{o}}_{\rnformat{d}}{}^\rnformat{7o}$.''

    --- \textcite[p.~4]{cutter1902harmonic}
\end{quote}

The numeric inversion syntax was already used by previous
theorists, however, the letters ${a, b, c , d}$ appear for
the first time in \textcite{cutter1902harmonic} among the
books surveyed. Cutter also seems to be the only author
acknowledging the existence of both notations, possibly
because his treatise was on harmonic analysis, rather than a
harmony textbook. Throughout the examples, however, only the
numeric inversions are used.

% \phdfigure[Inversions by Arabic numerals and by letters
% explained in
% \textcite{cutter1902harmonic}][1.0]{primary_sources/cutter1902harmonic004}

\phdfigure[\textcite{york1909practical}, where the inversion
by letter notation is
preferred][1.0]{primary_sources/york1909practical019}

The notation for chord inversions based on letters is less
common than the stacks of Arabic numerals. One place where
the letter notation was preferred was
\textcite{york1909practical}, where it appears prominently.
\reffig{primary_sources/york1909practical019} shows the
introduction of the chord inversion notation used by York.
