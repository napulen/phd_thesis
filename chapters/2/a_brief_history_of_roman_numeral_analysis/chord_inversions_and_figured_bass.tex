% Copyright 2022 Néstor Nápoles López

During the earlier \gls{rna} examples of the nineteenth
century, it was common to complement the notation with
figured bass. The figured bass was often clearly separated
from the Roman numerals, for example, by annotating one
below the staff and the other one above, as seen in
\reffig{primary_sources/hamilton1840cathecism044} or
\reffig{primary_sources/meister1852vollstandige32}.

In \reffig{primary_sources/bussler1878praktische063}, an
interesting example by \textcite{bussler1878praktische},
however, shows the Roman numerals and figured bass
annotations in the same analytical layer. One could argue
that this syntax is an early version of the modern chord
inversion syntax, where inversions are denoted by stacks of
Arabic numerals next to the Roman numeral.

\phdfigure[A single layer of chord annotations underneath the bass staff in \textcite{bussler1878praktische}, where Roman numerals and figured bass labels are intertwined][1.0]{primary_sources/bussler1878praktische063}

This practice was further developed by
\textcite{emery1879elements}, shown in
\reffig{primary_sources/emery1879elements051}. In Emery's
example, it is clearer that the stacks of Arabic numerals in
the \gls{rna} layer have a special meaning, as figured bass
annotations also appear above the staff. These stacks Arabic
numerals are indicators of special chords.

\phdfigure[Use of Roman numerals in \textcite{emery1879elements}, with notations for Neapolitan chords and augmented sixth chords][1.0]{primary_sources/emery1879elements051}
