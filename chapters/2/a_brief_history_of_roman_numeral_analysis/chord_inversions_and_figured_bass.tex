% Copyright 2022 Néstor Nápoles López

During the earlier \gls{rna} examples of the nineteenth
century, it was common to complement the notation with
figured bass. The figured bass was often clearly separated
from the Roman numerals, for example, by annotating one
below the staff and the other one above, as seen in Figures
\ref{fig:primary_sources/hamilton1840cathecism044},
\ref{fig:primary_sources/meister1852vollstandige32}, or
\ref{fig:primary_sources/emery1879elements051}.

In \reffig{primary_sources/bussler1878praktische063},
however, a differing example by
\textcite{bussler1878praktische} shows the Roman numerals
and figured bass annotations in the same analytical layer.
One could argue that this syntax is an early version of the
modern chord inversion syntax, where inversions are denoted
by stacks of Arabic numerals next to the Roman numeral. This
practice was further developed by
\textcite{emery1879elements}, shown in
\reffig{primary_sources/emery1879elements051}. In Emery's
example, it is clearer that the stacks of Arabic numerals in
the \gls{rna} layer have a special meaning, as figured bass
annotations also appear above the staff. These stacks of
Arabic numerals are indicators of special chords, which will
be discussed in
\refsubsec{neapolitans,augmentedsixths,andotherspecialchords}.

\phdfigure[A single layer of chord annotations underneath the bass staff in \textcite{bussler1878praktische}, where Roman numerals and figured bass labels are intertwined][1.0]{primary_sources/bussler1878praktische063}
\phdfigure[Use of Roman numerals and figured bass in \textcite{emery1879elements}. In the Roman numeral layer, stacks of Arabic numerals indicate augmented sixth chords. Additionally, a notation for Neapolitan chords is introduced][1.0]{primary_sources/emery1879elements051}


\textcite{shepard1896harmony} included figured bass
annotations next to the Roman numeral to indicate certain
chord inversions, notably, the \emph{cadential six-four}
chord, as shown in
\reffig{primary_sources/shepard1896harmony184}. This
notation was consistent even when there was no music
notation involved. In such case, Shepard would write the
Arabic numerals next to the Roman numeral in plain-text
form, as shown in
\reffig{primary_sources/shepard1896harmony117}.

\phdfigure[Shepard's Arabic-numeral inversions \textcite{shepard1896harmony}]{primary_sources/shepard1896harmony184}
\phdfigure[Shepard's Arabic-numeral inversions \textcite{shepard1896harmony}]{primary_sources/shepard1896harmony117}

Regarding numeric inversions, \textcite{chadwick1897harmony}
made this notation explicit.
\reffig{primary_sources/chadwick1897harmony012}
shows his explanation. This notation goes beyond triads,
including for example $\rn{V}\rnfourthree$, shown in
\reffig{primary_sources/chadwick1897harmony028}.

\phdfigure[Chadwick's Arabic-numeral inversions \textcite{chadwick1897harmony}]{primary_sources/chadwick1897harmony012}

\phdfigure[A dominant seventh chord in second inversion in \textcite{chadwick1897harmony}]{primary_sources/chadwick1897harmony028}

\phdfigure[\textcite{chadwick1897harmony} describing numeric inversions even in examples without music notation in them]{primary_sources/chadwick1897harmony064}

As with Shepard, the numeric inversions are visible even in
annotations without any music notation (see
\reffig{primary_sources/chadwick1897harmony064}).

\phdfigure[Inversions by Arabic numerals and by letters explained in \textcite{halm1900harmonielehre}]{primary_sources/cutter1902harmonic004}

Another notation for chord inversions adopted by several
theorists consists of the use of letters. In
\textcite{cutter1902harmonic}, both notations are explained
(see \reffig{primary_sources/cutter1902harmonic004}). The
numeric inversion syntax was already used by previous
theorists, however, the letters ${a, b, c , d}$ do not seem
to appear before this treatise. It is also unusual that
Cutter mentions both syntaxes in the same work, possibly
because this is a treatise on harmonic analysis, rather than
a harmony textbook (i.e., harmonization). Throughout the
examples, however, only the numeric inversions are used.

The numeric inversion notation could have started,
additionally to augmented sixth chords, with cadential
six-four chords. A clear example of this can be seen in
\textcite{loewengard1908lehrbuch}, where the cadential
six-four figure ($\rn{I}\rnsixfour$) has the numeric
inversion notation, but the $\rn{ii}\rnsix$ chord in the
same line does not (see
\reffig{primary_sources/loewengard1908lehrbuch045}).

\phdfigure[Missing first inversion in \textcite{loewengard1908lehrbuch}]{primary_sources/loewengard1908lehrbuch045}

The notation for chord inversions based on letters is less
common than the stacks of Arabic numerals. One place where
the letter notation was preferred was
\textcite{york1909practical}, where it appears prominently.
\reffig{primary_sources/york1909practical019} shows the
introduction of the chord inversion notation used by York.

\phdfigure[\textcite{york1909practical}, where the inversion by letter notation is preferred]{primary_sources/york1909practical019}