% Copyright 2022 Néstor Nápoles López

It was common to complement the Roman numeral notation with
figured bass. The figured bass notation was often clearly
separated from the Roman numerals, for example, notating one
notation below the staff and the other above, as in
\reffig{primary_sources/hamilton1840cathecism044} or
\reffig{primary_sources/meister1852vollstandige32}.

An interesting example by \textcite{bussler1878praktische}
is shown in
\reffig{primary_sources/bussler1878praktische063}, who
presents Roman numerals and figured bass annotations in the
same analytical layer. One could claim that this syntax is
an early version of the modern chord inversion syntax, where
inversions are denoted by stacks of Arabic numerals next to
the Roman numeral.

\phdfigure[A single layer of chord annotations underneath the bass staff in \textcite{bussler1878praktische}, where Roman numerals and figured bass labels are intertwined][1.0]{primary_sources/bussler1878praktische063}

This practice is further developed by
\textcite{emery1879elements}. 

% At first, Roman numerals did not indicate chord
% inversions. Later, the practice of interconnecting figured
% bass and Roman numeral annotations might have resulted in
% the notation used today to indicate chord inversions with
% Arabic numerals