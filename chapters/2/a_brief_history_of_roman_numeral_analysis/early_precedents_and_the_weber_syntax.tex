% Copyright 2022 Néstor Nápoles López

% This is \refsubsec{earlyprecedentsandthewebersyntax},
% which introduces the early precedents and the weber
% syntax.

% Copyright 2022 Néstor Nápoles López

% This is \refsubsec{theromannumeraltimeline}, which
% introduces the the roman numeral timeline.

Before the development of the \gls{rna} syntax, Roman
numerals were used to indicate scale-degree relationships.
For example, in two tables of \emph{Die Kunst der reinen
Satzes in der Musik} in \textcite{kirnberger1774kunst},
shown in \reffig{primary_sources/kirnberger1774kunst015}.

\phdfigure[Roman numerals in \textcite{kirnberger1774kunst}][0.5]{primary_sources/kirnberger1774kunst015}
\phdfigure[Roman numerals in \textcite{vogler1778grunde}]{primary_sources/vogler1778grunde021}

Later, Vogler introduced Roman numeral symbols underneath a
staff, which also referred to a scale degree
\parencite{vogler1778grunde, vogler1802handbuch}. These are
used a few times on both works.
\reffig{primary_sources/vogler1778grunde021} shows an
example in \textcite{vogler1778grunde}.

\phdfigure[Roman numerals in \textcite{weber1817versuch}. Keys are indicated with colons (e.g., mm. 1 and mm.5). Dominant seventh chords are indicated as $\rn{V}\rnseven$. Smaller Roman numerals indicate minor triads][1.0]{primary_sources/weber1817versuch205}

While it is difficult to credit someone with ``inventing''
\gls{rna}, the modern notation would probably not exist
without the precedent of \textcite{weber1817versuch}. Weber
extended the notation to indicate not only scale degrees but
their chord quality. A special notation separates diminished
triads (i.e., the seventh degree) from major and minor
chords; whereas major and minor are distinguished by the
size of the Roman numeral.\footnote{Nowadays, instead of
Roman numerals of different sizes, it is more common to see
upper- and lower-case Roman numerals to indicate major and
minor triads, respectively.} Weber introduced several other
traits of modern \gls{rna}. For example, changes of key are
indicated using a colon (``$\rn{:}$'') symbol and analyses
with several rows of keys and Roman numeral indications
appear in modulating passages. Finally, whereas Kirnberger
and Vogler used Roman numerals sporadically, Weber used the
notation extensively in the
\emph{Versuch einer geordneten Theorie der Tonsetzkunst}
\parencite{weber1817versuch}. Weber indicated dominant
seventh chords as $\rn{V}\rnseven$, which is a notation
thatis still in use today. Whereas others relied on figured
bass to explain their examples, Weber relied mostly on Roman
numerals.
