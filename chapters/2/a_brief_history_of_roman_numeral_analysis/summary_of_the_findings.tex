In summary, the survey revealed the following findings:

\phdparagraph{the weber syntax}
The role of \textcite{weber1817versuch} in defining the
basic characteristics of the notation was crucial.


\phdparagraph{adoption of the syntax}
After \textcite{weber1817versuch}, the Roman numeral
annotation syntax slowly gained popularity among English and
German theorists, achieving a more solid adoption by the end
of the 19\textsuperscript{th} century.

\phdparagraph{figured bass and roman numeral inversions}
 The Roman numeral notation was often accompanied with
figured bass. At first, Roman numerals did not indicate
chord inversions. Later, the practice of interconnecting
figured bass and Roman numeral annotations might have
resulted in the notation used today to indicate chord
inversions with Arabic numerals.

\phdparagraph{special chords}
Special symbols were introduced for certain chords. Notably,
\emph{neapolitans} received their own symbol (``$\rn{N}$'').
Sometimes ``competing'' syntaxes were used with the same
meaning. For example, some authors preferred a single quote
($\rn{'}$) symbol to denote augmented triads, whereas others
used a plus sign (``$\rn{+}$'').\footnote{Nowadays, the
``plus'' notation is more common.} An example of each of
these notations is shown in Figures
\ref{fig:richter1860lehrbuch34} and
\ref{fig:riemann1890katechismus064}, respectively.

\phdparagraph{the syntax for tonicizations}
The treatment of tonicizations was originally absent. Any
change of key was notated with a colon preceded of a new
reference key. Eventually, a few ways of notating common
tonicizations (e.g., \emph{dominant of the dominant})
emerged. These evolved into the slash syntax we use today,
although it is unclear who should be credited for the slash
notation.

\phdparagraph{the chord vocabulary}
Recently, the vocabulary of chords has been extended, new
additions consider, for example, the common-tone diminished
seventh, a chord that is often written as
$\rn{CT}\rndim\rnseven$. This also speaks of the continuing
evolution of the syntax.
