% Copyright 2022 Néstor Nápoles López

% This is \refsubsec{appliedchordsandtonicization}, which
% introduces the applied chords and tonicization.


%% EMERY's MODULATION Lastly, Emery's notation for the
% augmented sixth quality is the $\rn{+}$ symbol. Nowadays,
% used to denote augmented triads as well (e.g.,
% $\rn{c:III+}$ in the C harmonic minor scale). Up to this
% point, the notation for modulations is to indicate the
% changes of key at the level of the Roman numeral
% annotations. A full example of the notation can be seen in
% another excerpt by Emery, shown in
% \reffig{primary_sources/emery1879elements102}.
% \phdfigure[Modulations in \textcite{emery1879elements}.
% Notice that five keys are indicated within eight measures
% of music, an usually large number of key changes for
% modern practices. In the modern practice, some of these
% annotations (e.g., the two D-minor chords) would be
% expressed as
% \emph{tonicizations}][1.0]{primary_sources/emery1879elements102}


An important component of the modern \gls{rna} syntax is the
one used for tonicizations. Tonicizations are slight
deviations of key, which usually return to the original key,
without ``invoking'' a modulation. Nowadays, tonicizations
are generally written with a slash symbol, where the
numerator indicates the scale degree and the denominator
indicates the tonicized key. A common example is $\rn{V/V}$
or (dominant of the dominant). It is unclear where this
notation originally emerged, however, a similar notation was
introduced in \textcite{shepard1889how}. Shepard's notation,
which he calls ``[A]ttendant chords'', is presented as
``$\rn{[A] of V}$'' in the notation, shown in
\reffig{primary_sources/shepard1889how005}. This is
analogous to the modern $\rn{V/V}$. An advantage of the
modern notation is that other degrees other than $\rn{V}$
(in the numerator) can be annotated using the same
syntax.\footnote{\textcite{piston1941harmony}, in fact, used
Shepard's notation in this way, using a syntax of the form
``$\rn{V of V}$''.} In the explanations of his modulation
method, Shepard also introduces a modulation ``formula''
that resembles the fraction-like notation we use today (see
\reffig{primary_sources/shepard1889how010}). 

\phdfigure[Attendant chords in \textcite{shepard1889how}][0.8]{primary_sources/shepard1889how005}

\phdfigure[Modulation formula in \textcite{shepard1889how}][0.8]{primary_sources/shepard1889how010}


Halm can be considered another pioneer of the $\rn{V/V}$
notation. He used a similar notation in
\textcite{halm1900harmonielehre}, $\rn{V--V}$, to refer to
secondary dominant chords. However, the meaning of this
notation is not very concise. For instance, it is used in
the form of $\rn{I--V}$ and $\rn{I--IV}$ to denote chords
that act as tonics in one key and as dominants in another.
Yet, that notation would lead to a mistaken interpretation
in the case of $\rn{V--V}$, which should perhaps be written
as $\sharp{}\rn{II--V}$ using the same convention.
Nevertheless, the notation is similar to the modern way of
thinking about applied chords. An example of Halm's notation
is shown in
\reffig{primary_sources/halm1900harmonielehreXVII}

\phdfigure[Notation for secondary dominants in \textcite{halm1900harmonielehre}]{primary_sources/halm1900harmonielehreXVII}

Schenker adopted several notational conventions in
\emph{Neue Musikalische Theorien und Phantasien:
Harmonielehre} \parencite{schenker1906neue}. Throughout this
book, pivot chords that have multiple meanings in different
keys are usually expressed in Roman numerals. Schenker
sometimes uses the notation $\rn{VI=IV}$, and other times
$\rn{VI/IV}$. One of his specific examples can be seen in
\reffig{primary_sources/schenker1906neue190},
% \ref{fig:primary_sources/schenker1906neue079} and
% \ref{fig:primary_sources/schenker1906neue190}.

% \phdfigure[Schenker notation for applied
% chords]{primary_sources/schenker1906neue078}

% \phdfigure[Schenker notation for applied
% chords]{primary_sources/schenker1906neue079}

\phdfigure[Schenker notation for applied chords]{primary_sources/schenker1906neue190}

An interesting notation for secondary dominants appears in
\textcite{white1911harmonic}, where it seems to be
introduced particularly for diminished seventh chords. In
this notation, secondary dominants are presented within
parentheses, indicating the key they tonicize. It is clear
that these changes of key are short-lived, not affecting the
key of the next chord after the closing parenthesis. This is
consistent with a notation of \emph{tonicizations} and
\emph{modulations}, as seen in recent works of computational
music theory. An example of this notation is shown in
\reffig{primary_sources/white1911harmonic110}.

\phdfigure[Notation for key fluctuations, in parentheses, appearing in \textcite{white1911harmonic}]{primary_sources/white1911harmonic110}

% The notation of $\rn{V of V}$ is observed in
% \textcite{mokrejs1913lessons}, which is shown in
% \reffig{primary_sources/mokrejs1913lessons079}.

% \phdfigure[Applied chord syntax in
% \textcite{mokrejs1913lessons}]{primary_sources/mokrejs1913lessons079}

% \textcite{gilson1919traite} presented a
% numerator/denominator notation of Roman numerals, which is
% shown in \reffig{primary_sources/gilson1919traite049}.

% \phdfigure[Numerator/denominator notation of Roman
% numerals in
% \textcite{gilson1919traite}]{primary_sources/gilson1919traite049}

% \phdfigure[Secondary dominants with $\rn{VV}$ notation in
% \textcite{gilson1919traite}]{primary_sources/gilson1919traite090}

% The Roman numeral notation by Gilson is a mixture of
% Riemannian functions and scale-degree Roman numerals, an
% unconventional style in current workflows. Gilson also
% introduced the notation $\rn{VV}$ for \emph{dominant of
% the dominant} chords (see
% \reffig{primary_sources/gilson1919traite090}).


% The modern notation for tonicizations appears in
% \textcite{tischler1964practical}.

% The Shepard notation also shows up in
% \textcite{goldman1965harmony}.
