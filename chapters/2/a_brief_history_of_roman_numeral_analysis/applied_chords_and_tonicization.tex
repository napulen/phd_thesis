% Copyright 2022 Néstor Nápoles López

% This is \refsubsec{appliedchordsandtonicization},
% which introduces the applied chords and tonicization.

An important component of the modern Roman numeral analysis
syntax is the one used for tonicizations. Tonicizations are
slight deviations of key, which usually return to the
original key, without ``invoking'' a modulation. Nowadays,
tonicizations are generally written with a slash symbol,
where the numerator indicates the scale degree and the
denominator indicates the tonicized key. A common example is
$\rn{V/V}$ or (dominant of the dominant). It is unclear
where this notation originally emerged, however, a similar
notation was introduced by \textcite{shepard1889how}.
Shepard's notation, which he calls ``[A]ttendant chords'',
is presented as ``$\rn{[A] of V}$'' in the notation, shown
in \reffig{primary_sources/shepard1889how005}. This is
analogous to the modern $\rn{V/V}$. An advantage of the
modern notation is that other degrees other than $\rn{V}$
(in the numerator) can be annotated using the same
syntax.\footnote{\textcite{piston1941harmony}, in fact, used
Shepard's notation in this way, using a syntax of the form
``$\rn{V of V}$''.} In the explanations of his modulation
method, Shepard also introduces a modulation ``formula''
that resembles the fraction-like notation we use today (see
\reffig{primary_sources/shepard1889how010}). Lastly, Shepard
included figured bass annotations next to the Roman numeral
to indicate certain chord inversions, notably, the cadential
six-four chord, as shown in
\reffig{primary_sources/shepard1896harmony184}. This
notation was consistent even when there was no music
notation involved, where Shepard would write the Arabic
numerals next to the Roman numeral in plain-text form, as
shown in \reffig{primary_sources/shepard1896harmony117}.

\phdfigure[Attendant chords in \textcite{shepard1889how}]{primary_sources/shepard1889how005}

\phdfigure[Modulation formula in \textcite{shepard1889how}]{primary_sources/shepard1889how010}

\phdfigure[Shepard's Arabic-numeral inversions \textcite{shepard1896harmony}]{primary_sources/shepard1896harmony184}

\phdfigure[Shepard's Arabic-numeral inversions \textcite{shepard1896harmony}]{primary_sources/shepard1896harmony117}

Regarding numeric inversions, \textcite{chadwick1897harmony}
made this notation explicit.
\reffig{primary_sources/chadwick1897harmony012}
shows his explanation. This notation goes beyond triads,
including for example $\rn{V}\rnfourthree$, shown in
\reffig{primary_sources/chadwick1897harmony028}.

\phdfigure[Chadwick's Arabic-numeral inversions \textcite{chadwick1897harmony}]{primary_sources/chadwick1897harmony012}

\phdfigure[A dominant seventh chord in second inversion in \textcite{chadwick1897harmony}]{primary_sources/chadwick1897harmony028}

\phdfigure[The notation for augmented triads in \textcite{chadwick1897harmony}, which is similar to the one by \textcite{riemann1890katechismus}]{primary_sources/chadwick1897harmony053}

\phdfigure[\textcite{chadwick1897harmony} describing numeric inversions even in examples without music notation in them]{primary_sources/chadwick1897harmony064}

\phdfigure[Neapolitan in \textcite{chadwick1897harmony}]{primary_sources/chadwick1897harmony148}

Chadwick adopted the Riemann notation for augmented triads,
using $\rn{III+}$, as shown in
\reffig{primary_sources/chadwick1897harmony053}. As with
Shepard, the numeric inversions are visible even in
annotations without any music notation (see
\reffig{primary_sources/chadwick1897harmony064}). Similarly
to \textcite{emery1879elements}, Chadwick expressed
Neapolitan chords using a special figure $\rn{N}$. Chadwick
was explicit in expressing a Neapolitan as being in first
inversion, $\rn{N}\rnsix$, as shown in
\reffig{primary_sources/chadwick1897harmony148}.

\textcite{halm1900harmonielehre} can be considered another
pioneer of the $\rn{V/V}$ notation. He used a similar
notation $\rn{V--V}$ to refer to secondary dominant chords.
The meaning of this notation is not very clear, because
previously he uses it in the form of $\rn{I--V}$ and
$\rn{I--IV}$ to denote chords that act as a tonic in one key
and as a dominant in another key, but this notation does not
work for his $\rn{V--V}$, which should be written as
$\sharp{}\rn{II--V}$. Nevertheless, the notation is similar
to the modern way of thinking about applied chords. An
example of Halm's notation is shown in
\reffig{primary_sources/halm1900harmonielehreXVII}

\phdfigure[Notation for secondary dominants in \textcite{halm1900harmonielehre}]{primary_sources/halm1900harmonielehreXVII}

\phdfigure[Inversions by Arabic numerals and by letters explained in \textcite{halm1900harmonielehre}]{primary_sources/cutter1902harmonic004}

Another notation for chord inversions adopted by several
theorists consists of the use of letters. In
\textcite{cutter1902harmonic}, both notations are explained
(see \reffig{primary_sources/cutter1902harmonic004}). The
numeric inversion syntax was already used by previous
theorists, however, the letters ${a, b, c , d}$ do not seem
to appear before this treatise. It is also unusual that
Cutter mentions both syntaxes in the same work, possibly
because this is a treatise on harmonic analysis, rather than
a harmony textbook (i.e., harmonization). Throughout the
examples, however, only the numeric inversions are used.

Schenker adopted several notational conventions in
\emph{Neue Musikalische Theorien und Phantasien:
Harmonielehre} \parencite{schenker1906neue}. Throughout this
book, pivot chords that have multiple meanings in different
keys are usually expressed in Roman numerals. Schenker
sometimes uses the notation $\rn{VI=IV}$, and other times
$\rn{VI/IV}$. Some of his specific examples can be seen in
Figures \ref{fig:primary_sources/schenker1906neue190},
\ref{fig:primary_sources/schenker1906neue079} and
\ref{fig:primary_sources/schenker1906neue190}.

\phdfigure[Schenker notation for applied chords]{primary_sources/schenker1906neue078}

\phdfigure[Schenker notation for applied chords]{primary_sources/schenker1906neue079}

\phdfigure[Schenker notation for applied chords]{primary_sources/schenker1906neue190}

The numeric inversion notation could have started,
additionally to augmented sixth chords, with cadential
six-four chords. A clear example of this can be seen in
\textcite{loewengard1908lehrbuch}, where the cadential
six-four figure ($\rn{I}\rnsixfour$) has the numeric
inversion notation, but the $\rn{ii}\rnsix$ chord in the
same line does not (see
\reffig{primary_sources/loewengard1908lehrbuch045}).

\phdfigure[Missing first inversion in \textcite{loewengard1908lehrbuch}]{primary_sources/loewengard1908lehrbuch045}

The notation for chord inversions based on letters is less
common than the stacks of Arabic numerals. One place where
the letter notation was preferred was
\textcite{york1909practical}, where it appears prominently.
\reffig{primary_sources/york1909practical019} shows the
introduction of the chord inversion notation used by York.

\phdfigure[\textcite{york1909practical}, where the inversion by letter notation is preferred]{primary_sources/york1909practical019}

An interesting notation for secondary dominants appears in
\textcite{white1911harmonic}, where it seems to be
introduced particularly for diminished seventh chords. In
this notation, secondary dominants are presented within
parenthesis, indicating the key they tonicize. It is clear
that these changes of key are different from the other
indications of change, like the multirow analysis with a
second row of Roman numerals, which permanently change the
key in the next chord. This is consistent with a notation of
\emph{tonicizations} and \emph{modulations}, as seen in
recent works of computational music theory. An example of
this notation is shown in
\reffig{primary_sources/white1911harmonic110}.

\phdfigure[Notation for key fluctuations, in parentheses, appearing in \textcite{white1911harmonic}]{primary_sources/white1911harmonic110}

The notation of $\rn{V of V}$ is observed in
\textcite{mokrejs1913lessons}, which is shown in
\reffig{primary_sources/mokrejs1913lessons079}.

\phdfigure[Applied chord syntax in \textcite{mokrejs1913lessons}]{primary_sources/mokrejs1913lessons079}

\textcite{gilson1919traite} presented a
numerator/denominator notation of Roman numerals, which is
shown in \reffig{primary_sources/gilson1919traite049}.

\phdfigure[Numerator/denominator notation of Roman numerals in \textcite{gilson1919traite}]{primary_sources/gilson1919traite049}

\phdfigure[Secondary dominants with $\rn{VV}$ notation in \textcite{gilson1919traite}]{primary_sources/gilson1919traite090}

The Roman numeral notation by Gilson is a mixture of
Riemannian functions and scale-degree Roman numerals, an
unconventional style in current workflows. Gilson also
introduced the notation $\rn{VV}$ for \emph{dominant of the
dominant} chords (see
\reffig{primary_sources/gilson1919traite090}).

The Roman numeral syntax seems to be generally less popular
among French authors. For example, Koechlin sporadically
uses Roman numerals to indicate scale degrees of only the
bass note \textcite{koechlin1928traite}, shown in
\reffig{primary_sources/koechlin1928traite026}.

\phdfigure[The scale-degrees of \textcite{koechlin1928traite}, which refer to the only to the bass note and not the root of the chord]{primary_sources/koechlin1928traite026}

This practice is uncommon, as most of his contemporaries
think in terms of chord roots, not melodic scale degrees. In
contrast, figured bass notation seemed prominent among
French authors. In some English and German books, they often
refer to figured bass as the \emph{French}
system.\footnote{For example,
\textcite{norris1894practical}}

The modern notation for tonicizations appears in
\textcite{tischler1964practical}.

The Shepard notation also shows up in
\textcite{goldman1965harmony}. Goldman also employed the
symbols $\rn{Fr}$, $\rn{It}$ and $\rn{German}$ to refer to
the augmented sixth chords.
