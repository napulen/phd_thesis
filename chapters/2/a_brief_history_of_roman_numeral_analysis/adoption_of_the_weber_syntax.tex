% Copyright 2022 Néstor Nápoles López

% This is \refsubsec{adoptionofthewebersyntax}, which
% introduces the adoption of the weber syntax.

\phdfigure[Use of Roman numerals in \textcite{hamilton1840catechism}, indicating the scale degree of the bass, regardless of the chord root. In modern Roman numeral notation, these annotations would be written as $\rn{V}\rnsixfive$, $\rn{V}\rnfourthree$, $\rn{V}\rntwo$, $\rn{V}\rnfourthree$, and $\rn{V}\rnseven$, respectively][0.75]{primary_sources/hamilton1840cathecism044}

The notational system introduced by
\textcite{weber1817versuch} was not immediately adopted by
other theorists. One of the first adopters, an English
professor, was \textcite{hamilton1840catechism}. He,
together with several others throughout the years, did not
adopt the notation for major and minor chord qualities. All
scale degrees were indicated with a single Roman numeral
style, regardless of their chord qualities. In Hamilton's
annotations, Roman numerals indicate melodic scale degrees,
instead of chord roots, as shown in
\reffig{primary_sources/hamilton1840cathecism044}.

The next adopter of the Roman numeral notation, arguably the
first among the German theorists, was
\textcite{meister1852vollstandige}. Meister often
accompanied the Roman numeral notation with chord labels and
figured bass indications, as shown in
\reffig{primary_sources/meister1852vollstandige32}.

\phdfigure[Use of Roman numerals in \textcite{meister1852vollstandige}, accompanied by chord label and figured bass indications][1.0]{primary_sources/meister1852vollstandige32}

In order, the next authors who adopted Roman numerals were
Sechter, Richter, Tiersch, and Tracy.
\textcite{sechter1853grundsatze} mainly used a chord label
notation. In latter sections of the book, Roman numerals
were introduced in the musical examples, possibly to relate
the scale degree of a chord root to different key contexts.
An example is shown in
\reffig{primary_sources/sechter1853grundsatze103}.
\textcite{richter1860lehrbuch} did adopt the syntax for
major and minor chord qualities from
\textcite{weber1817versuch}, introducing also a single-quote
symbol ($\rn{'}$) to indicate augmented chords, shown in
\reffig{primary_sources/richter1860lehrbuch34}. In the
earlier treatise of \textcite{tiersch1868system}, the use of
Roman numerals is absent. However, in
\textcite{tiersch1874elementarbuch}, most of the annotations
provided in the musical examples are figured bass or Roman
numeral annotations. \textcite{tracy1878theory} adopted the
notation with some peculiarities, for example, using a seven
(``$\rn{7}$'') figure for all seventh chords except dominant
seventh chords, which were simply written as $\rn{V}$.

\phdfigure[Use of Roman numerals, underneath chord label annotations, in \textcite{sechter1853grundsatze}][1.0]{primary_sources/sechter1853grundsatze103}

\phdfigure[Adoption of the Weber syntax in \textcite{richter1860lehrbuch}, featuring a notation for augmented triads, $\rn{III'}$, which did not exist before][0.8]{primary_sources/richter1860lehrbuch34}

More authors followed in adopting Roman numerals during the
end of the nineteenth century and, by the beginning of the
twentieth century, it was more common than not to observe
\gls{rna} in emerging harmony textbooks. One exception was
among textbooks in French, where figured bass notation
seemed more prominent. A textbook in French by
\textcite{koechlin1928traite} uses Roman numerals
sporadically to indicate scale degrees of the bass note,
shown in \reffig{primary_sources/koechlin1928traite026}.
This practice, which was similar to the one by
\textcite{hamilton1840catechism}, was uncommon by the time
of \textcite{koechlin1928traite}, as most authors then used
Roman numerals to refer to the scale degree of the chord
root, not the bass. This ``archaic'' \gls{rna} practice of
Koechlin might suggest the importance that figured bass
notation had in the French language. In fact, some English
and German books referred to figured bass as the
\emph{French} system.\footnote{For example,
\textcite{norris1894practical}.} Beyond being the preferred
system in French textbooks, figured bass also heavily
influenced the evolution of \gls{rna} in other ways.

\phdfigure[The scale-degrees of \textcite{koechlin1928traite}, which refer to the bass note and not the root of the chord][1.0]{primary_sources/koechlin1928traite026}
