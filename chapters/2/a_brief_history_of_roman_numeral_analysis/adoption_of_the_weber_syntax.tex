% Copyright 2022 Néstor Nápoles López

% This is \refsubsec{adoptionofthewebersyntax}, which
% introduces the adoption of the weber syntax.

The notational system introduced by
\textcite{weber1817versuch} was not immediately adopted by
other theorists. One of the first adopters, an English
professor, was \textcite{hamilton1840catechism}. Hamilton,
and several others, did not adopt the notation for major and
minor chord qualities. All scale degrees were indicated with
a single Roman numeral style, regardless of their chord
qualities. In Hamilton's annotations, Roman numerals
indicate melodic scale degrees, instead of chord roots, as
shown in \reffig{primary_sources/hamilton1840cathecism044}.

\phdfigure[Use of Roman numerals in \textcite{hamilton1840catechism}, indicating the scale degree of the bass, regardless of the chord root. In modern Roman numeral notation, these annotations would be written as $\rn{V}\rnsixfive$, $\rn{V}\rnfourthree$, $\rn{V}\rntwo$, $\rn{V}\rnfourthree$, and $\rn{V}\rnseven$, respectively][0.75]{primary_sources/hamilton1840cathecism044}

The next adopter of the Roman numeral notation, this time a
German theorist, appears to be
\textcite{meister1852vollstandige}. Meister often
accompanied the Roman numeral notation with chord labels and
figured bass indications, as in
\reffig{primary_sources/meister1852vollstandige32}.

\phdfigure[Use of Roman numerals in \textcite{meister1852vollstandige}, accompanied by chord label and figured bass indications][1.0]{primary_sources/meister1852vollstandige32}

Other subsequent authors who adopted Roman numerals include
\parencite{sechter1853grundsatze, richter1860lehrbuch,
tiersch1874elementarbuch, tracy1878theory}.
\textcite{sechter1853grundsatze} mostly used a chord-label
notation. In latter parts of the treatise, Roman numerals
are introduced in the musical examples, possibly to relate
the scale degree of a chord root in different key contexts.
An example is shown in
\reffig{primary_sources/sechter1853grundsatze103}.

\phdfigure[Use of Roman numerals, underneath chord label annotations, in \textcite{sechter1853grundsatze}][1.0]{primary_sources/sechter1853grundsatze103}

\textcite{richter1860lehrbuch} did adopt the syntax for
major and minor chord qualities from
\textcite{weber1817versuch}, shown in
\reffig{primary_sources/richter1860lehrbuch34}.

In the earlier treatise of \textcite{tiersch1868system}, the
use of Roman numerals is nonexistent. However, in the latter
work \textcite{tiersch1874elementarbuch}, most of the
annotations provided in the musical examples are figured
bass or Roman numeral annotations.
\textcite{tracy1878theory} adopted the notation with some
peculiarities, for example, using a `7` for all except
Dominant seventh chords ($\rn{V}\rnseven$).

\phdfigure[Adoption of the Weber syntax in \textcite{richter1860lehrbuch}][1.0]{primary_sources/richter1860lehrbuch34}