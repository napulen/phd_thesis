% Copyright 2022 Néstor Nápoles López

% This is \refsec{abriefhistoryofromannumeralanalysis},
% which introduces the a brief history of roman numeral
% analysis.

Over the last 200 years, the \gls{rna} syntax has evolved in
an unregulated way, with Western music theorists proposing
new notations, and adopting the ones of previous theorists
at discretion. Beyond the recent formalization efforts of
researchers who digitize \gls{rna}
\parencite{huron1994humdrum, napoleslopez2017automatic,
neuwirth2018annotated, gotham2019romantext,
napoleslopez2020harmalysis, hentschel2021annotated}, there
is no ``standard'' way of writing Roman numerals. One way to
explain the modern traits of the notation is to observe,
chronologically, its practice over time.

In this section, I present a summary of the \gls{rna}
notation across nearly 150 primary sources between the late
18\textsuperscript{th} and early 20\textsuperscript{th}
centuries. The primary sources were collected from three
keyword searches in the
WorldCat\footnote{\href{https://www.worldcat.org/}{https://www.worldcat.org/}}
library catalog: ``harmony'', ``harmonie'', and
``harmonielehre''. These keywords were used to search for
books in the English, French, and German languages,
respectively. For books with multiple editions, the earliest
available edition was reviewed.\footnote{Physically
available at the McGill Marvin Duchow Library, or digitally
available in e-book form.} The full table of references is
available on \reftab{primary_sources_table}.

\phdtablefit[All primary sources reviewed in terms of their
\gls{rna} syntax][0.8\linewidth]{primary_sources_table}

The objective of reviewing the primary sources was to
observe the evolution of the \gls{rna} syntax across harmony
textbooks. For each book, I inspected the annotated musical
examples, occasionally observing annotations in the body of
the textbook (e.g., the tables in
\textcite{kirnberger1774kunst}). Although the survey was
centered around Roman numeral annotations, because many of
the musical examples included figured bass annotations, the
usage of figured bass notation was also documented. 
