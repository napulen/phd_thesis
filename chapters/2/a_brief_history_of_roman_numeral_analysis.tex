% Copyright 2022 Néstor Nápoles López

% This is \refsec{abriefhistoryofromannumeralanalysis},
% which introduces the a brief history of roman numeral
% analysis.

The \gls{rna} syntax evolved in an unregulated way, with
theorists proposing new notations, and adopting the ones of
previous theorists at discretion. Beyond recent efforts by
researchers who develop digital standards, there is no
``standard'' way of writing Roman numerals. In order to
explain the modern traits of the notation, the easiest way
is to observe, chronologically, its practice over time.

In this section, I present a summary of the Roman numeral
analysis notation across one hundred and forty-three primary
sources between the late eighteenth and early twentieth
centuries. The primary sources were collected from three
keyword searches in the
WorldCat\footnote{\href{https://www.worldcat.org/}{https://www.worldcat.org/}}
library catalog: ``harmony'', ``harmonie'', and
``harmonielehre''. These keywords were used to search for
books in the English, French, and German languages,
respectively. For books with multiple editions, the earliest
available edition was preferred.\footnote{Physically
available at the McGill Marvin Duchow Library, or digitally
available in e-book form.} The full table of references is
available on \reftab{primsources}.

\begin{tabular}{lll}
\textcite{kirnberger1774kunst} & \textcite{vogler1778grunde} & \textcite{vogler1802handbuch} \\
\textcite{weber1818versuch} & \textcite{logier1827system} & \textcite{hamilton1840catechism} \\
\textcite{fetis1844traite} & \textcite{lobe1850lehrbuch} & \textcite{meister1852vollstandige} \\
\textcite{sechter1853grundsatze} & \textcite{spencer1854rudimentary} & \textcite{southard1855course} \\
\textcite{bazin1857cours} & \textcite{volckmar1860harmonielehre} & \textcite{richter1860lehrbuch} \\
\textcite{reber1862traite} & \textcite{vonoettingen1866harmoniesystem} & \textcite{ouseley1868treatise} \\
\textcite{tiersch1868system} & \textcite{tiersch1874elementarbuch} & \textcite{tracy1878theory} \\
\textcite{bussler1878praktische} & \textcite{emery1879elements} & \textcite{kistler1879harmonielehre} \\
\textcite{clarke1880harmony} & \textcite{bowman1881harmony} & \textcite{durand1881traite} \\
\textcite{mangold1883harmony} & \textcite{coon1883harmony} & \textcite{riemann1883neue} \\
\textcite{jadassohn1883lehrbuch} & \textcite{oakey1884text} & \textcite{saintsaens1885harmonie} \\
\textcite{riemann1887systematische} & \textcite{broekhoven1889system} & \textcite{prout1889harmony} \\
\textcite{shepard1889how} & \textcite{jadassohn1890kunst} & \textcite{riemann1890katechismus} \\
\textcite{vivier1890traite} & \textcite{goetschius1892theory} & \textcite{goodrich1893goodrichs} \\
\textcite{buwa1893schule} & \textcite{norris1894practical} & \textcite{shepard1896harmony} \\
\textcite{chadwick1897harmony} & \textcite{gladstone1898fivepart} & \textcite{clarke1898system} \\
\textcite{boise1898harmony} & \textcite{werker1898theorie} & \textcite{cutter1899exercises} \\
\textcite{bridge1900course} & \textcite{halm1900harmonielehre} & \textcite{cutter1902harmonic} \\
\textcite{riemann1902grosse} & \textcite{shinn1904method} & \textcite{foote1905modern} \\
\textcite{schenker1906neue} & \textcite{heacox1907lessons} & \textcite{louis1907harmonielehre} \\
\textcite{capellen1908fortschrittliche} & \textcite{loewengard1908lehrbuch} & \textcite{gladstone1908manual} \\
\textcite{klauser1909nature} & \textcite{lavignac1909cours} & \textcite{vinee1909principes} \\
\textcite{york1909practical} & \textcite{white1911harmonic} & \textcite{eyken1911harmonielehre} \\
\textcite{gardner1912essentials} & \textcite{mokrejs1913lessons} & \textcite{kallenberg1913musikalische} \\
\textcite{molitor1913diatonischrhythmische} & \textcite{riemann1913handbuch} & \textcite{lenormand1913etude} \\
\textcite{gilson1914etude} & \textcite{spencer1915harmony} & \textcite{hull1915modern} \\
\textcite{leavitt1916practical} & \textcite{orem1916harmony} & \textcite{heacox1917keyboard} \\
\textcite{fowles1918harmony} & \textcite{robinson1918aural} & \textcite{anger1919treatise} \\
\textcite{watt1919foundations} & \textcite{foote1919modulation} & \textcite{deveaux1919les} \\
\textcite{gilson1919traite} & \textcite{ham1919rudiments} & \textcite{macpherson1920melody} \\
\textcite{kitson1920elementary} & \textcite{buck1920unfigured} & \textcite{koch1920aufbau} \\
\textcite{alchin1921applied} & \textcite{knorr1921aufgaben} & \textcite{dubois1921traite} \\
\textcite{klatte1922grundlagen} & \textcite{schenker1922neue} & \textcite{schoenberg1922harmonielehre} \\
\textcite{wedge1924keyboard} & \textcite{scholes1924beginners} & \textcite{mcconathy1927approach} \\
\textcite{haba1927neue} & \textcite{krehl1928theorie} & \textcite{koechlin1928traite} \\
\textcite{wedge1930applied} & \textcite{campbellwatson1930modern} & \textcite{morris1931foundations} \\
\textcite{schenker1935freie} & \textcite{barnes1937practice} & \textcite{jones1939harmony} \\
\textcite{piston1941harmony} & \textcite{hindemith1943concentrated} & \textcite{bairstow1945counterpoint} \\
\textcite{morris1946oxford} & \textcite{murphy1951creative} & \textcite{jacobs1958harmony} \\
\textcite{ottman1961elementary} & \textcite{ottman1961advanced} & \textcite{tischler1964practical} \\
\textcite{goldman1965harmony} & \textcite{mitchell1965elementary} & \textcite{siegmeister1965harmony} \\
\textcite{abraham1965harmonielehre} & \textcite{ulehla1966contemporary} & \textcite{schoenberg1967fundamentals} \\
\textcite{schoenberg1969structural} & \textcite{scheidt1975naturkundliche} & \textcite{mickelsen1977hugo} \\
\textcite{dreyer1977entwurf} & \textcite{delamotte1978harmonielehre} & \textcite{aldwell1978harmony} \\
\textcite{forte1979tonal} & \textcite{lester1982harmony} & \textcite{tunley1984harmony} \\
\textcite{kostka1984tonal} & \textcite{toutant1985functional} & \textcite{levy1985theory} \\
\textcite{carter2002harmony} & \textcite{swain2002harmonic} & \textcite{schenker2002tonwille} \\
\textcite{sarnecki2010harmony} & \textcite{roigfrancoli2011harmony} & 
\end{tabular}


The objective of reviewing the primary sources was to
observe the evolution of the Roman numeral analysis syntax
across harmony textbooks. Mostly the annotated musical
examples were revised, occasionally inspecting annotations
in the body of the textbook (e.g., the tables in
\textcite{kirnberger1774kunst}). The review was centered
around Roman numeral annotations. Because many of the
musical examples in the textbooks were either annotated with
figured bass or Roman numeral annotations, the usage of
figured bass notation was also documented. This process
revealed a few findings:

\begin{enumerate}
    \item The role of \textcite{weber1817versuch} in
    defining the basic characteristics of the notation was
    crucial
    \item After \textcite{weber1817versuch}, the Roman
    numeral annotation syntax slowly gained popularity among
    English and German theorists, achieving a more solid
    adoption by the end of the nineteenth century
    \item The Roman numeral notation was often accompanied
    with figured bass. At first, Roman numerals did not
    indicate chord inversions. Later, the practice of
    interconnecting figured bass and Roman numeral
    annotations might have resulted in the notation used
    today to indicate chord inversions with Arabic numerals
    \item Special symbols were introduced for certain
    chords. Notably, \emph{neapolitans} received their own
    symbol (``$\rn{N}$''). Sometimes ``competing'' syntaxes
    were used with the same meaining. For example, some
    authors preferred a single quote (``$\rn{'}$'') symbol
    to denote augmented triads, whereas others used a plus
    sign (``$\rn{+}$'')\footnote{Nowadays, the second
    notation is more common.}
    \item The treatment of tonicizations was originally the
    same as for modulations, using the colon syntax to set a
    new reference key. Eventually, a few ways to notate
    common key fluctuations of key became available. These
    evolved into the slash syntax we used today, although it
    is unclear who should be credited for the slash notation
    \item Recently, the vocabulary of chords has been
    extended, new additions consider, for example, the
    common-tone diminished seventh, a chord that is often
    written as $\rn{CT}\rndim\rnseven$
\end{enumerate}

The next sections describe these findings more thoroughly.
