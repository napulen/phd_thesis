% Copyright 2022 Néstor Nápoles López

% This is \refsec{digitizationofromannumeralannotations},
% which introduces the digitization of roman numeral
% annotations.

Because the Roman numeral notation evolved in a rather
unstructured way, there is no \emph{standard} way of writing
Roman numeral annotations. 
% Personally, studying with several harmony teachers from
% Guadalajara, Barcelona, and Montreal, I noticed that each
% of these teachers used a slightly different notation. A
% survey of the Harmony literature (see
% \reftab{primsources}) will reveal something similar among
% published works. 
Some of the common variations include:

\begin{itemize}
    \item Use of case-sensitive or case-insensitive numerals
    \item Different ways of denoting certain chords,
    notably, the so-called ``cadential six-four''
    chord\footnote{Particularly, there is currently a heated
    debate about annotating these chords as
    $\rn{I}\rnsixfour$ or $\rn{V}\rnsixfour$}
    \item Different notations for other chords (e.g.,
    Neapolitan written as either $\flat\rn{II}\rnsix$ or
    $\rn{N}\rnsix$)
    \item Different ways of notating inversions (using the
    letters a--d or stacks of Arabic numerals)
\end{itemize}

\guide{Case-sensitive versus case-insensitive Roman numerals.}
A clear separation of chord qualities (major and minor)
using case-sensitive Roman numerals has been proposed since
\textcite{weber1817versuch}. However, some analysts have
preferred to indicate the scale degrees using an upper-case
Roman numeral, regardless of the chord quality implied by
the chord. This is often fine for human analysts, as the
reader of the analysis will usually be able to complement
the information. Computers will not be able to disambiguate
this notation, however. In digital annotations, having
explicit information about the chord quality is
unequivocally better. Every digital representation considers
Roman numerals to be case-sensitive for that reason.

\phdfigure[Instances of the $\rn{V}\rnsixfour$, $\rn{I}\rnsixfour$, and $\rn{Cad}\rnsixfour$ chords, as suggested for digital representations][1.0]{v64_i64_and_cad64}

\guide{Cadential six-four chords.}
A tonic triad in second inversion is a chord frequently used
before a cadence. When it is used in this context (preceding
a $\rn{V}$ chord) it receives the name of \emph{cadential
six-four}. In order to indicate this chord using Roman
numerals, it is common to write it as $\rn{I}\rnsixfour$,
$\rn{V}\rnsixfour$, or $\rn{Cad}\rnsixfour$. Out of these,
unequivocally, the best representation for a digital
annotation is $\rn{Cad}\rnsixfour$, as it disambiguates the
chord as functioning in this specific way and not as a
passing chord (e.g., a second-inversion tonic or dominant
triads could be passing chords, which are neither considered
a cadential six-four). \reffig{v64_i64_and_cad64}
exemplifies the suggested use of $\rn{I}\rnsixfour$,
$\rn{V}\rnsixfour$, and $\rn{Cad}\rnsixfour$ chords in
machine-readable annotations, with all ambiguity removed.

\guide{Neapolitans, Augmented sixth chords, and other conventions.}
Neapolitan chords are a chromatic substitution of the
subdominant chord. The root of the chord is the flattened
second degree of the relevant major/minor key. For example,
the Neapolitan of $\rn{C major}$ (and $\rn{C minor}$) is
$\rn{D}\flat$. For this reason, it is either known as
$\flat\rn{II}$ or as $\rn{N}$.

\guide{Notating inversions.}
Inversions are generally notated using stacked Arabic
numerals. The conventions for common triad and seventh
chords are relatively standardized. These are indicated in
table \reftab{inversions}

\phdtable[Notations for chord inversions using Arabic
numerals or letters. Figures between square brackets are
optional]{inversions}

Inversions have also been indicated with letters. This
notation is common since the nineteenth century. It is also
the notation of the first digital system, **harm.

% The need for a standard
In the classroom setting, the flexibility of the notation is
arguably a desirable goal. Students may be encouraged to
develop their own ``style'' of tonal analysis, incorporating
aspects of voice-leading, motivic, and key analysis, as they
see fit. This is useful to extend the intrinsic limitations
of \gls{rna} to summarize tonal music.

Excessive flexibility may be an issue, however, in
computational work. Idiosyncratic and undocumented methods
of analysis lead to incompatible annotations. It is
unfeasible to assume that a single person will be able to
annotate a sufficiently large number of Roman numeral
analyses to be used for computational models. Thus,
compatibility between annotations and cooperation of
different analysts is necessary. This can only be achieved
by standardizing the notation, and the practice of
\gls{rna}. Several notational systems have emerged over the
years to attempt to solve this problem

In this section, I discuss the efforts that have been done
regarding the standardization of \gls{rna} into a digital
format.
