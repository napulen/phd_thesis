Roman numeral labels provide more information than chord
labels. Generally, a chord label indicates the root of a
chord and its quality. A Roman numeral label provides this
information as well, complemented with information about
keys, inversions, and functional roles of the chords.
Because Roman numerals are always relative to a key, the key
must be indicated at all times to disambiguate the meaning
of the Roman numeral. This is particularly helpful in Music
Information Retrieval, where key-estimation and
chord-recognition models can be developed using the same
Roman numeral annotations. Roman numerals indicate
inversions for triads and seventh chords. These are often
indicated with $\rnsix$ and $\rnsixfour$ in triads and
$\rnsixfive$, $\rnfourthree$, and $\rntwo$ in seventh
chords. Presumably, these stacks of Arabic numerals to
denote inversions evolved from figured bass, an argument I
discuss in \refsec{abriefhistoryofromannumeralanalysis}. In
the modern syntax, it is also customary to indicate
\emph{applied chords} or \emph{secondary dominants}. Once
the concept of \emph{tonicization} emerged, in the twentieth
century, this quickly became an important idea of tonal
music, particularly relevant to analyze the chromatic music
of the nineteenth century. Roman numerals facilitate the
indication of tonicizations using a ``$\rn{/}$'' symbol. For
example, the annotations $\rn{V}\rnsixfive\rn{/V}$ and
$\rn{vii}\rndim\rnfourthree\rn{/ii}$ in \reffig{rnandcl}
indicate a secondary dominant and a tonicization of the
G-minor key, respectively. With these, a musical analysis
can be more informative, indicating the analyst's point of
view of the musical key at a particular moment of the piece.

The next section explores the historical evolution of the
syntax, since its first introduction in the late eighteenth
century, to its modern notation.

% There are other subtle but important differences that come
% with Roman numeral analysis compared to chord labels.
% Think of a diminished seventh chord, say, F$\sharp$
% diminished seventh. The most usual context of that chord
% is to anticipate a resolution to a G minor triad. However,
% if the chord is in first inversion, then it is
% enharmonically equivalent to an A diminished seventh
% chord. It is tempting to write the chord as A diminished,
% rather than F$\sharp$ with A in the bass, however, these
% chords have two very different meanings contextually. This
% is evident in the presence of pitch spelling. The
% G$\flat$--G$\natural$ results in a strange voice leading,
% whereas the F$\sharp$--G is a customary leading-tone to
% tonic minor second. These subtle interactions of spelling
% are generally honored by the Roman numeral analysis
% notation. There are multiple examples that come to mind,
% such as Neapolitans, Augmented Sixth Chords, and
% \emph{Cadential six-four} chords, where the syntax of
% Roman numerals is helpful to provide a more thorough
% description of the annotated chord. These differences
% might seem negligible to some, but I would like to make a
% few arguments: (1) in tonal music theory textbooks, it is
% more common to see Roman numeral analysis notation than
% chord labels, which might suggest that the more granular
% notation is preferred (2) in machine learning contexts, as
% is the case for this dissertation, more information in the
% annotations can be really helpful, without this additional
% information, we could not attempt to learn a
% key-estimation model simultaneously using the same
% annotated data, (3) the syntax of Roman numeral analysis
% is highly compressed, it is generally not more
% time-consuming to write Roman numerals than it is to write
% chord labels, the time-consuming aspect of Roman numerals
% is the train of thought required to commit to a particular
% interpretation of the analysis (i.e., key context, chord,
% inversion, function of the chord), and that is precisely
% the thorough thought pattern that we want to capture in
% the data.
