\guide{How did Roman numeral analysis emerge?} It is
problematic to pinpoint a time and place where this
notational system appeared. It seems more accurate to say
that this notation did not ``appear'' but evolved, slowly,
from the sporadic use of Roman numerals indicating scale
degrees, to the complex syntax of applied chords and chord
inversions used today.

In this section, I summarize a brief timeline of this
evolution process. For describing the important
developments that lead to our modern Roman numeral
analysis notation, I rely on existing literature. For
example, the chapters on tonality of the Cambridge History
of Western Music
Theory~\parencite{christensen2002tonality,
christensen2002rameau, christensen2002nineteenthcentury,
christensen2002heinrich}. The historical events discussed
by previous scholars are complemented with a timeline of
Roman numeral analysis syntax that I collected from
different textbooks of the nineteenth century.

\guide{Important developments that lead to the modern
Roman numeral analysis syntax.} There are a few historical
figures who influenced the notation that we know today.
Rameau's fundamental bass, Vogler's first use of Roman
numerals, Weber's notation for Roman numerals with chord
qualities (big case for major triads, small case for minor
triads, circle postfix to denote diminished triads),
Riemann's concept of function and applied chords,
Schoenberg's take on function, Schenker's tonicization.

\guide{Rameau's fundamental bass.} When I was introduced
to tonal harmony in school, Rameau was introduced as the
\emph{father of Harmony}. When I took this course, I was
well aware of Rameau's music as well as Bach's. I used to
listen to the Harpsichord pieces of both composers, as the
harpsichord was my favorite instrument. Something that was
very confusing for me was how the music by previous
composers seemed to have ``chords''. How is it that Rameau
is the ``father of harmony''? I am pretty sure I have
heard chords in earlier composers. There seemed to be a
lot of misinformation and ``myths'' in music theory. Over
the years, I managed to confirm this over and over. The
\emph{meaning} of a certain key, the nonexistence of
``parallel fifths'' in music from the ``great masters'',
and on and on. Nowadays, I have reconciled my thoughts on
this particular topic, Rameau as the ``father of
Harmony''. The concept of ``chord root'' is taken for
granted nowadays, as a universal phenomenon of Western
tonal music. Even when approaching Fuxian counterpoint
theory, music teachers have for many years relied on the
concept of ``the root'' to analyze triads and seventh
chords formed from contrapuntal motion. This concept, the
chord root, is highly correlated with the idea of the
``fundamental bass'', introduced by Rameau. A good
explanation is found in~\textcite{christensen2002rameau}.
In the times of figured bass, a bass with an annotation of
``6'' indicated a harmony that was more closely linked to
``5'' than it was to its own ``chord root''. How we would
think of it today. For example, a C major triad is not as
related to an A minor triad in first inversion as it is to
a C major triad in first inversion or second inversion,
even though the bass of ``C:I'' and ``a:i6'' is
potentially the same. The root is important, more than the
bass. This is what the ``fundamental bass'' revolution
brought to music theory and, according to other scholars,
it changed everything. I agree, and I accept Rameau as an
essential figure in this process now.


It has been customary to divide the different tonal
theories of harmony of the eighteenth and nineteenth
centuries into two branches: \emph{scale-degree} theories
and \emph{functional} theories.

The difference between these groups is that the functional
theories tend to attempt explanations for the ``role'' or
``function'' that a chord has in a specific context,
whereas the scale-degree theories tend to focus on the
more objective chord ``root'', without too much
explanation of the context or the role of this scale
degree.

The modern Roman numeral analysis syntax tends to borrow
ideas from these two branches of theories, as it sometimes
indicates functional aspects. For example, the label
`V/V`, which is common in Roman numeral analysis could
also be indicated as `II'. Assuming that the notation is
case-sensitive, the upper case `II' indicates a major
triad on the second degree, which is the correct root and
chord quality of the chord indicated by `V/V'. By
indicating the chord as `V/V', implicitly, there is some
contextual information being introduced into the label.
Namely, the notation indicates that this chord has
possibly some relationship to a chord that will happen in
the future (a dominant), and it is acting as a dominant
chord for it (thus, ``dominant of the dominant'').

Although this hybrid nature of modern Roman numerals is
not often discussed, it seems customary nowadays in
textbooks, explanations, and general understanding of
tonal music theories of the past.


\guide{Vogler's first use of Roman numerals.} Except for
the nonexistence of zero, there is nothing special about
Roman numerals, per se. They are symbols that can be used
to count natural numbers. Vogler used them for this
purpose in 177X. More precisely, to refer to the seventh
degree of a major diatonic scale (i.e., the leading tone).
In a subsequent work in 1802, Vogler used Roman numerals
to denote each scale degree similarly. This was the first
use of Roman numerals for this purpose. An interesting
side effect of using the Roman numeral symbols for these
annotations, is that Vogler and other authors could also
use Arabic numerals to annotate figured bass annotations.
The evidence I have found in textbooks of the nineteenth
century suggests that it was common for authors to write
figured bass and Roman numerals for the same music
excerpt. Sometimes, these annotations would overlap (e.g.,
annotating certain chords, such as, augmented sixth
chords, using Arabic numerals (figured bass) rather than
Roman numerals). This was a very common practice I
observed. Arguably, this was the reason why the notation
evolved to denote chord inversions using Arabic numerals
nowadays.

\guide{Weber's notation for Roman numerals.} Weber is
credited with at least two important contributions to
music theory: (1) the \emph{tonal chart} table, which was
later revisited by Schoenberg and others to measure key
relationships, (2) the use of an early Roman numeral
notation, which can be considered an early version of the
notation used today. The syntax of Weber is more
sophisticated than the one used by Vogler in several
aspects. First, Weber was unequivocally using Roman
numerals to denote \emph{chords}, not just scale degrees.
The intention is clear because Weber used bigger
(nowadays, upper case) Roman numerals to denote major
triads, smaller (nowadays, lower case) Roman numerals to
denote minor triads, and a special symbol to denote
diminished triads, which is still in use today. That is,
the compact notation is now capable of encoding basic
chord qualities. The scale degrees can now be thought as
the roots of a chord, where the size (nowadays, case) of
the Roman numeral indicates the quality of the triad.

\guide{Riemann.} Although Rameau first introduced the
concepts of Dominant and SubDominant, Riemann introduced
them in the context of ``functions''. In Riemann's theory,
three chords (tonic, subdominant, and dominant) were
transformed into other chords by applying certain
operations. Although most of Riemann's theory has been
left out of modern Roman numeral syntax, his ideas of
function and, particularly, concepts of \emph{applied
chords} have arguably inspired some aspects of the modern
notation. For example, the use of annotations such as
`V/V' have a functional touch to them. While in
scale-degree theories it is customary to describe chords
based on the chord root that generates them, the notation
of \emph{applied chords} includes more information about
neighboring harmonies. I consider this a functional trait
of the notation, because it indicates not only the root
and quality of a given chord, but what is its role in the
harmonic context.

\guide{Schoenberg.} Schoenberg revisited the concept of
function. Schoenberg never used the ``V/V'' notation.

\guide{Schenker.} Schenker is a complicated character, not
only in the way he approached music theory but as well on
his political views. Nevertheless, his contributions to
music theory are undeniable, and best described by other
scholars, such as \textcite{christensen2002heinrich}. The
aspects of Schenker that are most relevant for this
chapter are his coining of the term ``tonicization'',
which is an important trait of modern Roman numeral
analysis syntax.

\guide{How does the evolution of Roman numerals look
like?}

\begin{itemize} \item According to Hyer, there are two
    branches of tonal music theory around the early
    nineteenth century: scale-degree theories and
    functional theories \item Scale-degree theories are
    best characterized by the work of Weber and Schenker
    \item Functional theories include those of Rameau and
    Riemann \item Roman numerals, nowadays unify concepts
    from these two theories, acting as a metalanguage
    \item Combination of theories is common, for example,
    Fuxian theory taught by first analyzing the harmonic
    context of the C.F., and other examples \item
    Arguable, Rameau's work influenced every other
    subsequent harmonic theory \item Rameau introduced the
    concepts of tonic, dominant, and subdominant chords
    \item However, he never used the term ``function'',
    which was established by Riemann \item Vogler
    introduced Roman numerals for the first time \item
    First, in 17XX, where he used them to denote the
    ``VII'' degree \item Then, 1802, he used them for
    every scale degree \item Weber firmly established
    Roman numerals \item Nowadays, Roman numerals have
    borrowed ideas from both branches of music theories
    \end{itemize}
