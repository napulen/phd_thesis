% Copyright 2022 Néstor Nápoles López

% This is \refsubsubsec{harmalysis},
% which introduces the harmalysis.

\guide{The harmalysis grammar}
\code{Harmalysis} is a Roman numeral analysis grammar
introduced by \textcite{napoleslopez2020harmalysis}.

The \code{harmalysis} grammar is based principally on
Huron‘s \code{**harm} syntax (see \refsec{**harm}). The syntax
is extended by borrowing elements from the RomanText format
(see \refsec{RomanText}), MuseScore‘s notation for Roman
numeral
analysis\footnote{\href{https://musescore.org/en/handbook/3/chord-symbols\#rna}{https://musescore.org/en/handbook/3/chord-symbols}},
and conventions observed in existing datasets of Roman
numeral analysis.  As  a result,  the harmalysis language is
a superset  of  the \code{**harm} syntax,  which includes
additional features and supports a wider range of customs of
harmonic analysis.

The main advantage of this syntax is that it is formally defined as a context-free grammar. Thus, the exact mathematical validation of the syntax is available for review and discussion, which might not be the case for representations defined by their textual explanations.
