\guide{The lack of a ``standard'' notation.}
There is no ``standard'' way of writing Roman numeral
analysis. Personally, I have studied or interacted with
several teachers/theorists using this technique in the
different places I have lived: M\'exico (Guillermo Salvador
Fern\'andez, Gamaliel Cano, Jose Luis Gonz\'alez Moya,
Alejandro Moreno), Barcelona (Montserrat Castro, Rafael Caro
Repetto), Montr\'eal (William Caplin).

Each of these teachers have used a slightly different
notation for Roman numeral analysis. A thorough analysis of
the literature of Harmony textbooks will reveal something
similar to the reader.

The differences include:
\begin{itemize}
    \item Use of case-sensitive or case-insensitive numerals
    \item Different ways of denoting certain chords,
    notably, the so-called ``cadential six-four'' chord
    \item Different notations for other chords (e.g.,
    Neapolitan and Augmented sixth chords)
    \item Different ways of notating inversions
\end{itemize}

\guide{Case-sensitive versus case-insensitive Roman numerals.}
A clear separation of chord qualities (major and minor)
using case-sensitive Roman numerals has been common since
Weber. This is, however, a more time-consuming notation for
the analyst. Thus, some analysts have preferred to indicate
the scale degrees using a Roman numeral, regardless of the
chord quality implied by the chord. This is often fine for
other human analysts, as the reader of the analysis will
usually be able to disambiguate the information. Computers
will not be able to disambiguate this notation, however. In
digital annotations, having explicit information about the
chord quality is unequivocally better. Every digital
representation considers Roman numerals to be case-sensitive
for that reason.

\guide{Cadential six-four chords.}
A tonic triad in second inversion is a frequent chord used
before a cadence (see \reffig{cadential64}). When in this
context (generally preceding a `V' chord) it receives the
name of \emph{Cadential six-four}. When denoting this chord
using Roman numerals, it is common to see it written as
`I64`, `V64`, and `Cad64`. Cad64 (or a similar token) is
unequivocally the best representation for a digital
annotation, as it disambiguates the chord as functioning in
this specific way (e.g., a second-inversion tonic triad
could also be a passing chord, which is not considered a
cadential six-four).

\guide{Neapolitans, Augmented sixth chords, and other conventions.}
Neapolitan chords are a chromatic substitution of the
subdominant chord. The root of the chord is the flattened
second degree of the relevant major/minor key. For example,
the Neapolitan of C major (and C minor) is D$\flat$ major.
For this reason, it is either known as $\flat$II or as N.

\guide{Notating inversions.}
Inversions are generally notated using stacked Arabic
numerals. The conventions for common triad and seventh
chords are relatively standardized.

\begin{itemize}
    \item 6
    \item 64
    \item 65
    \item 43
    \item 42 (or 2)
\end{itemize}

Inversions have also been indicated with letters. This
notation is common since the nineteenth century. It is also
the notation of the first digital system, **harm.

\guide{On the need for standardization.}
In the classroom setting, the flexibility of the notation is
arguably a desirable goal. Students may be encouraged to
develop their own ``style'' of tonal analysis, incorporating
aspects of voice-leading, motivic, and key analysis, as they
see fit. This is useful to extend the intrinsic limitations
of Roman numeral analysis to summarize tonal music.

Flexibility is, however, a problem in computational work. A
large and undocumented method of analysis leads to
incompatible and unreliable annotations. It is unfeasible to
assume that a single person will be able to annotate a
sufficiently large number of Roman numeral analyses to be
used for computational models. Thus, compatibility between
annotations and cooperation of different analysts is
necessary. This can only be achieved by standardizing the
notation, and the practice of Roman numeral analysis.
Several notational systems have emerged over the years to
attempt to solve this problem

In this section, I discuss the efforts that have been done
regarding the standardization of Roman numeral analysis into
a digital format.

\guide{The **harm syntax}
The **harm syntax was proposed by David Huron as part of the
Humdrum (**kern) format. This syntax includes a
comprehensive notation of triads, seventh chords, special
chords, and descriptive (arbitrary) chords using Roman
numeral notation. The notation is well documented, although
many aspects of the notation are missing from the
explanations.

\guide{The RomanText syntax}
This notation was first proposed by Tymoczko et al.
\parencite{tymoczko}. The syntax is clearly explained in the
conference publication. The most prominent aspect of this
notation is its software implementation as a module in the
music21 Python library.

\guide{The DCML syntax}
A notational system first introduced by
\parencite{neuwirth}.

\guide{The harmalysis grammar}
A Roman numeral analysis grammar introduced by N\'apoles
L\'opez and Fujinaga \parencite{napoleslopez}
