% Copyright 2021 Néstor Nápoles López

% This is the introduction to
% \refchap{introductiontoromannumeralanalysis}, which goes
% before any of its sections.

\gls{rna} is a common methodology in modern music theory
textbooks dealing with tonal music. The name derives from
the use of Roman numerals to denote scale degrees, which, in
turn, indicate diatonic chords in a certain musical key. The
musical key is usually prepended to the first Roman numeral
of the piece, using a case-sensitive note
letter\footnote{Generally, upper-case letters are used to
indicate major keys, and lower-case letters are used to
indicate minor keys.} followed by a colon. When there are
changes of key throughout the piece, these are indicated in
a similar way, with a new key letter followed by a colon,
prepended to a Roman numeral.

\gls{rna} is an alternative system to the, possibly more
popular, chord label system. It may be that \gls{rna} is
more common among tonal music theory textbooks, because the
changes of key and other indications of the \gls{rna} syntax
are helpful to break down a piece analytically.
\reffig{rnandcl} shows a fragment of music annotated with
both Roman numeral and chord label annotations.

\phdfigure[The first eight measures of Josephine Lang's Op.8
 No.1 ``Schmitterling'', annotated with \gls{rna} and chord
 labels underneath][1.0]{rnandcl}
