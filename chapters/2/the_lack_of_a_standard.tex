\guide{The lack of a ``standard'' notation.}
Because the Roman numeral notation evolved in a rather
``organic'' way, there is no ``standard'' way of writing
Roman numeral annotations.
% Personally, I studied or interacted with several
% teachers/theorists using this technique in the different
% places I have lived: M\'exico (Guillermo Salvador
% Fern\'andez, Gamaliel Cano, Jose Luis Gonz\'alez Moya,
% Alejandro Moreno), Barcelona (Montserrat Castro, Rafael
% Caro Repetto), Montr\'eal (William Caplin).
Personally, I studied Roman numeral analysis with several
teachers located in Guadalajara, Barcelona, and Montreal.
Each of these teachers used a slightly different approach. A
survey of the literature on Harmony will reveal something
similar in published works. Some of the differences include:
\begin{itemize}
    \item Use of case-sensitive or case-insensitive numerals
    \item Different ways of denoting certain chords,
    notably, the so-called ``cadential six-four'' chord;
    particularly, there is a heated debate about annotating
    these chords as $\rn{I}\rnsixfour$ or $\rn{V}\rnsixfour$
    \item Different notations for other chords (e.g.,
    Neapolitan written as either $\flat\rn{II}\rnsix$ or
    $\rn{N}\rnsix$)
    \item Different ways of notating inversions (using the
    letters a--d or stacks of Arabic numerals)
\end{itemize}

\guide{Case-sensitive versus case-insensitive Roman numerals.}
A clear separation of chord qualities (major and minor)
using case-sensitive Roman numerals has been common since
Weber. This is, however, a more time-consuming notation for
the analyst. Thus, some analysts have preferred to indicate
the scale degrees using a Roman numeral, regardless of the
chord quality implied by the chord. This is often fine for
other human analysts, as the reader of the analysis will
usually be able to disambiguate the information. Computers
will not be able to disambiguate this notation, however. In
digital annotations, having explicit information about the
chord quality is unequivocally better. Every digital
representation considers Roman numerals to be case-sensitive
for that reason.

\guide{Cadential six-four chords.}
A tonic triad in second inversion is a frequent chord used
before a cadence (see \reffig{cadential64}). When in this
context (generally preceding a $\rn{V}$ chord) it receives
the name of \emph{cadential six-four}. When indicating this
chord using Roman numerals, it is common to write it as as
$\rn{I}\rnsixfour$, $\rn{V}\rnsixfour$, or
$\rn{Cad}\rnsixfour$. $\rn{Cad}\rnsixfour$ (or a similar
token) is unequivocally the best representation for a
digital annotation, as it disambiguates the chord as
functioning in this specific way (e.g., a second-inversion
tonic triad could also be a passing chord, which is not
considered a cadential six-four).

\guide{Neapolitans, Augmented sixth chords, and other conventions.}
Neapolitan chords are a chromatic substitution of the
subdominant chord. The root of the chord is the flattened
second degree of the relevant major/minor key. For example,
the Neapolitan of C major (and C minor) is D$\flat$ major.
For this reason, it is either known as $\flat$II or as N.

\guide{Notating inversions.}
Inversions are generally notated using stacked Arabic
numerals. The conventions for common triad and seventh
chords are relatively standardized.

\begin{itemize}
    \item 6
    \item 64
    \item 65
    \item 43
    \item 42 (or 2)
\end{itemize}

Inversions have also been indicated with letters. This
notation is common since the nineteenth century. It is also
the notation of the first digital system, **harm.
