In this section, I present a summary of the Roman numeral
analysis notation across one hundred and forty-three primary
sources between the late eighteenth and early twentieth
centuries. The primary sources were collected from three
keyword searches in the
WorldCat\footnote{\href{https://www.worldcat.org/}{https://www.worldcat.org/}}
library catalog: ``harmony'', ``harmonie'', and
``harmonielehre''. These keywords were used to search for
books in the English, French, and German languages,
respectively. For books with multiple editions, the earliest
available edition was preferred.\footnote{Physically
available at the McGill Marvin Duchow Library, or digitally
available in e-book form.}

The objective of reviewing the primary sources was to
observe the evolution of the Roman numeral analysis syntax
across harmony textbooks. Mostly the annotated musical
examples were revised, occasionally inspecting annotations
in the body of the textbook (e.g., the tables in
\textcite{kirnberger1774kunst}). The review was centered
around Roman numeral annotations. Because many of the
musical examples in the textbooks were either annotated with
figured bass or Roman numeral annotations, the usage of
figured bass notation was also documented. This process
revealed a few interesting findings:

\begin{enumerate}
    \item After \textcite{weber1817versuch}, the Roman
    numeral annotation syntax quickly gained popularity
    among English authors, possibly faster than among German
    theorists.
    \item French authors did not seem to adopt the system as
    much as English and German authors, preferring figured
    bass annotations.
    \item Sometimes ``competing'' syntaxes exist for the
    same symbol. For example, some authors preferred the
    $\rn{'}$ symbol to denote augmented triads, whereas
    others used $\rn{+}$.\footnote{Nowadays, the second
    notation is more common.}
\end{enumerate}

The next section describes the timeline of Roman numeral
analysis syntax across the one hundred fourty-three
textbooks.
