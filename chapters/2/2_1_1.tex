\subsection{A brief history of Roman numeral analysis}
\label{sec:a_brief_history_of_roman_numeral_analysis}

\guide{What do we call Roman numeral analysis?}
Roman numeral analysis is a common methodology in contemporary music theory textbooks dealing with tonal music.
The name derives from the use of Roman numerals to denote scale degrees, which, in turn, indicate diatonic chords in a certain key context.

\guide{Why is Roman numeral analysis important compared to, say, chord labels?}
Roman numeral labels provide more information than chord labels.
Generally, a chord label indicates the root of a chord and its quality.
A Roman numeral label also indicates this information, however, the modern syntax of Roman numeral analysis provides more information than this.
For example, Roman numerals are always relative to a key, so the key must be indicated at all times to disambiguate the meaning of the Roman numeral.
This is a helpful property for passages of music that feature quick modulations or changes of key.
In the modern syntax, it is also customary to indicate \emph{applied chords} or \emph{secondary dominants}.
Once the concept of \emph{tonicization} emerged, in the twentieth century, this quickly became an important idea of tonal music, particularly relevant to analyze the chromatic music of the nineteenth century.
Roman numerals facilitate the indication of modulations as well as tonicizations.
With these, a musical analysis of a particular person can be more informative, for example, indicating how the analyst envisions the key context at a particular moment of the piece.
Roman numerals indicate inversions for triads and seventh chords.
Presumably, the Arabic numeral notations employed to denote inversions in Roman numerals evolved from figured bass.
However, there is generally a small vocabulary of numerical annotations appended to a Roman numeral, which indicate the inversion of the chord.
There are other subtle but important differences that come with Roman numeral analysis compared to chord labels.
Think of a diminished seventh chord, say, F$\sharp$ diminished seventh.
The most usual context of that chord is to anticipate a resolution to a G minor triad.
However, if the chord is in first inversion, then it is enharmonically equivalent to an A diminished seventh chord.
It is tempting to write the chord as A diminished, rather than F$\sharp$ with A in the bass, however, these chords have two very different meanings contextually.
This is evident in the presence of pitch spelling.
The G$\flat$--G$\natural$ results in a strange voice leading, whereas the F$\sharp$--G is a customary leading-tone to tonic minor second.
These subtle interactions of spelling are generally honored by the Roman numeral analysis notation.
There are multiple examples that come to mind, such as Neapolitans, Augmented Sixth Chords, and \emph{Cadential six-four} chords, where the syntax of Roman numerals is helpful to provide a more thorough description of the annotated chord.
These differences might not seem very important to some readers, but I would like to make a few arguments here: (1) in tonal music theory textbooks, it is more common to see Roman numeral analysis notation than chord labels, which might suggest that the more granular notation is preferred (2) in machine learning contexts, as is the case for this dissertation, more information in the annotations can be really helpful, without this additional information, we could not attempt to learn a key-estimation model simultaneously using the same annotated data, (3) the syntax of Roman numeral analysis is highly compressed, it is generally not more time-consuming to write Roman numerals than it is to write chord labels, the time-consuming aspect of Roman numerals is the train of thought required to commit to a particular interpretation of the analysis (i.e., key context, chord, inversion, function of the chord), and that is precisely the thorough thought pattern that we want to capture in the data.


\guide{How did Roman numeral analysis emerge?}
It is problematic to pinpoint a time and place where this notational system appeared.
It seems more accurate to say that this notation did not ``appear'' but evolved, slowly, from the sporadic use of Roman numerals indicating scale degrees, to the complex syntax of applied chords and chord inversions used today.

In this section, I summarize a brief timeline of this evolution process.
For describing the important developments that lead to our modern Roman numeral analysis notation, I rely on existing literature.
For example, the chapters on tonality of the Cambridge History of Western Music Theory~\parencite{christensen_tonality_2002, christensen_rameau_2002, christensen_nineteenth-century_2002, christensen_heinrich_2002}.
The historical events discussed by previous scholars are complemented with a timeline of Roman numeral analysis syntax collected from different textbooks of the nineteenth century.

\guide{Important developments that lead to the modern Roman numeral analysis syntax.}
There are a few historical figures who influenced the notation we know today.
Rameau's fundamental bass, Vogler's first use of Roman numerals, Weber's notation for Roman numerals with chord qualities (big case for major triads, small case for minor triads, circle postfix to denote diminished triads), Riemann's concept of function and applied chords, Schoenberg's take on function, Schenker's tonicization.

\guide{Rameau's fundamental bass.}
When I was introduced to tonal harmony in school, Rameau was introduced as the \emph{father of Harmony}.
When I took this course, I was well aware of Rameau's music as well as Bach's.
I used to listen to the Harpsichord pieces of both composers, as the harpsichord was my favorite instrument.
Something that was very confusing for me was how the music by previous composers seemed to have ``chords''.
How is it that Rameau is the ``father of harmony''? I am pretty sure I have heard chords in earlier composers.
There seemed to be a lot of misinformation and ``myths'' in music theory.
Over the years, I managed to confirm this over and over.
The \emph{meaning} of a certain key, the nonexistence of ``parallel fifths'' in music from the ``great masters'', and on and on.
Nowadays, I have reconciled my thoughts on this particular topic, Rameau as the ``father of Harmony''.
The concept of ``chord root'' is taken for granted nowadays, as a universal phenomenon of Western tonal music.
Even when approaching Fuxian counterpoint theory, music teachers have for many years relied on the concept of ``the root'' to analyze triads and seventh chords formed from contrapuntal motion.
This concept, the chord root, is highly correlated with the idea of the ``fundamental bass'', introduced by Rameau.
A good explanation is found in~\cite{christensen_rameau_2002}.
In the times of figured bass, a bass with an annotation of ``6'' indicated a harmony that was more closely linked to ``5'' than it was to its own ``chord root''.
How we would think of it today.
For example, a C major triad is not as related to an A minor triad in first inversion as it is to a C major triad in first inversion or second inversion, even though the bass of ``C:I'' and ``a:I6'' is potentially the same.
The root is important, more than the bass.
This is what the ``fundamental bass'' revolution brought to music theory and, according to other scholars, it changed everything.
I agree, and I accept Rameau as an essential figure in this process now.


It has been customary to divide the different tonal theories of harmony of the eighteenth and nineteenth centuries into two branches: \emph{scale-degree} theories and \emph{functional} theories.

The difference between these groups is that the functional theories tend to attempt explanations for the ``role'' or ``function'' that a chord has in a specific context, whereas the scale-degree theories tend to focus on the more objective chord ``root'', without too much explanation of the context or the role of this scale degree.

The modern Roman numeral analysis syntax tends to borrow ideas from these two branches of theories, as it sometimes indicates functional aspects.
For example, the label `V/V`, which is common in Roman numeral analysis could also be indicated as `II'.
Assuming that the notation is case-sensitive, the upper case `II' indicates a major triad on the second degree, which is the correct root and chord quality of the chord indicated by `V/V'.
By indicating the chord as `V/V', implicitly, there is some contextual information being introduced into the label.
Namely, the notation indicates that this chord has possibly some relationship to a chord that will happen in the future (a dominant), and it is acting as a dominant chord for it (thus, ``dominant of the dominant'').

Although this hybrid nature of modern Roman numerals is not often discussed, it seems customary nowadays in textbooks, explanations, and general understanding of tonal music theories of the past.




\begin{itemize}
    \item According to Hyer, there are two branches of tonal music theory around the early nineteenth century: scale-degree theories and functional theories
    \item Scale-degree theories are best characterized by the work of Weber and Schenker
    \item Functional theories include those of Rameau and Riemann
    \item Roman numerals, nowadays unify concepts from these two theories, acting as a metalanguage
    \item Combination of theories is common, for example, Fuxian theory taught by first analyzing the harmonic context of the C.F., and other examples
    \item Arguable, Rameau's work influenced every other subsequent harmonic theory
    \item Rameau introduced the concepts of tonic, dominant, and subdominant chords
    \item However, he never used the term ``function'', which was established by Riemann
    \item Vogler introduced Roman numerals for the first time
    \item First, in 17XX, where he used them to denote the ``VII'' degree
    \item Then, 1802, he used them for every scale degree
    \item Weber firmly established Roman numerals
    \item Nowadays, Roman numerals have borrowed ideas from both branches of music theories
\end{itemize}



% It is generally agreed that the first to introduce Roman numerals to indicate degrees was Vogler.
% Furthermore, the first to unambiguously introduce Roman numerals to denote chords was Weber~\parencite{weber_theory_2018}.
% Beyond these assertions, little is known about how the syntax evolved into the conventions followed nowadays.
% % In order to complement this information with evidence of the literature of the time, I include a table of music textbooks and their adoption of Roman numeral syntax.
% Although there is not an official standardization of Roman numeral analysis, modern annotations often feature the following characteristics:

% % $$
% % k:r^{q}^{x}_{y}/r
% % $$
% % where $k$ represents a key, $r$ a scale degree, $q$ a qualitative alteration (e.g., diminished or augmented), $x$ is the numerator of an inversion, $y$ another number for denoting certain inversions.

% \begin{enumerate}
%     \item A numeric inversion with one or two digits (e.g., $^6_5$, $^4_3$, $^2$, $^6_4$, $^6$)
%     \item A slash followed by a scale degree to denote an applied chord, secondary dominant, or tonicization (e.g., V/V)
%     \item A distinction between major and minor triads (e.g., I (major) i (minor))
% \end{enumerate}

% In order to better visualize this organic evolution of the syntax, I have compiled my observations across several harmony books starting in the late nineteenth century.
% In each textbook, I have indicated whether the chords are annotated using Roman numerals, and whether other modern traits are observed (e.g., inversions depicted with arabic numerals next to the Roman numerals, or secondary dominants presented as V/V).

% \begin{table}[]
% \begin{tabular}{rlllllll}
% \multicolumn{1}{l}{Year} & Book & Author & Language & Roman numerals? & Lower and upper case? & Numeric inversions? & Applied chords? \\
% 1832 & A treatise on harmony : written and composed for the use of the pupils at the Royal Conservatoire of Music in Paris by Catel & Charles-Simon Catel & English & No & - & Yes &  \\
% 1853 & Theory of musical composition, treated with a view to a naturally consecutive arrangement of topics by Godfrey Weber & Gottfried Weber & English & Yes & Yes & No & No \\
% 1854 & A rudimentary and practical treatise on music & Charles Child Spencer & English & No & No & No & No \\
% 1855 & Course of harmony : being a manual of instruction in the principles of thorough-bass and harmony, comp. from the works of the best writers & L. H. Southard & English & No & - & Yes & No \\
% 1860 & Harmonielehre & Wilhelm V. Volckmar & German & No & No & No & No \\
% 1865 & A cathecism of the rudiments of harmony and thorough bass & James A. Hamilton & English & Yes & No & Yes & No \\
% 1868 & A treatise on harmony & F. A. Gore Ouseley & English & No & No & No & No \\
% 1868 & System und methode der harmonielehre & Otto Tiersch & German & No & No & No & No \\
% 1878 & Theory and rudimental harmony & James Tracy & English & Yes & Yes & No & No \\
% 1879 & Elements of harmony & Stephen Emery & English & Yes & Yes & Yes* &  \\
% 1880 & Harmony on the inductive method & Hugh A. Clarke & English & No & No & No & No \\
% 1883 & Harmony & Carl Mangold & English & No &  & No & No \\
% 1883 & Harmony and Instrumentation & Oscar Coon & English & Yes & Yes & No (on upper staff) & No \\
% 1883 & Neue schule der melodik & Hugo Riemann & German & Scarce (for different purposes than chords, it seems) & No & No & No \\
% 1889 & Logier's system of and self instructor in the science of music, harmony, and practical composition & J. B. Logier & English & No & No & No & No \\
% 1889 & A System of Harmony for Teacher and Pupil & John Andrew Broekhoven & English & Yes & Yes & No &  \\
% 1889 & Harmony its theory and practice & Ebenezer Prout & English & No & No & No & No \\
% 1892 & The theory and practice of tone-relations & Percy Goetschius & English & Yes & Yes & Somewhat (sometimes mixed with figured bass) & Key regions under brackets \\
% 1893 & Goodrich's analytical harmony & A. J. Goodrich & English & No & No & No & No \\
% 1896 & A manual of harmony by S. Jadassohn & Salomon Jadassohn & English & Yes & Yes & No & No \\
% 1896 & Harmony Simplified & F. H. Shepard & English & Yes & Yes & No & YES! \\
% 1897 & Harmony: a course of study & G. W. Chadwick & English & Yes & Yes & Yes & No \\
% 1898 & Five-part harmony & Francis Edward Gladstone & English & No & No & No & No \\
% 1898 & A system of Harmony & H. A. Clarke & English & No & No & No & No \\
% 1899 & Exercises in Harmony & Benjamin Cutter & English & Yes &  & Yes & No \\
% 1899 & A system of Harmony & Cyrill Kistler & English & No* (used in explanations, but not in musical examples) & No & No & No \\
% 1902 & Harmonic Analysis & Benjaming Cutter & English & Yes & Yes & Yes! &  \\
% 1903 & Practical harmony : a comprehensive system of musical theory on a French basis by Homer A. Norris v.03 & Homer Norris & English & No & - & No & No \\
% 1904 & A method of teaching harmony & Frederick Shinn & English & No & No & No & No \\
% 1905 & Modern harmony in its theory and practice & Arthur Foote & English & Yes & Yes &  &  \\
% 1907 & Lessons of Harmony & Arthur Beacox & English & Yes & Yes & Yes! &  \\
% 1909 & The nature of music; original harmony in one voice by Julius Klauser & Julius Klauser & English & Yes & More or less & No &  \\
% 1911 & A Treatise on Harmony & Joseph Humfrey Anger Henry Clough-Leighter & English & Twa & No & No & No \\
% 1912 & Essentials of Music Theory Elementary & Carl E. Gardner & English & Yes & Yes & No (on upper staff) & No \\
% 1913 & Lessons in harmony & John Mokrejs & English & Yes & No & Yes & No \\
% 1914 & The evolution of Harmony & C. H. Kitson & English & No (only in headers, not in musical examples) & No & No & No \\
% 1915 & Harmony & S. Reid Spencer & English & Yes & Yes &  &  \\
% 1916 & Practical Lesson Plans in Harmony & Helen S. Leavitt & English & Yes & Yes & Yes! & No \\
% 1917 & The theory of harmony; an inquiry into the natural principles of harmony, with an examination of the chief systems of harmony from Rameau to the present day by Matthew Shirlaw & Matthew Shirlaw & English & Mostly no (appear at pp.360) & No & No & No \\
% 1917 & A short history of harmony by Charles Macpherson. With numerous musical illustrations & Charles Macpherson & English & ?? & ?? & ?? & ?? \\
% 1918 & Harmony in pianoforte-study & Ernest Fowles & English & No & No & No & No \\
% 1918 & Aural Harmony & Franklin Robinson & English & Yes & No & Yes & No \\
% 1919 & The Foundations of Music & Henry Watt & English & No & No & No & No \\
% 1920 & Melody and harmony : a treatise for the teacher and the student & Stewart Macpherson & English & Yes & No & No & No \\
% 1921 & Applied Harmony & Carolyn Alchin & English & Yes (inconsistently) & Yes (inconsistently) & No & No \\
% 1967 & Richter's Manual of harmony: a practical guide to its study, prepared especially for the Conservatory of music at Leipsic By Ernst Friedrich Richter & E. F. Richter & English & Yes & Yes & No &  \\
% \multicolumn{1}{l}{??} & Harmony and melody & Elie Siegmeister & English & ?? & ?? & ?? & ?? \\
% \multicolumn{1}{l}{??} & Modern harmony in its theory and practice & A. Eaglefield Hull & English & Scarce (pp. 79) & No & Yes (letter) & No \\
% \multicolumn{1}{l}{??} & Harmony Simplified, or the Theory of the Tonal Functions of Chords & Hugo Riemann & English & No & No & No & No \\
% \multicolumn{1}{l}{??} & Harmony Book for Beginners & Preston Ware Orem & English & Yes & Yes & Yes & No \\
% \multicolumn{1}{l}{??} & The Rudiments of Music and Elementary Harmony & Albert Ham & English & No & No & No & no \\
% \multicolumn{1}{l}{??} & Text Book of Harmony & George Oakey & English & No & No & No & No \\
% \multicolumn{1}{l}{??} & A manual of harmony for schools &  & English &  &  &  &
% \end{tabular}
% \caption{}
% \label{tab:books}
% \end{table}