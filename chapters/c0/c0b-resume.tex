\chapter*{R\'esum\'e}
\addcontentsline{toc}{chapter}{R\'esum\'e}
\label{chap:chap0-res}

L'une des façons les plus courantes d'analyser un morceau de musique tonale est l'analyse des chiffres romains.
Cela nécessite l'inspection de plusieurs attributs liés aux accords et aux tonalités.
Les accords peuvent être inspectés en fonction de leurs propriétés : racine, qualité, inversion et fonction.
Les clés peuvent être inspectées en termes de leur portée temporelle comme les modulations ou les tonifications.
Chacun de ces attributs (ou tâches) de l'analyse des chiffres romains peut être modélisé de manière isolée.
Cependant, des recherches récentes ont montré que l'analyse simultanée de plusieurs tâches tonales conduit à des modèles MIR plus robustes.
Ceci a motivé la recherche de modèles multitâches pour l'analyse des chiffres romains.
Dans cette thèse, je développe cette ligne de recherche en :
(1) améliorant le processus de curation des données pour les ensembles de données existants ;
(2) développant une nouvelle technique d'augmentation des données pour les modèles d'analyse des chiffres romains ;
(3) améliorant la conception des réseaux neuronaux récurrents convolutionnels existants ;
et (4) l'extraction de plus de tâches tonales à partir des annotations de numéraux romains.
En combinant ces idées, j'ai formé un nouveau modèle d'analyse des numéraux romains.
Entre autres applications, cela facilitera la recherche avancée dans les collections de musique.
Par exemple, la recherche par progression d'accords ou par trajectoires de modulation.