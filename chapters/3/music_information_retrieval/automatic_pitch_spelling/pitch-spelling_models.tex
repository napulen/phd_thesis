% Copyright 2021 Néstor Nápoles López

% This is \refsubsubsec{pitch-spellingmodels}, which
% introduces the pitch-spelling models.

% \guide{Longuet-Higgins (1976)}
\textcite{longuethiggins1976perception} presented what is
arguably the first pitch-spelling algorithm. It is limited
to work on monophonic melodies. Longuet-Higgins considered
that any note could be assigned a number, $p$, based on
3-variables that related the note with respect to
\emph{middle C}: the distance in perfect fifths, the
distance in major thirds, and the octave. Using that
methodology, Longuet-Higgins extended the notation to add a
``sharpness'' feature, $q$, which was defined based on the
distance in fifths and thirds to \emph{middle C}, irrelevant
of the octave. This ``sharpness'' value, in conjunction with
the \gls{midi} note number, was used to indicate the spelling of
the note.

% \guide{Cambouropoulos (2003)}
After a gap of around 25 years, a series of new algorithms
were developed by different researchers independently,
almost at the same time. \textcite{cambouropoulos2003pitch}
presents an approach that heavily relies in intervals. The
algorithm uses a shifting overlapping window (suggested to
be of 9 to 12 pitches long). The window moves along the
sequence of pitches, from left to right, until the sequence
concludes. For each window, the pitch-spelling process
optimizes two aspects: 1) that notes make the minimum use of
accidentals (something that Cambouropoulos refers to as
\emph{notational parsimony}), and 2) the avoidance of 8
classes of augmented and diminished intervals (something
that Cambouropoulos refers as \emph{interval optimization}).
The algorithm was evaluated on Mozart piano sonatas and
Chopin Waltzes.

% \guide{Chew and Chen (2003)}
After its introduction in her dissertation, the Spiral Array
model by \textcite{chew2000towards} was used for multiple
tasks that involved tonal analysis, for example, chord
labeling, key-finding, and pitch-spelling. The Spiral Array
is a spatial representation of tonality that allows a
sequence of pitches to be \emph{positioned} in
the---theoretical---tonal space, which facilitates the
computation of tonal features in different ways. Regarding
the pitch-spelling problem, three different algorithms were
proposed by \textcite{chew2003determining} that were built
on top of each other. In the paper, their evaluation of the
algorithms restricted to few movements of Beethoven's piano
sonatas.

% \guide{Temperley and Sleator (2004)}
Throughout the late 1990s and early 2000s, Temperley
developed a series of algorithms for modelling different
aspects of musical structure (i.e., meter, phrasing,
counterpoint, harmony, key, and pitch-spelling). These
models were implemented by Daniel Sleator in the
\emph{Melisma Music Analyzer} and explained in detail in the
book by \textcite{temperley2004cognition}. Respecting the
pitch-spelling problem, the approach by Temperley consists
of a rule-based system based on his concepts of \gls{tpc} and the \emph{line of fifths}.
Additionally, the algorithm depends on the metrical analysis
provided by the Melisma Music Analyzer, before extracting
the spelling of the notes.

% \guide{Meredith (2003)}
\textcite{meredith2003pitch} introduced an algorithm that
receives \gls{midi} note numbers and onset times as ordered pairs
and determines the spelling of the notes through 2 stages.
The first stage consists of a rule-based system with 8 rules
that are carried out in sequence. After the first stage, the
algorithm has determined a spelling for the notes. The
second stage, consists of the correction of the outputs
given by the first stage. The corrections performed account
for neighbouring notes or passing notes which could have
been erroneously predicted by the first stage. In a
subsequent study, \textcite{meredith2005comparing} compared
its pitch-spelling algorithm against other pitch spelling
algorithms, determining that his algorithm had the highest
accuracy.

% \guide{Stoddard et al. (2004)}
\textcite{stoddard2004welltempered} proposed a new algorithm
for pitch-spelling. Their approach is data-driven, which
requires ground-truth spelling information in order to be
trained. Additionally, their algorithm runs on top of the
probabilistic framework for harmonic analysis by
\textcite{raphael2003harmonic}, and it is conditioned by the
accuracy of that harmonic analysis model. The algorithm was
evaluated over a total of 22,593 notes, scoring 96.87\%
accuracy. During their study, they found that the detection
and resolution of voice-leading considerations (i.e., the
relationship between a note and its immediate neighbours)
was the most important feature to consider in a
pitch-spelling algorithm.

% \guide{Teodoru and Raphael (2007)}
\textcite{teodoru2007pitch} tackled the problem by assuming
that the spelling of a note depends on voice leading and
local harmony. They observed principles and guidelines for
spelling notes from music theory textbooks
\parencite{aldwell1978harmony,rimskykorsakov2005practical}
and attempted to build a probabilistic model that
\emph{captures} these relationships through the use of
Markov chains. In their evaluation, they showed that their
model outperformed every other model, including the one in
\textcite{meredith2006ps13}. Nevertheless, an important
drawback from their approach is that it assumes that the
music will be provided as individual voices, which is not a
reasonable assumption for real-world \gls{midi} inputs and,
therefore, limits the application of their algorithm in
practice. In their study, they found that estimating the
time-varying key (local key) is fundamental for pitch
spelling. Additionally, they consider that the relationship
between local keys and spelling is bidirectional. That is,
spelling can help influence the choice of key as much as the
key can influence the choice of spelling.
