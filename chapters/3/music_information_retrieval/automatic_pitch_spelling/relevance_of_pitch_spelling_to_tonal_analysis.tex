% Copyright 2021 Néstor Nápoles López

This is
\refsubsubsec{relevanceofpitchspellingtotonalanalysis},
which introduces the relevance of pitch spelling to tonal
analysis.

% Pitch spelling is likely to be correlated with tonal
% analysis. Consider the following example. If during a
% musical performance, a pianist presses the key corresponding
% to pitch class ``8'' in the piano keyboard (shown in Figure
% \ref{fig:Q7_1}), how can we know if the note played by the
% performer was an A$\flat$ or a G$\sharp$?\footnote{Assume an
% equal-tempered setting, with no special considerations
% regarding how the piano might be tuned. That is, the
% distinction between the A$\flat$ and G$\sharp$ notes that we
% are looking for is of a semantic nature, not of a tuning
% nature.}

% % \begin{figure}[h] \centering %
% % \includegraphics[width=0.6\textwidth]{figures/Q7_1.png} %
% % \caption{A note playing in a keyboard. It could be %
% % either G$\sharp$ or A$\flat$.} \label{fig:Q7_1} %
% % \end{figure}

% A first approach could be to determine the spelling of the
% note based on the musical key of the piece where the note
% was played. Such information can be automatically obtained
% using a \gls{gke} model with a reasonable degree of
% accuracy. Nevertheless, our pitch-spelling model now
% requires information about the key in order to make its
% predictions.

% The key model could predict that the piece is in, for
% example, \emph{A minor}. ``Is it a piece in A minor? Then
% the note is probably a G$\sharp$.''

% % \begin{figure}[h] \centering %
% % \includegraphics[width=0.8\textwidth]{figures/Q7_2.png} %
% % \caption{J. S. Bach's BWV 784, mm. 14-15. The piece is %
% % in A minor, however, during measure 15, the pitch % class
% % number 8 is spelled as an A$\flat$.} % \label{fig:Q7_2}
% % \end{figure}

% Figure \ref{fig:Q7_2} shows an adversarial example where the
% pitch class number 8 is spelled as A$\flat$ in a piece that
% is originally in \emph{A minor}. A person familiarized with
% Western tonal music could guess that similar examples as the
% one shown in Figure \ref{fig:Q7_2} will occur whenever the
% music is modulating or tonicizing a different key.

% Assuming that the awareness of modulations and tonicizations
% throughout the piece (which we generally refer to as
% \emph{local keys}) will mitigate our problems when
% predicting the spelling of a note, seems to be a reasonable
% guess. Instead of requiring to know the \emph{global key} of
% the piece, now, our model requires to know the \emph{local
% keys} throughout the piece. Furthermore, there are other
% circumstances that affect the spelling of a note, such as
% the use of non-chord tones or the harmonic context. One
% example of the implications of non-chord tones is, as shown
% in Figure \ref{fig:Q7_3}, the spelling of a chromatic
% neighbouring note, which should probably have a different
% note name than the \emph{real} note.

% % \begin{figure}[h] \centering %
% % \includegraphics[width=0.8\textwidth]{figures/Q7_3.png} %
% % \caption{Two notations (A and B) of a chromatic %
% % neighbouring note. Regardless of the harmonic and key %
% % context, the second note should probably be spelled as %
% % in version B.} \label{fig:Q7_3} \end{figure}

% One example of a complicated harmonic context is the
% presence of a German augmented sixth, for which Teodoru and
% Raphael write \parencite{teodoru2007pitch}:

% \begin{italicquotes}
% Other situations require a deeper notion of the harmonic
% state than provided by the local key, as in the German
% augmented sixth chord, which seems nearly impossible to
% spell correctly without recognizing it as such.
% \end{italicquotes}

% Overall, designing a pitch-spelling algorithm poses
% important challenges in different musical fronts, but it
% also provides researchers with the opportunity of putting in
% practice different analytical models (e.g., local key, chord
% labeling, and non-chord detection) for solving a common
% task. Furthermore, unlike many of those models which suffer
% from highly ambiguous evaluations, pitch spelling presents a
% simple way of assessing whether our computational models
% understand the musical context or not. That is, the note is
% either G$\sharp$ or A$\flat$, and there is only one right
% answer.

% In the following section, a survey is provided with
% different pitch-spelling algorithms that have been developed
% throughout the years.
