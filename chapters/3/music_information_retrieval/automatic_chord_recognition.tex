% Copyright 2021 Néstor Nápoles López

% This is \refsubsec{automaticchordrecognition}, which
% introduces the automatic chord recognition.


\gls{acr} has been explored thoroughly in the field of
\gls{mir}. \gls{acr} systems typically seek to predict the
root and quality of the chords throughout a piece of music
via either an audio or a symbolic representation. A more
specific type of chordal analysis, particularly relevant for
Western classical music, is functional harmony, or Roman
numeral analysis. The main difference between \gls{acr} and
functional harmony is that the latter requires other
adjacent tasks to be solved simultaneously, notably
including detection and identification of key changes
(modulations \parencite{feisthauer2020estimating,
schreiber2020local} and tonicizations
\parencite{napoleslopez2020local}).

Partly due to the repertoires typically targeted, Roman
numeral analysis also involves handling of some additional
`special chords' such as Neapolitan and Augmented sixth
chords.

For a summary of general \gls{acr} strategies, see Pauwels
et al. \parencite{pauwels201920}.
