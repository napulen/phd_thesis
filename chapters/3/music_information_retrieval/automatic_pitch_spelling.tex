% Copyright 2021 Néstor Nápoles López

% This is \refsubsec{automaticpitchspelling}, which
% introduces the automatic pitch spelling.

A pitch-spelling model is an algorithm that predicts the
original spelling that a note had in a musical score when
only the pitch-class, octave, and duration of the note are
provided as inputs to the model.\footnote{Some researchers
have also found helpful to know the \emph{voice} (or
\emph{stream}, in psychoacoustics) to which the note belongs
\parencite{teodoru2007pitch}. However, this information is
unavailable in most \gls{midi} files available online.
Therefore, algorithms of this kind should preferrably not
rely on voice information.}

\guide{A brief survey of pitch-spelling algorithms}

Compared to other \gls{mir} tasks, there are fewer
pitch-spelling algorithms available in the literature. One
caveat here is that, because pitch-spelling is an important
feature for commercial software dealing with the conversion
of \gls{midi} files into music scores (e.g., music notation
editors), other algorithms may exist among commercial
applications. However, I focus on discussing the published
algorithms.

One argument for the relevance of pitch spelling in
\glspl{rna} is that recent models have benefitted from
spelling information \parencite{micchi2020not}. This is
feasible for \gls{musicxml} and similar input
representations, but not for \gls{midi} inputs, which
arguably account for the vast majority of the existing
digital symbolic music files today. Pitch spelling in
parallel (or as a preprocessing step) to \glspl{rna} will
thus be a viable approach in future models. This is also
true for implementing \glspl{rna} models in the audio
domain, which will likely deal with chromagram
representations (without spelling).
