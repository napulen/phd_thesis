% Copyright 2021 Néstor Nápoles López

% This is \refsubsec{automaticpitchspelling}, which
% introduces the automatic pitch spelling.

A pitch-spelling model is an algorithm that predicts the
original spelling that a note had in a musical score when
only the pitch-class, octave, and duration of the note are
provided as inputs to the model.\footnote{Some researchers
have also found helpful to know the voice (\emph{stream}, in
psychoacoustics) in which the note is sounding
\parencite{teodoru2007pitch}. However, this information is
beyond what can be reliably obtained from real-world digital
music files. Therefore, algorithms of this kind should work
regardless of voice information being provided or not.}

\guide{A brief survey of pitch-spelling algorithms}

Compared to other \gls{mir} tasks, there have not been as
many attempts to solve the problem of pitch spelling.
Nevertheless, it is likely that there are other solutions to
the problem, in commercial applications, which are not
described in the literature and in this survey. This is due
to the fact that pitch-spelling is a highly relevant problem
for commercial software that deals with the conversion of
\gls{midi} files into music scores, for example, most music
notation editors.

The reason why I consider pitch spelling to be a relevant
problem for Roman numeral analysis is because recent models
have benefitted from spelling information
\parencite{micchi2020not}. This approach works for MusicXML
inputs and similar symbolic representations, but not for
\gls{midi} inputs, which arguably account for the vast
majority of the existing digital symbolic music files today.
Pitch spelling in parallel (or as a preprocessing step) will
be necessary for Roman numeral analysis then. This is also
true for implementing Roman numeral analysis models in the
audio domain.
