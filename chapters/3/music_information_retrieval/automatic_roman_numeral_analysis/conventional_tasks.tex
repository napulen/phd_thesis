% Copyright 2022 Néstor Nápoles López

\textcite{chen2018functional} proposed breaking a \gls{rna}
label into a set of multitask classification configuration
based on the following tonal features: local key, primary
degree, secondary degree, inversion, chord quality, and
chord root. This approach has been adopted in other research
\parencite{chen2019harmony, micchi2020not,micchi2021deep}.

\phdparagraph{local key}

The (local) key describes the reference key used to
disambiguate the Roman numeral scale degrees. It is arguably
one of the most important features because if it is
mislabeled, then all \gls{rna} labels will be mislabeled.

In the initial multitask learning model by
\textcite{chen2018functional}, this task had 24 output
classes. Each of the classes represented the pitch class of
the tonic, which were encoded twice to account for major
mode keys and minor mode keys.

In a subsequent model by \textcite{micchi2020not}, the
number of classes was extended to 30, as the spelling of the
tonic pitch was taken into account (i.e., $C\sharp$ $\neq$
$D\flat$ major).


\phdparagraph{primary degree}

The primary degree of the chord represents the ``numerator''
part of a Roman numeral. For example, in the annotation
$\rn{V}\rnseven{}/\rn{v}$, the primary degree is
$\rn{V}\rnseven{}$. In the multitask learning model by
\textcite{chen2018functional}, this task had 21 classes.

\phdparagraph{secondary degree}

The secondary degree represents the ``denominator'' part of
a Roman numeral label. For example, in the annotation
$\rn{V}\rnseven{}/\rn{v}$, the secondary degree is $\rn{V}$.
In the multitask learning model by
\textcite{chen2018functional}, this task had 21 classes.
Secondary degrees only appear in the context of a
tonicization. Thus, the majority of the \gls{rna}
annotations will have an empty secondary degree, and a
special class to denote this empty secondary degree needs to
be included. The scarcity of secondary degrees also results
in a highly imbalanced distribution of examples, which is
problematic for training.

\phdparagraph{inversion}

The inversion of the chord indicates the note acting as the
bass of the chord. In the \gls{rna} system, the arrangement
of the notes above the bass is irrelevant for the
annotation. For example, the first two chord realizations
shown in \reffig{inversion} have an equivalent Roman numeral
label, despite the different arrangement of the upper
voices. However, modifying the bass modifies the inversion,
regardless of the arrangement of the upper voices remaining
unchanged.

\phdfigure[Example of chord inversions in the \gls{rna}
syntax. The inversion changes when the bass changes,
regardless of the arrangement of the upper
voices]{inversion}

The inversion task generally requires 4 output classes: root
position (no inversion), 1st inversion, 2nd inversion, and
3rd inversion (only for seventh chords).

\phdparagraph{quality}

The chord quality task depends greatly on the chord
vocabulary. The vocabulary introduced in
\textcite{chen2018functional} considered 10 classes. The one
by \textcite{micchi2020not} considered a similar number.
This classification also suffers from a large class
imbalance, because the vast majority of chords found in
tonal music are major triads.

\phdparagraph{root}

The chord root task, as the name implies, predicts the pitch
of the chord's root. In the multiclass classification
problem introduced by \textcite{chen2018functional}, there
are 12 possible chord roots (i.e., as many as pitch
classes). In the work by \textcite{micchi2020not}, this was
extended to 35 classes (see
\refsubsubsec{encodingnoteswithspelling}), as spelling was taken
into account. The system was trained to predict any root
within two flats and two sharps. 
