% Copyright 2022 Néstor Nápoles López

% This is \refsubsubsec{romannumeralanalysismodels}, which
% introduces the roman numeral analysis models.

\phdparagraph{pioneering works}

The pioneering work in \gls{arna} is the one by
\textcite{winograd1968linguistics}. This model was evaluated
over music by Bach. It required a preliminary, hand-made,
representation of the score as a sequence of four-part
perfect chords. This representation was processed by a model
implemented in the \gls{lisp} programming language. The
hand-made representation required \emph{nonchord}s to be
removed before processing the input with the model, a
crucial limitation, because this process was complex and
could not be automated. In addition to this limitation,
\textcite{temperley1997algorithm} provided insight about
other limitations in Winograd's model, for example, its
incapacity to deal with arpeggios and contrapuntal textures.

After Winograd, an \emph{expert system} model was introduced
by Maxwell, first in his PhD dissertation
\parencite{maxwell1984artificial}, and subsequently in a
book chapter \parencite{maxwell1992expert}. This model is
one of the best examples of rule-based systems for
\glspl{mir}. The model consisted of fifty five rules that
reduced vertical sonorities into chord sequences, and
decided the location of changes of key. Initially, the rules
of the system were short and concise:

\begin{italicsquote}
    Perfect and imperfect consonant intervals constitute a
    consonant interval. Every other is a dissonant interval.
\end{italicsquote}

However, as new rules were introduced, they became
increasingly more complex and cryptic:

\begin{italicsquote}
    If the goal chord falls on a strong beat and it is a
    major triad or major-minor seventh, and the root
    movement from the pre-cadence is an ascending or
    descending perfect fifth or major second or a descending
    minor second, and when the root motion is a descending
    fifth, the pre-cadence is not a potential dominant, and
    when the root motion is an ascending fifth the
    pre-cadence is triadic, then the pseudo-cadence is a
    half cadence, and its strength increases by 10.
\end{italicsquote}

The system was also criticized by Temperley and others for
its use of arbitrary or ``hardcoded'' values to determine
the strength of a cadence. An example shown in the last
sentence of the previous rule.

Regardless of the complex rules, Maxwell's model provided
convincing music-theoretical outputs when analyzing three
movements of the Bach Six French Suites: the sarabande from
Suite No.~1, the minuet from Suite No.~2 and the gavotte
from Suite No.~5. The pieces featured distinct types of
complexities: four-part harmony with several
\gls{nonchord}s, 2-voice contrapuntual textures, and a
varying contrapuntual texture, respectively. Unfortunately,
because these are only three musical examples by the same
composer, it is unlikely that the system would generalize to
other compositions. \textcite{temperley1997algorithm} listed
the limitations of Winograd and Maxwell's approaches,
summarizing them in the following:

\begin{itemize}
    \item Sequences of notes that are not displayed
    simultaneously (vertically), as arpeggiations of chords.
    \item Missing pitches in the spelling of a full chord
    that can be deduced from the context.
    \item Ornamental notes. Maxwell proposes specific rules
    to deal with these notes, but according to Temperley,
    neither Maxwell's or Winograd's are good enough to
    correctly detect ornamental notes.
\end{itemize}

\phdparagraph{the melisma music analyzer}

The first end-to-end \gls{arna} system can be attributed to
the contributions of Temperley, Sleator, and Sapp. First,
\textcite{temperley1997algorithm} proposed an algorithm to
find harmonic roots, which was complemented with preference
rules for meter analysis \parencite{temperley1999modeling}
and a key estimation model \parencite{temperley1999whats}.
The collective algorithms were published by
\textcite{temperley2004cognition} and implemented in
collaboration with Sleator as a suite of programs called
\gls{melisma}.\footnotelink{https://www.link.cs.cmu.edu/music-analysis/}
The system operated with intermediate representations for
both inputs and outputs. Sapp filled in the remaining gap
with additional stages that allowed \gls{melisma} to digest
\gls{humkern} inputs and generate \gls{humharm} \gls{rna}
annotations, aligning them with the original score. The full
end-to-end workflow was facilitated by a program called
\emph{tsroot} \parencite{sapp2009tsroot} and used to
generate fully automatic \gls{rna} labels of the music
scores in the \emph{KernScores} database
\parencite{sapp2005online}.

After the first version of \gls{melisma}, which is based on
preference rules, Temperley favored probabilistic models in
subsequent work. Notably, \textcite{temperley2009unified}
proposed a Bayesian approach to simultaneously model
harmonic analysis, meter induction, and melodic
seggregation. This motivated the \emph{Melisma Music
Analyzer (version
2)},\footnotelink{http://davidtemperley.com/melisma-v2/}
however, this program was not extended to provide \gls{rna}
annotations.

\phdparagraph{probabilistic and grammar-based models}

% Another important probabilistic approach that emerged to
% solve the problem of functional harmonic analysis was that
% of Christopher Raphael \cite{raphael2003harmonic}. This
% model from Raphael is one of the few pure-functional
% harmonic analysis approaches, oriented towards the
% analysis of common-practice music. The idea is simple and
% some of his assumptions simplify the parameters of the
% model. Some constant that remains as in previous models is
% the fact that it gets rid of all the pitch-spelling
% information and replaced for solely pitch information.
% This model, unlike Temperley, does not try to reconstruct
% the pitch spelling information back by any algorithmic
% means, and in the words of Raphael, it is an obvious
% extension to the model.

% Another quite important effort in the statistical domain
% includes the work from Martin Rohrmeier
% \cite{rohrmeier2008statistical}, who uses a heuristic
% method of segmentation. Analyzes distributions of single
% pitch-class-sets, chord classes and pitch-class-sets
% transitions. One of the goals of Rohrmeier was the
% research of \emph{syntacticality} in harmony. The final
% goal is to produce descriptive analyses of harmonic
% structure based on an empirical approach. The choice of
% the corpus is, similarly to others, music from Joahann
% Sebastian Bach. In his case, chorales, because they
% constitute a large and coherent set of pieces regarding
% style and composition technique. Rohrmeier claims that
% this work is pioneer in the statistical analysis of a
% corpus for the purpose of finding features of tonal
% harmony. According to the results of Rohrmeier, the most
% frequent occurrences of pitch-class-sets in this music are
% those of tonic, dominant and subdominant chords. This
% would be expected and helps to reinforce those
% scale-degree theories that describe harmonic movement with
% transition tables of scale degrees. The results from this
% research, apart from being interesting in confirming
% music-theory beliefs regarding common chords and
% transitions in tonal music, could also be replicated in a
% different corpus to target its particular common chords
% and transitions.

% \subsection{Grammar-based} Three years after his
% statistical work in Bach Chorales, Martin Rohrmeier
% brought back the use of grammars to study the underlying
% structure of musical harmony \cite{rohrmeier2011towards},
% which inspired future works by other researchers in the
% field, specially towards analyzing jazz music. In this
% work, Rohrmeier claims that the structure of harmonic
% progressions exceeds the simplicity of a markovian
% transition table, and he proposes a set of
% phrase-structure grammar rules. For this purpose, the
% hierarchical analysis from the music-theory approach done
% by \cite{kostka1995tonal} is presented, with the belief
% that it can be brought to a closer formalization.
% Rohrmeier presents a tree representation of a chord
% sequence, using two principles: \begin{itemize} \item
% Chords have dependencies and the existence of one
% sometimes requires the existence of another \item Chords
% have functions, these functions can be realized by a set
% of chords. \end{itemize} The system comprises 27 rules for
% the generation of a grammar, this helps to model
% common-practice music as well as jazz music. The output of
% this algorithm is a hierarchical tree of the functions and
% dependencies of the chords. The level of detail from this
% work to model every special case of the harmonic language
% is remarkable, as careful attention has been put trying to
% comprehend distinct kinds of cadences, chords and
% modulations in the model.

% This work was eventually retaken and implemented by Bas de
% Haas \cite{de2013automatic} using chord labels as input
% for the system.

Notable subsequent studies include an \gls{hmm}-based
approach for functional harmony in
\textcite{raphael2004functional}, a method based on dynamic
programming in \textcite{illescas2007harmonic}, an
application of \gls{pcset}s to search for chord
``syntacticality'' in \textcite{rohrmeier2008statistical},
grammar-based approaches in
\textcite{magalhaes2011functional, quick2016learning,
harasim2018generalized, harasim2019harmonic}, and a
generative syntax to parse chord progressions featuring
modulations in \textcite{rohrmeier2011towards}.

\phdparagraph{deep learning models}

More recently, deep neural networks have become the
preferred tool for approaching this problem.
\textcite{chen2018functional} were the first to introduce
\gls{mtl} \parencite{ruder2017overview} to the problem as a
suitable way for the neural network to share representations
between related tonal tasks. Chen and Su's model consists of
a bidirectional \gls{lstm} \parencite{hochreiter1997long}
followed by task-specific dense layers, which implement the
\gls{mtl} aspect. In this work, the authors also introduced
the \gls{bps} dataset for evaluating such models. The
\gls{mtl} layout outperformed single-task configurations and
it has continued to prove the best-performing approach in
subsequent deep learning studies. In subsequent work, the
same authors have adopted Transformer-based networks to deal
with functional harmony and \gls{acr}
\parencite{chen2019harmony, chen2021attend}. The work with
these networks has explored the capability of the attention
mechanisms to improve the performance of \gls{acr}, paying
special consideration to chord segmentation and its
evaluation.

\textcite{micchi2020not} proposed a \emph{DenseNet}-like
\parencite{huang2017densely} \gls{cnn}, followed by a
recurrent component. The recurrent component consists of a
bidirectional \gls{gru} \parencite{cho2014learning}
connected to task-specific dense layers, similar to those of
\textcite{chen2018functional}. In their experiments, the
\emph{DenseNet}-like convolutions outperformed dilated
convolutions and a \gls{gru} by itself (i.e., with pooling
instead of the convolutional blocks). Micchi et al.~also
demonstrated the positive effect of using \emph{pitch
spelling} in the inputs and outputs. This provided at least
two advantages: a more informative output (e.g., not only
the correct key, but the correct spelling between two
enharmonic keys), and an increased (theoretical) number of
transpositions available for data augmentation. In
\textcite{micchi2021deep}, the same architecture was
extended to include a \gls{nade} layer, which models a
dependency between the \gls{mtl} layers, improving the
overall performance of the network.

\textcite{mcleod2021modular} proposed a new network
architecture for \gls{rna}. In their work, they depart from
the multitask learning approach used by previous
researchers, substituting it with a modular workflow where
the different tonal attributes are computed in sequential
stages of the model.

Around the same time that \textcite{micchi2021deep} and
\textcite{mcleod2021modular} have proposed their models, we
have also introduced an earlier version of the neural
network presented in this \thesisdiss{}
\parencite{napoleslopez2021augmentednet}.

The method in this dissertation follows the line of research
based on \gls{mtl}. This methodology is described below.
