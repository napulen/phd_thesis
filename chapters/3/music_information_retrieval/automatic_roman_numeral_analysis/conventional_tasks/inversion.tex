% Copyright 2022 Néstor Nápoles López

The inversion of the chord indicates the note acting as the
bass of the chord. In the \gls{rna} system, the arrangement
of the notes above the bass is irrelevant for the
annotation. For example, the first two chord realizations
shown in \reffig{inversion} have an equivalent Roman numeral
label, despite the different arrangement of the upper
voices. However, modifying the bass modifies the inversion,
regardless of the arrangement of the upper voices remaining
unchanged.

\phdfigure[Example of chord inversions in the \gls{rna}
 syntax. The inversion changes when the bass changes,
 regardless of the arrangement of the upper
 voices]{inversion}

 The inversion task generally requires 4 output classes:
 root position (no inversion), 1st inversion, 2nd inversion,
 and 3rd inversion (only for seventh chords).
