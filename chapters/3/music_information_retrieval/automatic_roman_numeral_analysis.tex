% Copyright 2021 Néstor Nápoles López

% This is \refsubsec{automaticromannumeralanalysis}, which
% introduces the \glspl{rna}.

\gls{rna} refers to the syntax used to annotate chords,
which evolved from the sporadic use of Roman numerals in the
late eighteenth century to the complex analysis system used
today (see \refchap{introductiontoromannumeralanalysis}). In
computational contexts, \glspl{rna} has been thought,
starting with \textcite{temperley1997algorithm}, as a
``chord-finding plus key-finding'' problem. In recent years,
\textcite{chen2018functional} subdivided the problem further
into six sub-tasks: chord quality, chord root, local key,
inversion, primary degree, and secondary degree. As an
\gls{mir} problem, \glspl{rna} can be expressed as the task
of correctly predicting enough features in order to
reconstruct the correct Roman numeral labels. This problem
is challenging for various reasons (see
\refsec{challenges}). In the next section, I present a
survey of existing approaches to solve this problem.
