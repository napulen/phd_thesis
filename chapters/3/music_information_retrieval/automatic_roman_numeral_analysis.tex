% Copyright 2021 Néstor Nápoles López

% This is \refsubsec{automaticromannumeralanalysis}, which
% introduces the automatic roman numeral analysis.

One of the most common ways to analyze a piece of tonal
music is through Roman numeral analysis. This requires the
inspection of several attributes related to chords and keys.
Chords can be inspected in terms of their properties: root,
quality, inversion, and function. Keys can be inspected in
terms of their temporal scope as modulations or
tonicizations\footnote{An explanation of these concepts is
presented in \textcite{napoleslopez2020local}}. As an \gls{mir}
problem, functional harmony can be expressed as the task of
correctly predicting enough features in order to reconstruct
the correct Roman numeral labels. From a computational
perspective, predicting such labels is challenging
given that the model has to predict multiple features
correctly and simultaneously. In the past, \gls{mir} researchers
have reconstructed Roman numeral annotations by predicting
six sub-tasks: chord quality, chord root, local key,
inversion, primary degree, and secondary degree
\parencite{chen2018functional, micchi2020not}.
