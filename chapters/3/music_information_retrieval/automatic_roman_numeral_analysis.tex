% Copyright 2021 Néstor Nápoles López

% This is \refsubsec{automaticromannumeralanalysis}, which
% introduces the automatic roman numeral analysis.

One of the most common ways to analyze a piece of tonal
music is through Roman numeral analysis. This requires the
inspection of several attributes related to chords and keys.
Chords can be inspected in terms of their properties: root,
quality, inversion, and function. Keys can be inspected in
terms of their temporal scope as modulations or
tonicizations~\cite{napoleslopez2020local}. Each of these
attributes (or tasks) of Roman numeral analysis can be
modeled in isolation. As a result, many models for automatic
key and chord analysis exist in the Music Information
Retrieval (MIR) literature. Recent improvements have been
achieved by Roman numeral analysis models that predict
chords and keys simultaneously~\cite{chen2021attend,
micchi2020not}. This happens via multitask learning, a
technique where a machine learning model solves several
problems at once~\cite{ruder2017overview}.

Thus, as a machine learning problem, functional harmony can
be expressed as the task of correctly predicting enough
features in order to reconstruct the original Roman numeral
label.


\guide{Roman Numeral Analysis.}
The analytical process of functional harmony is commonly
described through Roman numeral annotations. This annotation
system is particularly popular in Western music theory for
the analysis of `common-practice' tonal music. Roman numeral
annotations encode a great deal of information about
tonality, in a compact syntax. For instance, an annotation
like  \texttt{C:viio65/V} includes an account of the local
key (here C), the quality of the chord (diminished seventh),
the chord inversion (first), and nature of any tonicization
(optional, here true: of the dominant).

\guide{Problem.}
From a computational perspective, predicting such
annotations is challenging given that the model has to
predict multiple features correctly and asimultaneously. In
the past, MIR researchers have reconstructed Roman numeral
annotations by predicting six sub-tasks: chord quality,
chord root, local key, inversion, primary degree, and
secondary degree \parencite{chen2018functional,
micchi2020not}.

Thus, as a machine learning problem, functional harmony can
be expressed as the task of correctly predicting enough
features in order to reconstruct the original Roman numeral
label.

Recent efforts in this area have seen a great
standardization of the notation and conversion routines,
\parencite{gotham2019romantext} which in turn has
facilitated the amassing of a relatively large meta-corpus
of Roman numeral analyses \parencite{gotham2019romantext}.
However, despite these developments and the wider resurgence
of interest in the field, the performance of functional
harmony models for predicting full Roman numeral labels
remains relatively low.
