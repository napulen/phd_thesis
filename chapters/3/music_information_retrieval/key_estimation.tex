% Copyright 2021 Néstor Nápoles López

% This is \refsubsec{keyestimation}, which introduces the
% key estimation.

Identifying the musical key is a fundamental task in the
analysis of tonal music. It is often a preliminary or
concurrent step to other common musicological tasks like
harmonic analysis and cadence detection. In particular, the
knowledge of the musical key can help a music analyst to
find boundaries in a musical piece, interpret the role of
notes and chords, or suggest a musical form to which the
analyzed piece conforms. Due to its importance, key
estimation is a well-studied research topic in \gls{mir},
and multiple key-analysis algorithms have emerged during the
last decades. Broadly, there are two types of key-estimation
algorithms: those that find the \emph{main key} of the piece
(hereafter \gls{gke} models) and those that find the changes
of key \emph{within} the piece (hereafter \gls{lke} models).
The global key is related with the key of the piece or work,
for example, as written in the title of the score.
% Generally, that relationship is well-defined.
A local key is related to the music-theoretical concepts
that explain changes of keys, such as \emph{modulations} and
\emph{tonicizations}.\footnote{And maybe others, such as
\emph{modal mixture}, \emph{applied chords}, and
\emph{secondary dominants}.}
% However, the definition of a local key is still somewhat
% elusive, a point that I will revisit in
% \refsubsubsec{local-keyestimation}.

In this section, I describe the different \gls{gke} and
\gls{lke} models that have been presented in the \gls{mir}
literature over the years.
