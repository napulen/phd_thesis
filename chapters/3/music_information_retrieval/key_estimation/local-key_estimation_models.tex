% Copyright 2021 Néstor Nápoles López

% This is \refsubsubsec{local-keyestimationmodels}, which
% introduces the local-key estimation models.


Contrary to the global-key estimation approaches, local-key
estimation models have a relatively recent history.

% Audio
\textcite{purwins2000new} introduced a method for tracking
changes of key in audio signals using cq-profiles, which are
calculated with the constant Q filter bank. Their goal is to
track the tone center and its variation during the piece.
Their references annotate both modulations and tonicizations
but consider that the ground truth is the one indicated by
the tonicizations.

% symbolic
\textcite{chew2002spiral} measured the distance from a
sequence of pitches to a key using the \emph{spiral array}.
The succession of keys is then modeled as a sequence of
\emph{boundaries} dividing the score in different key areas.

% Audio
\textcite{chai2005detection} designed a model based on a
Hidden Markov Model (HMM) to detect changes of key. They
describe the term \emph{modulation} as ``the change of key
at some point''. Their model detects, at first, the tonal
center, and then, the mode of the key.

% Audio
\textcite{catteau2007probabilistic} introduced a model for
scale and chord recognition, assuming that there is a
correspondence between a major scale and a major key, and
between a harmonic minor scale and a minor key. Their model
is based on the key profiles by
\textcite{temperley1999whats} and Lerdahl's \emph{tonal
pitch space} \parencite{lerdahl2005tonal}.

\textcite{izmirli2007localized} introduced a model to find
local keys from audio sources, based on non-negative matrix
factorization for segmentation. Izmirli also attempted to
disambiguate modulations and tonicizations in the following
manner: ``Secondary functions and tonicizations are heard as
short deviations from the well-grounded key in which they
appear---although the boundary between modulation and
tonicization is not clear cut. A modulation unambiguously
instigates a shift in the key center''.
% This work is also, to our knowledge, the first time that
% the term \emph{local keys} has been mentioned in an MIR
% publication.

\textcite{papadopoulos2009local} adopted a similar approach
to Izmirli for audio local-key estimation. Their model
attempts to segment the score based on the points of
modulation. They introduced key dependencies on the harmonic
and metric structures of global-key-finding methods, in
order to convert them into local-key-finding ones.
% They also discussed the need for more data in chords and
% local key.

\textcite{rocher2010concurrent} introduced a model that
provides (chord, key) duples for each audio frame of an
input excerpt. The model is based on a graph and the
\emph{best-path} estimation method. For evaluating key
distances, they used the key profiles by
\textcite{temperley1999whats}. The authors alluded to the
term modulation when discussing their key predictions.

\textcite{mearns2011automatically} used an HMM to estimate
modulations over audio transcriptions of Bach chorales. The
HMM is trained with chord progressions. The emission
probability distributions are obtained from two tables with
the probabilities of chords existing in a given key. These
tables are based on the work by Schoenberg and Krumhansl.
Applied chords (i.e., tonicizations) are not described in
these charts, therefore, the authors do not deal with
tonicizations.

\textcite{pauwels2014combining} presented a probabilistic
framework for the simultaneous estimation of chords and keys
from audio. They mention the importance of ``integrating
prior musical knowledge'' into a local-key-estimation
system, however, they do not allude to the terms modulation
and tonicization. The same year, \textcite{weis2014chroma}
proposed an audio scale estimator. They argue that this
estimator can help to determine the local tonality based on
G\'{a}rdonyi's scale analysis method. They did not use the
term tonicization, however, they discussed ``short-time
local modulations'', which resemble tonicizations.

Machine learning approaches, especially using neural
networks, have recently gained popularity in MIR research,
including key estimation. Independently,
\textcite{chen2018functional, chen2019harmony} and \textcite{micchi2020not} designed models that estimate
local keys as well as roman numeral analysis annotations.
Tonicization information is implied by the roman numeral
analysis annotations.

N\'apoles L\'opez et al. introduced a model to find changes
of key (local-key estimation) as well as the main key of a
piece (global-key estimation), using an HMM
\textcite{napoleslopez2019key}. The model is also capable of
working with symbolic and audio data. They do not allude to
the terms modulation or tonicization, always referring to
their predictions as \emph{local keys}.

One of the most recent models for finding changes of key is
by \textcite{feisthauer2020estimating}, which has
been designed to detect modulations in Classical music. It
uses three proximity measures established from pitch
compatibility, tonality anchoring, and tonality proximity.
The model computes the cost of being in a key on a given
beat, and estimates the succession of keys using dynamic
programming techniques.
