% Copyright 2021 Néstor Nápoles López

% This is \refsubsubsec{local-keyestimationmodels}, which
% introduces the local-key estimation models.


Contrary to the global-key estimation approaches, local-key
estimation models have a relatively recent history.

% Audio
\textcite{purwins2000new} introduced a method for tracking
changes of key in audio signals using cq-profiles, which
were calculated with the constant-Q filter bank. Their goal
was to track the tone center and its variation during the
piece. Their references annotate both modulations and
tonicizations but considered that the ground truth was the
one indicated by the tonicizations.

% symbolic
\textcite{chew2002spiral} measured the distance from a
sequence of pitches to a key using the \emph{spiral array}.
The succession of keys was then modeled as a sequence of
\emph{boundaries} dividing the score in different key areas.

% Audio
\textcite{chai2005detection} designed a model based on a
Hidden Markov Model (\gls{hmm}) to detect changes of key.
They described the term \emph{modulation} as ``the change of
key at some point''. Their model detected first the tonal
center and then the mode of the key.

% Audio
\textcite{catteau2007probabilistic} introduced a model for
scale and chord recognition, assuming that a correspondence
between a major scale and a major key, and between a
harmonic minor scale and a minor key. Their model was based
on the key profiles by \textcite{temperley1999whats} and
Lerdahl's \emph{Tonal Pitch Space}
\parencite{lerdahl2005tonal}.

\textcite{izmirli2007localized} proposed a model to find
local keys in audio signals. As a pre-processing step, the
model adapted the tuning of the signal before computing the
spectral analysis and the chroma features. In a later stage,
non-negative matrix factorization was used to segment
contiguous chroma vectors, identifying potential local keys.
The model identified segments that were candidates for
unique local keys in relation to the neighboring key
centers. The model was evaluated in three datasets: 17 pop
songs with at least one modulation, excerpts from the
beginning of 152 classical music pieces from the Naxos
website, and the examples from the
\textcite{kostka1984tonal} harmony textbook. In the paper,
the results for three different evaluation methods were
provided for each of the datasets. Izmirli also attempted to
disambiguate modulations and tonicizations in the following
manner:

\begin{quote}
    Secondary functions and tonicizations are heard as short
    deviations from the well-grounded key in which they
    appear---although the boundary between modulation and
    tonicization is not clear cut. A modulation
    unambiguously instigates a shift in the key center.
\end{quote}

This work is also, to the best of my knowledge, the first
one where the term \emph{local key} was mentioned in \gls{mir} research.

\textcite{papadopoulos2009local} adopted a similar approach
to \textcite{izmirli2007localized} for audio local-key
estimation. Their model attempts to segment the score based
on the points of modulation. They introduced key
dependencies on the harmonic and metric structures of
global-key-finding methods, in order to convert them into
local-key-finding ones. The method extends on previous work
\parencite{papadopoulos2008simultaneous}, adding a stage of
local key estimation using an \gls{hmm}. The observations of
the model are derived from different key profiles:
\textcite{krumhansl1982tracing},
\textcite{temperley1999whats}, and a flat diatonic key
profile (a uniform distribution for diatonic pitch-classes
and \emph{zero} elsewhere). The system was evaluated in five
piano sonatas by Mozart, where the local keys and chords
were manually annotated. The maximum accuracy achieved was
80.22\%


\textcite{rocher2010concurrent} introduced a model that
provides (chord, key) duples for each audio frame of an
input excerpt. The model is based on a graph and the
\emph{best-path} estimation method. For evaluating key
distances, they used the key profiles by
\textcite{temperley1999whats}. The authors alluded to the
term modulation when discussing their key predictions.

\textcite{mearns2011automatically} used an \gls{hmm} to
estimate modulations over audio transcriptions of Bach
chorales. The \gls{hmm} is trained with chord progressions.
The emission probability distributions are obtained from two
tables with the probabilities of chords existing in a given
key. These tables are based on the work by Schoenberg and
Krumhansl. Applied chords (i.e., tonicizations) are not
described in these charts, therefore, the authors do not
deal with tonicizations.

\textcite{pauwels2014combining} presented a probabilistic
framework for the simultaneous estimation of chords and keys
from audio. They mention the importance of ``integrating
prior musical knowledge'' into a local-key-estimation
system, however, they do not allude to the terms modulation
and tonicization. The same year,
\textcite{weis2014chromabased} proposed an audio scale
estimator. They argue that this estimator can help to
determine the local tonality based on G\'{a}rdonyi's scale
analysis method. They did not use the term tonicization,
however, they discussed ``short-time local modulations'',
which resemble tonicizations.

Machine learning approaches, especially using neural
networks, have recently gained popularity in \gls{mir}
research, including key estimation. Independently,
\textcite{chen2018functional, chen2019harmony} and
\textcite{micchi2020not} designed models that estimate local
keys as well as roman numeral analysis annotations.
Tonicization information is implied by the roman numeral
analysis annotations.

N\'apoles L\'opez et al. introduced a model to find changes
of key (local-key estimation) as well as the main key of a
piece (global-key estimation), using an \gls{hmm}
\textcite{napoleslopez2019keyfinding}. The model is also
capable of working with symbolic and audio data. They do not
allude to the terms modulation or tonicization, always
referring to their predictions as \emph{local keys}.

One of the most recent models for finding changes of key is
by \textcite{feisthauer2020estimating}, which has been
designed to detect modulations in Classical music. It uses
three proximity measures established from pitch
compatibility, tonality anchoring, and tonality proximity.
The model computes the cost of being in a key on a given
beat, and estimates the succession of keys using dynamic
programming techniques.
