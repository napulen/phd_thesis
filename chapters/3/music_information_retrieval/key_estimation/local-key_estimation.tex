% Copyright 2021 Néstor Nápoles López

% This is \refsubsubsec{local-keyestimation}, which
% introduces the local-key estimation.

Changes of key may belong to different categories. In music
theory, terms like \emph{modulation} and \emph{tonicization}
are helpful for interpreting the context of a change of key.
In \gls{mir}, it is common to describe algorithms that model
\emph{changes of key} as \gls{lke} algorithms. The
\emph{local keys} being the predictions that these models
generate. A local key is related to the concepts of
\emph{modulation} and \emph{tonicization}. However, it is
difficult to understand the distinction between these three
terms.

In order to clarify this nomenclature, we investigated the
relationship between the local keys, modulations, and
tonicizations of the same musical fragment
\parencite{napoleslopez2020local}. From this work, I take
the definitions of local keys, modulations, and
tonicizations, which might clarify what exactly an \gls{lke}
algorithm predicts.

\phdparagraph{local keys}
Are the predictions of the musical key provided by a
\gls{lke} algorithm. These predictions are given at a finer
level of granularity than the entire piece (e.g., notes,
onsets, fixed-duration timesteps, audio frames, etc.). In
principle, no music-theoretical meaning is inferred from
them. They may coincide with modulations or tonicizations.

\phdparagraph{modulation}
Is the change from one key to another. We refer to the
initial key as the \emph{departure} key, and the second key
as the \emph{destination} key.

\phdparagraph{tonicization}
Is a brief deviation to a different key, usually with the
intention of emphasizing a certain scale degree or harmony.
The tonicization often returns to the original key briefly
after the deviation.
