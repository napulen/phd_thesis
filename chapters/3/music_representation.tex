% Copyright 2021 Néstor Nápoles López

% This is \refsec{musicrepresentation}, which introduces the
% music representation.

% Taken verbatim from comps Q4

The musical information of interest requires a digital
representation before any \gls{mir} research can be done.
According to \textcite{muller2015music}, there are generally
three types of music representations: digital sheet music
images, symbolic music representations, and audio
representations.

Digital sheet music images consist of the digital version of
printed musical scores. Image representations are useful to
distribute musical scores among musicians, or to print them
on paper. However, accessing the musical content of the
scores (e.g., note names, durations, key signatures, or time
signatures) is quite a difficult task, which usually
involves the development of complex \gls{omr} systems
\parencite{calvozaragoza2020understanding}.

Symbolic music representations also often refer to digital
representations of sheet music. The main difference is that
symbolic representations are encoded in machine-readable
formats, where the musical content is readily available for
computational analysis. Examples of such representations
include \gls{humkern}, Lilypond, \gls{mei}, \gls{midi}, and
\gls{musicxml}.

Audio representations refer to digital representations of
acoustic sound waves. These representations are popular in
digital media, because they more closely resemble the
musical experience that most users want to consume. For
example, as a musical performance streamed using a
music-streaming service. Many \gls{mir} tasks operate on
audio data because of the importance of audio
representations in the daily experience of music.

An \glspl{rna} algorithm is closely related to a few
``satellite'' tasks: \gls{acr}, automatic key estimation,
and \gls{midi} pitch spelling. Over the years, there have
been numerous proposed chord and key estimation algorithms
for symbolic and audio music representations. These
representations encode music in different ways (music
notation and acoustic signals, respectively) and do not
contain the same information nor operate at the same
semantic level. Usually, an algorithm focuses in a single
type of digital music representation but algorithms that
operate in both representations are possible.\footnote{See,
for example, our symbolic-and-audio key-estimation model in
\textcite{napoleslopez2019keyfinding}.}

For the most part, \glspl{rna} models operate with symbolic
music data. Thus, \refsubsec{symbolicmusicformats} presents
the characteristics of some of the most common symbolic
music formats. 

% However, the techniques could potentially be extended to
% the audio domain in the future.
% \refsubsec{symbolicvs.audiorepresentations} discusses the
% implications and existing work on this.
