% Copyright 2021 Néstor Nápoles López

% This is \refsubsubsec{humdrum(**kern)}, which introduces
% the humdrum(**kern).

\phdparagraph{origin of **kern}

The Humdrum toolkit is a family of forty-two data
representations,\footnote{The full list is available in:
\href{https://www.humdrum.org/rep/}{https://www.humdrum.org/rep/}}
from which the \gls{humkern} representation for musical
scores is the most widely used. The toolkit was developed in
the early 1990s and first thoroughly described by
\textcite{huron1994humdrum} in the \emph{Humdrum Toolkit:
Reference Manual}. As it often happens, the same word (in
this case, ``Humdrum''), can refer to multiple things:

\begin{itemize}
    \item Humdrum is a meta-format or family of
    representations, which can encode diverse types of data,
    such as musical scores (\code{**kern}), harmonic
    analysis annotations (\code{**harm}), or lyrics
    (\code{**text}).
    \item Humdrum is the name of a collection of software
    tools, also known as the Humdrum toolkit. These tools
    are originally \code{Perl} scripts, which have now been
    extended with the \code{C/C++} tools \emph{Humdrum
    Extra}\footnote{\href{https://github.com/craigsapp/humextra}{https://github.com/craigsapp/humextra}}
    and
    \emph{Humlib}.\footnote{\href{https://github.com/craigsapp/humlib}{https://github.com/craigsapp/humlib}}
    \item Humdrum is also used to refer to a file containing
    data encoded in a Humdrum representation. For example,
    the Humdrum files in the \emph{KernScores} library
    described by \textcite{sapp2005online}.
\end{itemize}

\phdparagraph{applications of **kern}

The \code{**kern} symbolic music representation is very
compact, and tailored toward music analysis. The content can
be typed by a human encoder and analyzed with the toolkit
scripts. An extensive list of applications is presented by
\textcite{sapp2011computational}.

\phdparagraph{humdrum in roman numeral analysis}

Humdrum is an extremely relevant format for \gls{rna} for
two reasons:

\begin{enumerate}
    \item It already provides support for \gls{rna} using
    the Humdrum(\code{**harm}) syntax, which is also the
    first digital representation for \gls{rna} annotations.
    \item Historically, the first end-to-end system for
    \glspl{rna} (see
    \refsubsec{automaticromannumeralanalysis}) was developed
    to operate with Humdrum(\code{**kern}) scores,
    generating automatic Humdrum(\code{**harm}) annotations
    for them.
\end{enumerate}
