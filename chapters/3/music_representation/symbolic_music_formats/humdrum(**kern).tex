% Copyright 2021 Néstor Nápoles López

% This is \refsubsubsec{humdrum(**kern)}, which introduces
% the humdrum(**kern).

\phdparagraph{origin of **kern}

The Humdrum toolkit is a family of forty-two data representations\footnote{The full list is available in: \href{https://www.humdrum.org/rep/}{https://www.humdrum.org/rep/}}, from
which the \code{**kern} representation for musical scores is
the most widely used. The toolkit was developed in the early
1990s and first thoroughly described in the \emph{Humdrum
Toolkit: Reference Manual} \parencite{huron1994humdrum}.

As it is often the case with technologies, the same word,
``Humdrum'', represents multiple things:

\begin{itemize}
    \item Humdrum is a meta-format or family of
    representations, which can encode diverse types of data,
    such as musical scores (\code{**kern}), harmonic
    analysis annotations (\code{**harm}), or lyrics
    (\code{**text})
    \item Humdrum is the name of a collection of software
    tools, known also as the Humdrum toolkit. These tools
    are originally \code{Perl} scripts, which have now been
    extended with \code{C} tools known as Humdrum
    Extra\footnote{\href{https://github.com/craigsapp/humextra}{https://github.com/craigsapp/humextra}}
    and
    Humlib\footnote{\href{https://github.com/craigsapp/humlib}{https://github.com/craigsapp/humlib}}.
    \item Humdrum is also used to refer to a file containing
    data encoded in a Humdrum representation. For example,
    the Humdrum files in the KernScores library
    \parencite{sapp2005online} \end{itemize}

\phdparagraph{applications of **kern}

The \code{**kern} symbolic music representation is very
compact, and tailored toward music analysis. The content can
be typed by a human encoder and analyzed with the toolkit
scripts. An extensive list of applications is presented by \textcite{sapp2011computational}.

\phdparagraph{humdrum in roman numeral analysis}
