% Copyright 2021 Néstor Nápoles López

% This is \refsubsubsec{mei}, which introduces the mei.

The \gls{mei} is described by \textcite{hankinson2011music}
as a ``community-driven effort to define guidelines for
encoding musical documents in a machine-readable
structure.'' As part of those guidelines, a family of music
formats have been developed over the years. In its flavor
for common Western music notation, the \gls{mei} format
describes musical content using an \gls{xml}-based syntax,
which emphasizes musical semantics.

\phdparagraph{origin of MEI}

The \gls{mei} community (and format) was first presented by
\textcite{roland2002music}, where it proposed to replicate
the efforts of the \gls{tei} in the musical domain. Starting
in 2013, a yearly
conference\footnote{\href{https://music-encoding.org/conference/}{https://music-encoding.org/conference/}}
organized by the \gls{mei} community has perpetuated the
involvement of different people in the development of new
guidelines for music encoding
\parencite{crawford2016review}.

\phdparagraph{applications of MEI}

The \gls{mei} format has gained popularity in digital
collections and libraries. Thanks to tools like Verovio, the
\gls{mei} music engraver by \textcite{pugin2014verovio},
several music libraries have utilized the \gls{mei} format
to encode their collections.

\phdparagraph{MEI in roman numeral analysis}

Although there is no specific \gls{rna} recommendation for \gls{mei} files, the \gls{mei} standard supports a generic \code{<harm>}\footnotelink{https://music-encoding.org/guidelines/v4/elements/harm.html} element, which can be used to annotate Roman numeral annotations.
