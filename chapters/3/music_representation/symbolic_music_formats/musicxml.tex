% Copyright 2021 Néstor Nápoles López

% This is \refsubsubsec{musicxml}, which introduces the
% musicxml.

\phdparagraph{origin of musicxml}

MusicXML is an \gls{xml}-based format for representing
common Western music notation, introduced by
\textcite{good2001musicxml}. It is designed mostly as an
exchange format between music notation software. However, it
is also useful for other applications, such as music
analysis and retrieval. Originally, the MusicXML standard
was heavily inspired by the MuseData and Humdrum (see
\refsubsubsec{humdrum(**kern)}) representations, adapting
them into an \gls{xml} context.


\phdparagraph{applications of musicxml}

Because MusicXML is designed to represent music of the
seventeenth century onwards, it has gained popularity as the
standard exchange format among commercial applications, such
as Dorico\footnotelink{https://www.steinberg.net/dorico/},
Finale\footnotelink{https://www.finalemusic.com/},
MuseScore\footnotelink{https://musescore.org/en},
Sibelius\footnotelink{https://www.avid.com/sibelius}, and
others.


\phdparagraph{musicxml in roman numeral analysis}

As of version 4.0 of the MusicXML format, there is a dedicated \code{<numeral>} element\footnotelink{https://www.w3.org/2021/06/musicxml40/musicxml-reference/elements/numeral/} to encode \gls{rna} annotations.
