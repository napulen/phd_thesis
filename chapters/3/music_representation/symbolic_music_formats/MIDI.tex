% Copyright 2021 Néstor Nápoles López

% This is \refsubsubsec{midi}, which introduces the midi.

\phdparagraph{origin of MIDI}

The \gls{midi} format is a specification to transmit musical
information among hardware devices and software
applications. It was originally presented as the
\emph{Universal Synthesizer Interface} by
\textcite{smith1981usi}. The specification was designed to
interconnect different synthesizers, facilitating
compatibility among manufacturers and increasing sales. By
the end of 1982, several other manufacturers joined the
initiative, and the specification was renamed to \gls{midi}
\parencite{moog1986midi}.

\phdparagraph{applications of MIDI}

\gls{midi} is an ubiquitous format for digital musical
instruments. It is mostly designed to capture a musical
performance, for example, transmitting the pressed keys (and
the velocity at which they were pressed) of a keyboard
controller into a \gls{daw}.

Because it is designed for musical performance, it can be
generated in real-time and transmitted across hardware and
software applications. For the same reason, it is also a
very limited format, generally unable to capture music
notation information, such as ties, note durations, pitch
spelling, or phrasing. This limitation is usually important
in automatic analysis models that rely on more information
than pitch class and octave.


\phdparagraph{MIDI in roman numeral analysis}

Compared to other formats, such as Humdrum, MusicXML, or \gls{mei}, \gls{midi} is a relatively inconvenient format for any \gls{rna} workflow. For example, as an input, it does not provide reliable pitch spelling information, which is crucial in recent \glspl{rna} models.
