% Copyright 2021 Néstor Nápoles López

% This is \refsubsec{symbolicmusicformats}, which introduces
% the symbolic music formats.

\textcite{muller2015music} defines symbolic music
representations as:

\begin{quote}
[Symbolic representations] refer to any machine-readable
data format that explicitly represents musical entities.
These musical entities may range from timed note events, as
is the case of MIDI files, to graphical shapes with attached
musical meaning, as is the case of music engraving systems.
\end{quote}

% TODO: Review Selfridge-Field's Beyond MIDI definition.
% Might be better.

The musical entities often correspond to notes, rests, time
signatures, beams, accidentals, slurs, and other musical
symbols. For the same reason that ``musical symbols'' can be
as diverse as timed note events or graphical shapes with
musical meaning, there is also a wide range of symbolic
music formats. Some formats focus more on the engraving
aspect of music (e.g., Lilypond or MEI), whereas others
focus more on the performative aspect of music (e.g., MIDI).

Depending on the MIR problem of interest, some of these
symbols might be more useful than others. For example, most
MIR systems for chord and key recognition require pitches
and durations but not beams or slurs. Luckily, most symbolic
music formats are able to encode the musical symbols that
are useful in automatic Roman numeral analysis systems.
There are important differences, nevertheless.
