% Copyright 2021 Néstor Nápoles López

% This is \refsubsec{feedforwardnetworks}, which introduces
% the feed forward networks.

An \gls{ann} is a machine learning algorithm that models
arbitrary functions by automatically learning \emph{weights}
(also known as \emph{parameters}) connecting the different
nodes of the neural network. Generally, a nonlinear
activation function is applied to such weights, introducing
a nonlinear behavior in the neural network that allows it to
learn functions of higher complexity, which a linear model
could not possibly learn. The learning of the weights is not
achieved by programming task-specific rules but instead, by
identifying simple characteristics of the training examples
and extending them into more complex, more abstract
characteristics through the \emph{backpropagation}
algorithm. The process of decomposing an example into a
combination of simpler features is known as
\emph{representation learning}, and it is one of the main
ideas that differentiate an \gls{ann} from other classes of
machine learning.

The study of \glspl{ann} started around the 1940s, and it
has been known through different names throughout the years
\parencite{goodfellow2016deep}.

The research on \glspl{ann} can be traced back to the 1940s,
when a bio-inspired \emph{neuron} model was introduced
\parencite{mcculloch1943logical}. This neuron allowed to
model very simple functions by manually setting the weights
that connected the input into the neuron. This idea was
later extended to propose the Perceptron
\parencite{rosenblatt1958perceptron} and Adaline
\parencite{widrow1960adaptive} models, which were able to
automatically derivate such weights from the data. Although
these models showed promise, their popularity decreased
significantly when it was demonstrated that they could not
learn relatively simple functions, like the \emph{XOR}
function \parencite{minsky1972perceptrons}. This wave of
research (1940--1960) is often referred as the
\emph{cybernetics} wave of neural networks research
\parencite{goodfellow2016deep}.

Following the wave of cybernetics, another wave extended
around 1980--1990, colloquially known as
\emph{connectionism}. During the work of the
\emph{connectionists},\footnote{Scientists of this time
period and research field} the research community benefited
from the development of the current form of the
\emph{backpropagation} algorithm
\parencite{rumelhart1988learning}. The backpropagation
algorithm became (and remains) an elemental process in the
training of neural networks, which allows to propagate the
error throughout the network by making use of the
\emph{chain rule}. Finding the derivatives of each parameter
in the network, the values of such parameters can be updated
in the ``right direction'' (against the gradient) to
decrease the error in the next batch of training examples.
This facilitates the automatic training of large and
complicated neural networks, with a variety of layers,
neurons, and nonlinear activation functions. Even though
this and other improvements made neural networks a promising
area of research, they were still very difficult to train in
practice (partly due to the difficulty of finding a good
initialization of the weights) and were typically
outperformed by domain-knowledge techniques. This resulted
in many scientists losing interest in the technology for a
while.

Finally, a third wave of research started around 2006, when
new methods for training neural networks were introduced
\parencite{hinton2006fast}. These new methods not only
facilitated the training of neural networks but the training
of much larger neural networks. The interest in
such larger architectures extended, and in a historical
evaluation of the \emph{ImageNet} dataset
\parencite{deng2009imagenet}, in 2012, a neural network
widely outperformed the most sophisticated methods of
computer vision \parencite{krizhevsky2012imagenet}. This had
an important effect in the way that neural networks were
perceived by the research community and motivated their
application into different problems and fields of study. We
know this last wave of research as \emph{deep learning}, and
it is currently an active and growing wave of research
across many fields. Around this umbrella term of \emph{deep
learning}, many state-of-the-art machine learning techniques
have been developed and continue to be improved.

\guide{Deep learning and \gls{mir}} After the growing
interest for neural networks in the wave of deep learning
research, many new models, architectures, and applications
have been proposed and put into practice in recent years.
%achieving good results and generating subfields of research
%within the umbrella term of deep learning.
Among the most important innovations to the original neural
network architectures, we can consider \glspl{cnn},
\glspl{rnn}, and Transformer networks.

\guide{Other deep learning architectures in \gls{mir}}
Throughout the years, the solutions presented in the
research field of deep learning have been applied to
multiple \gls{mir} tasks. The most popular ones have already
been discussed, however, other applications include the
historical \gls{som} in the early 2000s
\parencite{kiernan2000scorebased, harford2003automatic},
\glspl{dbn} \parencite{hamel2010learning,
schmidt2013learning, chacon2014developing,
raczynski2010multiple, battenberg2012analyzing,
herwaarden2014predicting, zhou2015chord}, and deep
feedforward networks \parencite{cherla2014multiple,
liang2015contentaware, dawson2018keyfinding, valk2018deep}.
