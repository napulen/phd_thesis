% Copyright 2021 Néstor Nápoles López

% This is \refsubsec{recurrentneuralnetworks}, which
% introduces the recurrent neural networks.

\glspl{rnn} are a type of neural networks designed for
dealing with sequential data. Unlike most other neural
network architectures, \glspl{rnn} do not assume that inputs
are \emph{independent} from each other and, therefore, they
update their parameters considering not only the current
input to the network but also the previous inputs processed
by the network.

\glspl{rnn} were introduced after the backpropagation
algorithm \parencite{rumelhart1988learning} was extended
into the \gls{bptt} algorithm, around 1988
\parencite{werbos1988generalization,
werbos1990backpropagation}. Nevertheless, the difficulty of
training such \gls{rnn} architectures made them unfeasible
in practical applications before the invention of the Long
Short-Term Memory (LSTM) architecture
\parencite{hochreiter1997long}. They were popularized during
the \emph{deep learning} wave of research and, since then,
achieved state-of-the-art performance in many tasks
involving sequential data (e.g., speech recognition, natural
language processing, music).

Throughout the years, different strategies have been
proposed to design \glspl{rnn}, for example, connecting the
output of one time step into the next time step (Jordan
RNN), connecting the hidden state of one time step to the
next (Elman RNN), and training the network in both
directions \parencite{schuster1997bidirectional}.

It is also not uncommon to see several of these techniques
combined in a single architecture, for example, a
\gls{cblstm}.

In \gls{mir}, \glspl{rnn} and hybrid \gls{crnn} models have
been quite popular in numerous tasks, for example, onset
detection \parencite{eyben2010universal}, chord recognition
\parencite{boulangerlewandowski2013audio, sigtia2016endend,
sears2018evaluating}, voice separation
\parencite{huang2014singingvoice}, music transcription
\parencite{sigtia2014rnnbased}, tempo estimation
\parencite{bock2015accurate}, beat and downbeat tracking
\parencite{bock2016joint, krebs2016downbeat}, music
generation \parencite{liu2016predicting, liang2017automatic,
lim2017chord}, music transcription
\parencite{rigaud2016singing, sigtia2016endend,
southall2016automatic, vogl2016recurrent,
southall2017automatic, vogl2017drum, basaran2018main},
\gls{omr} \parencite{calvozaragoza2017onestep,
wel2017optical, calvozaragoza2018cameraprimus}, sequence
modelling \parencite{ycart2017study}, mood detection
\parencite{delbouys2018music}, and instrument recognition
\parencite{gururani2018instrument}.



% Recurrent Neural Networks (RNNs) are a special type of
% ANNs where the inputs are not considered to be independent
% but part of a sequence. This distinction allows the
% network to model time-varying processes and tasks that
% require sequence-to-sequence models (e.g., language,
% music, and weather). An additional benefit of RNN
% archictures is that, unlike other types of deep learning
% networks such as Convolutional Neural Networks (CNNs),
% RNNs can process inputs of arbitrary length. This has made
% CNNs the most robust choice for fixed-size grid-like data
% (e.g., images) and RNNs a popular choice for tasks where
% the length of the input is unknown.

% Although modern RNNs have achieved state-of-the-art
% accuracy in many tasks that involve sequential inputs,
% this was not easily achieved and required many years of
% innovations. For example, RNNs were famously unstable and
% difficult to train.

% \guide{Training RNNs}

% Over the years, RNNs have gained a reputation as
% architectures that are difficult to train
% \parencite{pascanu2013difficulty}, given that many
% attempts of training such models have failed. For example,
% in 1994, experiments showed that as the span of
% dependencies that need to be captured by the RNN
% increases, the probability of successfully training the
% network via the Stochastic Gradient Descent (SGD)
% optimizer rapidly reaches 0 for sequences of 10 or 20 time
% steps \parencite{bengio1994learning}.

% The main difficulty of training an RNN is something known
% as the \emph{vanishing} and \emph{exploding} gradients.
% When the gradients vanish---something very common---the
% network is unable to learn long-term dependencies and can
% only model the dependencies of inputs that were provided a
% few time steps in the past (rather than modeling the
% entire sequence, which is usually the goal). When the
% gradients explode---something less common but more
% damaging---the convergence of the training is compromised
% because the large value of the gradient occludes the
% search of a local minimum by the optimizer algorithm.

% One of the reasons why RNNs are more prone to vanishing
% and exploding gradients is their heavy reliability on
% \emph{parameter sharing}. Other architectures (e.g. CNNs)
% also share parameters between different neurons in order
% to reduce the number of parameters and, hence, the effort
% of training the network. RNNs, however, make a heavier use
% of parameter sharing, as their input is usually a
% fixed-size vector designed to receive the time steps of
% the input sequence, one at a time. This design decision
% implies two things: 1) the parameters of the network are
% shared for every input of the sequence, 2) the parameters
% need to be updated for every time step of every input
% sequence.

% The second of these design decisions is what mostly
% contributes to the vanishing and exploding gradients. In
% other architectures, it is expected that the parameters
% will be updated once per training example, while in RNNs
% the parameters are updated several times per training
% example.

% When the RNN is updating its shared parameters, it is very
% easy that these parameters grow out of control (explode)
% or become close to zero (vanishing), similar to how a
% scalar number would explode or vanish when raised by a
% large exponent (e.g., $10^{1000}$ or $0.5^{1000}$).

% By making use of modern solutions, however, RNNs have been
% able to successfully learn long-term dependencies (in the
% order of hundreds or thousands of time steps
% \parencite{hochreiter1997long}), making them practical for
% many tasks. There have been many proposed solutions
% \parencite{elhihi1995hierarchical, yildiz2012revisiting,
% jaeger2012long}, however, the most successful ones are the
% \emph{gated RNNs}. Particularly, the Long Short-Term
% Memory (LSTM) and Gated Recurrent Unit (GRU)
% architectures.
