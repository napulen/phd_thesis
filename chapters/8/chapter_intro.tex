% Copyright 2022 Néstor Nápoles López

This annex chapter presents a few technical ideas about
\gls{rna} that are relevant in various parts of the
dissertation (e.g., Chapters 4--6). Thus, it made sense to
make it a self-contained chapter that could be referenced
throughout other chapters.

The first sections of the appendix describe the vocabularies
used throughout the project.
\refsec{thevocabularyofromannumerals} commences the
discussion of the method by first defining the vocabulary of
valid Roman numeral labels.\footnote{This vocabulary refers
to the Roman numeral ``numerators'' of an annotation. For
example, the $V$ in $\rn{V}/\rn{ii}$.}
\refsec{thevocabularyofmusicalkeys} introduces the musical
keys can be theoretically recognized by \gls{arna} system.
This vocabulary of keys also has the purpose of constraining
the vocabulary of chords (\gls{pcset}s).
\refsec{thevocabularyofpitch-classsets} introduces the
vocabulary of \gls{pcset}s that exist within the system.
\refsec{analgorithmtoresolveromannumeralsfromapitch-classsetandkey}
introduces an algorithm to resolve a Roman numeral label
from the vocabulary, given a valid \gls{pcset} and key from
the vocabulary. \refsec{analgorithmtoresolvetonicizations}
introduces an algorithm to handle the situations where the
Roman numeral does not exist for a given \gls{pcset} and
key. In that case, a tonicization is forced, which will
result in a valid Roman numeral label tonicizing a scale
degree of the given key.
