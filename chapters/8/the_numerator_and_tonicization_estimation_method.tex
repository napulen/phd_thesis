% Copyright 2022 Néstor Nápoles López

This section describes the \gls{natem} algorithm, which is
useful to retrieve a Roman numeral numerator $\elnum$ and
tonicization scale degree $\elden$ when only a \gls{pcset}
$\elpcset$ and key $\elkey$ are known. Thus, \gls{natem} can
be defined as the function in \refeq{natem}.

\begin{equation}
    \label{eq:natem}
    \gls{natem}(\elpcset, \elkey) = \elnum\text{/}\elden
\end{equation}

The retrieval process consists of two steps: if possible,
acquire the numerator $\elnum$ from the \gls{pcset}
$\elpcset$ and key $\elkey$; if this is not possible
directly, force a tonicization and to retrieve a
tonicization $\elden$ such that $\elden_\elkey \in
\setkey_\elpcset$. Both steps are described below.

