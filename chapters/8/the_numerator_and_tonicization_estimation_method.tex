% Copyright 2022 Néstor Nápoles López

This section describes the \gls{natem} algorithm, which is
useful to retrieve a Roman numeral numerator $\elnum$ and
tonicization $\elden$ from a \gls{pcset} $\elpcset$ and key
$\elkey$, as shown in \refeq{natem}. This is useful when a
machine learning system only generates chord labels and
keys, which is often the case. 

\begin{equation}
    \label{eq:natem}
    \gls{natem}(\elpcset, \elkey) = \elnum\text{/}\elden
\end{equation}

The retrieval process can be summarized in two steps. The
first step is to acquire, if possible, the numerator
$\elnum$ from the given $\elpcset$ and $\elkey$ and assume a
tonicization $\elden = \rnI$ or $\elden = \rni$, depending
on $\elkey$'s mode. If this is not possible directly, the
second, optional step, is to force a tonicization to
retrieve a tonicization $\elden$ such that $\elden_\elkey
\in \setkey_\elpcset$.

