% Copyright 2022 Néstor Nápoles López

One of the problems with the previous representation of 35
features is that the it cannot encode any notes beyond two
flats or two sharps. Additionally, the resulting vector is
almost three times larger than the common representation
based on pitch classes, which has 12 features:\footnote{When
enharmonic equivalence is assumed in a \gls{12tet} system, a
set of 12 pitch classes spans all the note classes of the
Western chromatic scale.}

\begin{equation}
    \{0, 1, 2, 3, 4, 5, 6, 7, 8, 9, 10, 11\}
\end{equation}

The representation, first proposed in
\textcite{napoleslopez2021augmentednet}, reduces some of the
problems with the representation of 35 features. Instead of
encoding \gls{sharp} and \gls{flat} notes explicitly, it
extends the pitch class vector representation, by also
encoding the generic note letter name:

\begin{equation}
    \label{eq:pitch_spelling_19}
    \vocabps_{19} = \{0, 1, 2, 3, 4, 5, 6, 7, 8, 9, 10, 11\} 
    \cup \{C, D, E, F, G, A, B\}
\end{equation}


\begin{equation}
    \label{eq:pitch_spelling_19_magnitude}
    | \vocabps_{19} | = 19
\end{equation}

Note that in this representation, in addition to the pitch
class, a spelled note has a corresponding letter. For
example, C\musSharp{} is represented by the duple $(1,
\text{C})$, where as D\musFlat{} is represented by the same
pitch class but a different generic note letter, $(1,
\text{D})$. D$\musNatural{}$ has the same generic note
letter but a different pitch class, $(2, \text{D})$, and so
on. Using this representation, a spelled note, beyond two
flats and two sharps can be encoded with a 19-feature
vector. Notice, however, that this vector is a two-hot
encoding representation.
