% Copyright 2022 Néstor Nápoles López

A Roman numeral numerator $\elnum$ is the only compulsory
symbol in any \gls{rna} label. The
other---optional---symbols being a key $\elkey$ and a
tonicization $\elden$.

A Roman numeral numerator always indicates a chord relative
to a key. When a tonicized key $\elden$ is provided, the
numerator $\elnum$ is relative to this key, when $\elden$ is
omitted, the numerator $\elnum$ is relative to the key
$\elkey$.

There are 31 valid Roman numeral labels. These are indicated
in \reftab{rn_vocabulary}.

\phdtable[Vocabulary of valid Roman numeral labels for this
\gls{rna} method]{rn_vocabulary}


\begin{equation}
    \setnum_{C} = \{ \rnCad, \: \rnV, \: \rnviio,
    \rnVsev, \rnN, \: \rnIt, \: \rnFr, \: \rnGer \: \}
\end{equation}

\begin{equation}
    \setnum_{M} = \setnum_{C} \cup \{ \: \rnI, \: \rnii, \: 
    \rniii, \: \rnIV, \: \rnvi, \rnIsev, \: \rniisev, \: 
    \rniiisev, \: \rnIVsev, \: \rnvisev, \: \rnviihosev, \rnVaug \}
\end{equation}

\begin{equation}
    \setnum_{m} = \setnum_{C} \cup  \{ \rni, \: \rniio, \: 
    \rnIIIaug, \: \rniv, \: \rnVI, \rnisev, \: \rniihosev, \: 
    \rnIIIaugsev, \: \rnivsev, \: \rnVIsev, \: \rnviiosev \}
\end{equation}

\begin{equation}
    \setnum = \setnum_{M} \cup \setnum_{m}
\end{equation}

\begin{equation}
    | \setnum | = 31
\end{equation}

                

The Roman numeral vocabulary consists mainly of diatonic
harmonies that can be constructed from a major or harmonic
minor scale. A few chords are exceptions to this rule, \:
which I indicated as ``special chords.''

Each of the chords in the vocabulary is explained below.


