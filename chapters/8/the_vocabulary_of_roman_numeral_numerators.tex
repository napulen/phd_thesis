% Copyright 2022 Néstor Nápoles López

If we consider a Roman numeral label as a ``fraction-like''
notation of the form 

\begin{equation}
    \frac{\eta}{\delta}
\end{equation}

then the numerator $\eta$ represents a chord, and the
denominator $\delta$ represents a tonicized key. For
instance, in the annotation
$\keyC$:$\rnviihosev{}$/$\rn{V}$, the $\delta = \rnV$ is the
key of the dominant, $\keyG$, and $\rho=\rnviihosev$ is a
chord of the tonicized key, an F$\musSharp$ half-diminished
seventh chord.

The mode of the tonicized key depends on the case of the
denominator's Roman numeral. For example, if the annotation
was $\keyc$:$\rnviiosev{}$/$\rnii$ instead, the deonminator
$\rnii$ is the key of $\keyd$. In this case, the numerator
$\rnviiosev$ is a chord of $\keyd$, a C$\musSharp$
diminished seventh chord.

There are 31 valid Roman numeral labels. These are indicated
in \reftab{rn_vocabulary}.

\phdtable[Vocabulary of valid Roman numeral labels for this
\gls{rna} method]{rn_vocabulary}


\begin{equation}
    \vocabnum_{C} = \{ \rnCad, \: \rnV, \: \rnviio,
    \rnVsev, \rnN, \: \rnIt, \: \rnFr, \: \rnGer \: \}
\end{equation}

\begin{equation}
    \vocabnum_{M} = \vocabnum_{C} \cup \{ \: \rnI, \: \rnii, \: 
    \rniii, \: \rnIV, \: \rnvi, \rnIsev, \: \rniisev, \: 
    \rniiisev, \: \rnIVsev, \: \rnvisev, \: \rnviihosev, \rnVaug \}
\end{equation}

\begin{equation}
    \vocabnum_{m} = \vocabnum_{C} \cup  \{ \rni, \: \rniio, \: 
    \rnIIIaug, \: \rniv, \: \rnVI, \rnisev, \: \rniihosev, \: 
    \rnIIIaugsev, \: \rnivsev, \: \rnVIsev, \: \rnviiosev \}
\end{equation}

\begin{equation}
    \vocabnum = \vocabnum_{M} \cup \vocabnum_{m}
\end{equation}

\begin{equation}
    | \vocabnum | = 31
\end{equation}

                

The Roman numeral vocabulary consists mainly of diatonic
harmonies that can be constructed from a major or harmonic
minor scale. A few chords are exceptions to this rule, \:
which I indicated as ``special chords.''

Each of the chords in the vocabulary is explained below.


