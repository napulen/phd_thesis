% Copyright 2022 Néstor Nápoles López

After a key $\elkey$ has been established in at least one
annotation, a Roman numeral numerator $\elnum$ is the only
compulsory symbol in any subsequent \gls{rna} label. The
other---optional---symbols being a new key $\elkey$ a
tonicization $\elden$, and an inversion $\elinv$.

A Roman numeral numerator always indicates a chord relative
to a key. When a tonicized key $\elden$ is provided, the
numerator $\elnum$ is relative to this key. However, when
$\elden$ is omitted, the numerator $\elnum$ is relative to
the key $\elkey$, which should have been indicated in at
least the first \gls{rna} label of a piece. One additional
simplification is to assume that if a tonicization $\elden$
is not indicated for a particular \gls{rna} label, then
$\elden = \elkey$. Following this assumption, then the chord
indicated by $\elnum$ is always relative to $\elden$. The
vocabulary of chords allowed for $\elnum$ depends on the
mode of the key $\elden$.

Some of the numerators are common to both major and minor
modes. These are shown in \refeq{common_numerators}. 

\begin{equation}
    \label{eq:common_numerators}
    \begin{split}
    \setnum_{M/m} &= \{ \rnCad, \: \rnV, \: \rnviio,
    \rnVsev, \rnN, \: \rnIt, \: \rnFr, \: \rnGer \: \} \\
    | \setnum_{M/m} | &= 8
    \end{split}
\end{equation}

The set of numerators for major and minor keys consist of
the union between the numerators that are common to both
modes $\setnum_{M/m}$ and the numerators that are exclusive
of a given mode. The set of numerators for major keys,
$\setnum_{M}$, is shown in \refeq{major_numerators}, and the
set for minor keys, $\setnum_{m}$ is shown in
\refeq{minor_numerators}. If $\elden$ is a major key, there
are 20 possible values for $\elnum$; if $\elden$ is a minor
key, there are 19 possible values for $\elnum$.

\begin{equation}
    \label{eq:major_numerators}
    \begin{split}
    \setnum_{M} &= \setnum_{C} \cup \{ \: \rnI, \: \rnii, \: 
    \rniii, \: \rnIV, \: \rnvi, \rnIsev, \: \rniisev, \: 
    \rniiisev, \: \rnIVsev, \: \rnvisev, \: \rnviihosev, \rnVaug \} \\
    | \setnum_{M} | &= 20
    \end{split}
\end{equation}

\begin{equation}
    \label{eq:minor_numerators}
    \begin{split}
    \setnum_{m} &= \setnum_{C} \cup  \{ \rni, \: \rniio, \: 
    \rnIIIaug, \: \rniv, \: \rnVI, \rnisev, \: \rniihosev, \: 
    \rnIIIaugsev, \: \rnivsev, \: \rnVIsev, \: \rnviiosev \} \\
    | \setnum_{m} | &= 19
    \end{split}
\end{equation}

The complete vocabulary of Roman numeral numerators
$\setnum$, shown in \refeq{all_numerators}, comprises all
numerators for major and minor modes, including the set of
numerators shared by both modes. Thus, $\setnum$ indicates
the set of Roman numeral numerators where $\forall \: \elnum
\in \setnum$. Notice that the cardinality of the vocabulary
is of 31 Roman numeral numerators, because the duplicate
numerators ($\setnum_{M/m}$) are only counted once.

\begin{equation}
    \label{eq:all_numerators}
    \begin{split}
    \setnum &= \setnum_{M} \cup \setnum_{m} \\
    | \setnum | &= 31
    \end{split}
\end{equation}

Regarding the meaning of the 31 numerators in the
vocabulary, these correspond mainly to diatonic harmonies
that can be constructed from a major or harmonic minor
scale. A few chords are exceptions to this rule, \: which
are indicated as ``special chords.'' \reftab{rn_vocabulary}
summarizes the 31 numerators and their chord context. An
explanation of all the chords is provided below.

\phdtable[Vocabulary $\setnum$ of valid Roman numeral
numerators]{rn_vocabulary}


