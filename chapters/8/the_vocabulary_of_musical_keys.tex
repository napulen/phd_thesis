% Copyright 2022 Néstor Nápoles López

If all enharmonic major-and-minor keys were collapsed into
the same key class, then there would be a set of 24 keys.
However, if two enharmonic keys are considered different
keys, then the set of keys becomes an infinite set. In the
first case, the keys can be arranged in a circular
structure, such as the circle of fifths, because eventually
the classes will repeat. In the second case, the keys can be
arranged in a \emph{line-of-fifths}
\parencite{temperley2000line}, as shown in
\reftab{line_of_fifths_keys}.

\phdtablefit[A line of fifths for major (top) and minor
(bottom) keys]{line_of_fifths_keys}

Designing a computational system that distinguishes between
enharmonic keys is easier if the vocabulary of keys is
bounded. Furthermore, many of the keys in the infinite exist
in theory, but would rarely (or never) occur in practice.
For example, the key of $C\musDoubleSharp$ major.

In this dissertation, I bounded the vocabulary of
major-and-minor keys to be finite. The set of available keys
was determined in a data-driven way. By inspecting the
examples provided in the aggregated dataset of \gls{rna}
annotations (see \refsec{aggregatingallavailabledatasets}),
it was determined that most annotations lie within the range
of keys between $\keyBbb$ and $\keyDs$ for the major keys,
and $\keygb$ and $\keybs$ for the minor keys. Thus, the
vocabulary of keys considered in this dissertation spans all
the keys in that range, as shown in \refeq{keys}. This
results in 38 keys.

\begin{equation}
    \label{eq:keys}
    \begin{split}
    \setkey &= \{ \keyBbb, \keygb, \keyFb, \keydb, \dotsc,
    \keyGs, \keyes, \keyDs, \keybs \}\\
    | \setkey | &= 38
    \end{split}
\end{equation}
