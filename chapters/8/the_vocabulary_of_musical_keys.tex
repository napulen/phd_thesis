% Copyright 2022 Néstor Nápoles López

If all enharmonic major-and-minor keys were collapsed into
the same key class, then there would be a set of 24 keys.
However, if two enharmonic keys are considered different
keys, then the set of keys becomes an infinite set. For the
first case, the keys can be arranged in a circular
structure, such as the circle of fifths, because eventually
the classes will repeat. For the second case, the keys can
be arranged in a \emph{line-of-fifths}
\parencite{temperley2000line}.

For a computational system, distinguishing between
enharmonics is easier if the vocabulary of keys is bounded.
Furthermore, many of those keys exist in theory, but would
rarely (or never) occur in practice. For example, the key of
$C\musDoubleSharp$ major.

In the proposed system, the vocabulary of keys was bounded.
The number of available keys was determined in a data-driven
way. By inspecting the examples provided in the aggregated
dataset of \gls{rna} annotations (see
\refsec{aggregatingallavailabledatasets}), it was determined
that most annotations lie within the range of keys between
$B\musDoubleFlat$ and $D\musSharp$ for the major keys, and
$g\musFlat$ and $b\musSharp$ for the minor keys. Thus, the
vocabulary of keys considered in this \thesisdiss{} spans
all the keys in that range. This results in 38 keys.

\begin{equation}
    \setkey = \{ \keyBbb, \keygb, \keyFb, \keydb, \dotsc,
    \keyGs, \keyes, \keyDs, \keybs \}
\end{equation}

\begin{equation}
    | \setkey | = 38
\end{equation}
