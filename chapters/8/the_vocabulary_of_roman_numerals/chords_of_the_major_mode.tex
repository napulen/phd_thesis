% Copyright 2022 Néstor Nápoles López

Except for $\rn{V}$, $\rn{vii}\rndim$, and a few other
chords, the \gls{rna} numerators constructed from the major
mode are exclusive of a major key. That is, they do not
exist in a minor key. Those chords are explained in this
section.

\phdparagraph{major-mode triads}

There are 5 Roman numerals generated from the diatonic
triads of the major mode: $\rnI$, $\rnii$, $\rniii$,
$\rnIV$, and $\rnvi$.


\phdparagraph{major-mode seventh chords}

There are 6 Roman numerals generated from the diatonic
seventh chords of the major mode: $\rnIsev$, $\rniisev$,
$\rniiisev$, $\rnIVsev$, $\rnvisev$, and $\rnviihosev$.


\phdparagraph{major-mode augmented dominant}

A chord often found in the dataset is an augmented dominant
chord. This chord is also common in the minor mode, however,
it is an enharmonic of the $\rnIIIaug$ triad of that mode.
Thus, the \gls{pcset} resolution collides for these two
chords. The way this was addressed in the system was to
consider $\rnVaug$ as a chord exclusive of the major mode and
$\rnIIIaug$ as a chord exclusive of the minor mode. When the
system is trying to resolve a \gls{pcset} that conforms to
an augmented triad, it will be resolved as either $\rnIIIaug$
or $\rnVaug$ depending on the resolution of the key. 
