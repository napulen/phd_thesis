% Copyright 2022 Néstor Nápoles López

One of the main uses of the \gls{rna} notation is to
indicate chords that occur in special tonal contexts. The
chords considered here are examples of those special
contexts.

\phdparagraph{neapolitan chord}

A \gls{neapolitan} triad is the major triad that forms from
the flatted second degree in either mode,
$\rnformat{\musFlat}\rn{II}$. It is often used to substitute
a ``subdominant'' chord, for example, the $\rnIV$ of a major
key. It is also common to find it in first inversion, with
the third as the bass.

\phdparagraph{augmented sixth chords}


