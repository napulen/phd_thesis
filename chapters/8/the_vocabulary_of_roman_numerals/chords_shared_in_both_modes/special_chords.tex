% Copyright 2022 Néstor Nápoles López

One of the main uses of the \gls{rna} notation is to
indicate chords that occur in special tonal contexts. The
chords considered here are examples of those special
contexts.

\phdparagraph{neapolitan chord}

A \gls{neapolitan} triad is the major triad that forms from
the flatted second degree in either mode,
$\rnformat{\musFlat}\rn{II}$. It is often used to substitute
a ``subdominant'' chord, for example, the $\rnIV$ of a major
key. It is also common to find it in first inversion, with
the third as the bass.

\phdparagraph{augmented sixth chords}

There are three kinds of \gls{augsix} chords: $\rnIt$,
$\rnFr$, and $\rnGer$.

These chords share commonalities. For example, they all
feature an interval of an augmented sixth between the sixth
degree (minor mode) and the raised fourth degree. The main
difference between the three types of chords lies in the
additional pitches that complete their configurations. An
$\rnIt$ chord is a triad, adding the root of the key as part
of the chord. For example, in the key of $c$, the $\rnIt$
chord is made of the notes $\{F\musSharp A\musFlat C\}$.
