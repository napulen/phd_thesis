% Copyright 2022 Néstor Nápoles López

If we consider a Roman numeral label to be a
``fraction-like'' expression, such as the one shown in
\refeq{rn_expression}, then the symbol $\elkey$ preceding
the colon represents a key, the numerator $\elnum$
represents a chord, and the denominator $\elden$ represents
a tonicized key.

\begin{equation}
    \label{eq:rn_expression}
    \elkey : \elnum \: / \: \elden
\end{equation}

Thus, in the annotation shown in \refeq{rn_example1}, the
preceding symbol $\elkey=\rn{C}$ indicates the key of
$\keyC$; the denominator $\elden = \rnV$ indicates a
tonicization of $\elkey$'s dominant, $\keyG$; and
$\elnum=\rnviihosev$ indicates a chord relative to the
tonicized key $\elden$, an F$\musSharp$ half-diminished
seventh chord.

\begin{equation}
    \label{eq:rn_example1}
    \rnwkeyton{C}{\rnviihosev{}}{\rnV{}}
\end{equation}


The mode (major or minor) implied by the keys $\elkey$ and
$\elden$ is indicated with a case-sensitive notation. For
example, in the annotation shown in \refeq{rn_example2},

\begin{equation}
    \label{eq:rn_example2}
    \rnwkeyton{C}{\rnviiosev}{ii}
\end{equation}

the denominator $\elden=\rnii$ indicates a key of $\keyd$.
In this case, the numerator $\elnum = \rnviiosev$ indicates
a chord that is relative to the key $\elden$, namely, a
C$\musSharp$ diminished seventh chord.

Tonicizations are optional and do not occur often. If there
is no symbol $\elden$ in the annotation, it can be assumed
to be the same key as $\elkey$. For example, in the
annotation

\begin{equation}
    \rn{G}:\rnVsev
\end{equation}

$\elden = \elkey = \keyG$. The ``modulation'' key\footnote{I
like to refer to key $\elkey$ as the ``modulation'' key.
However, it is also common to refer to it as the ``local
key'' among the \gls{mir} literature
\parencite{napoleslopez2020local}} $\elkey$ is mandatory for
the numerator $\elnum$ to have any meaning, but it can be
omitted in subsequent annotations if it remains unchanged.
For example, in two subsequent \gls{rna} annotations

\begin{equation}
    \rn{G}:\rnVsev \;\;\;\; \rnI
\end{equation}

the symbols are $\elkey=\keyG$, $\elden=\keyG$, and
$\elnum=\rnVsev$, for the first annotation, and
$\elkey=\keyG$, $\elden=\keyG$, and $\elnum=\rnI$, for the
second annotation.

In most instances, a Roman numeral label will be of the form
$\elnum$, with $\elkey$ symbols appearing either at the
beginning of the piece or when a modulation occurs, and
$\elden$ symbols appearing when there is a tonicized key
$\elden \neq \elkey$.

These general rules allow the interpretation of virtually
any \gls{rna} label. However, one of the confusing aspects
of the notation is perhaps the vocabulary of $\elnum$,
$\elkey$, and $\elden$ symbols that one can encounter in the
notation, and the meaning of those when interpreted as
chords.

The \refappendix{amethodforsystematicromannnumeralanalysis}
indicates the vocabularies of musical symbols $\setnum$,
$\setkey$, and $\setden$, among others. Such that

\begin{equation}
    \begin{split}
    \elnum \in \setnum \\
    \elkey \in \setkey \\
    \elden \in \setden
    \end{split}
\end{equation}

This complementary information facilitates the
interpretation of \gls{rna} labels in the context of this
dissertation.
