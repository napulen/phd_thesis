% Copyright 2022 Néstor Nápoles López

If we consider a Roman numeral label to be a
``fraction-like'' expression of the form

\begin{equation}
    \kappa : \eta \: / \: \delta
\end{equation}

then the symbol $\kappa$ preceding the colon represents a
key, the numerator $\eta$ represents a chord, and the
denominator $\delta$ represents a tonicized key. For
instance, in the annotation 

\begin{equation}
    \rn{C}:\rnviihosev{}/\rnV
\end{equation}

the preceding symbol $\kappa=\rn{C}$ is the key of $\keyC$;
the denominator $\delta = \rnV$ is a tonicization of the
dominant of key $\kappa$, $\keyG$; and $\eta=\rnviihosev$ is
a chord of the tonicized key $\delta$, an F$\musSharp$
half-diminished seventh chord.

The mode (major or minor) of the keys implied by $\kappa$
and $\delta$ is indicated by a case-sensitive notation. For
example, if the annotation was instead

\begin{equation}
    \rn{C}:\rnviiosev{}/\rnii
\end{equation}

the denominator $\delta=\rnii$ is the key of $\keyd$. In
this case, the numerator $\rnviiosev$ is a chord of $\keyd$,
a C$\musSharp$ diminished seventh chord.

Tonicizations are optional and infrequent, if there is no
symbol $\delta$ in the annotation, it can be assumed to be
the same key as $k$

\begin{equation}
    \rn{G}:\rnVsev
\end{equation}

in this case, $\delta = \kappa = \keyG$. The ``modulation''
key $\kappa$ is mandatory for the numerator $\eta$ to have
any meaning, but it can be omitted in subsequent annotations
if it remains to be the same. For example, in the subsequent
\gls{rna} annotations

\begin{equation}
    \rn{G}:\rnVsev \;,\; \rnI
\end{equation}

the symbols are $\kappa=\keyG$, $\delta=\keyG$, and
$\eta=\rnVsev$, for the first annotation, and
$\kappa=\keyG$, $\delta=\keyG$, and $\eta=\rnI$, for the
second annotation.

in most instances, a Roman numeral label will be of the form
$\eta$. In this case, the tonicization 
