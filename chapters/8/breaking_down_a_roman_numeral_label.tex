% Copyright 2022 Néstor Nápoles López

If we consider a Roman numeral label to be a
``fraction-like'' expression of the form

\begin{equation}
    \kappa : \eta \: / \: \delta
\end{equation}

then the symbol $\kappa$ preceding the colon represents a
key, the numerator $\eta$ represents a chord, and the
denominator $\delta$ represents a tonicized key. For
instance, in the annotation 

\begin{equation}
    \rn{C}:\rnviihosev{}/\rnV
\end{equation}

the preceding symbol $\kappa=\rn{C}$ indicates the key of
$\keyC$; the denominator $\delta = \rnV$ indicates a
tonicization of the dominant of key $\kappa$, $\keyG$; and
$\eta=\rnviihosev$ indicates a chord of the tonicized key
$\delta$, an F$\musSharp$ half-diminished seventh chord.

The mode (major or minor) of the keys implied by $\kappa$
and $\delta$ is indicated by a case-sensitive notation. For
example, if the annotation was instead

\begin{equation}
    \rn{C}:\rnviiosev{}/\rnii
\end{equation}

the denominator $\delta=\rnii$ indicates a key of $\keyd$.
In this case, the numerator $\rnviiosev$ indicates a chord
that is relative to the key $\delta$, namely, a C$\musSharp$
diminished seventh chord.

Tonicizations are optional and do not occur often. If there
is no symbol $\delta$ in the annotation, it can be assumed
to be the same key as $\kappa$

\begin{equation}
    \rn{G}:\rnVsev
\end{equation}

in this case, $\delta = \kappa = \keyG$. The ``modulation''
key\footnote{I like to refer to key $k$ as the
``modulation'' key. However, it is also common to refer to
it as the ``local key'' among the \gls{mir} literature
\parencite{napoleslopez2020local}} $\kappa$ is mandatory for
the numerator $\eta$ to have any meaning, but it can be
omitted in subsequent annotations if it remains unchanged.
For example, in two subsequent \gls{rna} annotations

\begin{equation}
    \rn{G}:\rnVsev \;\;\;\; \rnI
\end{equation}

the symbols are $\kappa=\keyG$, $\delta=\keyG$, and
$\eta=\rnVsev$, for the first annotation, and
$\kappa=\keyG$, $\delta=\keyG$, and $\eta=\rnI$, for the
second annotation.

In most instances, a Roman numeral label will be of the form
$\eta$, with $\kappa$ symbols appearing either at the
beginning of the piece or when a modulation occurs, and
$\delta$ symbols appearing when there is a tonicized key
$\delta \neq \kappa$.
