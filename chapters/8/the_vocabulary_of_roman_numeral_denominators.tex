% Copyright 2022 Néstor Nápoles López

A denominator $\elden$ in a \gls{rna} label indicates a
tonicized key. The tonicized key is always relative to the
current key $\elkey$, which should be known at any given
moment of the piece.

The tonicized key $\elden$ is indicated with case-sensitive
Roman numerals, where the Roman numeral indicates the scale
degree, and the case of the Roman numeral indicates the
mode. When the scale degree is not a diatonic scale degree
of the key $\elkey$, an accidental can be used to alter the
scale degree. Thus, the vocabulary $\setden$ that satisfies
$\forall \; \elden \in \setden$ comprises all case-sensitive
scale degrees relative to key $\elkey$. This vocabulary is
shown in \refeq{all_denominators}.

\begin{equation}
    \label{eq:all_denominators}
    \setden = \{ \musDoubleFlat, \musFlat, \musNatural, 
    \musSharp, \musDoubleSharp \} \cup \{ \rnI, \rn{II}, \rn{III}, \rnIV, \rnV, \rnVI, \rn{VII} \} \cup \{ \rni, \rnii, \rniii, \rniv, \rn{v}, \rnvi, \rn{vii} \}
\end{equation}
