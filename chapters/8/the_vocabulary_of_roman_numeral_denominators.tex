% Copyright 2022 Néstor Nápoles López

A denominator $\elden$ in a \gls{rna} label indicates a
tonicized scale degree. The tonicized scale degree is always
relative to the current key $\elkey$, which should be known
at any given moment of the piece.

The tonicization $\elden$ is indicated with case-sensitive
Roman numerals, where the Roman numeral indicates the scale
degree, and the case of the Roman numeral indicates the
mode. When the scale degree is not a diatonic scale degree
of the key $\elkey$, an accidental can be used to alter it
accordingly. Thus, the vocabulary $\setden$ that satisfies
$\forall \; \elden \in \setden$ comprises all case-sensitive
scale degrees relative to key $\elkey$. This vocabulary is
shown in \refeq{all_denominators}.

\begin{equation}
    \label{eq:all_denominators}
    \setden = \{ \musDoubleFlat, \musFlat, \musNatural, 
    \musSharp, \musDoubleSharp \} \times \{ \rni, \rnI, \rnii, \rn{II}, \rniii, \rn{III}, \rniv, \rnIV, \rn{v}, \rnV, \rnvi, \rnVI, \rn{vii}, \rn{VII} \}
\end{equation}
