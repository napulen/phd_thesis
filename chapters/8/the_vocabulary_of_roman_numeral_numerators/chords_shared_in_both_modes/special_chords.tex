% Copyright 2022 Néstor Nápoles López

One of the main uses of the \gls{rna} notation is to
indicate chords that occur in special tonal contexts. The
chords considered here are examples of those special
contexts.

\phdparagraph{neapolitan chord}

A \gls{neapolitan} triad is the major triad that forms from
the flatted second degree in either mode,
$\rnformat{\musFlat}\rn{II}$. It is often used to substitute
a ``subdominant'' chord, for example, the $\rnIV$ of a major
key. It is also common to find it in first inversion, with
the third as the bass.

\phdparagraph{augmented sixth chords}

There are three kinds of \gls{augsix} chords: $\rnIt$,
$\rnFr$, and $\rnGer$. These chords share commonalities. For
example, they all feature an interval of an augmented sixth
between the sixth degree (lowered sixth degree, if in major
mode) and the raised fourth degree. The main difference
between the three types of chords lies in the additional
pitches that complete their configurations. An $\rnIt$ chord
is a triad, adding the root of the key as part of the chord.
For example, in the key of $C$, the $\rnIt$ chord is made of
the notes $\{F\musSharp$, $A\musFlat$, $C\}$. Using the same
key as an example, the $\rnFr$ chord includes the second
degree, $\{F\musSharp$, $A\musFlat$, $C$, $D\}$. The
$\rnGer$ chord, instead of the second degree, includes the
third degree (lowered third degree, if in major mode). Thus,
$\{F\musSharp$, $A\musFlat$, $C$, $E\musFlat\}$.

The \gls{pcset} configuration of $\rnIt$ chords is unique.
That is, there is no other Roman numeral numerator $\elnum$
with the same \gls{pcset} $\elpcset$ of an $\rnIt$ chord,
except for the same $\rnIt$ in an enharmonic key. This is
not true for $\rnFr$ and $\rnGer$. A $\rnFr$ chord has the
same \gls{pcset} configuration as another $\rnFr$ in a
different (non-enharmonic) key. A $\rnGer$ chord has the
same \gls{pcset} configuration as a $\rnVsev$ chord in a
different key context. These properties need to be
considered when designing the chord vocabulary, as the
subtle distinction of one chord (\gls{pcset}) in different
keys, results in a very distinct \gls{rna} label.
