% Copyright 2022 Néstor Nápoles López

A chord occasionally found in the dataset is an augmented
dominant chord. This chord is also common in the minor mode,
however, it is an enharmonic of the $\rnIIIaug$ triad of
that mode. In order for the algorithm described in
\refsec{analgorithmtoresolveromannumeralsfromapitch-classsetandkey}
to work, it is required that a given \gls{pcset} $\elpcset$
has only one Roman numeral numerator $\elnum$ association in
a given key, a condition that would not be met for
$\rnIIIaug$ and $\rnVaug$ coexisting in a minor key. One way
to address this is to consider $\rnVaug$ a chord exclusive
of the major mode and $\rnIIIaug$ a chord exclusive of the
minor mode. When the method proposed in
\refsec{analgorithmtoresolveromannumeralsfromapitch-classsetandkey}
is trying to resolve a \gls{pcset} $\elpcset$ that conforms
to an augmented triad, it will be resolved as either
$\rnIIIaug$ or $\rnVaug$ depending on the mode of the key.
