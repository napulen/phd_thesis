% Copyright 2022 Néstor Nápoles López

A Roman numeral numerator $\elnum \in \setnum$ is said to be
in ``root position'' if the root of its chord is also acting
as the bass. If one of the other chord tones is taking the
role of the bass, the Roman numeral numerator is said to be
in either a first, second, or third
inversion.\footnote{Third inversions only occur if $\elnum$
indicates a seventh chord, as a third inversion requires a
chord with at least four notes.} Thus, the vocabulary of
inversions $\setinv$ for a numerator $\elnum$ comprises the
root position and three possible inversions, as shown in
\refeq{inversion_vocabulary}.

\begin{equation}
    \label{eq:inversion_vocabulary}
    \setinv = \{ 0, 1, 2, 3 \}
\end{equation}

However, the annotation of the inversion in a \gls{rna}
label is not written with the inversion number, but with a
stack of Arabic numerals.\footnote{See
\refsubsec{chordinversionsandfiguredbass} for further
discussion on why this is the case.} The precise notation
depends on whether the numerator $\elnum$ is a triad or a
seventh chord, and it is summarized in \reftab{inversions}.

\phdtable[Notation for the vocabulary $\setinv$ of
inversions depending on the characteristics of
$\elnum$]{inversions}
