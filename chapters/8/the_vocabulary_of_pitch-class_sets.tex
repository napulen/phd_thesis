% Copyright 2022 Néstor Nápoles López

Given the set $\setpc$ of pitch classes in the Western
chromatic scale $\setpc = \{0, 1, 2, 3, 4, 5, 6, 7, 8, 9,
10, 11 \}$, a pitch class set (\gls{pcset}) is, in the
context of this dissertation, a subset $\elpcset \subset
\setpc$ comprising the pitch classes of a given triad or
seventh chord. Several \gls{rna} annotations may share the
same \gls{pcset}, as shown in \refeq{pcset_collision}.

\begin{equation}
    \label{eq:pcset_collision}
    \begin{split}
        \text{pcset}(\rnwkey{C}{\rnviihosev}) &= \{2, 5, 9, 11 \}\\
        \text{pcset}(\rnwkey{a}{\rniihosev}) &= \{2, 5, 9, 11 \}
    \end{split}
\end{equation}

One of the advantages of formulating a vocabulary of chords
$\setpcset$ based on \gls{pcset}s, instead of a vocabulary
of chords $\setrna$ based on \gls{rna} labels is that,
because of the collisions of many chords in terms of their
\gls{pcset}, $| \setpcset | \ll | \setrna |$. The main
disadvantage of this is that, because several \gls{rna}
labels will result in the same \gls{pcset}, it is difficult
to retrieve the original \gls{rna} back from the
\gls{pcset}. Nevertheless, using the musical key and the
algorithm described in
\refsec{analgorithmtoresolveromannumeralsfromapitch-classsetandkey},
it is possible to retrieve the original numerator of the
\gls{rna} annotation. Thus, I assume that a smaller
vocabulary of chords $\setpcset$, which is based on
\gls{pcset}s, will be useful to generate \gls{rna} labels if
the \gls{pcset} $\elpcset$ and key $\elkey$ can be predicted
by a machine learning model.

The vocabulary of \gls{pcset}s is constructed by combining
all keys $\elkey \in \setkey$ with the Roman numeral
numerators $\elnum \in \setnum$, and extracting the
\gls{pcset} of the resulting chord. However, because there
are Roman numeral numerators that are exclusive of the major
or minor mode, the approach shown in \refeq{pcsets} is
preferred, where this process is divided among major and
minor keys. The union of both sets for major and minor keys
results in a set of 121 \gls{pcset}s.

% In a \gls{12tet} system, the same collection of pitch
% classes may have different chord labels and/or \gls{rna}
% labels. For example, the set of pitch classes $\{0$, $4$,
% $7\}$ may be called a ``C major'' chord, or a
% ``B$\musSharp$ major'' chord; in \gls{rna}, it is called
% $\rnV$ if it appears in the context of the $F$ major key,
% or $\rnN$ if it appears in the context of the $b$ minor
% key. However, in all of those instances, the collection of
% pitch classes is the same. The attribute that
% disambiguates the different names of that pitch class set
% is the key context.

% The proposed method assumes that a \gls{rna} label can
% always be disambiguated from a \gls{pcset} and a key. For
% each pair of \gls{pcset} and key attributes, there is at
% most one associated \gls{rna} label.\footnote{There could
% be pairs of \gls{pcset} and key for which there is no
% \gls{rna} label associated.}

% When the bounded vocabulary of keys (see
% \refsec{thevocabularyofmusicalkeys}) and bounded
% vocabulary of Roman numerals (see
% \refsec{thevocabularyofromannumerals}) are combined, the
% resulting vocabulary is a bounded vocabulary of
% \gls{pcset} classes.

\begin{equation}
    \label{eq:pcsets}
    \begin{split}
    \setpcset_M &= \{ \elpcset \mid \elpcset \text{ is the \gls{pcset} of }
    \elkey\text{:}\elnum, \text{ and }
    \elkey \in \setkey_M \text{ and } \elnum \in \setnum_M  \} \\
    \setpcset_m &= \{ \elpcset \mid \elpcset \text{ is the \gls{pcset} of }
    \elkey\text{:}\elnum, \text{ and }
    \elkey \in \setkey_m \text{ and } \elnum \in \setnum_m  \} \\
    \setpcset &= \setpcset_M \cup \setpcset_m \\
    | \setpcset | &= 121 \\    
    \end{split}
\end{equation}
