% Copyright 2022 Néstor Nápoles López

\begin{equation}
    \setpcset = \sum_{k}^{| \setkey |} \sum_{n}^{| \setnum_{\mu(k)} |} \rho(k, n)
\end{equation}

\begin{equation}
    | \setpcset | = 121
\end{equation}

In a \gls{12tet} system, the same collection of pitch
classes may have different chord labels and/or \gls{rna}
labels. For example, the set of pitch classes $\{0$, $4$,
$7\}$ may be called a ``C major'' chord, or a ``B$\musSharp$
major'' chord; in \gls{rna}, it is called $\rnV$ if it
appears in the context of the $F$ major key, or $\rnN$ if it
appears in the context of the $b$ minor key. However, in all
of those instances, the collection of pitch classes is the
same. The attribute that disambiguates the different names
of that pitch class set is the key context.

The proposed method assumes that a \gls{rna} label can
always be disambiguated from a \gls{pcset} and a key. For
each pair of \gls{pcset} and key attributes, there is at
most one associated \gls{rna} label.\footnote{There could be
pairs of \gls{pcset} and key for which there is no \gls{rna}
label associated.}

When the bounded vocabulary of keys (see
\refsec{thevocabularyofmusicalkeys}) and bounded vocabulary
of Roman numerals (see
\refsec{thevocabularyofromannumerals}) are combined, the
resulting vocabulary is a bounded vocabulary of \gls{pcset}
classes.

This results in a set of 121 pitch class sets.

\begin{equation}
    \begin{split}
    \setpcset_M &= \{ \elpcset \mid \elpcset \text{ is the \gls{pcset} of } \elkey\text{:}\elnum, \text{ and }
    \elkey \in \setkey_M \text{ and } \elnum \in \setnum_M  \} \\
    \setpcset_m &= \{ \text{pcset}(\elkey\text{:}\elnum) \mid 
    \elnum \in \setnum_m \text{ and } \elkey \in \setkey_m \} \\ 
    \setpcset &= \setpcset_M \cup \setpcset_m \\
    | \setpcset | &= 121 \\    
    \end{split}
\end{equation}
