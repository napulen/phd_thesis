% Copyright 2022 Néstor Nápoles López

Consider the case of $\Gamma(\elpcset, \elkey)$ when
$\elpcset = \pcsetcdFa$ and $\elkey = \keyC$. In that case,
$\elpcset$ has an associated numerator $\elnum \in \setnum$
with the subset of keys shown in \refeq{key_not_found}.
Thus, that subset of keys is $\setkey_\elpcset$, as shown in
\refeq{key_not_found2}. 

\begin{equation}
    \label{eq:key_not_found}
    \elpcset = \pcsetcdFa \text{ in key } \elkey =
    \begin{cases}
        \keyfs & \quad \elnum = \rnGer \\
        \keyFs & \quad \elnum = \rnGer \\
        \keygb & \quad \elnum = \rnGer \\
        \keyg  & \quad \elnum = \rnVsev \\
        \keyGb & \quad \elnum = \rnGer \\
        \keyG  & \quad \elnum = \rnVsev \\
    \end{cases}
\end{equation}

\begin{equation}
    \label{eq:key_not_found2}
    \setkey_\elpcset = \{ \keyfs, \keyFs, \keygb, \keyg, \keyGb, \keyG \}
\end{equation}

However, we can see that for the given key $\elkey = \keyC$,
$\elkey \not\in \setkey_\elpcset$. In this case, a
tonicization $\elden$ will be computed, such that $\elden
\in \setkey_\elpcset$. The computed tonicization $\elden$
must be a ``closely related'' key to $\elkey$, in order for
the tonicization to be reasonable.

One way to compute a ``closely related'' key is to use a
metric of key distance between $\elkey$ and all keys
$\elden$ that satisfy $\elden \in \setkey_\elpcset$. The
tonicization $\elden$ with the minimal distance to $\elkey$
is a good candidate for a tonicization.

The key distance metric used is the euclidean distance $d$
between $\elkey$ and $\elden$ in the Weber tonal chart of
keys \parencite{weber1818versuch}. The chart is arranged in
the layout shown in \reftab{weber_tonal_chart}.

\phdtable[Configuration of the Weber tonal chart for a given
key $\elkey$]{weber_tonal_chart}

Thus, the estimation of tonicization $\elden$ is shown in
\refeq{tonicization_estimation}. For the example above with
$\elkey = \keyC$, this results in the key-distance
estimations shown in
\reftab{tonicization_estimation_example}.

\begin{equation}
    \label{eq:tonicization_estimation}
    \min_{\forall \elden \in \setkey_\elpcset} d(\elkey, \elden)
\end{equation}

\phdtable[Distance between given key $\elkey$ and
tonicizations $\elden \in \setkey_\elpcset$. The $\elden$
tonicization with the smallest distance is
highlighted]{tonicization_estimation_example}




% An experienced analyst would perhaps assume that the chord
% is referring to a \emph{dominant of the dominant} chord,
% $C:\rnVsev/V$. Thus, related to the interpretation
% $G:\rnVsev$ but in a tonicized $G$ major key. Most of the
% times that interpretation would be correct, however, it is
% not trivial to disambiguate the tonicization context
% programmatically.

% One way to find a reasonable tonicization given the key
% context is to use a key distance algorithm to find the
% tonicization that would minimize the distance between the
% key provided and the tonicization.
