% Copyright 2022 Néstor Nápoles López

A \gls{pcset} $\rho$ has an interpretation as one of the
Roman numeral numerators $\eta \in \vocabnum$ in a subset of
keys $\vocabkey_{\rho} \subset \vocabkey$. Such that:

\begin{equation}
    \vocabkey_{\rho} = \{\kappa \mid \kappa \in \vocabkey,
     \: \text{and $\rho$ has an interpretation in key $\kappa$}\}
\end{equation}

Given a pitch class set $\rho \in \vocabpcset$ and a key
$\kappa \in \vocabkey$, the function $\Gamma(\rho, \kappa) =
\eta$ where $\eta \in \vocabnum \cup \: \{ \emptyset \}$ is
either a Roman numeral numerator from the bounded vocabulary
or the special symbol $\chi$ if there is no Roman numeral
associated with the ($\rho$, $\kappa$) pair.

The function $\Gamma$ can be implemented as a lookup table
that returns $\chi$ when the duple ($\rho$, $\kappa$) is not
found, and the Roman numeral numerator $\eta \in \vocabnum$
otherwise.

When $\Gamma(\rho, \kappa) = \chi$, the Roman numeral
numerator cannot be retrieved directly. However, it can be
retrieved if a tonicization is forced in the system (i.e.,
the key changes for that particular chord).

% Let $\Gamma(\rho, \kappa)$ be a lookup table that returns
% a Roman numeral $\eta \in \chi$ for all pairs of ($\rho$,
% $\kappa$) for which there is a Roman numeral association. 
