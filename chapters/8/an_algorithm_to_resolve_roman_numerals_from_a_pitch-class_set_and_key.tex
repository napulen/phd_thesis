% Copyright 2022 Néstor Nápoles López

Given a pitch class set $\rho$ that is part of the finite
vocabulary of 121 pitch class sets and a key $\kappa$ that
is part of the finite vocabulary of 38 keys, the function
$\Gamma(\rho, \kappa) = \eta$ where $\eta$ is a Roman
numeral numerator from the bounded vocabulary of 31 Roman
numerals or the special symbol $\chi$ if there is no Roman
numeral associated with that ($\rho$, $\kappa$) pair.

The function $\Gamma$ can be implemented as a lookup table
that returns $\chi$ when the duple ($\rho$, $\kappa$) is not
found, and the Roman numeral $\eta$ otherwise.

When the value returned by $\Gamma = \chi$, the resolution
of the Roman numeral label can be done if a tonicization is
forced (i.e., change of key for that particular chord).

% Let $\Gamma(\rho, \kappa)$ be a lookup table that returns
% a Roman numeral $\eta \in \chi$ for all pairs of ($\rho$,
% $\kappa$) for which there is a Roman numeral association. 
