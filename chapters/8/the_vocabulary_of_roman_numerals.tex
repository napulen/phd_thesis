% Copyright 2022 Néstor Nápoles López

There are 31 valid Roman numeral labels. These are indicated
in \reftab{rn_vocabulary}.

\phdtable[Vocabulary of valid Roman numeral labels for this
\gls{rna} method]{rn_vocabulary}

\begin{equation}
    \vocabnum_{M} = \{ \: \rnI, \: \rnii, \: \rniii, \: \rnIV, \: \rnvi,
                \rnIsev, \: \rniisev, \: \rniiisev, \: \rnIVsev, \: 
                \rnvisev, \: \rnviihosev, \rnVaug,
                \rnCad, \: \rnV, \: \rnviio,
                \rnVsev,
                \rnN, \: \rnIt, \: \rnFr, \: \rnGer \: \}
\end{equation}


\begin{equation}
    \vocabnum_{m} = \{ \rni, \: \rniio, \: \rnIIIaug, \: \rniv, \: \rnVI,
                \rnisev, \: \rniihosev, \: \rnIIIaugsev, \: \rnivsev, \: \rnVIsev, \: \rnviiosev,
                \rnCad, \: \rnV, \: \rnviio,
                \rnVsev,
                \rnN, \: \rnIt, \: \rnFr, \: \rnGer \: \}
\end{equation}

\begin{equation}
    \vocabnum = \vocabnum_{M} \cup \vocabnum_{m}
\end{equation}

                

The Roman numeral vocabulary consists mainly of diatonic
harmonies that can be constructed from a major or harmonic
minor scale. A few chords are exceptions to this rule, \:
which I indicated as ``special chords.''

Each of the chords in the vocabulary is explained below.


