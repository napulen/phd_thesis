% Copyright 2022 Néstor Nápoles López

If we consider a \gls{rna} label to be a ``fraction-like''
expression, such as the one shown in \refeq{rn_expression},
then the symbol $\elkey$ preceding the colon represents a
key, the numerator $\elnum$ represents a chord, and the
denominator $\elden$ represents a tonicized scale degree.

\begin{equation}
    \label{eq:rn_expression}
    \elkey : \elnum \: / \: \elden
\end{equation}

Thus, in the annotation shown in \refeq{rn_example1}, the
preceding symbol $\elkey=\rn{C}$ indicates the key of
$\keyC$; the denominator $\elden = \rnV$ indicates a
tonicization of $\elkey$'s dominant, $\keyG$; and
$\elnum=\rnviihosev$ indicates a chord relative to the
tonicized key $\elden$, an F$\musSharp$ half-diminished
seventh chord.

\begin{equation}
    \label{eq:rn_example1}
    \rnwkeyton{C}{\rnviihosev{}}{\rnV{}}
\end{equation}


The mode (major or minor) implied by the keys $\elkey$ and
$\elden$ is indicated with a case-sensitive notation. For
example, in the annotation shown in \refeq{rn_example2}, the
denominator $\elden=\rnii$ indicates a key of $\keyd$. In
this case, the numerator $\elnum = \rnviiosev$ indicates a
chord that is relative to the key of the denominator
$\elden$, namely, a C$\musSharp$ diminished seventh chord.

\begin{equation}
    \label{eq:rn_example2}
    \rnwkeyton{C}{\rnviiosev}{ii}
\end{equation}

Tonicizations are optional and do not occur often. If there
is no symbol $\elden$ found in the annotation, it can be
assumed that $\elden = \elkey$. For example, in the
annotation shown in \refeq{rn_example3}, $\elden = \elkey =
\keyG$.

\begin{equation}
    \label{eq:rn_example3}
    \rnwkey{G}{\rnVsev}
\end{equation}

The ``modulation'' key\footnote{I like to refer to key
$\elkey$ as the ``modulation'' key. However, it is also
common to refer to it as the ``local key'' in the \gls{mir}
literature \parencite{napoleslopez2020local}.} $\elkey$ is
mandatory for the numerator $\elnum$ to have any meaning,
but it can be omitted in subsequent annotations if it
remains unchanged. For example, in the subsequent \gls{rna}
annotations shown in \refeq{rn_three_annotations}, which are
read from left to right, the symbols can be interpreted as
in \reftab{rn_three_annotations}.

\begin{equation}
    \label{eq:rn_three_annotations}
    \rnwkey{a}{\rniv} \qquad \rnwton{\rnVsev}{V} \qquad \rn{V}
\end{equation}

\phdtable[Interpretation of the \gls{rna} annotations in
\refeq{rn_three_annotations}]{rn_three_annotations}

In most instances, a \gls{rna} label will be made of only a
numerator $\elnum$, with keys $\elkey$ appearing either at
the beginning of the piece or when modulations occur, and
denominators $\elden$ appearing when there is a tonicized
key, such that $\elden \neq \elkey$.

These general rules allow the interpretation of virtually
any \gls{rna} label. However, one of the confusing aspects
of the notation is perhaps the vocabulary of $\elnum$,
$\elkey$, and $\elden$ symbols that one can encounter in the
notation, and the meaning of those when interpreted as
chords.

The next sections describe the vocabularies of musical
symbols $\setnum$, $\setkey$, and $\setden$, among others,
that satisfy $\forall \; \elnum \in \setnum$, $\forall \;
\elkey \in \setkey$, and $\forall \; \elden \in \setden$ for
all the annotations presented throughout the chapters. This
complementary information facilitates the interpretation of
\gls{rna} labels in the context of this dissertation.
