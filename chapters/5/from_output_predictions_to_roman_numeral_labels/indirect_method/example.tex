% Copyright 2022 Néstor Nápoles López

For the following example, assume that the \gls{mtl}
configuration provides the following outputs only:
\gls{harmonicrhythm7}, \gls{bass35},  \gls{localkey38}, and
\gls{pcset121}. Using these three predictions, a \gls{rna}
label can be generated with the help of the \algorithmrn{}
algorithm.

The \gls{pcset} $\elpcsetpred \in \setpcset$ corresponds to
the prediction of the \gls{pcset121} classifier, the key
$\elkeypred \in \setkey$ is the prediction of the
\gls{localkey38} classifier, and the onsets $\elonsetpred
\in \setonset$ are used to filter the timesteps where
\gls{rna} labels need to be generated.
\reftab{indirectmethod_rna_start} shows a reduction of the
table of predictions. This is similar to the simplification
done in the direct method in
\reftab{crnn_predictions_onsets_cut}

\phdtable[Table of predictions in the example of the
indirect method]{indirectmethod_rna_start}

\phdparagraph{resolving the numerator and tonicization}

Given the inputs $\elpcsetpred$ and $\elkeypred$, the
\algorithmrn{} provides the outputs shown in
\reftab{indirectmethod_rna_natem}.

\phdtable[Table of predictions in the example of the
indirect method]{indirectmethod_rna_natem}

