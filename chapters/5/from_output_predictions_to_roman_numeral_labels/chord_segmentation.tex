% Copyright 2022 Néstor Nápoles López

As expected, the \gls{crnn} provides predictions every
\gls{32nd} note, however, most of these timesteps should not
be considered \gls{rna} labels. Therefore, the chord
segmentation of the chords should be determined, in order to
decide which rows of predictions should be considered the
location of a \gls{rna} label. This is done by interpreting
the onset predictions $\elonsetpred \in \setonset$ retrieved
from the \gls{harmonicrhythm7} classifier.

After filtering out all rows that are not \gls{rna} onsets,
the table shrinks to the one shown in
\reftab{crnn_predictions_onsets}.

\phdtable[Predictions from the \gls{crnn} that were considered the onset of a \gls{rna} label by the \gls{harmonicrhythm7} classifier]{crnn_predictions_onsets}

% This requires an approach for chord segmentation, where the
% onset predictions $\elonset \in \setonset$ starting from the
% predictions of the \gls{crnn} is to decide the chord
% segmentation. That is, which of the 640 timesteps should be
% considered the location of a \gls{rna} label. This is the
% first problem to be solved.
