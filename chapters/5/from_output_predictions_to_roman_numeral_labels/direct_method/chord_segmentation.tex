% Copyright 2022 Néstor Nápoles López

The \gls{crnn} provides predictions every \gls{32nd} note.
Yet, most of these timesteps should not have a \gls{rna}
label generated. Therefore, the location of the chords
should be determined (i.e., the chord segmentation). In
order to determine which rows of predictions should be
considered \gls{rna} labels, the onset predictions
$\elonsetpred \in \setonset$ retrieved from the
\gls{harmonicrhythm7} classifier are used to filter the
predictions of the \gls{crnn}. After filtering out all rows
that are not \gls{rna} onsets, the number of rows in the
table of predictions is reduced, as shown in
\reftab{crnn_predictions_onsets_cut}. One \gls{rna} label
will be generated for each of these rows. In addition to the
reduced number of rows, all other classification tasks
except for \gls{harmonicrhythm7}, \gls{localkey38},
\gls{tonicization38}, \gls{bass35}, and \gls{rn31} have been
removed for brevity.

\phdtable[Predictions from the \gls{crnn} that were
considered the onset of a \gls{rna} label by the
\gls{harmonicrhythm7}
classifier]{crnn_predictions_onsets_cut}
