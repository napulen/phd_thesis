% Copyright 2022 Néstor Nápoles López


In the indirect method, instead of retrieving predictions
for $\elnum$ and $\elden$ directly, these are estimated from
a \gls{pcset} $\elpcset \in \setpcset$ and key $\elkey \in
\setkey$. The estimation is done using a new algorithm
called Numerator and Tonicization Estimation Method
(\gls{natem}). This algorithm provides a Roman numeral
numerator given the \gls{pcset} and key. In instances where
no Roman numeral numerator is suitable for the \gls{pcset}
and key provided, the algorithm searches for an appropriate
tonicization that fits the given tonal context. For example,
given the \gls{pcset} $\elpcset = \pcsetdFa$ and key $\elkey
= \keyC$, the algorithm would return a ``dominant of the
dominant'' annotation of the form $\rnwton{\rnV}{V}$. The
technical details of the \gls{natem} algorithm are described
in \refsec{thenumeratorandtonicizationestimationmethod}. The
indirect method, which is facilitated by the computations of
the \gls{natem} algorithm, is useful in several situations: 

\begin{enumerate}
    \item When running an evaluation with several \gls{arna}
    models that have a distinct set of classification tasks,
    as in \refsec{evaluationagainstpreviousmodels}.
    \item When several classification tasks need to be
    combined to retrieve the chord (e.g., the $\elpcset$
    implied by combining the \gls{satb35} classifiers).
    \item When there is no information available other than
    a chord label and a key, which occurs in some methods.
\end{enumerate}

In this case, the \gls{natem} algorithm, is used to retrieve
the numerator $\elnum$ and tonicization $\elden$. The
generation of the \gls{rna} label is done as shown in
\refeq{rna_indirect}.

\begin{equation}
    \label{eq:rna_indirect}
    \begin{split}
        \elnumpred / \eldenpred &= \gls{natem}(\elpcset, \elkey) \\
        \elrnapred &= \elkey : \elnumpred / \elden 
    \end{split}
\end{equation}

Note that the indirect method does not retrieve the
inversion $\elinv$. If the inversion of the chord is of
interest, there needs to be a classifier for it. The
\gls{natem} algorithm can be used to obtain Roman numeral
numerators and tonicizations only. 


% multiclass classification tasks of the \gls{augmentednet}
% to generate the \gls{rna} label. The first step is to use
% the predictions of the \gls{harmonicrhythm7} task to
% decide the location of the chord onsets. Once the location
% of timesteps that represent \gls{rna} annotation are
% known, the next step is to use the predictions of the
% relevant multitask classifiers.

% \begin{enumerate} \item[] The prediction of the
%     \gls{localkey38} task becomes the key $\elkey$ \item[]
%     The prediction of the \gls{rn31} task becomes the
%     Roman numeral numerator $\elnum$ \item[] The
%     prediction of the \gls{tonicization38} becomes the key
%     implied by the tonicization $\elden_\elkey$; if
%     $\elden_\elkey \neq \elkey$, then it needs to be
%     converted into a scale degree $\elden$ that is
%     relative to key $\elkey$ \end{enumerate}

% The inversion $\elinv$ requires a slightly more involved
% process. First, using the other components of the
% \gls{rna} annotation, the chord implied by the numerator
% $\elnum$ needs to be converted into a sequence of notes,
% ordered from the root. Then, retrieving the prediction
% $\elbass \in \setps_{35}$ of the \gls{bass35} classifier,
% the location of $\elbass$ is searched within the sequence
% of notes of the chord. If the note is found, then the
% index of the bass $\elbass$ in the sequence of notes is
% the inversion $\elinv$. With these features, a \gls{rna}
% label can be generated, as shown in \refeq{rna_direct},
% where $\elrnapred$ indicates that the generated \gls{rna}
% label is an automatic prediction from the \gls{arna}
% model. This process is repeated for every chord onset.

% \begin{equation} \label{eq:rna_direct} \elrnapred = \elkey
%     : \elnum^{\elinv} \; / \; \elden \end{equation}



% In addition to the direct method that can be used with the
% \gls{mtl} predictions of this model, it might be useful to
% generate Roman numeral labels when only a chord label and
% key are provided. Some methods, such as the one in
% \textcite{mcleod2021modular}, produce chord labels and
% keys. Thus, a numerator $\elnum$ and a tonicization
% $\elden$ need to be estimated from the chord label and
% key. In that situation, an alternative method is provided,
% based on the \algorithmrn{} algorithm, described in
% \refsec{thenumeratorandtonicizationestimationmethod}. This
% was the method used to estimate full \gls{rna} labels from
% the predictions of the model in
% \textcite{mcleod2021modular, chen2021attend} during the
% evaluation (which is explained in
% \refsec{evaluationagainstpreviousmodels})

% The method follows the same principles described in the
% direct method, with the following differences: the
% numerator $\elnumpred$ and tonicization $\eldenpred$ are
% not directly available from the predictions of the neural
% network. Instead, they are retrieved from the chord label
% and key using the \algorithmrn{} algorithm, as shown in
% \refeq{calling_natem}.

% \begin{equation} \label{eq:calling_natem} (\elnumpred,
%     \eldenpred) = \algorithmrn{}(\elpcset, \elkey)
%     \end{equation}

% In \refeq{calling_natem}, the \gls{pcset} $\elpcset \in
% \setpcset$ is the representation of the chord label as a
% set of pitch classes. It is easy to obtain this set by
% first deriving the notes that comprise the chord label,
% and then listing their pitch classes, as shown
% \refeq{indirect_method_pcset}. This information is
% generally available in any \gls{acr} approach, regardless
% of the syntax of the chord labels.


% \begin{equation} \label{eq:indirect_method_pcset} Gmin
%     \implies \{\pitchG, \pitchBb, \pitchD\} \implies (7,
%     10, 2) \end{equation} 

% After the \gls{pcset} $\elpcsetpred$ is obtained from the
% machine learning model, the numerator $\elnumpred$ and
% tonicization $\eldenpred$ can be retrieved with the
% \algorithmrn{} algorithm. Afterwards, the process to
% generate the \gls{rna} label is the same as in the direct
% method. In order to illustrate the indirect method, an
% example will be computed from the predictions of the
% \gls{pcset121} of the \gls{crnn} proposed here.

