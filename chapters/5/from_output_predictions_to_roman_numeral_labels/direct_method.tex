% Copyright 2022 Néstor Nápoles López

The \gls{mtl} configuration of the model is tightly coupled
with the vocabularies defined for $\setkey$ (see
\refsec{thevocabularyofmusicalkeys}), $\setnum$ (see
\refsec{thevocabularyofromannumeralnumerators}), $\setden$
(see \refsec{thevocabularyofromannumeraldenominators}), and
$\setinv$ (see
\refsec{thevocabularyofarabicnumeralinversions}). Thus, the
output labels of the \gls{arna} model can be generated as
shown in \refeq{rna_pred}. 

\begin{equation}
    \label{eq:rna_pred}
    \elrnapred_{\elonsetpred} = \elkeypred : \elnumpred^{\elinvpred} \; / \; \eldenpred
\end{equation}

In \refeq{rna_pred}, $\elkeypred \in \setkey$ is the key
prediction of the \gls{localkey38} classifier, $\elnumpred
\in \setnum$ is the Roman numeral numerator prediction of
the \gls{rn31} classifier, $\eldenpred \in \setden$ is the
tonicization prediction of the \gls{tonicization38}
classifier, $\elinvpred \in \setinv$ is an inversion
extracted from the prediction of the \gls{bass35}
classifier, and $\elrnapred \in \setrna$ is the final
\gls{rna} label prediction.


The \gls{crnn} provides predictions every \gls{32nd} note.
Yet, most of these timesteps should not have a \gls{rna}
label generated. Therefore, the location of the chords
should be determined (i.e., the chord segmentation). In
order to determine which rows of predictions should be
considered \gls{rna} labels, the onset predictions
$\elonsetpred \in \setonset$ retrieved from the
\gls{harmonicrhythm7} classifier are used to filter the
predictions of the \gls{crnn}. After filtering out all rows
that are not \gls{rna} onsets, the number of rows in the
table of predictions is reduced, as shown in
\reftab{crnn_predictions_onsets}. One \gls{rna} label will
be generated for each of these rows.

\phdtable[Predictions from the \gls{crnn} that were
considered the onset of a \gls{rna} label by the
\gls{harmonicrhythm7} classifier]{crnn_predictions_onsets}

% This requires an approach for chord segmentation, where
% the onset predictions $\elonset \in \setonset$ starting
% from the predictions of the \gls{crnn} is to decide the
% chord segmentation. That is, which of the 640 timesteps
% should be considered the location of a \gls{rna} label.
% This is the first problem to be solved.
