% Copyright 2022 Néstor Nápoles López


The direct method consists of using the multiclass
classification tasks of the \gls{augmentednet} to generate
the \gls{rna} label. The first step is to use the
predictions of the \gls{harmonicrhythm7} task to decide the
location of the chord onsets. Once the location of timesteps
that represent \gls{rna} annotations are known, the next
step is to use the predictions of the relevant multitask
classifiers.

\begin{enumerate}
    \item[] The prediction of the \gls{localkey38} task
    becomes the key $\elkey$.
    \item[] The prediction of the \gls{rn31} task becomes
    the Roman numeral numerator $\elnum$.
    \item[] The prediction of the \gls{tonicization38}
    becomes the key implied by the tonicization
    $\elden_\elkey$; if $\elden_\elkey \neq \elkey$, then it
    needs to be converted into a scale degree $\elden$ that
    is relative to key $\elkey$.
\end{enumerate}

The inversion~$\elinv$ requires a slightly more involved
process. First, using the other components of the \gls{rna}
annotation, the chord implied by the numerator~$\elnum$
needs to be converted into a sequence of notes, arranged in
order from the root upwards. Then, retrieving the
prediction~$\elbass \in \setps_{35}$ of the \gls{bass35}
classifier, the location of~$\elbass$ is searched within the
sequence of notes of the chord. If the note is found, then
the index of the bass~$\elbass$ in the sequence of notes is
the inversion~$\elinv$. With these features, an \gls{rna}
label can be generated, as shown in \refeq{rna_direct},
where~$\elrnapred$ indicates that the generated \gls{rna}
label is an automatic prediction from the \gls{arna} model.
This process is repeated for each of the chord onsets
predicted by the \gls{harmonicrhythm7} classifier.

\begin{equation}
    \label{eq:rna_direct}
    \elrnapred = \elkey : \elnum^{\elinv} \; / \; \elden
\end{equation}


% The \gls{mtl} configuration of the model is tightly
% coupled with the vocabularies defined for $\setkey$ (see
% \refsec{thevocabularyofmusicalkeys}), $\setnum$ (see
% \refsec{thevocabularyofromannumeralnumerators}), $\setden$
% (see \refsec{thevocabularyofromannumeraldenominators}),
% and $\setinv$ (see
% \refsec{thevocabularyofarabicnumeralinversions}). Thus,
% the output labels of the \gls{arna} model can be generated
% as shown in \refeq{rna_pred}. 

% \begin{equation} \label{eq:rna_pred}
%     \elrnapred_{\elonsetpred} = \elkeypred :
%     \elnumpred^{\elinvpred} \; / \; \eldenpred
%     \end{equation}

% In \refeq{rna_pred}, $\elkeypred \in \setkey$ is the key
% prediction of the \gls{localkey38} classifier, $\elnumpred
% \in \setnum$ is the Roman numeral numerator prediction of
% the \gls{rn31} classifier, $\eldenpred \in \setden$ is the
% tonicization prediction of the \gls{tonicization38}
% classifier, $\elinvpred \in \setinv$ is an inversion
% indirectly obtained from the prediction of the
% \gls{bass35} classifier, and $\elrnapred \in \setrna$ is
% the final \gls{rna} label prediction. In the string form
% of the \gls{rna} label, the tonicization $\eldenpred$
% needs to be encoded as a relative scale degree of the key
% $\elkeypred$. Furthermore, the inversion $\elinvpred$
% needs to be determined from the prediction of the
% \gls{bass35} classifier in the context of the chord
% implied by $\elnumpred$. Thus, the inversion is going to
% be the last step of the method. The rest of the variables
% are direct predictions from the \gls{crnn}. However, one
% issue that is yet to be addressed before generating the
% labels is which of the 640 timesteps should be considered
% the location of a \gls{rna} label. 

% This requires an approach for chord segmentation, where
% the onset predictions $\elonset \in \setonset$ starting
% from the predictions of the \gls{crnn} is to decide the
% chord segmentation. That is, which of the 640 timesteps
% should be considered the location of a \gls{rna} label.
% This is the first problem to be solved.
