% Taken verbatim from AugmentedNet

\guide{Inputs}

\textbf{Reference note per timestep.}
The input to the network consists of a sequence of
timesteps, which are sampled from the score at symbolically
regular note duration values. In this study, we use the
thirty-second note (`demisemiquaver') as this atomic value
(i.e.~eight timesteps per quarter note in the score) in
order to match the most fine-grained frame sampling seen in
previous work. The length of the sequence is set by a fixed
number of timesteps. Following Micchi et al., we set that
number at 640 frames (or 80 quarter notes) per sequence
example.
%Each timestep contains a vector of bass and chroma
%features.

\textbf{Bass and spelled chroma features.}
We also follow the Micchi et al.~input representation of
pitch as a vector containing bass spelling and spelled
`chroma features'\footnote{A chroma feature representation
is typically a vector of 12 pitch classes, where the
activation of each pitch class at a given timestep is
indicated in a many-hot encoding fashion (symbolic data), or
as continous values (audio data). Due to the similarity of
this input representation to chroma features, except for the
spelling aspect, we refer to them as \emph{spelled} chroma
features.} at every timestep, however the details differ in
subtle but important ways. In the Micchi et al.
representation, each timestep has 70 features: 35 for the
bass and 35 for the chroma features. We consolidate this
information in 38 features: 19 for the bass, and 19 for the
chroma features. The reduction in number of features is due
to an alternative encoding of pitch spelling, described
below.

\textbf{Encoding the pitch spelling.}
%In previous work, Micchi et al. \cite{micchi2020not} have
%encoded pitch spellings as a one-hot encoded vector with 35
%features, which encodes pitch spellings up to two sharps or
%two flats of every note letter (7 * 5 = 35).
We split the representation of a pitch spelling into two
components: the pitch class (0--12) and the generic note
letter (A--G). Each spelled pitch thus leads to a two-hot
encoded vector with 19 features (1 of 12 pitch classes, and
1 of 7 note names). This reduces the number of parameters in
the network without any observable compromise in
performance.

\guide{Separating the inputs.}
Furthermore, the bass and spelled chroma features are
connected to the network independently and concatenated only
after they have passed through their own convolutional
blocks. In preliminary experiments, we discovered that this
separation of the inputs (bass and spelled chroma features)
was beneficial to the learning process. Thus, two parallel
convolutional blocks are computed.
