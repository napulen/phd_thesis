% Copyright 2022 Néstor Nápoles López

The steps in the previous sections describe the methods to
reduce a symbolic music file into a fixed-length sequence of
pitch values, which are sampled at regularly-spaced
intervals of \gls{32nd} notes. Using these sequences and the
vector representations described in
\refsubsec{encodingpitchspelling} and
\refsubsec{encodingduration}, the score can be encoded into
a set of \emph{input representations}, which are dispatched
to the trainable layers of the neural network. 

The system currently considers three input representations,
\gls{bass19}, \gls{chroma19}, and \gls{duration14},
described below.

% The corresponding encodings are shown in
% \reffig{bass_chroma_separation}.

% \phdfigureproxy[Two encodings of the same input. The first
% encoding highlights the lowest-sounding note and the
% second one highlights all the notes (i.e., spelled pitch
% classes) sounding at a given time]{bass_chroma_separation}
