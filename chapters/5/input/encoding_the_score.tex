% Copyright 2022 Néstor Nápoles López

The steps in the previous sections describe the methods to
reduce a symbolic music file into a fixed-length sequence of
pitch values. As mentioned in
\refsubsec{samplingofthescore}, the score in the symbolic
music file is stripped from some of its original information
when it is encoded for the neural network model.
Furthermore, the score is sampled at regular note intervals
of \gls{32nd} notes, encoding the musical information at
each of those timesteps. This section describes this
encoding process.

Using these sequences and the vocabularies described in
\refsec{thevocabularyofnoteswithspelling} and
\refsec{thevocabularyofmeasure,note,andchordonsets}, the
score can be encoded into a set of \emph{input
representations}, which are dispatched to the trainable
layers of the neural network.

The system currently considers three input representations,
\gls{bass19}, \gls{chroma19}, and \gls{duration14},
described below.

% The corresponding encodings are shown in
% \reffig{bass_chroma_separation}.

% \phdfigureproxy[Two encodings of the same input. The first
% encoding highlights the lowest-sounding note and the
% second one highlights all the notes (i.e., spelled pitch
% classes) sounding at a given time]{bass_chroma_separation}
