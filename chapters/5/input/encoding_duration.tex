% Copyright 2022 Néstor Nápoles López

A representation for note duration is often needed in inputs
and output representations of the neural network. For
example, in the \gls{duration14} input representation (see
\refsubsec{Duration14}) or the \gls{harmonicrhythm7} output
representation (see \refsubsubsec{HarmonicRhythm7}). In
order to encode this kind of information, the following
representation is proposed, with 7 features.

\begin{equation}
    \{\varnothing,\; \musThirtySecond,\; \musSixteenth,\; \musEighth,\; 
    \musQuarter,\; \musHalf,\; \musWhole \}
\end{equation}

In this representation, $\varnothing$ describes an
\emph{onset}, and each of the subsequent classes represents
the time elapsed since the onset, measured in note
durations. In the inputs to the neural network, the
representation is used to encode measure and note onsets
(see \refsubsec{measureandnoteonsets}). In the outputs of
the neural network, it is used to predict the onset of
\gls{rna} labels (see \refsubsubsec{HarmonicRhythm7}). These
cases will be discussed more in detail in
\refsubsec{Duration14} and \refsubsubsec{HarmonicRhythm7}.
