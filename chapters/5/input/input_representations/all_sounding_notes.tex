% Copyright 2022 Néstor Nápoles López

The second input representation, \gls{chroma19}, encodes all
the sounding note classes at each timestep. In
\textcite{napoleslopez2021augmentednet}, we colloquially
referred to it as the ``spelled chroma'' input
representation. This is because of the similarity between
this representation and the commonly used ``chromagram''
representation in audio. The main difference being, beyond
being used in the symbolic domain, that chromagram features
often ignore note spelling, whereas in this case spelling is
taken into account. The encoding of \gls{chroma19} is also
done using the pitch spelling duples discussed in
\refsubsubsec{19two-hotencoding}. However, in this case, the
representation results in a ``multi'' two-hot encoding, as
each note will result in a two-hot encoding (i.e., several
spelled notes will be encoded in the same timestep). An
example of this encoding is shown in
\reffig{chroma_encoding}.

\phdfigureproxy[An encoding of the spelled chroma input]{chroma_encoding}
