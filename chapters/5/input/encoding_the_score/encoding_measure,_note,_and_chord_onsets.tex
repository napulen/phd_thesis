% Copyright 2022 Néstor Nápoles López

A way to encode onsets is often needed in inputs and outputs
of an \gls{arna} model. For example, to indicate where a
measure, note, or chord begins. 
% In the \gls{duration14} input representation (see
% \refsubsubsec{measureandnoteonsets}) or the
% \gls{harmonicrhythm7} output representation (see
% \refsubsubsec{HarmonicRhythm7}). 
In order to encode this kind of information, the following
representation is proposed, with 7 features.

\begin{equation}
    \label{eq:onset_7}
    \setonset = \{\checkmark,\; \musThirtySecond,\; \musSixteenth,\; 
    \musEighth,\; \musQuarter,\; \musHalf,\; \musWhole \}
\end{equation}

\begin{equation}
    \label{eq:onset_7_magnitude}
    | \setonset | = 7
\end{equation}

In this encoding method, $\varnothing$ describes an
\emph{onset}, and each of the subsequent classes represents
the time elapsed since the onset, measured in note
durations. In the input representation \gls{duration14}, it
is used to encode measure and note onsets (see
\refsubsubsec{measureandnoteonsets}). In the output
representation \gls{harmonicrhythm7}, it is used to predict
the onset of \gls{rna} labels (see
\refsubsec{HarmonicRhythm7}).

The two methods discussed for encoding pitch spellings and
onsets, based on the vocabularies $\setps_{19}$ and
$\setonset$, respectively, are used in the three \emph{input
representations} dispatched to the neural network.
