% Copyright 2022 Néstor Nápoles López

The second input representation, \gls{chroma19}, encodes all
the sounding note classes at each timestep. In
\textcite{napoleslopez2021augmentednet}, we colloquially
referred to this one as the ``spelled chroma'' input
representation. This is due to its similarity with the
commonly used ``chromagram'' representation in the audio
domain. The main difference being, beyond operating in the
symbolic domain, that chromagram features often ignore note
spellings, whereas here they are taken into account. The
encoding of \gls{chroma19} is also done using the
pitch-spelling duples discussed in
\refsubsubsec{19two-hotencoding}. However, in this case, the
representation results in a ``multi'' two-hot encoding, as
each note will result in a two-hot encoding, but several
spelled notes may be encoded per timestep. An example of
this encoding is shown in \reffig{chroma_encoding}.

\phdfigure[An encoding of the \gls{chroma19} representation
in Clara Schumann's Op. 13 No. 2, mm. 1--4]{chroma_encoding}
