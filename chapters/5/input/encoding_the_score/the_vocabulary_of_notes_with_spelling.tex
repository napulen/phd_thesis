% Copyright 2022 Néstor Nápoles López

Most tonal music analysis neural network models collapse all
enharmonic spellings of the same note, as shown in
\refeq{enharmonic_equivalence}.

\begin{equation}
    \label{eq:enharmonic_equivalence}
    \pitchFs = \pitchGb \quad \text{enharmonic equivalence}
\end{equation}

\begin{equation}
    \label{eq:enharmonic_nonequivalence}
    \pitchFs \neq \pitchGb \quad \text{enharmonic nonequivalence}
\end{equation}

In the \gls{arna} model proposed here, however, the spelling
of a pitch is encoded and taken into account, as shown in
\refeq{enharmonic_nonequivalence}. Two methods are described
to encode pitch spellings in this way.

%  and encode each of those spelling classes as one
% dimension in a vector. This method has been used in
% \textcite{micchi2020not, micchi2021deep}. An alternative
% method, proposed here, is to encode the pitch class and
% spelling of the pitch class. Both methods are described
% next.

\phdparagraph{35 one-hot encoding}

Perhaps the easiest method to encode a pitch spelling is to
assume that only a number of accidentals are allowed next to
a generic note letter. The use of \gls{sharp} and \gls{flat}
accidentals is very common. The use of \gls{doublesharp} and
\gls{doubleflat} accidentals is less common. Any accidental
sharper than \gls{doublesharp} or flatter than
\gls{doubleflat} is rare. Thus, a reasonable vocabulary for
spelled pitch classes\footnote{Note that the octave is
ignored.} is one with 35 classes, from $\pitchCbb$ to
$\pitchBx$, as shown in \refeq{pitch_spelling_35}.

\begin{equation}
    \label{eq:pitch_spelling_35}
    \begin{split}
    \setps{}_{35} = \{ & \pitchCbb, \pitchDbb, \pitchEbb, \pitchFbb, \pitchGbb, \pitchAbb, \pitchBbb, \\
    & \pitchCb, \pitchDb, \pitchEb, \pitchFb, \pitchGb, \pitchAb, \pitchBb, \\
    & \pitchC, \pitchD, \pitchE, \pitchF, \pitchG, \pitchA, \pitchB, \\
    & \pitchCs, \pitchDs, \pitchEs, \pitchFs, \pitchGs, \pitchAs, \pitchBs, \\
    & \pitchCx, \pitchDx, \pitchEx, \pitchFx, \pitchGx, \pitchAx, \pitchBx \}
    \end{split}
\end{equation}

This method was used in \textcite{micchi2020not,
micchi2021deep} to encode the inputs of \gls{arna} models.
Using this approach, a 35-dimensional pitch-spelling vector
is encoded every timestep, indicating the active pitch (or
pitches, if several) among the 35 available classes. One of
the limitations of this method is that it cannot encode any
notes sharper than a \gls{doublesharp} or flatter than a
\gls{doubleflat}. Additionally, the resulting vector is
almost three times larger than an encoding based on pitch
classes, which only requires 12 classes, as shown in
\refeq{pitch_classes}.\footnote{When enharmonic equivalence
is assumed in a \gls{12tet} system, a set of 12 pitch
classes spans all the note classes of the Western chromatic
scale.}

\begin{equation}
    \label{eq:pitch_classes}
    \mathcal{C} = \{0, 1, 2, 3, 4, 5, 6, 7, 8, 9, 10, 11\}
\end{equation}

\phdparagraph{19 two-hot encoding}


Another encoding method, first proposed in
\textcite{napoleslopez2021augmentednet}, proposes an
alternative way to encode pitch spellings. Instead of
encoding \gls{sharp} and \gls{flat} notes explicitly, it
extends the pitch class vector representation, by
concurrently encoding a generic note letter class, as shown
in \refeq{pitch_spelling_19}.

\begin{equation}
    \label{eq:pitch_spelling_19}
    \setps_{19} = \mathcal{C}
    \times \{ \pitchC, \pitchD, \pitchE, \pitchF, \pitchG, \pitchA, \pitchB \}
\end{equation}

% \begin{equation} \label{eq:pitch_spelling_19_magnitude} |
%     \setps_{19} | = 19 \end{equation}

This encoding method results in duples of the form $(p, n)$,
where $c \in \mathcal{C}$ and $n \in \{\pitchC, \pitchD,
\pitchE, \pitchF, \pitchG, \pitchA, \pitchB\}$. For example,
C\musSharp{} is represented by the duple $(1, \text{C})$,
whereas D\musFlat{} is represented by the same pitch class
but a different generic note letter, $(1, \text{D})$.
D$\musNatural{}$ has the same generic note letter but a
different pitch class, $(2, \text{D})$, and so on. Using
this representation, a spelled note, beyond two flats and
two sharps can be encoded with a 19-feature vector. Notice,
however, that this vector is a two-hot encoding
representation. This means that any given timestep, each
pitch spelling will be encoded as two active classes: one
for the pitch class and one for the generic note letter. The
input representations described in
\refsubsec{inputrepresentations} employ the latter encoding
method.
