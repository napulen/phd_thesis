% Copyright 2022 Néstor Nápoles López

When two spellings of the same \gls{midi} note number are
distinguished, they can be encoded as different dimensions
in a vector. The use of \gls{sharp} and \gls{flat}
accidentals is very common. The use of \gls{doublesharp} and
\gls{doubleflat} accidentals is less common. Any accidental
sharper than \gls{doublesharp} or flatter than
\gls{doubleflat} is rare. Thus, a reasonable representation
for spelled pitch classes\footnote{Note that the octave is
ignored.} is a representation with 35 features:

\begin{equation}
    \label{eq:pitch_spelling_35}
    \setps{}_{35} = \{\musDoubleFlat{}, \musFlat{}, 
    \musNatural{}, \musSharp{}, \musDoubleSharp{} \}
    \times 
    \{C, D, E, F, G, A, B \}
\end{equation}

\begin{equation}
    \label{eq:pitch_spelling_35_magnitude}
    | \setps{}_{35} | = 35
\end{equation}

The input representation shown in \refeq{pitch_spelling_35}
is the one used in \textcite{micchi2020not, micchi2021deep}.
In their representation, a 35-dimensional pitch spelling
vector was encoded per timestep, with only one active
spelling class (one-hot encoding) per timestep.
