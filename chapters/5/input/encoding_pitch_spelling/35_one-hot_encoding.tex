% Copyright 2022 Néstor Nápoles López

When two spellings of the same \gls{midi} note number are
distinguished, they can be encoded as different dimensions
in a vector. The use of sharp (\musSharp{}) and flat
(\musFlat{}) accidentals is very common. The use of double
sharp (\musDoubleSharp{}) and double flat (\musDoubleFlat{})
accidentals is less common. Any accidental sharper than
\musDoubleSharp{} or flatter than \musDoubleFlat{} is rare.
Thus, a reasonable representation for spelled pitch
classes\footnote{Note that the octave is ignored.} is a
representation with 35 features:

\begin{equation}
    \label{eq:pitch_spelling_35}
    \{\musDoubleFlat{}, \musFlat{}, \musNatural{}, \musSharp{},
    \musDoubleSharp{} \}
    \times 
    \{C, D, E, F, G, A, B \}    
\end{equation}

The input representation shown in \refeq{pitch_spelling_35}
is the one used in \textcite{micchi2020not, micchi2021deep}.
In their representation, a 35-dimensional pitch spelling
vector was encoded per timestep, with one spelling class
(one-hot encoding) per timestep.
