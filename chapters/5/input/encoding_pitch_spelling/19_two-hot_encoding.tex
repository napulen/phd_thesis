% Copyright 2022 Néstor Nápoles López

One of the problems with the previous representation of 35
features is that the it cannot encode any notes beyond two
flats or two sharps. Additionally, the resulting vector is
almost three times larger than a similar vector encoding
pitch classes without spelling:

\begin{equation}
    pitchClasses = {0, 1, 2, 3, 4, 5, 6, 7, 8, 9}
\end{equation}

The representation proposed here reduces some of these
problems. When an enharmonic equivalence is assumed, there
is usually a set of 12 pitch classes used to represent all
the chromatic notes of a Western scale. In addition to this
set, known as ``pitch classes'', notes with a spelling have
a corresponding letter (e.g., $C\sharp$ is pitch class 1,
with the name $C$). Using this information, a pitch class
vector can be complemented with the note letter
representation of a particular note. This results in 19
features (12 pitch classes, plus 7 note letter names).
