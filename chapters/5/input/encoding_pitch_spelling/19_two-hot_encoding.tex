% Copyright 2022 Néstor Nápoles López

One of the problems with the previous representation of 35
features is that the it cannot encode any notes beyond two
flats or two sharps. Additionally, the resulting vector is
almost three times larger than the common representation
based on pitch classes, which has 12 features:

\begin{equation}
    \{0, 1, 2, 3, 4, 5, 6, 7, 8, 9, 10, 11\}
\end{equation}

The representation proposed here reduces some of the
problems with the representation of 35 features. Instead of
encoding sharps and flats explicitly, it extends the vector
pitch class representation, by also encoding the generic
note letter name:

\begin{equation}
    \{0, 1, 2, 3, 4, 5, 6, 7, 8, 9, 10, 11\} \cup \{C, D, E, F, G, A, B\}
\end{equation}

When enharmonic equivalence is assumed in a \gls{12tet}
system, a set of 12 pitch classes spans all the notes of the
Western chromatic scale. In addition to belonging to one of
these pitch classes, a spelled note has a corresponding
letter (e.g., $C\sharp$ is pitch class 1, with the name
$C$). Using this information, a pitch class vector can be
complemented with the generic note letter representation of
a particular note. This results in 19 features (12 pitch
classes, plus 7 note letter names).
