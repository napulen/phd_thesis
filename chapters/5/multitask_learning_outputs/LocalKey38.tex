% Copyright 2022 Néstor Nápoles López

The \gls{localkey38} is a classification task that predicts
the $\elkey$ component of a \gls{rna} annotation. It uses
the $\setkey$ vocabulary to encode one of the 38 key classes
available. In the \gls{rna} annotations of the ground truth,
a key $\elkey$ must be indicated at least once in the piece
(generally, at the beginning of the piece). If there are no
modulations, this will be the key throughout for all
timesteps. The \gls{localkey38} tends to remain unchanged for
longer periods of time than the \gls{tonicization38} task,
however, this depends on the specific dataset and the
musical conventions of the annotator. The prediction of the
\gls{localkey38} task can be used directly as the $\elkey$
part of the \gls{rna} annotation generated by the \gls{arna}
model. This will be discussed in \refsubsec{directmethod}.


% represents the key segmentations of the piece, such that a
% given timestep is interpreted in terms of the
% \gls{localkey38} in turn. % This is equivalent to the
% local key (see % \refsubsubsec{localkey}) in other
% approaches. The vocabulary of keys, $\setkey$ (see
% \refsec{thevocabularyofmusicalkeys}), comprises 38
% classes. The 38 classes are the result of inspecting all
% the modulations and tonicizations across the aggregated
% training set.
