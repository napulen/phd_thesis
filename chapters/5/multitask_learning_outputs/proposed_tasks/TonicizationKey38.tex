% Copyright 2022 Néstor Nápoles López

Generally, in other work \textcite{chen2018functional,
chen2019harmony, micchi2020not, micchi2021deep,
mcleod2021modular} there is only one ``key'' task to be
addressed. Applied chords and other fluctuations of key are
handled through the secondary degree (see
\refsubsubsec{secondarydegree}) or a similar task.

In the proposed tasks of this \thesisdiss{}, an alternative
approach is proposed based on two layers of key analysis.
The \gls{localkey38} task represents the key of a region of
music. By default, any Roman numeral label is interpreted in
reference to the \gls{localkey38}. However, for cases of
tonicization, a different key is encoded. The scope of this
key is uniquely for the current chord, and it is considered
a brief fluctuation of key. This is referred as the
\gls{tonicization38}.
