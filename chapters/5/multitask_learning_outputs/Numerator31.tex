% Copyright 2022 Néstor Nápoles López

The \gls{rn31} is a classification task that predicts the
numerator $\elnum$ of a \gls{rna} annotation. In a similar
way than the \gls{localkey38}, the \gls{rn31} can be used
directly to generate the $\elnum$ component of a \gls{rna}
annotation from the predictions of the \gls{crnn}. This will
be discussed further in \refsubsec{directmethod}.

% In a similar way than \gls{pcset121}, the \gls{rn31} task
% is also a generated vocabulary of chords. This task is
% equivalent to the \emph{primary degree} (see
% \refsubsubsec{conventionaltasks}). However, as there is no
% ``secondary degree'' (which has been replaced by the
% \gls{tonicization38} task), the name ``Roman numeral
% numerator'' was chosen, alluding also to the fact that
% secondary degrees are related to the ``fraction'' notation
% employed in \gls{rna} to indicate tonicizations.

% The task is constrained to only 31 classes. In
% \parencite{napoleslopez2021augmentednet}, the number of
% Roman numeral numerators was left unbounded during data
% preprocessing. The same paper also discussed that a subset
% of 75 classes spanned ~98\% of the annotations found in
% the training set. Here, that number has been constrained
% further, to allow only 31 possible Roman numeral classes.
