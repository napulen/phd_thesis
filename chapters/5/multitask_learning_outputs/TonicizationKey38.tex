% Copyright 2022 Néstor Nápoles López

In previous methods, such as \textcite{chen2018functional,
chen2019harmony, micchi2020not, micchi2021deep,
mcleod2021modular} there is generally only one ``Key'' task
to solve by the machine learning model. Applied chords and
other fluctuations of key are handled either by the
secondary degree task \parencite{chen2021, micchi2021deep}
or not handled at all \parencite{mcleod2021modular}.

In the proposed tasks of this dissertation, an alternative
approach is proposed based on two layers of key analysis.
The \gls{localkey38} task represents the key of a region of
music. By default, any Roman numeral label is interpreted in
reference to the \gls{localkey38}. However, for cases of
tonicization, a different key is encoded. The scope of a
tonicization key is uniquely the current chord, and it is
considered an brief fluctuation of key. This is referred as
the \gls{tonicization38}. The vocabulary of keys is the same
one used for \gls{localkey38}, $\vocabkey$.
