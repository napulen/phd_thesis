% Copyright 2022 Néstor Nápoles López

% In previous methods, such as \textcite{chen2018functional,
% chen2019harmony, micchi2020not, micchi2021deep,
% mcleod2021modular} there is generally only one task
% related to the ``key'' to solve by the machine learning
% model. Applied chords and other fluctuations of key are
% handled either by a ``secondary degree'' task
% \parencite{chen2021attend, micchi2021deep} or not modeled
% explicitly \parencite{mcleod2021modular}.

The \gls{tonicization38} task used in this \gls{mtl}
configuration is equivalent to the ``secondary degree''
tasks in other approaches
\parencite{chen2021attend,micchi2021deep}. However, instead
of encoding the tonicization as a scale degree $\elden \in
\setden$, the tonicization is encoded as a key
$\elden_\elkey \in \setkey$. 

% In the proposed tasks of this dissertation, an alternative
% approach is proposed based on two layers of key analysis.
% The \gls{localkey38} task represents the key of a region
% of music. By default, any Roman numeral label is
% interpreted in reference to the \gls{localkey38}. However,
% for cases of tonicization, a different key is encoded. The
% scope of a tonicization key is always the current
% timestep, and it is considered an instantaneous
% fluctuation of key (i.e., it does not affect future
% timesteps). This other task is referred as the
% \gls{tonicization38}. The vocabulary of keys is the same
% one used for \gls{localkey38}, $\setkey$.
