% Copyright 2022 Néstor Nápoles López

The \gls{bass35} class represents the chord tone acting as
the bass. Pitches of up to two flats and two sharps are
possible output classes. Notice that throughout this
\thesisdiss{}, the ``bass'' is considered a perceptual
phenomenon, which might not correspond with the lowest
sounding note encoded as input representation \gls{bass19}.
\reffig{bass_encoding} is a good example of this situation.
In the original score, the two-voice pattern of the bass in
the measures 1 and 2 could easily persuade the listener to
think that the lower G note is the bass throughout measure
3, however, it is only the lowest-sounding note during the
first eighth note. Thus, the \emph{bass} always refers to
the lowest note of the annotated chord.

The first four output representations, \gls{bass35},
\gls{tenor35}, \gls{alto35}, and \gls{soprano35}, are all
related to the realization of the annotated chord in a
\gls{closed-position}. Collectively, I refer to these four
classifiers as the \gls{satb35} tasks. Note that these four
tasks encode as the target class a spelled note. However,
unlike the \gls{bass19} and \gls{chroma19} input
representations, the spelled notes in the \gls{satb35}
classifiers have been encoded using the $\vocabps_{35}$
vocabulary, instead of $\vocabps_{19}$. This was done in
order to use the same loss function (sparse categorical
cross entropy) across all the multitask learning tasks,
which would not be possible for the two-hot encoding
required in $\vocabps_{19}$.

\reffig{satb_all} shows the realization of the annotated
chords that defines the target class of each classifier.
\reffig{satb_bass} shows an example of the \gls{bass35}
target label for this fragment of music.

\phdfigure[Example of the ``SATB'' classifiers, where each
classifier learns to predict one of the four notes in the
\gls{closed-position} realization of the chord]{satb_all}

\phdfigure[Example of the \gls{bass35} encoding for the
chords in \reffig{satb_all}]{satb_bass}


% \footnote{Note that the term ``bass'' is problematic. A
% person may hear a note as the ``bass'', and that may be or
% not the lowest-sounding note in the score at a given
% moment. When referring to the encoding of the input, I am
% always talking about the ``lowest-sounding note in the
% score'', but ``bass'' is used for brevity.}
