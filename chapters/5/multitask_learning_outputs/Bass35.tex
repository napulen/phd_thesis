% Copyright 2022 Néstor Nápoles López

The \gls{bass35} class represents the chord tone acting as
the bass. Pitches with up to \gls{doubleflat} and
\gls{doublesharp} accidentals are possible output classes,
because the $\setps_{35}$ method is used here, unlike the
$\setps_{19}$ in the input representations. The motivation
for this dual use of pitch spelling representations is to
use the same loss function for all multiclass classification
problems in the \gls{mtl} configuration. The duple encoding
required by the $\setps_{19}$ method requires a different
loss function, which was avoided as an implementation
practicality. Notice also that throughout this
\thesisdiss{}, the ``bass'' is considered a perceptual
phenomenon, which may or may not correspond with the
lowest-sounding note encoded in the input representation
\gls{bass19}. \reffig{bass_encoding} is a good example of
this situation. In the original score, the two-voice pattern
of the bass in measures 1 and 2 could easily persuade the
listener to think (or hear) the lower G note as the bass
throughout measure 3, however, it is only the
lowest-sounding note during the first eighth note. Thus, it
can simultaneously be the \emph{bass} of the chord
throughout measure 3, but not the lowest-sounding note.

The first four output representations, \gls{bass35},
\gls{tenor35}, \gls{alto35}, and \gls{soprano35}, are all
related to the realization of the annotated chord in a
\gls{closed-position}. Collectively, I refer to these four
classifiers as the \gls{satb35} tasks. Note that these four
tasks encode as the target class a spelled note. However,
unlike the \gls{bass19} and \gls{chroma19} input
representations, the spelled notes in the \gls{satb35}
classifiers have been encoded using the $\setps_{35}$
vocabulary, instead of $\setps_{19}$. This was done in order
to use the same loss function (sparse categorical cross
entropy) across all the multitask learning tasks, which
would not be possible for the two-hot encoding required in
$\setps_{19}$.

\reffig{satb_all} shows the realization of the annotated
chords that defines the target class of each classifier.
\reffig{satb_bass} shows an example of the \gls{bass35}
target label for this fragment of music.

\phdfigure[Example of the \gls{satb35} classifiers, where
each classifier learns to predict one of the four notes in
the \gls{closed-position} realization of the
chord]{satb_all}

\phdfigure[Example of the \gls{bass35} encoding for the
chords in \reffig{satb_all}]{satb_bass}


% \footnote{Note that the term ``bass'' is problematic. A
% person may hear a note as the ``bass'', and that may be or
% not the lowest-sounding note in the score at a given
% moment. When referring to the encoding of the input, I am
% always talking about the ``lowest-sounding note in the
% score'', but ``bass'' is used for brevity.}
