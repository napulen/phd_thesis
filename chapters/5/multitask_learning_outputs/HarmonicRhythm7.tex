% Copyright 2022 Néstor Nápoles López

The last classification task is the \gls{harmonicrhythm7}.
One of the main problems of the \gls{mtl} configuration,
especially with the larger number of tasks proposed here, is
that it is difficult to infer the segmentation of the chord
labels, because the different classifiers may segment the
score differently (e.g., if the \gls{pcset121} predicts a
change of chord in timestep $t$, but the \gls{rn31} does
not).

There have been multiple solutions proposed to deal with
this problem. \textcite{chen2021attend} explored a different
neural network architecture and metrics to improve the
segmentation. \textcite{micchi2021deep} proposed a
\gls{nade} layer to model the interdependency of the
different classification tasks. \textcite{mcleod2021modular}
and \textcite{wu2021melody} tackled the problem of chord
segmentation (or harmonic rhythm) by implementing dedicated
models. 

Here, another multiclass classifcation problem is proposed
instead. This has the advantage that its implementation is
much simpler than a dedicated model, and the training can be
done jointly with the other tasks in an end-to-end fashion.

The encoding of the \gls{harmonicrhythm7} is based on the
method for encoding onsets described in
\refsubsubsec{encodingmeasure,note,andchordonsets}, with a
vocabulary of 7 classes. The way that chord onsets are
encoded is by assigning one of those classes as a chord
onset and all others as the ``time elapsed since the last
chord'', in a similar way to how it is done in the
\gls{duration14} input representation (see
\refsubsubsec{measureandnoteonsets}). When the
\gls{harmonicrhythm7} predicts a chord change at timestep
$t$, the predictions of all other classifiers at timestep
$t$ are used to generate the \gls{rna} output label.
