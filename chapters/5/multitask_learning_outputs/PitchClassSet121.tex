% Copyright 2022 Néstor Nápoles López

The \gls{pcset121} is a classification task that predicts
the \gls{pcset} $\elpcset$ of a \gls{rna} annotation. The
\gls{pcset} can also be derived from the \gls{satb35} tasks.
However, this approach may lead to a set of predictions that
do not form a valid \gls{pcset} $\elpcset \in \setpcset$.
The advantage of the \gls{pcset121} is that it is always
bounded to a valid set of \gls{pcset}s, which is useful as a
``fallback'' method to always generate a chord annotation,
regardless of the behaviour of other tasks.

% Just as \gls{rna} lacks a generally agreed standardization,
% so do chord labels. Some standards have been proposed over
% the years, such as the one by \textcite{harte2005automatic},
% however, not all annotators adopt the same standard. Thus,
% the same chord can have multiple names. For example, a chord
% \textbf{C:}$\rn{ii}\rnsixfive{}$ (Dmin7 in first inversion)
% may sometimes be called an ``F6'' chord in certain musical
% circles. One way to disambiguate chords is to describe them
% as sets of pitch classes. The use of a mathematical set
% indicates that duplicate pitches are not taken into account,
% and making it a set of pitch classes indicates that
% enharmonic chords are collapsed into the same class. This
% collapses instances like Dmin7 and F6, but also more
% complicated examples, such as
% \textbf{c:}$\rn{Ger}\rnsixfive{}$ and
% \textbf{D\musFlat{}:}$\rn{V}\rnseven$, which are also
% enharmonic.

% A vocabulary of pitch class sets is proposed, where each
% pitch-class set represents a class of chord, collapsing all
% of its enharmonic spellings. The proposed vocabulary
% consists of 121 pitch-class sets, which were generated by
% automatically generating combinations of Roman numeral
% chords in different keys, and collecting the minimal set of
% pitch-class sets that spans all of the Roman numeral chords.

% \reffig{pcset_vocabulary} an example of the vocabulary of
% pitch-class sets.

% \phdfigureproxy[A graph of pitch class sets in different
%  key contexts, which results in 121
%  classes]{pcset_vocabulary}
