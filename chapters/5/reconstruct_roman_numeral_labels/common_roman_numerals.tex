% Copyright 2022 Néstor Nápoles López

In \textcite{napoleslopez2021augmentednet}, the \gls{rna}
labels were reconstructed using either the ``conventional''
method or a feature called the \emph{CommonRomanNumerals75}.

The \emph{CommonRomanNumerals75} indicated the 75-most
common classes of Roman numerals labels. These labels were
stripped from the inversion information, but not of
secondary degrees. That is, the list of 75 labels spanned
the 75-most common Roman numeral strings collapsed to their
root position form.

In this case, the reconstruction of Roman numerals was
slightly different.

\begin{equation}
\begin{split}
K[t] = localKey(t) \\
Rn^{75}[t] = commonRomanNumerals75(t) \\
I[t] = inversion(t)  \\
\end{split}
\end{equation}

Then 

\begin{equation}
\begin{split}
romanNumeral[t] = K[t] + Rn^{75}[t] + I[t]
\end{split}
\end{equation}
