% Copyright 2022 Néstor Nápoles López

The method by \textcite{chen2018functional} to reconstruct
the \gls{rna} labels relied on five of the six conventional
tasks described in \refsubsec{conventionaltasks}. In their
paper, \textcite{chen2018functional} did not discuss how
these predictions were processed to obtain Roman numeral
labels all the way to the output, however, it is possible to
use them in the following way.

Let $t$ be each timestep of the sequence and

\begin{equation}
K[t] = local_key(t)

D^p[t] = primary_degree(t)

D^s[t] = secondary_degree(t)

Q[t] = chord_quality(t)

I[t] = inversion(t)
\end{equation}

if $D^s[t] \neq null$ then

\begin{equation}
Rn^den[t] = D^s[t]

k = pitch_of(D^s[t], K[t])
\end{equation}

else
\begin{equation}
k = K[t]

Rn^num[t] = pitch_of(D^p[t], k) + Q[t] + I[t] 
\end{equation}
