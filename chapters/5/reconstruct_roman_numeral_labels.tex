% Copyright 2022 Néstor Nápoles López

Although the predictions provided by the multitask learning
model are important for an \gls{arna} system, there is
another important step, which consists of using those
predictions to reconstruct a Roman numeral label.

There is not one way to do this, because multiple tasks are
equivalent and provide similar information. For example, the
\emph{chord inversion} and the \gls{bass35} tasks, or the
\emph{secondary degree} and the \gls{tonicization38}. These
different tasks, however, often involve different musical
assumptions, or a different processing of other predictions.

In this section, three methods for processing the
predictions are presented. The first method, dubbed the
``conventional'' method, is the one introduced by
\textcite{chen2018functional}. The second method is a method
dubbed the ``common Roman numerals'' method, introduced in
\textcite{napoleslopez2021augmentednet}. Lastly, the third
method is the proposed method in this \thesisdiss{}, which
makes use of the nine multitask learning predictions
presented in \refsubsec{proposedtasks}.

\phdfigure[All the pitch-class sets associated with the key of C major]{pcset_key_to_rn}
\phdfigure[All the keys where the pitch-class set (0, 4, 7) has a role]{pcset_pcset_to_rn}
\phdfigure[All the keys that have a \gls{rna} of $\rn{I}$]{pcset_rn_to_key}
