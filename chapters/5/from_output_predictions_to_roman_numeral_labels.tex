% Copyright 2022 Néstor Nápoles López

Although the predictions provided by the \gls{crnn} are
crucial to determine the \gls{rna} labels generated by the
\gls{arna} model, there is another important step left,
which is to turn those predictions into \gls{rna} labels.
% Two methods for interpreting the predictions were
% designed: a preference rule and an approach based on
% cosine similarity and the algorithm in
% \refsec{analgorithmtoresolveromannumeralsfromapitch-classsetandkey}.

As mentioned in
\refsec{thestructureofaromannumeralanalysislabel}, a Roman
numeral label $\elrna \in \setrna$ could be defined as a
string with four components: a key $\elkey \in \setkey$, a
numerator $\elnum \in \setnum$, a tonicization (denominator)
$\elden \in \setden$, and an inversion $\elinv \in \setinv$,
as shown in \refeq{rna}. Thus, if those four pieces of
information are known, a \gls{rna} label that conforms to
this structure can be generated.

\begin{equation}
    \label{eq:rna}
    \elrna = \elkey : \elnum^{\elinv} \; / \; \elden
\end{equation}

\phdtable[Predictions from the \gls{crnn}]{crnn_predictions}

The \gls{mtl} configuration of the model is highly coupled
with the vocabularies defined for $\setkey$ (see
\refsec{thevocabularyofmusicalkeys}), $\setnum$ (see
\refsec{thevocabularyofromannumeralnumerators}), $\setden$
(see \refsec{thevocabularyofromannumeraldenominators}), and
$\setinv$ (see
\refsec{thevocabularyofarabicnumeralinversions}). Thus, the
output labels of the \gls{arna} model can be generated as
shown in \refeq{rna_pred}, where $\elkeypred \in \setkey$ is
the key prediction of the \gls{localkey38} classifier,
$\elnumpred \in \setnum$ is the Roman numeral numerator
prediction of the \gls{rn31} classifier, $\eldenpred \in
\setden$ is the tonicization prediction of the
\gls{tonicization38} classifier, $\elinvpred \in \setinv$ is
an inversion extracted from the prediction of the
\gls{bass35} classifier, and $\elrnapred \in \setrna$ is the
final \gls{rna} label prediction.

\begin{equation}
    \label{eq:rna_pred}
    \elrnapred_{\elonsetpred} = \elkeypred : \elnumpred^{\elinvpred} \; / \; \eldenpred
\end{equation}

The different components of the \gls{rna} string have a
hierarchical structure. The tonicization $\elden$ is
relative to the key $\elkey$. The numerator $\elnum$ is
relative to the tonicized key $\elden$. The inversion
$\elinv$ depends on the numerator $\elnum$, because the
syntax of the inversion is different for triads and for
seventh chords. Taking the hierarchical structure into
account, the components of the \gls{rna} label need to be
resolved in the following order:

\begin{enumerate}
    \item Key $\elkey$
    \item Tonicization $\elden$
    \item Numerator $\elnum$
    \item Inversion $\elinv$
\end{enumerate}

The following section describes the process of generating a
sequence of \gls{rna} labels given the predictions shown in
\reftab{crnn_predictions}. 

In addition to the direct method that can be used with the
\gls{mtl} predictions of this model, it might be useful to
generate Roman numeral labels when only a chord label and
key are provided. Some methods, for example, the one in
\textcite{mcleod2021modular} produce chord labels and keys,
thus a numerator $\elnum$ and a tonicization $\elden$ need
to be estimated from the chord label and key. In that
situation, an alternative method is provided, based on the
\algorithmrn{} algorithm, described in
\refsec{analgorithmtoresolveromannumeralsfromapitch-classsetandkey}.
This was the method used to estimate full \gls{rna} labels
from the predictions of the model in
\textcite{mcleod2021modular, chen2021attend} during the
evaluation (which is explained in
\refsec{evaluationagainstpreviousmodels})



The \gls{crnn} provides predictions every \gls{32nd} note.
Yet, most of these timesteps should not have a \gls{rna}
label generated. Therefore, the location of the chords
should be determined (i.e., the chord segmentation). In
order to determine which rows of predictions should be
considered \gls{rna} labels, the onset predictions
$\elonsetpred \in \setonset$ retrieved from the
\gls{harmonicrhythm7} classifier are used to filter the
predictions of the \gls{crnn}. After filtering out all rows
that are not \gls{rna} onsets, the number of rows in the
table of predictions is reduced, as shown in
\reftab{crnn_predictions_onsets}. One \gls{rna} label will
be generated for each of these rows.

\phdtable[Predictions from the \gls{crnn} that were
considered the onset of a \gls{rna} label by the
\gls{harmonicrhythm7} classifier]{crnn_predictions_onsets}

% This requires an approach for chord segmentation, where
% the onset predictions $\elonset \in \setonset$ starting
% from the predictions of the \gls{crnn} is to decide the
% chord segmentation. That is, which of the 640 timesteps
% should be considered the location of a \gls{rna} label.
% This is the first problem to be solved.


% Consider the following predictions coming from the
% \gls{crnn} for the same musical fragment shown in
% \reffig{schumann}.


% The design of the algorithm served two purposes: to
% standardize the resolution of tonicizations in \gls{rna}
% labels, and to guarantee that a \gls{rna} label would be
% generated even when the predictions of the network
% resulted nonsensical.



% There is not a unique way to do this, because multiple
% tasks provide overlapping information. For example, the
% \gls{satb35} tasks and the \gls{pcset121} task can both be
% used to retrieve the chord. These different tasks are
% often advantageous in different musical situations. They
% might also require a different postprocessing.

% Thanks to the approach followed to resolve \gls{rna}
% labels from \gls{pcset}s and keys, these are the main two
% pieces of information needed from the predictions of the
% neural network to generate a final \gls{rna} string. There
% are, however, a few exceptions.

% The \gls{satb35}, \gls{pcset121}, and \gls{rn31} tasks can
% all be used to retrieve the \gls{pcset}.

% The rule-based system works as a series of conditions.

% The \gls{satb35} tasks are preferred as the source of the
% chord. That is, if the pitches form a valid chord in the
% vocabulary, that chord is assumed to be the 
