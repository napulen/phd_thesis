% Copyright 2022 Néstor Nápoles López

Although the predictions provided by the \gls{crnn} are
crucial to determine the \gls{rna} labels generated by the
\gls{arna} model, there is another important step left,
which is to turn those predictions into \gls{rna} labels.
% Two methods for interpreting the predictions were
% designed: a preference rule and an approach based on
% cosine similarity and the algorithm in
% \refsec{analgorithmtoresolveromannumeralsfromapitch-classsetandkey}.

As mentioned in
\refsec{thestructureofaromannumeralanalysislabel}, a Roman
numeral label $\elrna \in \setrna$ located at onset
$\elonset \in \setonset$ could be defined as a string with
four components: a key $\elkey \in \setkey$, a numerator
$\elnum \in \setnum$, a tonicization (denominator) $\elden
\in \setden$, and an inversion $\elinv \in \setinv$, as
shown in \refeq{rna}. Thus, if those four pieces of
information are known (five if including the onset), a
\gls{rna} label that conforms to this structure can be
generated.

\begin{equation}
    \label{eq:rna}
    \elrna_{\elonset} = \elkey : \elnum^{\elinv} \; / \; \elden
\end{equation}

\phdtable[Predictions from the \gls{crnn}]{crnn_predictions}

The \gls{mtl} configuration of the model is highly coupled
with the vocabularies defined for $\setkey$ (see
\refsec{thevocabularyofmusicalkeys}), $\setnum$ (see
\refsec{thevocabularyofromannumeralnumerators}), $\setden$
(see \refsec{thevocabularyofromannumeraldenominators}), and
$\setinv$ (see
\refsec{thevocabularyofarabicromannumeralinversions}). Thus,
the output labels of the \gls{arna} model can be generated
as shown in \refeq{rna_pred}, where $\elonsetpred \in
\setonset$ is the onset prediction of the
\gls{harmonicrhythm7} classifier, $\elkeypred \in \setkey$
is the key prediction of the \gls{localkey38} classifier,
$\elnumpred \in \setnum$ is the Roman numeral numerator
prediction of the \gls{rn31} classifier, $\eldenpred \in
\setden$ is the tonicization prediction of the
\gls{tonicization38} classifier, $\elinvpred \in \setinv$ is
an inversion extracted from the prediction of the
\gls{bass35} classifier, and $\elrnapred \in \setrna$ is the
final \gls{rna} label prediction.

\begin{equation}
    \label{eq:rna_pred}
    \elrnapred_{\elonsetpred} = \elkeypred : \elnumpred^{\elinvpred} \; / \; \eldenpred
\end{equation}

The different components of the \gls{rna} string have a
hierarchical structure. The tonicization $\elden$ is
relative to the key $\elkey$. The numerator $\elnum$ is
relative to the tonicized key $\elden$. The inversion
$\elinv$ depends on the numerator $\elnum$, as the syntax of
the inversion is different for triads than for seventh
chords. Taking the hierarchical structure into account, the
components of the \gls{rna} label need to be resolved in the
following order:

\begin{enumerate}
    \item Onset $\elonset$
    \item Key $\elkey$
    \item Tonicization $\elden$
    \item Numerator $\elnum$
    \item Inversion $\elinv$
\end{enumerate}

The following sections describe the process of generating a
sequence of \gls{rna} labels considering the predictions
shown in \reftab{crnn_predictions}.


% Consider the following predictions coming from the
% \gls{crnn} for the same musical fragment shown in
% \reffig{schumann}.



% The key $\elkey$ corresponds to the prediction of the
% \gls{localkey38} classification task. The numerator
% $\elnum$ corresponds to the prediction of the \gls{rn31}
% classification task. The tonicization $\elden$ corresponds
% to the prediction of the \gls{tonicization38}
% classification task. The inversion $\elinv$ corresponds to
% the prediction of the \gls{bass35} note in the context of
% the numerator $\elnum$. An initial approach could be to
% use the predictions of the \gls{localkey38}, \gls{rn31},
% and \gls{tonicization38} tasks to generate the chord, and
% then \gls{bass35} to retrieve the inversion.

% In order to generate the final \gls{rna} labels, an
% algorithm was designed to resolve them based on three
% pieces of information: \begin{enumerate} \item a
% \gls{pcset} $\rho \in \setpcset$ \item a key $k \in
% \setkey$ \item an inversion $i \in$ $\{$ $0$, $1$, $2$,
% $3$ $\}$ \end{enumerate}

% With these three pieces of information, the algorithm
% described in
% \refsec{analgorithmtoresolveromannumeralsfromapitch-classsetandkey}
% will produce a Roman numeral numerator $\elnum \in
% \setnum$.

% The purpose of the algorithm is to retrieve a numerator
% $\elnum$, however, this information is already present in
% the \gls{rn31} task, which is one of the classification
% outputs of the \gls{mtl} configuration. Why do we need an
% algorithm, then?

% The design of the algorithm served two purposes: to
% standardize the resolution of tonicizations in \gls{rna}
% labels, and to guarantee that a \gls{rna} label would be
% generated even when the predictions of the network
% resulted nonsensical.



% There is not a unique way to do this, because multiple
% tasks provide overlapping information. For example, the
% \gls{satb35} tasks and the \gls{pcset121} task can both be
% used to retrieve the chord. These different tasks are
% often advantageous in different musical situations. They
% might also require a different postprocessing.

% Thanks to the approach followed to resolve \gls{rna}
% labels from \gls{pcset}s and keys, these are the main two
% pieces of information needed from the predictions of the
% neural network to generate a final \gls{rna} string. There
% are, however, a few exceptions.

% The \gls{satb35}, \gls{pcset121}, and \gls{rn31} tasks can
% all be used to retrieve the \gls{pcset}.

% The rule-based system works as a series of conditions.

% The \gls{satb35} tasks are preferred as the source of the
% chord. That is, if the pitches form a valid chord in the
% vocabulary, that chord is assumed to be the 
