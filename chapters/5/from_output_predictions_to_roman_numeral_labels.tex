% Copyright 2022 Néstor Nápoles López

Although the predictions provided by the \gls{crnn} are
crucial to determine the \gls{rna} labels generated by the
\gls{arna} model, there is another important step left,
which is to turn those predictions into \gls{rna} labels. In
order to generate the final \gls{rna} labels, an algorithm
was designed to resolve them based on three pieces of
information: 
\begin{enumerate}
    \item a \gls{pcset} $\rho \in \setpcset$
    \item a key $k \in \setkey$
    \item an inversion $i \in$ $\{$ $0$, $1$, $2$, $3$ $\}$
\end{enumerate}

With these three pieces of information, the algorithm
described in
\refsec{analgorithmtoresolveromannumeralsfromapitch-classsetandkey}
will produce a Roman numeral numerator $\eta \in \setnum$.

There is not a unique way to do this, because multiple tasks
provide overlapping information. For example, the
\gls{satb35} tasks and the \gls{pcset121} task can both be
used to retrieve the chord. These different tasks are often
advantageous in different musical situations. They might
also require a different postprocessing.

Thanks to the approach followed to resolve \gls{rna} labels
from \gls{pcset}s and keys, these are the main two pieces of
information needed from the predictions of the neural
network to generate a final \gls{rna} string. There are,
however, a few exceptions.

The \gls{satb35}, \gls{pcset121}, and \gls{rn31} tasks can
all be used to retrieve the \gls{pcset}.
