% Copyright 2022 Néstor Nápoles López

The second convolutional block processes all the sounding
note classes of each timestep. Colloquially, I refer to this
input as the ``spelled chroma'' input. This is because of
the similarity between this representation and the commonly
used ``chromagram'' representation in audio. The main
difference being, beyond being used in the symbolic domain,
that chromagram features often collapse enharmonic spellings
of the same pitch class, whereas in this case spelling is
taking into account. The encoding of the spelled chroma
input is done using the pitch spelling representation
discussed in \refsubsubsec{19two-hotencoding}. However, in
this case, the representation results in a multi-hot
encoding, as each spelled pitch class will result in a
two-hot encoding timestep (but several pitch classes will be
encoded in the same timestep). An example of this encoding
is shown in \reffig{chroma_encoding}.

\phdfigureproxy[An encoding of the spelled chroma input]{chroma_encoding}
