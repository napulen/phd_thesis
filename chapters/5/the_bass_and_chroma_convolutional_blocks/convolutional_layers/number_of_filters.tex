% Copyright 2022 Néstor Nápoles López

In a \gls{cnn} layer, the number of filters represents the
number of ``patterns'' that the network will learn from the
input. In the visual domain, these patterns could result in,
for example, edge-detection filters. In the musical domain,
the possible patterns are related with combinations of pitch
classes, pitch names, or onsets.

In this architecture, the number of filters is halved for
each consecutive \gls{cnn} layer. 

\begin{equation}
    \label{eq:filters_kernel}
    f_l = 2^{L - 1 - l}
    k_l = 2^{l}
\end{equation}

That is, the network is
allowed to capture more short-term patterns (i.e., with a
lower kernel size) and fewer long-term patterns (i.e., with
a higher kernel size). Preliminary experiments showed that
prioritizing short-term patterns in the \gls{cnn} layers
facilitated the network to learn certain features, such as
the ones related with chord segmentation, whereas
longer-term patterns benefitted key-finding tasks.
