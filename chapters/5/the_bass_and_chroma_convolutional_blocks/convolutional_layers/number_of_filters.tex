% Copyright 2022 Néstor Nápoles López

In a \gls{cnn} layer, the number of filters represents the
number of ``patterns'' that the network will detect in the
signal. In the visual domain, these patterns could result
in, for example, an edge-detection filter. In the musical
domain, the possible patterns are related with combinations
of pitch classes, pitch names, or durations.

In this architecture, the number of filters is halfed for
each consecutive \gls{cnn} layer. That is, the network is
allowed to capture more short-term patterns (i.e., with a
lower kernel size) and less long-term patterns (i.e., with a
higher kernel size). In preliminary experiments, I found
that this helped to model certain features, such as the
\gls{harmonicrhythm7} more effectively.
