% Copyright 2022 Néstor Nápoles López

The \gls{bass35} class represents the chord tone acting as
the bass. Pitches of up to two flats and two sharps are
possible output classes. Notice that throughout this
\thesisdiss{}, the ``bass'' is considered a perceptual
phenomenon, which might not correspond with the lowest
sounding note encoded as input representation \gls{bass19}.
\reffig{bass_adversarial_example} shows an example of why
the annotated/perceived bass might differ from the
lowest-sounding note at a given timestep.

\phdfigureproxy[Example of a ``bass'', which differs from
the lowest-sounding note]{bass_adversarial_example}

The first four output representations, \gls{bass35},
\gls{tenor35}, \gls{alto35}, and \gls{soprano35}, are all
related to the realization of the annotated chord in a
\gls{closed-position}. \reffig{closed_position_bass} shows
an example of the \gls{bass35} target label for a fragment
of music.

\phdfigureproxy[Example of the ``bass'' target pitch in a
\gls{closed-position} realization of a chord
annotation]{closed_position_bass}


% \footnote{Note that the
% term ``bass'' is problematic. A person may hear a note as
% the ``bass'', and that may be or not the lowest-sounding
% note in the score at a given moment. When referring to the
% encoding of the input, I am always talking about the
% ``lowest-sounding note in the score'', but ``bass'' is used
% for brevity.}
