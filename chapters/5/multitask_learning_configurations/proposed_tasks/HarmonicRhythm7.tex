% Copyright 2022 Néstor Nápoles López

The last task in this section is the \gls{harmonicrhythm7}.
One of the main problems of the multitask learning
configuration, especially with the larger number of tasks
proposed here, is that it is difficult to resolve the
segmentation of the chord labels, because the different
predictors may segment the score differently.

There have been multiple solutions proposed to deal with
this problem. \textcite{chen2021attend} explored a different
neural network architecture and metrics to improve the
segmentation, \textcite{micchi2021deep} proposed a
\gls{nade} layer to model the interdependency of the
different classification tasks, \textcite{mcleod2021modular}
and  \textcite{wu2021melody} tackled the problem of chord
segmentation (harmonic rhythm) by implementing dedicated
models. 

Here, another multitask classifcation problem is proposed
instead. This has the advantage that its implementation is
much simpler than a dedicated model, and the training can be
done jointly with the other tasks in an end-to-end fashion.
The overhead in the time it takes to train the network is
negligible and, more importantly, it does not seem to be
detrimental to the other classifcation problems. On the
contrary, it seems to improve the overall segmentation of
the other tasks, possibly because the network learns
representations that minimize the loss across all the tasks,
including the chord segmentation (harmonic rhythm).

Initially, this problem was addressed in
\textcite{napoleslopez2021augmentednet} as a binary
classification problem, where timesteps that coincided with
a chord onset were marked as 1s and the remaining timesteps
as 0s.

However, this approach lead to a highly imbalanced
distribution of classes, a recurring problem that has been
important in other tasks. In order to compensate for the
imbalanced distribution of classes, an alternative method is
proposed, based on 7 classes. The task of harmonic rhythm is
now phrased as ``annotate the duration elapsed since the
last chord''. This encoding makes use of the duration
encoding mentioned in \refsubsec{encodingduration}.
