% Copyright 2022 Néstor Nápoles López

It is often claimed that \glspl{rna} is ``chord finding plus
key finding''. A chord can be classified based on the four
\gls{closed-position} tasks presented before: \gls{bass35},
\gls{tenor35}, \gls{alto35}, and \gls{soprano35}. However,
this approach may lead to a set of predictions that do not
form a valid chord in the vocabulary. Another approach for
defining a chord is to collapse it into a set of pitch
classes.

Just as \gls{rna} lacks a generally agreed standardization,
so do chord labels. Some standards have been proposed over
the years, such as the one by \textcite{harte2005automatic},
however, not all annotators adopt the same standard. Thus,
the same chord can have multiple names. For example, a chord
C:$\rn{ii}\rnsixfive{}$ (Dmin7 in first inversion) may
sometimes be called an ``F6'' chord in certain musical
circles. One way to disambiguate chords is to describe them
as sets of pitch classes. The use of a mathematical set
indicates that duplicate pitches are not taken into account,
and making it a set of pitch classes indicates that
enharmonic chords are collapsed into the same class. This
collapses instances like Dmin7 and F6, but also more
complicated examples, such as c:$\rn{Ger}\rnsixfive{}$ and
D\musFlat{}:$\rn{V}\rnseven$, which are also enharmonic.

A vocabulary of pitch class sets is proposed, where each
pitch-class set represents a class of chord, collapsing all
of its enharmonic spellings. The proposed vocabulary
consists of 121 pitch-class sets, which were generated by
automatically generating combinations of Roman numeral
chords in different keys, and collecting the minimal set of
pitch-class sets that spans all of the Roman numeral chords.

\reffig{pcset_vocabulary} an example of the vocabulary of
pitch-class sets.

\phdfigureproxy[A graph of pitch class sets in different key
 contexts, which results in 121 classes]{pcset_vocabulary}
