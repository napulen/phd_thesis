% Copyright 2022 Néstor Nápoles López

One of the main contributions of this \thesisdiss{} is the
output representation of chords and tonal features. In
previous work \parencite{napoleslopez2021augmentednet}, we
showed that having additional tasks is sometimes beneficial,
especially in combination with the synthesized training
examples described in
\refsubsec{synthesisofartificialexamples}.

In this section, the ``conventional'' method of arranging
the multitask classification layers is compared against the
proposed method in this \thesisdiss{}.

The next section describes the final step for annotating
\gls{rna} labels, which involves processing the labels
predicted by the multitask classifiers.

% The proposed multitask layer configuration consists of nine
% multiclass classification problems: \gls{bass35},
% \gls{tenor35}, \gls{alto35}, \gls{soprano35},
% \gls{pcset121}, \gls{rn31}, \gls{localkey38},
% \gls{tonicization38}, and \gls{harmonicrhythm7}. The tasks
% have codenamed with their number of output classes
% appended.\footnote{I opted for this naming convetion to more
% easily track the evolution of the multiclass classification
% problems. Some of them, such as the \gls{pcset121} or
% \gls{rn31} went through various revisions, as the chord
% vocabulary was adjusted during data exploration and
% experiments. The names, definitions, and number of classes
% presented in this \thesisdiss{} represent the latest
% definition of these problems.}

% Each of these tasks is subsequently used to address
% different aspects of the \gls{rna} label reconstruction.
