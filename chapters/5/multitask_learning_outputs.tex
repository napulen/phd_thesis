% Copyright 2022 Néstor Nápoles López

After the recurrent layers, the last component of the
\gls{crnn} is a set of nine classification tasks, which are
arranged in a \gls{mtl} configuration. The \gls{mtl}
configuration is perhaps one of the main contributions of
this dissertation, as it defines the tonal attributes of
chords and keys that will be predicted by the network. In
previous work \parencite{napoleslopez2021augmentednet}, we
showed that having additional tasks is sometimes beneficial
to the performance of the network, especially in combination
with the synthesized training examples described in
\refsubsec{synthesisofartificialtrainingexamples}.

The proposed \gls{mtl} configuration consists of nine
multiclass classification problems: \gls{bass35},
\gls{tenor35}, \gls{alto35}, \gls{soprano35},
\gls{pcset121}, \gls{rn31}, \gls{localkey38},
\gls{tonicization38}, and \gls{harmonicrhythm7}. The tasks
are codenamed with their number of output classes
appended.\footnote{I opted for this naming convention to
more easily track the evolution of the multiclass
classification problems. Some of them, such as the
\gls{pcset121} or \gls{rn31} went through various revisions,
as the vocabularies were adjusted during data exploration
and experiments. The names, definitions, and number of
classes presented in this dissertation represent the latest
definition of these problems.} Each of the output tasks is
also tightly coupled with the musical vocabularies and
encoding methods introduced throughout the dissertation.
More specifically, the following vocabularies and encoding
methods:

\begin{enumerate}
    \item[] $\setps_{35}$ - The encoding method for spelled
    pitches with 35 classes, described in
    \refsubsubsec{encodingnoteswithspelling}.
    \item[] $\setonset$ - The encoding method for onsets
    with 7 classes, described in
    \refsubsubsec{encodingmeasure,note,andchordonsets}.
    \item[] $\setpcset$ - The vocabulary of 121
    \gls{pcset}s, described in
    \refsec{thevocabularyofpitch-classsets}.
    \item[] $\setnum$ - The vocabulary of 31 Roman numeral
    numerators, described in
    \refsec{thevocabularyofromannumeralnumerators}.
    \item[] $\setkey$ - The vocabulary of 38 keys, described
    in
    \refsec{thevocabularyofmusicalkeys}
\end{enumerate}

Each of these tasks is subsequently used to generate the
final \gls{rna} annotations in string form, such as
$\rnwton{\rnviiosev}{ii}$. The first four output
representations, \gls{bass35}, \gls{tenor35}, \gls{alto35},
and \gls{soprano35}, are all related to the realization of
the annotated chord in a \gls{closed-position}.
Collectively, I refer to these four classifiers as the
\gls{satb35} tasks. Note that these four tasks encode as the
target class a spelled note. However, unlike the
\gls{bass19} and \gls{chroma19} input representations, the
spelled notes in the \gls{satb35} classifiers have been
encoded using the $\setps_{35}$ vocabulary, instead of
$\setps_{19}$. This was done in order to use the same loss
function (sparse categorical cross entropy) across all the
\gls{mtl} tasks, which would not be possible for the two-hot
encoding required in $\setps_{19}$. \reffig{satb_all} shows
the realization of the chords in the ground truth, which
define the target class of each \gls{satb35} classifier.

\phdfigure[Example of the \gls{satb35} classifiers, where
each classifier learns to predict one of the four notes in
the \gls{closed-position} realization of the
chord]{satb_all}
