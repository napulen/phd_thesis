% Copyright 2022 Néstor Nápoles López

After the recurrent layers, the last component of the
\gls{crnn} is a set of nine classification tasks, which are
arranged in a \gls{mtl} configuration. The \gls{mtl}
configuration is perhaps one of the main contributions of
this dissertation, as it defines the tonal attributes of
chords and keys that will be predicted by the network. In
previous work \parencite{napoleslopez2021augmentednet}, we
showed that having additional tasks is sometimes beneficial
to the performance of the network, especially in combination
with the synthesized training examples described in
\refsubsec{synthesisofartificialtrainingexamples}.

The proposed \gls{mtl} configuration consists of nine
multiclass classification problems: \gls{bass35},
\gls{tenor35}, \gls{alto35}, \gls{soprano35},
\gls{pcset121}, \gls{rn31}, \gls{localkey38},
\gls{tonicization38}, and \gls{harmonicrhythm7}. The tasks
are codenamed with their number of output classes
appended.\footnote{I opted for this naming convention to
more easily track the evolution of the multiclass
classification problems. Some of them, such as the
\gls{pcset121} or \gls{rn31} went through various revisions,
as the vocabularies were adjusted during data exploration
and experiments. The names, definitions, and number of
classes presented in this dissertation represent the latest
definition of these problems.} Each of these tasks is
subsequently used to generate the final \gls{rna}
annotations in string form, such as
$\rnwton{\rnviiosev}{ii}$.
