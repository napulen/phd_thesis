% Copyright 2022 Néstor Nápoles López

According to the count performed in this \thesisdiss{}, the
\gls{haydnsun} dataset has a total duration in quarter notes
of 9,095, across 2,921 measures. After preprocessing and
preparation, the \gls{abc} dataset contributed 5,357
\gls{rna} chord annotations, which have an average harmonic
rhythm of 1.7 quarter notes ($\musQuarter$).

The distribution of the \gls{rna} annotations and their
inversions are shown in \reffig{haydnsun_rn_inversion}.
Something surprising among the \gls{rna} distribution of
this dataset is that all of the $\rnIt$ and $\rnGer$
annotations have been encoded as in ``root position'', which
is unusual for these types of chords. Most likely, this
could be a mistranslation of the annotations for this
particular dataset. Due to the time-consuming nature of such
translations, an exploration of the translation workflow is
left for future work. If the annotations are correctly
labeling a $\rnIt$ or $\rnGer$ chord, they will remain
useful for classifying the appropriate \gls{pcset} and key
context.


\phdfigure[All the \gls{rna} labels taken from the
\gls{haydnsun} dataset. Each bar indicates the counts of the
Roman numeral class in different
inversions]{haydnsun_rn_inversion}



\phdfigure[Intersection between the \gls{pcset}s in this
dataset and the \gls{pcset}s in all other
datasets][0.5]{haydnsun_pcset_venn}



\phdfigure[All the keys spanned by the \gls{rna} annotations
of the \gls{haydnsun} dataset. For each key, the counts
indicate which ones correspond to modulations (local key
regions) and tonicizations ]{haydnsun_key_mod_ton}
