% Copyright 2022 Néstor Nápoles López

According to the count performed in this \thesisdiss{}, the
\gls{bps} dataset has a total duration in quarter notes of
23,540, across 7,080 measures. After preprocessing and
preparation, the \gls{bps} dataset contributed 10,584
\gls{rna} chord annotations, which have an average harmonic
rhythm of 2.22 quarter notes ($\musQuarter$).

The distribution of the \gls{rna} annotations and their
inversions are shown in \reffig{bps_rn_inversion}. The
\gls{bps} dataset is one of the two datasets (the other one
being \gls{haydnsun}) that do not provide examples of
$\rnCad$ chords. It is unlikely that this means there is no
presence of them, but instead that they have been annotated
as either $\rn{I}\rnsixfour$ or $\rn{V}\rnsixfour$. This is
one of the reasons why the $\rnCad$ symbol is useful,
because then these chords can be disambiguated.

\phdfigure[All the \gls{rna} labels taken from the \gls{bps}
dataset. Each bar indicates the counts of the Roman numeral
class in different inversions]{bps_rn_inversion}

\phdfigure[All the keys spanned by the \gls{rna} annotations
of the \gls{bps} dataset. For each key, the counts indicate
which ones correspond to modulations (local key regions) and
tonicizations ]{bps_key_mod_ton}

\reffig{bps_key_mod_ton} shows the distribution of the keys
in the \gls{bps} dataset. The dataset spans a total of 34
keys out of the 38 in the vocabulary. The distribution of
keys is slightly skewed towards the ``flatter'' side of the
line-of-fifths, up to the last key in the vocabulary,
\keyBff{}. The occurrence of the flatter keys is mostly in
the form of tonicizations. 
