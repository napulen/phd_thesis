% Copyright 2022 Néstor Nápoles López

According to the count performed in this \thesisdiss{}, the
\gls{haydnsun} dataset has a total duration in quarter notes
of 2,107, across 548 measures. After preprocessing and
preparation, the \gls{abc} dataset contributed 1,415
\gls{rna} chord annotations, which have an average harmonic
rhythm of 1.49 quarter notes ($\musQuarter$).

The distribution of the \gls{rna} annotations and their
inversions are shown in \reffig{kmt_rn_inversion}. This
dataset lacks examples of \gls{rna} classes more than any
other dataset considered. The majority of the annotations
being $\rnI$, $\rni$, $\rnV$, and $\rnVsev$ chords. However,
there is a considerable amount of $\rnCad$ annotations and
an unusually larger number of \gls{neapolitan} ($\rnN$)
chords. Because of the nature of this dataset (i.e.,
examples of modulations in music theory textbooks), it is
possible that the focus of the examples has been on
modulations across different keys, rather than exotic chord
progressions. The distribution of keys shown below supports
this ideas.

% \phdfigure[All the \gls{rna} labels taken from the
% \gls{kmt} dataset. Each bar indicates the counts of the
% Roman numeral class in different
% inversions]{kmt_rn_inversion}



\phdfigure[Intersection between the \gls{pcset}s in this
dataset and the \gls{pcset}s in all other
datasets][0.5]{kmt_pcset_venn}



\phdfigure[All the keys spanned by the \gls{rna} annotations
of the \gls{kmt} dataset. For each key, the counts indicate
which ones correspond to modulations (local key regions) and
tonicizations ]{kmt_key_mod_ton}

