% Copyright 2022 Néstor Nápoles López

According to the count performed in this \thesisdiss{}, the
\gls{kmt} dataset has a total duration in quarter notes of
2,107, across 548 measures. After preprocessing and
preparation, the \gls{kmt} dataset contributed 1,415
\gls{rna} chord annotations, which have an average harmonic
rhythm of 1.49 quarter notes ($\musQuarter$).

The distribution of the \gls{rna} annotations and their
inversions are shown in \reffig{kmt_rn_inversion}. This
dataset has the smallest \gls{rna} vocabulary among all
datasets considered. The majority of the annotations being
$\rnI$, $\rni$, $\rnV$, and $\rnVsev$ chords. However, there
are nearly 50 examples of $\rnCad$ annotations and an
unusually large number, nearly 70, of \gls{neapolitan}
chords,~$\rnN$. Due to the nature of this dataset, which
consists of examples of modulations in music theory
textbooks, it is possible that the focus of the examples has
been on modulations across different keys, rather than
exotic chord progressions. The distribution of keys
discussed below supports this idea.

\phdfigure[All the \gls{rna} labels taken from the \gls{kmt}
dataset. Each bar indicates the counts of the Roman numeral
class in different inversions]{kmt_rn_inversion}

\phdfigure[All the keys spanned by the \gls{rna} annotations
of the \gls{kmt} dataset. For each key, the counts indicate
which ones correspond to modulations (local key regions) and
tonicizations ]{kmt_key_mod_ton}

\reffig{kmt_key_mod_ton} shows the distribution of the keys
in the \gls{kmt} dataset. The dataset spans a total of 37
keys out of the 38 in the vocabulary, missing only the
occurrence of the key of \keyBbb{}. Among all datasets, this
is perhaps the most interesting in terms of the key
distribution. It was discussed above that the vocabulary of
Roman numerals in this dataset was visibly smaller than in
all other datasets. The opposite happens with the
distribution of keys, especially when considering the
presence of a key in the context of a modulation. This
dataset is more uniform and the use of the vocabulary more
extensive. This is probably because the dataset is a
compilation of textbook examples of modulations.
Particularly, one of the sources of this dataset,
\textcite{reger1904supplement}, contains an extensive set of
modulations in different, sometimes unusual, keys. This is
reflected in the key vocabulary observed here.
