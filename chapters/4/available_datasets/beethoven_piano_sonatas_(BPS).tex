% Copyright 2022 Néstor Nápoles López

% This is \refsubsec{beethovenpianosonatas(bps)}, which
% introduces the beethoven piano sonatas (bps).

The \gls{bps} dataset was introduced by
\textcite{chen2018functional}. It consists of annotations
for all the first movements of piano sonatas by Beethoven.
The annotations were provided by an expert musicologist,
according to the authors. The process for encoding the
annotations was divided in four steps: 1) identify the local
key, 2) decide the segmentation of the chords (i.e., where
one chord ends and the next one begins), 3) labeling the
chord, taking into account the absence of chord tones
and/or presence of nonchord tones, and 4)
specifying the chord inversion. Out of these steps, the
authors mention that step 3 was particularly complicated,
because of the factors that need to be taken into account. 

The \gls{bps} dataset contains 32 files (i.e., piano sonata
movements), 86,950 notes, 7,394 Roman numeral labels, and
531 key modulations.