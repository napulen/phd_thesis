% Copyright 2022 Néstor Nápoles López

According to the count performed in this \thesisdiss{}, the
\gls{abc} dataset has a total duration in quarter notes of
48,034, across 15,746 measures. After preprocessing and
preparation, the \gls{abc} dataset contributed 29,427
\gls{rna} chord annotations, which have an average harmonic
rhythm of 1.63 quarter notes ($\musQuarter$).

The distribution of the \gls{rna} annotations and their
inversions are shown in \reffig{abc_rn_inversion}. As
expected, all of the $\rnCad$ annotations correspond to a
``second inversion'' position. While not all of the datasets
provide this type of annotation (cadential six-four chords)
this dataset provides nearly 110 examples of them. The
\gls{abc} dataset is also one of the few datasets (the
others being \gls{tavern} and \gls{wir}) that provide
examples of $\rnIIIaugsev$ chords, which is the least common
type of Roman numeral of the vocabulary.

\phdfigure[All the \gls{rna} labels taken from the \gls{abc}
dataset. Each bar indicates the counts of the Roman numeral
class in different inversions]{abc_rn_inversion}

Regarding the distribution of the keys, the \gls{abc}
dataset spans a total of 36 keys out of the 38 in the
vocabulary, making it one of the datasets with more
representation of the keys in the vocabulary. It is only
outnumbered by the \gls{kmt} and \gls{wir}, which span 37
keys each. Most of the keys in \gls{abc} lie within the
center of the line of fifths (i.e., few accidentals), as
expected. However, there is an unusually large count of the
\keyDs key.

\phdfigure[All the keys spanned by the \gls{rna} annotations
of the \gls{abc} dataset. For each key, the counts indicate
which ones correspond to modulations (local key regions) and
tonicizations ]{abc_key_mod_ton}

