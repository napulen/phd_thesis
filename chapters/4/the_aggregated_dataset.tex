% Copyright 2022 Néstor Nápoles López

The distribution of the \gls{rna} annotations and their
inversions are shown in \reffig{aggregated_rn_inversion}.

\phdfigure[All the \gls{rna} labels in the aggregated
dataset. Each bar indicates the counts of the Roman numeral
class in different inversions]{aggregated_rn_inversion}

\phdfigure[All the keys spanned in the aggregated dataset.
For each key, the counts indicate which ones correspond to
modulations (local key regions) and tonicizations
]{aggregated_key_mod_ton}

\reffig{aggregated_key_mod_ton} shows the distribution of
the keys in the aggregated dataset. 

After the data aggregation process, the next step is to
split the data into training, validation, and test sets that
can be used to run supervised learning experiments.
