% Copyright 2022 Néstor Nápoles López

The distribution of the \gls{rna} annotations and their
inversions are shown in \reffig{aggregated_rn_inversion}.

\phdfigure[All the \gls{rna} labels in the aggregated
dataset. Each bar indicates the counts of the Roman numeral
class in different inversions]{aggregated_rn_inversion}

\phdfigure[All the keys spanned in the aggregated dataset.
For each key, the counts indicate which ones correspond to
modulations (local key regions) and tonicizations
]{aggregated_key_mod_ton}

\reffig{aggregated_key_mod_ton} shows the distribution of
the keys in the aggregated dataset. 

% As noted throughout \refsec{availabledatasets}, the
% available datasets for \glspl{rna} are encoded in
% different digital formats, which are sometimes very
% different to each other. Because \gls{rna} is not a
% standardized practice across music schools, countries, or
% languages, the distinctions between formats may be beyond
% technical, they might also be musically distinct and
% encode chords using different musical conventions.

% This introduces some problems, both musical and technical,
% which need to be addressed to aggregate all the datasets
% into a unified set.
% \refsubsec{standardizingthenotationbetweendatasets}
% describes in more detail some of the issues to overcome in
% order to aggregate the datasets. Then,
% \refsubsec{data-curationmetrics} describes a solution to
% some of these problems, namely, a set of objective metrics
% that semi-automate the detection, correction, and quality
% assurance of the data.

% After the curation process, the only remaining step is to
% split the data into chunks that can be used for running
% supervised learning experiments. That step is described in
% \refsubsec{generatingtraining,validation,andtestsplits}.

% Once all datasets have been aggregated, standardized, and
% split into three sets, the dataset is ready for training
% the model and running the experiments.

% The last step, described in \refsec{dataaugmentation}, is
% to perform data augmentation on the training set.

% Before that, in this section, I describe the
% characteristics of the aggregated training, validation,
% and test splits of the dataset.

% \phdfigure[All the \gls{rna} labels taken from the
% aggregated dataset. Each bar indicates the counts of the
% Roman numeral class in different
% inversions]{aggregated_rn_inversion}

% \phdfigure[All the keys spanned by the \gls{rna}
% annotations of the aggregated dataset. For each key, the
% counts indicate which ones correspond to modulations
% (local key regions) and tonicizations
% ]{aggregated_key_mod_ton}
