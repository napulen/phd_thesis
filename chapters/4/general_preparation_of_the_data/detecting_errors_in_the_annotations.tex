% Copyright 2022 Néstor Nápoles López

% Some of the problems observed when standardizing existing
% datasets were that: 1) the annotations and scores were
% misaligned, 2) several chords were mislabeled (due to
% errors in the conversion or errors in the annotations
% themselves), and 3) the inversions of the chords were
% incorrectly annotated.

% Reviewing these issues manually requires the inspection of
% the musical scores and annotations, which is extremely
% time consuming. A semi-automatic workflow facilitated the
% manual inspection by pinpointing scores and portions of
% the score that were potentially containing errors. 

The process of inspecting each dataset to ensure that they
can be aggregated is very time consuming, and generally
leads to several corrections of errors. Performing this work
manually is unrealistic, as each of these datasets involves,
presumably, hundreds of hours of work by experts in tonal
harmony, which can doubtly be reviewed by a single person.

In order to lessen this effort, I developed a semi-automatic
workflow to identify potential issues with the quality of an
annotation file in a dataset. This semi-automatic workflow
is based on three objective metrics. The metrics are based
on common corrections that I needed to perform in datasets
that I aggregated manually, during the first days of the
project.

