% Copyright 2022 Néstor Nápoles López



\phdparagraph{alignment between scores and annotations}

When the annotations and scores are not perfectly aligned,
this results in a very negative effect, where a model learns
the wrong chord associations for a musical context. I found
this one of the most pressing problems to be addressed when
aggregating the available datasets. However, it is also very
difficult to detect.

An initial approach is to verify that the duration and
numbering of a measure matches between the annotation and
score files. If a measure number changes inconsistently
between the two, this might indicate a misalignment.

\phdparagraph{pitch correspondence between chord annotations and scores}

Although the presence of \gls{nonchord}s and absence of
chord tones can make it difficult to automatically detect
whether a chord has been mislabeled, there is usually some
shared pitch content between the annotation and the score.

Whenever the correspondence of the pitch content and the
chord annotation is unusually low, this might be an
indication that the chord has been mislabeled. Designing a
routine to compare this pitch correspondence helps detecting
such instances (e.g., a $\pitchC$ chord has been mislabeled as a
$C\sharp$ chord, which is unambiguosly incorrect). The most
common version of this problem is an annotator indicating
the wrong key at some point of the \gls{rna} annotations,
and all subsequent chords have an unusually low pitch
correspodence with the score (i.e., they are relative to the
wrong key).

\phdparagraph{correspondence between lowest-sounding note and bass of the chord}


One of the most common errors done by harmonic analysis
annotators (even if they are experts) is to mislabel the
inversion of the chord. This happens likely because the
annotators pay less attention to this property of the chord
than, for example, the root. In certain genres, such as
chorales or string quartets, a higher voice may cross the
bass momentarily, confusing the annotator into thinking that
the lowest sounding note of the chord is in the bass voice,
when it is in fact not the case.

These inconsistencies can be verified in a similar way to
the pitch correspondence. An unusually low match between the
annotated bass and the one in the score may indicate that
the annotator mislabeled the inversion of the chord.
