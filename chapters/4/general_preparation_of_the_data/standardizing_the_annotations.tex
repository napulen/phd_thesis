% Copyright 2022 Néstor Nápoles López

In order to facilitate the aggregation of all datasets,
\gls{romantext} was chosen as the ``container'' digital
format for all \gls{rna} annotations. However, the
underlying vocabulary of Roman numerals is the one described
in \refsec{thevocabularyofromannumerals}, which consists of
31 classes of Roman numerals. This helps to perform the
supervised learning workflow in a more controlled manner
during training and evaluation.

Regarding \gls{romantext}, there are a few motivations for
choosing it as the container format for all annotation
files:

\begin{enumerate} 
    \item It is a stand-alone format, which does not require
    access to the musical score; this is useful for datasets
    where the scores are not provided, in which the
    annotations can be aligned with a third-party score
    \item An existing effort has already been done in the
    \gls{wir} dataset
    \parencite{gotham2019romantext, gotham2022openscore}
    to convert other datasets into \gls{romantext}. Thus,
    taking advantage of that effort is desirable 
    \item The \gls{romantext} has been closely integrated to
    the music21 Python library
    \parencite{cuthbert2010music21}, which is one of the
    most advanced software tools for processing musical
    scores, and facilitates the integration of annotations
    with \gls{musicxml} files, which is the symbolic music
    format of most music scores 
\end{enumerate}

Beyond the conversion to \gls{romantext}, one of the
challenges of aggregating the different datasets is that
errors are introduced at different stages of the process.
Some of the problems observed were that: 1) the annotations
and scores were misaligned, 2) several chords were
mislabeled (due to errors in the conversion or errors in the
annotations themselves), and 3) the inversions of the chords
were incorrectly annotated.


