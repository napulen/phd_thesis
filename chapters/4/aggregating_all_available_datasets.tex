% Copyright 2022 Néstor Nápoles López

As noted throughout \refsec{availabledatasets}, the
available datasets for \glspl{rna} are encoded in different
digital formats, which are sometimes very different to each
other. Because \gls{rna} is not a standardized practice
across music schools, countries, or languages, the
distinctions between formats may be beyond technical, they
might also be musically distinct and encode chords using
different musical conventions.

This introduces some problems, both musical and technical,
which need to be addressed to aggregate all the datasets
into a unified set.
\refsubsec{standardizingthenotationbetweendatasets}
describes in more detail some of the issues that need to be
overcomed in the aggregation process. Furthermore,
\refsubsec{data-curationmetrics} introduces a set of
objective metrics that have facilitated me the detection,
correction, and quality assurance of the data.

After the curation process, the only remaining step is to
split the data into chunks that can be used for running
supervised learning experiments. That step is described in
\refsubsec{generatingtraining,validation,andtestsplits}.