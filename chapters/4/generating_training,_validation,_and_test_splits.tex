% Copyright 2022 Néstor Nápoles López

In supervised learning machine learning methods, it is
customary to evaluate the generalization of a model on an
unseen portion of data. This is necessary as models tend to
\gls{overfit} on the data they are trained with.

Furthermore, there are several ways to designate that
portion of unseen data, such as: \emph{k-fold} cross
validation; a training and validation sets; or a training,
validation, and test sets. 

In this research, I opted for training, validation, and test
splits. The training split is the data used for training the
model. The validation split is to verify the generalization
of the model while tuning hyper-parameters, designing the
neural network architecture, or making other decisions
affecting the model. The test set is an unseen portion of
data that remains unused throughout the entire process of
designing model, and it is used once (and only once) to
evaluate the generalization of the final model.

In some datasets, such as \gls{bps}, these splits were
already provided by other researchers. In those instances,
the splits were respected for direct comparison with their
models. When not specified, the splits were generated
randomly, providing around 70\% of the data for training,
15\% for validation, and 15\% for testing.

Once all datasets have been aggregated, standardized, and
split into three sets, the dataset is ready for training the
model and running the experiments.

The last step, described in \refsec{dataaugmentation}, is to
perform data augmentation on the training set.
