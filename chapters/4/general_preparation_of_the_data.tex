% Copyright 2022 Néstor Nápoles López


The \gls{arna} system developed for this dissertation was
trained from publicly available \gls{rna} annotations in a
digital format. 

\gls{arna} datasets often  differ in their size, format, and
musical genre. One of the main characteristics that
distinguishes them from, for example, chord labeling
datasets, is that \gls{rna} annotations require a clear
indication of the local key (see \refsubsec{keyestimation}).
Thus, knowing the key is a necessity for any dataset to be
considered a \gls{rna} dataset. Preferrably, a full Roman
numeral annotation in string form as described by one of the
digital standards (described in
\refsubsec{symbolicmusicformats}). However, this is not
always provided, or the format of the \gls{rna} string
notation in one digital standard is not c

ompatible with another digital standard. For this reason,
the approach taken in this dissertation is to standardize
all the annotations based on two pieces of information: the
key(s) provided, and the \gls{pcset} representation of the
chord annotation. This is achieved through the method
described in
\refappendix{amethodforsystematicromannumeralanalysis}.
