% Copyright 2022 Néstor Nápoles López

The \gls{arna} system developed for this dissertation,
\gls{augmentednet}, was trained and evaluated with publicly
available digital \gls{rna} annotations. 

\gls{arna} datasets often  differ in their size, format, and
musical genre. One of the main characteristics that
distinguishes them from, for example, chord labeling
datasets, is that \gls{rna} annotations require a clear
indication of the key (see \refsubsec{keyestimation}). Thus,
knowing the key is a necessity for any dataset to be
considered a \gls{rna} dataset. Preferrably, a full Roman
numeral annotation in string form, such as the ones
described in the digital standards (see
\refsubsec{symbolicmusicformats}) is provided as well.
However, this is not always the case, or the \gls{rna}
strings in one digital standard may be incompatible with the
ones of another digital standard. For this reason, the
approach taken in this dissertation is to standardize all
the annotations based on two pieces of information: the
key(s) provided, and the \gls{pcset} representation of the
chord annotation. This is achieved through the method
described in
\refappendix{amethodforsystematicromannumeralanalysis}.

% As noted throughout \refsec{availabledatasets}, the
% available datasets for \gls{arna} are encoded in different
% digital formats, which are sometimes very different to
% each other. Because \gls{rna} is not a standardized
% practice across music schools, countries, or languages,
% the distinctions between formats may be beyond technical,
% they might also be musically distinct and encode chords
% using different musical conventions.

% This introduces some problems, both musical and technical,
% which need to be addressed to aggregate all the datasets
% into a unified set.
% \refsubsec{standardizingthenotationbetweendatasets}
% describes in more detail some of the issues to overcome in
% order to aggregate the datasets. Then,
% \refsubsec{data-curationmetrics} describes a solution to
% some of these problems, namely, a set of objective metrics
% that semi-automate the detection, correction, and quality
% assurance of the data.
