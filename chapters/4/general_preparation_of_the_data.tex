% Copyright 2022 Néstor Nápoles López


The \gls{arna} system developed for this dissertation was
trained and evaluated with publicly available digital
\gls{rna} annotations. 

\gls{arna} datasets often  differ in their size, format, and
musical genre. One of the main characteristics that
distinguishes them from, for example, chord labeling
datasets, is that \gls{rna} annotations require a clear
indication of the local key (see \refsubsec{keyestimation}).
Thus, knowing the key is a necessity for any dataset to be
considered a \gls{rna} dataset. Preferrably, a full Roman
numeral annotation in string form, such as the ones
described in the digital standards (see
\refsubsec{symbolicmusicformats}) is provided as well.
However, this is not always the case, or the \gls{rna}
strings in one digital standard may be incompatible with the
ones of another digital standard. For this reason, the
approach taken in this dissertation is to standardize all
the annotations based on two pieces of information: the
key(s) provided, and the \gls{pcset} representation of the
chord annotation. This is achieved through the method
described in
\refappendix{amethodforsystematicromannumeralanalysis}.
