% Copyright 2022 Néstor Nápoles López

According to the count performed in this \thesisdiss{}, the
\gls{tavern} dataset has a total duration in quarter notes
of 40,899, across 15,534 measures. After preprocessing and
preparation, the \gls{tavern} dataset contributed 24,544
\gls{rna} chord annotations, which have an average harmonic
rhythm of 1.67 quarter notes ($\musQuarter$).

The distribution of the \gls{rna} annotations and their
inversions are shown in \reffig{tavern_rn_inversion}. The
\gls{tavern} dataset is one of the few datasets (the others
being \gls{abc} and \gls{wir}) that provide examples of
$\rnIIIaugsev$ chords, which are the least common type of
chord in the vocabulary.

\phdfigure[All the \gls{rna} labels taken from the
\gls{tavern} dataset. Each bar indicates the counts of the
Roman numeral class in different
inversions]{tavern_rn_inversion}

\phdfigure[All the keys spanned by the \gls{rna} annotations
of the \gls{tavern} dataset. For each key, the counts
indicate which ones correspond to modulations (local key
regions) and tonicizations ]{tavern_key_mod_ton}

\reffig{tavern_key_mod_ton} shows the distribution of the
keys in the \gls{tavern} dataset. The dataset spans a total
of 27 keys out of the 38 in the vocabulary. After \gls{mps},
it has the second-smallest key vocabulary among the datasets
presented here. The vocabulary is also slightly skewed
towards the ``flatter'' side of the \gls{lineoffifths}. The
``sharpest'' key that appears in a modulation is $\keyfs{}$ (3
sharps), whereas the flattest key is $\keyCb{}$ (7 flats).
