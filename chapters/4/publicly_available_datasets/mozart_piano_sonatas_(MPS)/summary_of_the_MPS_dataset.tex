% Copyright 2022 Néstor Nápoles López

According to the count performed in this \thesisdiss{}, the
\gls{mps} dataset has a total duration in quarter notes of
22,305, across 7,465 measures. After preprocessing and
preparation, the \gls{mps} dataset contributed 15,865
\gls{rna} chord annotations, which have an average harmonic
rhythm of 1.41 quarter notes ($\musQuarter$).

The distribution of the \gls{rna} annotations and their
inversions are shown in \reffig{mps_rn_inversion}. In this
dataset, all occurrences of $\rnN$ chords are in first
inversion, which is not the case for any other dataset.
Although this is also the most common occurrence of
\gls{neapolitan} chords, so it is consistent with the
musical practice of the period. The \gls{mps} dataset is
also one of two datasets lacking any examples of $\rnIIIaug$
chords (the other one being \gls{kmt}). Unlike \gls{kmt},
however, \gls{mps} does present examples of $\rnVaug$
chords. This is an interesting case, because a $\rnIIIaug$
chord is an enharmonic of a $\rnVaug$ chord in the same key,
as both are made of the same \gls{pcset} but with different
note spellings. For this reason, when designing the
vocabulary, I decided to make $\rnIIIaug$ an exclusive chord
of the minor mode, and $\rnVaug$ an exclusive chord of the
major mode, so both labels can coexist.\footnote{See
\refappendix{amethodforsystematicromannumeralanalysis} for
further discussion on $\rnIIIaug$ and $\rnVaug$ chords.}
What this indicates here, is that all occurrences of
augmented triads found in \gls{mps} occur in a major-key
context, which results in only $\rnVaug$ labels. Perhaps
this suggests that Mozart only used augmented triads in
major keys, although that musicological claim would require
further inspection beyond the scope of this dissertation.

\phdfigure[All the \gls{rna} labels taken from the \gls{mps}
dataset. Each bar indicates the counts of the Roman numeral
class in different inversions]{mps_rn_inversion}

\phdfigure[All the keys spanned by the \gls{rna} annotations
of the \gls{mps} dataset. For each key, the counts indicate
which ones correspond to modulations (local key regions) and
tonicizations ]{mps_key_mod_ton}

\reffig{mps_key_mod_ton} shows the distribution of the keys
in the \gls{mps} dataset. The dataset spans a total of 26
keys out of the 38 in the vocabulary, making it the dataset
with the smallest key vocabulary, among the datasets
presented here. As expected, most of the keys found in this
dataset lie within the center of the \gls{lineoffifths}. The
most extreme keys in the dataset are \keydb{} and \keyFs{},
both in the context of tonicizations.
