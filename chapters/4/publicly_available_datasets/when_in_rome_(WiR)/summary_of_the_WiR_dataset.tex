% Copyright 2022 Néstor Nápoles López

According to the count performed in this \thesisdiss{}, the
\gls{wir} dataset has a total duration in quarter notes of
29,951, across 9,099 measures. After preprocessing and
preparation, the \gls{wir} dataset contributed 17,734
\gls{rna} chord annotations, which have an average harmonic
rhythm of 1.69 quarter notes ($\musQuarter$).

The distribution of the \gls{rna} annotations and their
inversions are shown in \reffig{wir_rn_inversion}. This
dataset is one of the two datasets (the other one being
\gls{abc}) that provides at least one example of every Roman
numeral class in the vocabulary. In addition to that, the
\gls{wir} dataset is the only dataset that provides examples
in third inversion of every seventh chord in the vocabulary,
and the only dataset providing examples of any inversions at
all in a $\rnIIIaugsev$ chord.

\phdfigure[All the \gls{rna} labels taken from the \gls{wir}
dataset. Each bar indicates the counts of the Roman numeral
class in different inversions]{wir_rn_inversion}

\phdfigure[All the keys spanned by the \gls{rna} annotations
of the \gls{wir} dataset. For each key, the counts indicate
which ones correspond to modulations (local key regions) and
tonicizations ]{wir_key_mod_ton}

\reffig{wir_key_mod_ton} shows the distribution of the keys
in the \gls{wir} dataset. The dataset spans a total of 37
keys out of the 38 in the vocabulary $\setkey$ (see
\refsec{thevocabularyofmusicalkeys}). This is as many keys
as \gls{kmt} and more than any other dataset. The occurrence
of keys around the center of the \gls{lineoffifths} is
higher and it slowly smooths toward the extremes. The only
key of the vocabulary which is missing from this dataset is
$\keybs{}$. The distribution of modulations appears to be
less uniform than in \gls{kmt}. It is possible that this is
because the examples of \gls{wir} are real musical examples,
instead of textbook examples of modulations.
