% Copyright 2022 Néstor Nápoles López

According to the count performed in this \thesisdiss{}, the
\gls{haydnsun} dataset has a total duration in quarter notes
of 9,095, across 2,921 measures. After preprocessing and
preparation, the \gls{haydnsun} dataset contributed 5,357
\gls{rna} chord annotations, which have an average harmonic
rhythm of 1.7 quarter notes ($\musQuarter$).

The distribution of the \gls{rna} annotations and their
inversions are shown in \reffig{haydnsun_rn_inversion}.
Something surprising among the \gls{rna} distribution of
this dataset is that all of the $\rnIt$ and $\rnGer$
annotations have been encoded as in ``root position,'' which
is unusual for these types of chords. It is more likely that
this indicates a mistranslation of the annotations for this
particular dataset. Due to the cascading effects that a
re-translation of the dataset could have in other areas of
this dissertation (e.g., experimental evaluation), a
possible re-translation of the dataset is left for future
work. As long as the annotations are correctly labeling
$\rnIt$ or $\rnGer$ chords, they will remain useful for
classifying the appropriate \gls{pcset} and key contexts,
despite the inversions being incorrect.

\phdfigure[All the \gls{rna} labels taken from the
\gls{haydnsun} dataset. Each bar indicates the counts of the
Roman numeral class in different
inversions]{haydnsun_rn_inversion}

\phdfigure[All the keys spanned by the \gls{rna} annotations
of the \gls{haydnsun} dataset. For each key, the counts
indicate which ones correspond to modulations (local key
regions) and tonicizations ]{haydnsun_key_mod_ton}

\reffig{haydnsun_key_mod_ton} shows the distribution of the
keys in the \gls{haydnsun} dataset. The dataset spans a
total of 31 keys out of the 38 in the vocabulary $\setkey$ (see \refsec{thevocabularyofmusicalkeys}) The
occurrence of keys around the center of the
\gls{lineoffifths} is higher, as expected. The keys on the
extremes occur mostly in the context of tonicizations and
for one or two chord annotations, which is consistent with a
brief tonicization throughout a piece of music.
