% Copyright 2022 Néstor Nápoles López

% This is \refsubsec{mozartpianosonatas(mps)}, which
% introduces the mozart piano sonatas (mps).

The \gls{mps} dataset was introduced by
\textcite{hentschel2021annotated}. The dataset consists of
musical scores in \gls{musicxml} format, as well as annotations of
the harmony and cadence information. The format of the
annotations is similar to the one in the revised (i.e.,
version 2) \gls{abc} dataset.

\phdparagraph{MPS preprocessing}

The scores were preprocessed before aggregated to the
dataset.

The first step was to detect collisions in the position of
two annotations. Collisions happen because the ``Chord''
objects in MuseScore do not encode a position when they are
exported to \gls{musicxml}. The location of the annotations is
inferred from the surrounding note. However, in instances
where an annotation is added to a longer note (e.g., half
way of a whole note), the position is impossible to retrieve
from the \gls{musicxml} file. One way to solve this problem is to
create an additional part in MuseScore with a sufficient
note/rest onsets to accomodate the annotation's location. I
processed all the scores that required this modification
(all but 8 scores in the dataset). This process removed
the majority of the collisions, however, a few of them
persisted. Additional collisions happen when a measure is
divided by a repetition bar (e.g., in anacrusic music). In
these instances, the offset of the annotation resets for the
portion of the measure beyond the repetition bar and it
collides with the annotations before the repetition bar.
These second round of collisions can be removed by removing
repetition bars, effectively avoiding to reset the offset of
the annotations within a single measure.
