% Copyright 2022 Néstor Nápoles López

In order to facilitate the aggregation of all datasets,
\gls{romantext} was chosen as the ``container'' digital
format for all \gls{rna} annotations. However, the
underlying tokens accepted in each of those \gls{romantext}
files was reduced to a smaller vocabulary of chords, so that
the process of training a model resulted in a more
controllable vocabulary of chord classes.

There are two motivations for choosing \gls{romantext} as
the container format for all annotation files:

\begin{enumerate}
    \item An existing effort has been already done in the
    \gls{wir} dataset \parencite{gotham2019romantext,
    gotham2022openscore} to convert other datasets into
    \gls{romantext}. Thus, taking advantage of that effort
    is desirable
    \item The \gls{romantext} has been closely integrated to
    the music21 Python library
    \parencite{cuthbert2010music21}, which is one of the
    most advanced software tools for processing musical
    scores, and facilitates the integration of annotations
    and \gls{musicxml} files
\end{enumerate}