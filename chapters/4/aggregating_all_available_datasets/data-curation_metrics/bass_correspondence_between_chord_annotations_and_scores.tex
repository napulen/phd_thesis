% Copyright 2022 Néstor Nápoles López

One of the most common errors done by harmonic analysis
annotators (even if they are experts) is to mislabel the
inversion of the chord. This happens likely because the
annotators pay less attention to this property of the chord
than, for example, the root. In certain genres, such as
chorales or string quartets, a higher voice may cross the
bass momentarily, confusing the annotator into thinking that
the lowest sounding note of the chord is in the bass voice,
when it is in fact not the case.

These inconsistencies can be verified in a similar way to
the pitch correspondence. An unusually low match between the
annotated bass and the one in the score may indicate that
the annotator mislabeled the inversion of the chord.