% Copyright 2022 Néstor Nápoles López

Once all datasets have been aggregated, standardized, and
split into three sets, the dataset is ready for training the
model and running the experiments.

The last step, described in \refsec{dataaugmentation}, is to
perform data augmentation on the training set.

Before that, in this section, I describe the characteristics
of the aggregated training, validation, and test splits of
the dataset.

\phdfigure[All the \gls{rna} labels taken from the
aggregated dataset. Each bar indicates the counts of the
Roman numeral class in different
inversions]{aggregated_rn_inversion}



\phdfigure[Intersection between the \gls{pcset}s in this
dataset and the \gls{pcset}s in all other
datasets][0.5]{aggregated_pcset_venn}



\phdfigure[All the keys spanned by the \gls{rna} annotations
of the aggregated dataset. For each key, the counts indicate
which ones correspond to modulations (local key regions) and
tonicizations ]{aggregated_key_mod_ton}
