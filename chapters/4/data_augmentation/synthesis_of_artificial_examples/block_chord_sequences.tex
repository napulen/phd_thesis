% Copyright 2022 Néstor Nápoles López

In harmony exercises like the one shown in
\reffig{hindemith_exercise}, the goal is often to
\gls{realize} the chords in a four-part harmonization that
respects the voice-leading rules explained in the textbook,
as shown in \reffig{voice_leading_sequence}. 

\phdfigureproxy[A possible solution to the first harmony
exercise proposed
\reffig{hindemith1943concentrated}]{voice_leading_sequence}

Although it is technically possible to computationally
generate such harmonizations with awareness of voice-leading
rules, this is more expensive to compute using a rule-based
voice-leading
algorithm.\footnotelink{https://github.com/napulen/romanyh}
Moreover, in the neural network architecture of this
dissertation (see \refchap{modeldesign}), the voice-leading
information is removed, because the octave information of
the notes is removed. For that reason, an alternative
approach is to start the synthesized training examples with
block chord sequences of chords, such as the ones shown in
\reffig{block_chord_sequence}.

\phdfigureproxy[Another version of
\reffig{voice_leading_sequence}, without any consideration
for voice leading rules]{block_chord_sequence}
