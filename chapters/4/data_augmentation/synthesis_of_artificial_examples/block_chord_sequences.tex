% Copyright 2022 Néstor Nápoles López

In harmony exercises like the one shown in
\reffig{hindemith_exercise}, the goal is often to
\gls{realize} the chords in a four-part harmonization that
respects the voice-leading rules explained in the textbook,
as shown in \reffig{voice_leading_sequence}. 

\phdfigureproxy[A possible solution to the first harmony
exercise proposed
\reffig{hindemith1943concentrated}]{voice_leading_sequence}

Although it is possible to algorithmically generate such
harmonizations with awareness of voice-leading rules, this
is computationally expensive using a rule-based
voice-leading
algorithm.\footnotelink{https://github.com/napulen/romanyh}
Moreover, in the neural network architecture of this
dissertation (see \refchap{modeldesign}), the voice-leading
information is not taken into account, because the octave of
the notes is not encoded. For that reason, an alternative
approach is to generated synthesized training examples with
block chord sequences of chords, such as the ones shown in
\reffig{block_chord_sequence}. These block chords are useful
to indicate what are the notes of the chord and which of
those notes is the lowest-sounding note. 

We introduced a data-augmentation approach of this kind in
\textcite{napoleslopez2020harmonic}, where it showed an
improved accuracy when predicting the chords in a dataset of
Bach chorales. Nevertheless, in preliminary experiments with
the datasets presented in
\refsec{publiclyavailabledatasets}, this approach was not
sufficient to work with the more complicated textures found
in piano sonatas or string quartets.

\phdfigureproxy[Another version of
\reffig{voice_leading_sequence}, without any consideration
for voice leading rules]{block_chord_sequence}
