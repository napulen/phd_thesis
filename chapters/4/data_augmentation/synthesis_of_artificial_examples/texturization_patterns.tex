% Copyright 2022 Néstor Nápoles López

The synthesized scores using the new data-augmentation
approach have all a homorhythmic ``block-chord'' texture.
Thus, they lack the textural complexity of polyphonic or
homophonic pieces of music, which are often found in the
aggregated dataset.

For this reason, a texturization strategy was developed to
process the block-chord annotations, turning them in more
realistic data-augmentation examples.

I explored three texturization patterns, which were designed
based on the results obtained in preliminary experiments. 

\phdparagraph{alberti bass}

An \emph{Alberti bass} pattern was used to turn block chords
into melodic lines.

\phdparagraph{bass split}

A pattern where the bass was separated from the upper notes
served to inform the network of the bass and/or inversion of
the chord.

\phdparagraph{syncopation}

A pattern where the bass was played in syncopation, after
the chord onset served to inform the network that the
location of the bass could be different from the location of
the chord onset.
