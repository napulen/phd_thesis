% Copyright 2022 Néstor Nápoles López

In tonal music theory textbooks, exercises like the one
shown in \reffig{hindemith_exercise} are common. Notice that
the exercise indicates the key, time signature, duration of
the chords (i.e., harmonic rhythm), and Roman numeral
labels. Moreover, no indication of the arrangement of the
chords is given, which is left to the student to decide.

\phdfigure[Harmony exercise in
\textcite{hindemith1943concentrated}, where a student
\glspl{realize} the chords indicated by the Roman numerals
and note durations]{hindemith_exercise}

These served as an inspiration for developing a
data-augmentation technique based on using existing
annotations to \gls{realize} the chords of a generated
musical score.
