% Copyright 2022 Néstor Nápoles López

In tonal music theory textbooks, exercises like the one
shown in \reffig{hindemith_exercise} are common. Notice that
the exercise indicates the key, time signature, duration of
the chords (i.e., harmonic rhythm), and Roman numeral
labels. Moreover, no indication of the arrangement of the
chords is given, which is left to the student to decide. 

\phdfigure[Harmony exercise in
\textcite{hindemith1943concentrated}, where a student
\glspl{realize} the chords indicated by the Roman numerals
and note durations]{hindemith_exercise}

The information provided in the exercise is similar to the
one contained in a digitized \gls{rna} file in the
\gls{romantext} (or similar) format. Thus, an ``exercise''
can be synthesized from the annotations and aggregated to
the dataset as a new training example for the same set of
\gls{rna} annotations.

This approach was implemented as a data-augmentation
workflow to increase the number of examples in the training
set.
