% Copyright 2022 Néstor Nápoles López

In harmony exercises like the one shown in
\reffig{hindemith_exercise}, the goal is often to
\gls{realize} the chords in a four-part harmonization that
respects the voice-leading rules explained in the textbook,
as shown in \reffig{synthesis_voiceleading}. 

\phdfigure[A possible solution to the first harmony exercise
proposed
\reffig{hindemith_exercise}]{synthesis_voiceleading}

Although it is possible to algorithmically generate such
harmonizations with awareness of voice-leading rules, this
is computationally expensive using a rule-based
voice-leading
algorithm.\footnotelink{https://github.com/napulen/romanyh}
Moreover, in the neural network architecture of this
dissertation, \gls{augmentednet} (see
\refchap{modeldesign}), the voice-leading information is not
taken into account, because the octave of the notes is not
encoded. For that reason, an alternative approach is to
synthesize training examples as simple block chords, such as
the ones shown in \reffig{synthesis_blockchord}. These block
chords are useful because they indicate the notes of the
chord and which of those notes is at the bottom (i.e., the
lowest-sounding one). 

We introduced a data-augmentation approach of this kind in
\textcite{napoleslopez2020harmonic}, where it showed an
improved accuracy when predicting the chords in a dataset of
Bach chorales when compared to using only the ``real''
training set. Nevertheless, in preliminary experiments with
the datasets presented in
\refsec{publiclyavailabledatasets}, this approach was not
sufficient to work with the more complicated textures found,
for example, in piano sonatas and string quartets.

\phdfigure[Another version of
\reffig{synthesis_voiceleading} with block chords, without
any consideration for voice leading
rules]{synthesis_blockchord}

A solution to this problem, originally proposed by co-author
Mark Gotham in \textcite{napoleslopez2021augmentednet} was
to texturize the block chords to approximate the textures
found in more complicated music. I approached this solution
by implementing texturization patterns on top of the block
chords.
