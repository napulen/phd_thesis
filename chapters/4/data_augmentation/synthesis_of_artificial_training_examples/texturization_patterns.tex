% Copyright 2022 Néstor Nápoles López

The block chords have all a homorhythmic texture. Thus, they
lack the textural complexity of polyphonic or homophonic
pieces of music, which are often found in the aggregated
dataset.

For this reason, an automatic texturization approach is
proposed to process block-chord annotations, turning them in
more realistic data-augmentation examples.

I explored three texturization patterns, which were designed
to deal with specific problems found in the real examples:
labeling arpeggios and chords implied in a melodic line,
labeling chords where the bass is played in isolation from
the upper notes of the chord, and labeling chords where the
bass is temporally displaced from the location of the chord.
The patterns designed to deal with these problems,
respectively, are called the \emph{Alberti bass} pattern,
the \emph{bass split} pattern, and the \emph{syncopation}
pattern.

\phdparagraph{alberti bass}

An \emph{Alberti bass} pattern was used to turn block chords
into melodic lines.

\phdparagraph{bass split}

A pattern where the bass was separated from the upper notes
served to inform the network of the bass and/or inversion of
the chord.

\phdparagraph{syncopation}

A pattern where the bass was played in syncopation, after
the chord onset served to inform the network that the
location of the bass could be different from the location of
the chord onset.
