% Copyright 2022 Néstor Nápoles López

The block chords have all a homorhythmic texture. Thus, they
lack the textural complexity of polyphonic or homophonic
pieces of music, which are often found in the aggregated
dataset.

For this reason, an automatic texturization approach is
proposed to process block-chord annotations, turning them in
more realistic data-augmentation examples.

I explored three texturization patterns, which were designed
to deal with specific problems found in the real examples:
labeling arpeggios and chords implied in a melodic line,
labeling chords where the bass is played in isolation from
the upper notes of the chord, and labeling chords where the
bass is temporally displaced from the location of the chord.
The patterns designed to deal with these problems,
respectively, are called the \emph{Alberti bass} pattern,
the \emph{bass split} pattern, and the \emph{syncopation}
pattern.

\phdparagraph{alberti bass}

This pattern consists of a 4-note melodic pattern with a
pitch contour of
\emph{low}-\emph{high}-\emph{middle}-\emph{high}. It is used
to turn block chords into melodic lines. The goal is to
occasionally play chords using a monophonic texture.

\phdfigure[A random texturization of some of the block
chords in \reffig{synthesis_blockchord} with an
\emph{Alberti bass} pattern]{synthesis_alberti}

\phdparagraph{bass split}

This is pattern where the original chord duration is divided
by half, playing the bass in the first half, and the
remaining notes in the second. The goal is to occasionally
separate the bass from all other notes.

\phdfigure[A random texturization of some of the block
chords in \reffig{synthesis_blockchord} with a \emph{Bass
split} pattern]{synthesis_basssplit}

\phdparagraph{syncopation}

This is a pattern where the highest note is played first,
followed by the rest of the notes, played in syncopation.
The goal is to occasionally shift the onset of the bass from
the onset of the chord. It is intended to provide the
trainable system with examples where the location of the
bass is different from the location of the chord onset.

\phdfigure[A random texturization of some of the block
chords in \reffig{synthesis_blockchord} with a
\emph{Syncopation} pattern]{synthesis_syncopation}

In the experiments performed in this dissertation, these
patterns were applied randomly and recursively to the
annotations. A hypothetical training example synthesized
from the Hindemith annotations is shown in
\reffig{synthesis_recursive}. This example combines the
three patterns.\footnote{One of the possibilities is also to
leave the block chord unmodified, as depicted in
\reffig{synthesis_recursive}.}

\phdfigure[A texturized training example that was
synthesized from the annotations in the Hindemith exercise
of \reffig{hindemith_exercise}. The example includes the
three texturization patterns applied
randomly]{synthesis_recursive}

The results of this data-augmentation technique, with and
without the use of key transpositions is discussed in
\refsec{effectsofdataaugmentation}.
