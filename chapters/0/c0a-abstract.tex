\chapter*{Abstract}
\addcontentsline{toc}{chapter}{Abstract}
\label{chap:chap0-abs}

One of the most common ways to analyze a piece of tonal music is through Roman numeral analysis.
This requires the inspection of several attributes related to chords and keys.
Chords can be inspected in terms of their properties: root, quality, inversion, and function.
Keys can be inspected in terms of their temporal scope as modulations or tonicizations.
Each of these attributes (or tasks) of Roman numeral analysis can be modeled in isolation.
However, recent research has found that analyzing several tonal tasks simultaneously leads to more robust MIR models.
This has motivated the research of multitask models for Roman numeral analysis.
In this dissertation, I extend this line of research by:
(1) improving the data-curation process for existing datasets;
(2) developing a new data-augmentation technique for Roman numeral analysis models;
(3) improving the design of existing convolutional recurrent neural networks;
and (4) extracting more tonal tasks from the Roman numeral annotations.
Combining these ideas, I trained a new Roman numeral analysis model.
Among other applications, this will facilitate advanced searching in music collections.
For example, searching by chord progressions or by modulation trajectories.