% Copyright 2022 Néstor Nápoles López

In a large number of \gls{mir} works, transposition has been
used as a data augmentation technique to overcome the
scarcity of data. When pitch spelling is taken into account,
it requires the additional consideration that transpositions
into two enharmonic keys are distinct from each other. This
was first investigated in \textcite{micchi2020not}, where
the transposition takes into account the spelling of the
notes and the vocabulary of the key modulations within the
piece. The approach here is to make sure that all
transpositions done over a piece result in key modulations
and tonicizations that fit within the vocabulary of keys
$\setkey$ (see \refsec{thevocabularyofmusicalkeys}).
\reftab{augmentation_transposition} shows the parameters and
training time of this experiment. Note that in two datasets
with the same number of files, \gls{tavern} and \gls{mps},
the augmentation results in a different number of
transpositions. This is related to the possible number of
transpositions where the modulations lie within the
vocabulary $\setkey$.

\phdtable[Parameters of the experiment with
data-augmentation by
\emph{transposition}]{augmentation_transposition}
