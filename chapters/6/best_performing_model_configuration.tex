% Copyright 2022 Néstor Nápoles López

\phdtable[Performance obtained in the ablation studies,
compared to the baseline configuration of the
network]{ablation}

\reftab{ablation} shows a summary of the performance of the
different configurations in the ablation experiments. Some
of these results confirm the expected results. For example,
removing the onset information affects the
\gls{harmonicrhythm7} more than any other change in the
network, indicating that measure onsets are important for a
better chord segmentation. Removing the \gls{chroma19} input
representation has drastic effects on the prediction of
chords (e.g., \gls{pcset121}, \gls{rn31}), but surprisingly
also in the \gls{soprano35} task. Removing the bottom note
affects the performance of the \gls{bass35} task, as
expected, but also surprisingly the \gls{tenor35} and
\gls{alto35} as well.

Overall, the biggest drop in performance happens when the
recurrent layers of the network are removed. This has a
notable effect on the \gls{localkey35} and
\gls{tonicization35} tasks, which are generally the tasks
with the longer-term dependencies. 

All of the modifications in the ablation study have a
negative performance compared to the baseline model, except
for the dense linear layer. This was a surprising result,
because that did not seem to be the case in preliminary
experiments. It could be that data augmentation has an
effect in the performance of the baseline model against the
linear dense variation. This will be explored further.
