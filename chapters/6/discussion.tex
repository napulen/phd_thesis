% Copyright 2022 Néstor Nápoles López

An \gls{arna} model is more than a neural network
architecture. It should be seen as an end-to-end model that
consists both of machine learning components and musical
decisions. When I started this project, I assumed that the
goal was to create a ``machine learning model'' that would
solve the problem. However, from the beginning, parsing
different datasets in different annotation formats soon made
me realize that domain knowledge was necessary to clean the
datasets, standardize their musical assumptions, define the
chord vocabulary, and guide the design of the network. This
is true for the output of the model as well. For example,
determining the number of classification outputs and how
these are defined, the number of key layers in the musical
analysis, or the spelling of certain Roman numerals (e.g.,
that infamous $\rn{I}\rnsixfour$ vs $\rn{V}\rnsixfour$). All
of these aspects shape the design of a \gls{arna} model.
Such a model is a unified model that spans both technical
and musical considerations.

Out of the ideas explored in the comparison process, the
\gls{natem} algorithm seems particularly promising to me in
future experiments. The design of tonal music analysis
models is open to the interpretation of the researchers
implementing them. This makes it very difficult to compare
them. Perhaps the distribution of \reftab{rare_chords}
summarizes my current thoughts on the design of the
\gls{arna} models. Even if a model is not competitive in
most metrics, it may be useful in certain situations, or
adopt certain design patterns that result beneficial for
certain types of music. Thus, it is worth the effort to
search for tools that allow these models to ``communicate''
with each other, and obtain feedback on the things they do
well and do poorly. Taking the performance on the $\rnFr$ as
an example, noticing the peculiar intervallic configuration
of the chord and designing a representation that exploits it
might be the difference between a null performance and
perfect performance on a chord that happens 0.02\% of the
time.

% \reffig{haydn_comparison} shows an example analysis by
% various models compared. As expected by the results
% presented in the table of rare chords, the proposed model
% is the only one able to detect the \gls{augsix} chord in
% the excerpt. There, the model by
% \textcite{mcleod2021modular} proposes a label of
% g:$\rn{VI}\rnsix$, which is a reasonable guess when
% ignoring the C$\musSharp$ in the first violin. However,
% this intervallic relationship between E$\musFlat$4 and
% C$\musSharp$5 is precisely the indication that would lead
% an analyst to label the chord as an \gls{augsix}. Those
% situations are rare, but also important. They separate a
% ``naive'' \gls{arna} model from a ``useful'' one. The
% results in \reftab{rare_chords} indicate that we are far
% from useful musical awareness, but there is some progress
% towards it.

% \phdfigure[Comparison of recent \gls{arna}
%  models]{haydn_comparison}
