% Copyright 2022 Néstor Nápoles López

An \gls{arna} system is more than a neural network
architecture. It should be seen as an end-to-end system that
consists both of machine learning components and musical
decisions. When I started this project, I started it under
the assumption that the goal was to create a ``machine
learning system'' that would solve the problem. However,
from the beginning, parsing different datasets in different
annotation formats soon made me realize that domain
knowledge was necessary to clean the datasets, standardize
their musical assumptions, define the chord vocabulary, and
guide the design of the network. This is true for the output
of the system as well. For example, determining the number
of classification outputs and how these are defined, the
number of key layers in the musical analysis, or the
spelling of certain Roman numerals (e.g., that infamous
$\rn{I}\rnsixfour$ vs $\rn{V}\rnsixfour$). All of these
aspects shape the design of a \gls{arna} system. Such a
system is a unified model that spans both technical and
musical decisions.

\reffig{haydn_comparison} shows an example analysis by
various models compared. As expected by the results
presented in the table of rare chords, the proposed model is
the only one able to detect the \gls{augsix} chord in the
excerpt. There, the model by \textcite{mcleod2021modular}
proposes a label of g:$\rn{VI}\rnsix$, which is a reasonable
guess when ignoring the C$\musSharp$ in the first violin.
However, this intervallic relationship between E$\musFlat$4
and C$\musSharp$5 is precisely the indication that would
lead an analyst to label the chord as an \gls{augsix}. Those
situations are rare, but also important. They separate a
``naive'' \gls{arna} system from a ``useful'' one. The
results in \reftab{rare_chords} indicate that we are far
from useful musical awareness, but there is some progress
towards it.

\phdfigure[Comparison of recent \gls{arna}
 systems]{haydn_comparison}
