% Copyright 2022 Néstor Nápoles López

A \glspl{rna} system is more than a neural network
architecture. It should be seen as an end-to-end system that
consists both of machine learning components and musical
decisions. When I started this project, I started it under
the assumption that the goal was to create a ``machine
learning system'' that would solve the problem. However,
from the beginning, parsing different datasets in different
annotation formats soon made me realize that domain
knowledge was necessary to clean the datasets, standardize
their musical assumptions, define the chord vocabulary, and
guide the following steps. This is true for the output of
the system as well. The way classification outputs are
defined, the number of key layers in the musical analysis,
or the spelling of certain Roman numerals (e.g., that
infamous $\rn{I}\rnsixfour$ vs $\rn{V}\rnsixfour$). All of
these aspects shape the design of a \glspl{rna} system. Such
a system is a unified model that spans both technical and
musical decisions. It should arguably be evaluated as such.

\phdfigure[Comparison of recent \glspl{rna}
 systems]{haydn_comparison}
