% Copyright 2022 Néstor Nápoles López

An \gls{arna} model is more than a neural network
architecture. It should be seen as an end-to-end model that
consists both of machine learning components and musical
decisions. When I started this project, I assumed that the
goal was to create a ``machine learning model'' that would
solve the problem. However, from the beginning, parsing
different datasets in different annotation formats soon made
me realize that domain knowledge was necessary to clean the
datasets, standardize their musical assumptions, define the
chord vocabulary, and guide the design of the network. This
is true for the output of the model as well. For example,
determining the number of classification outputs and how
these are defined, the number of key layers in the musical
analysis, or the spelling of certain Roman numerals (e.g.,
that infamous $\rn{I}\rnsixfour$ vs $\rn{V}\rnsixfour$). All
of these aspects shape the design of an \gls{arna} model.

Out of the ideas explored in the comparison process, the
\gls{natem} algorithm seems particularly promising to me in
future experiments. The design of tonal music analysis
models is open to the interpretation of the researchers
implementing them. This makes it very difficult to compare
them. Perhaps the distribution of \reftab{rare_chords}
summarizes my current thoughts on the design of the
\gls{arna} models. Even if a model is not competitive in
most metrics, it may be useful in certain situations, or
adopt certain design patterns that result beneficial for
certain types of music or certain types of chords. Thus, it
is worth the effort to search for tools that allow these
models to ``communicate'' with each other, and obtain
feedback on the things they do well and do poorly. Taking
the performance on the $\rnFr$ as an example, noticing the
peculiar intervallic configuration of the chord and
designing a representation that exploits it might be the
difference between recognizing all $\rnFr$ or none. 

\phdfigurefullpage[Comparison of the annotations provided by
a human analyst and various \gls{arna} models. The musical
excerpt is from Haydn's Op. 20 No. 3 - IV, a piece in the
\gls{haydnsun} test set. The annotations from the \gls{arna}
models that differ from the human analyst's are marked in
red]{haydn_comparison}

In order to complement the quantitative results presented in
this section, \reffig{haydn_comparison} shows a musical
excerpt analyzed by the various \gls{arna} systems. For
reference, the example also shows the human annotations
(``ground truth'') in the dataset. The key of the excerpt is
$\keyg{}$, which all of the models except \gls{melisma}
predict correctly. 

\paragraph{Analysis of the Melisma model}

In general, the \gls{melisma} model seems to present more
flaws than all other \gls{arna} models. The most important
perhaps being mislabelling the key. After that, it misses
the initial annotation of the minor tonic triad (m.1, beat
1). The model also has a tendency for long streams of chord
annotations (e.g., m. 3), which greatly differs from the
harmonic rhythm described by the human annotator and the
other \gls{arna} models. 

\paragraph{Analysis of the Chen and Su model}

After the \gls{melisma} model, the \textcite{chen2021attend}
model provides better predictions. Importantly, all but one
of the annotations (the $\rnIt\rnsix$ chord on measure 8,
beat 2) in \textcite{chen2021attend} are either tonic or
dominant chords. This speaks about the bias that exists for
the Roman numeral classes that appear more often in the
dataset, as shown in \reftab{rare_chords}. 

\paragraph{Analysis of the Micchi et al. model}

The \textcite{micchi2021deep} model is less biased towards
tonic and dominant classes. For example, it recognizes
instances of diminished chords (e.g., m.~2, beat 3). The
model seems to struggle with the chord segmentation, as
chords are often in odd locations. For example, the
precipitated \gls{neapolitan} (m. 9, beat 3), or the tonic
triad at the beginning of the second system (m.6, beat 2.5).
This is perhaps because the musical excerpt is anacrusic,
and it makes it more complicated to align the score and
annotations. 

\paragraph{Analysis of the McLeod and Rohrmeier model}

The \textcite{mcleod2021modular} model is fairly similar to
the human annotator. Two drawbacks of this model is that it
often predicts the wrong inversion (e.g., measure 1, beat 3;
or measure 6, beat 3). In addition to this, it is unable to
recognize the $\rnGer$ chord in measure 8, beat 2. The
proposed model in this dissertation, \gls{augmentednet} is
the only model that seems to be able to recognize that
chord, the $\rnGer$ \gls{augsix} in measure 8. Other models
provide either an $\rnIt$ label, or in the case of
\textcite{mcleod2021modular}, a label of $\rn{VI}\rnsix$,
which is a reasonable prediction when ignoring the
$\pitchC{}\musSharp$ in the first violin. However, this
intervallic relationship between $\pitchE{}\musFlat$4 and
$\pitchC{}\musSharp$5 is precisely what characterizes the
\gls{augsix} chord. The prediction of $\rnIt$ here is
expected in \textcite{chen2021attend} and
\textcite{micchi2021deep}, as these models collapse the
various flavours of \gls{augsix} chords into one category,
which makes it difficult (maybe impossible) to distinguish
between a $\rnIt$ and $\rnGer$. Notice also that there are
two errors in the human analysis. In measure 2, beat 1, the
analyst indicated a $\rni\rnsix$ label, however, because the
viola crosses the violoncello at that point, the chord is in
fact in root position. All \gls{arna} models unequivocally
get this annotation right. The second error in the
annotation lies in the inversion of the $\rnGer$ chord,
indicated by the annotator as ``root position.'' The
\gls{augmentednet} model gets the annotation in the proper
inversion, according to the convention for chord inversions
used in the \gls{romantext} format.\footnote{This particular
set of chords, \gls{augsix}, often have differing regarding
their inversions.}

% The string quartet example features a polyphonic texture,
% which would be difficult to analyze by a simple model
% (e.g., one that assumes the absence)