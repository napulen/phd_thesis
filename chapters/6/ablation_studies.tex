% Copyright 2022 Néstor Nápoles López

It is difficult to understand what kind of representations a
deep learning model learns during training. Ablation studies
are an useful way to inspect the contribution of its
different components. Ablation studies refer to a concept
from neuroscience related to the removal of components of an
organism \parencite{meyes2019ablation}. Applied to deep
learning models, this refers to experiments where components
of the model are removed or modified. The purpose of the
modifications is to observe the effects in the performance
of the model, as well as to understand the contributions of
the different parts of the model. This section introduces a
series of experiments to evaluate the performance of the
model after modifying or removing some of the components of
the neural network. The experiments are divided by sections
of the neural network: input representations, convolutional
layers, dense layers, and recurrent layers.

The ablation studies were run over the aggregated training
dataset using 5-fold cross-validation. There was no use of
data augmentation, neither in the form of transposition nor
synthetic files, whose effects was evaluated separately in
\refsec{effectsofdataaugmentation}.
