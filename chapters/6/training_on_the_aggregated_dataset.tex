% Copyright 2022 Néstor Nápoles López

\refsec{ablationstudies} introduced a baseline \gls{crnn}
model and a series of experiments to explore its underlying
components. This baseline model was subsequently used in
\refsec{effectsofdataaugmentation} to explore the effects of
data-augmentation techniques when training the model on
seven publicly available datasets. In the experiments of
\refsec{effectsofdataaugmentation}, the model was trained on
each dataset individually. However, the best performance of
a deep learning model is often achieved when more data is
used to train it.

This section reports the results of training the
\gls{augmentednet} model with the aggregated dataset,
consisting of all the training data across the seven
publicly available datasets. Furthermore, the model was
evaluated on the test portion of each individual dataset.
The results are reported in \reftab{final_results_dataset}.
In this experiment, both data-augmentation techniques,
\emph{transposition} and \emph{synthesis}, were used in
combination. 
% Due to the large training time (16 hours), only one
% experiment was conducted on the aggregated dataset.

\phdtable[Validation accuracy of the model trained with the
aggregated dataset. The validation accuracy is reported in
the test set of each dataset and for each of the 9
classification tasks of the model]{final_results_dataset}
