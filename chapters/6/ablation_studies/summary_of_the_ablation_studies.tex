% Copyright 2022 Néstor Nápoles López

\phdtablefith[Performance obtained in the ablation studies,
compared to the baseline configuration of the network. The
``Baseline'' row shows the average accuracy obtained in each
task across the 5-fold cross validation. Each row of the
ablation studies shows the difference in accuracy between
the value obtained in the experiment and the baseline. The
biggest drop in performance for each task column is
highlighted in bold font]{ablation}

\reftab{ablation} shows a summary of the performance of the
different configurations in the ablation experiments. Some
of these results confirm the expected results. For example,
removing the \gls{duration14} input representation affects
the \gls{harmonicrhythm7} more than any other change in the
model configuration. This might indicate that measure onsets
are important for a better chord segmentation. Removing the
\gls{chroma19} input representation has noticeable effects
on the prediction of chords (e.g., \gls{pcset121},
\gls{rn31}). Unexpectedly, it also affects the
\gls{soprano35} task more than any other \gls{satb35} task.
On the contrary, removing the \gls{bass19} input
representation affects the performance of the \gls{tenor35}
and \gls{alto35} tasks, in addition to the \gls{bass35},
which was the expected outcome.

Overall, as predicted, the biggest drop in performance
happens when the recurrent layers of the network are
removed. The recurrent layers have a notable effect on the
``key'' tasks: \gls{localkey38} and \gls{tonicization38}.
This supports the hypothesis that a longer musical context
is needed to estimate musical keys, in comparison to
estimating chords. It is interesting that among the chord
tasks, however, \gls{rn31} is the one affected the most by
the removal of the recurrent layers. This might be because
Roman numeral numerators are relative to the key, and thus
sensitive to the key context.

In the row of the baseline model, the standard deviation
($\sigma$) across the 5-fold cross validation is provided
for reference. All the differences in performance
highlighted in \reftab{ablation} are at least 2 standard
deviations below the performance of the baseline.
Furthermore, most modifications in the ablation studies have
a negative effect compared to the baseline model, which is
consistent with the observations during preliminary
experiments designing the network. One exception is the use
of a linear dense layer, which appears to be slightly above
the baseline in most tasks. However, the advantage of the
linear dense variation fades away after introducing data
augmentation, with an average drop of $-2.7\%$ across all
tasks compared to the baseline. This brings an important
point about the ablation studies presented here, which is
that they are useful to provide hints about the
contributions of the different components of the network,
but do not provide information about how the modifications
scale with more data. That is, the design of a neural
network architecture is importantly an empirical process,
which requires continuous experimentation in different
scenarios. For this reason, the baseline model presented in
this section was used in subsequent experiments, as it is
the version of the network that went through more
experiments and datasets of different sizes. The ablation
studies nevertheless confirm certain contributions in a
controlled experiment where every modification (including
the baseline) received the same amount of data.