% Copyright 2022 Néstor Nápoles López

In the baseline model, the convolutional blocks process the
three inputs independently, which are later concatenated.
The convolutional blocks for all three inputs are similar:
they have the same number of layers ($n=6$, see
\refsubsubsec{numberofconvolutionalfilters}) and use the
same strategy to adjust the kernel size (see
\refsubsubsec{kernelsize}) and number of filters (see
\refsubsubsec{numberoffilters}) of each layer. The
experiments in this section explore modifications to the
convolutional blocks or the convolutional layers within.
\refsubsubsec{singleconvolutionalblock} explores the
performance of the neural network with a single
convolutional block, stacking the three inputs into a single
vector before dispatching them to the convolutional block.
\refsubsubsec{constantnumberoffilters} explores the
performance of the neural network with a constant number of
filters in each convolutional layer, testing the hypothesis
that capturing short-term patterns is preferred. The last
experiment in \refsubsubsec{noconvolutionallayers} explores
the effects of removing all convolutional layers (and
blocks) altogether.
