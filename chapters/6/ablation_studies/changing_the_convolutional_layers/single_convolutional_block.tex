% Copyright 2022 Néstor Nápoles López

The number of convolutional blocks is defined by the number
of input representations sent to the neural network. In the
baseline model, there are three convolutional blocks, one
for each of the \gls{bass19}, \gls{chroma19}, and
\gls{duration14} input representations. An experiment is
proposed to explore the effects of using a unique
convolutional block. That is, learning convolutional filters
from all the inputs at once, instead of independently. This
is achieved by concatenating (i.e., stacking) all the inputs
into a single vector per timestep, and processing the
resulting sequence with a single convolutional block, as
shown in \reffig{ablation5}.

\phdfigure[Modification proposed in the ablation study,
where all the input representations are concatenated and
processed by a single convolutional block]{ablation5}
