% Copyright 2022 Néstor Nápoles López

Chords often occur at the beginning of measures. The
\gls{duration14} input representation provides the network
with this kind of structural information. Namely, the
timesteps where a new measure starts, and the timesteps
where a new note onset starts. This information complements
the pitch-related information provided by the other tasks,
\gls{bass19} and \gls{chroma19}. An ablation study is
proposed here, where \gls{duration14} input representation
is removed. The hypothesis is that this will be detrimental
to the performance of the \gls{crnn}, particularly, the
chord segmentation. The position of the chords is determined
from the predictions of the \gls{harmonicrhythm7} task.
Thus, losing performance in this task results in worse chord
segmentation. aThe modifications proposed in the ablation
study are shown in \reffig{ablation4}.

\phdfigure[Modifications proposed in the fourth ablation
study]{ablation4}
