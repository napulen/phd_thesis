% Copyright 2022 Néstor Nápoles López

In \refsubsubsec{encodingnoteswithspelling}, a new method to
encode pitch spelling was proposed using a 19-dimensional
vector. This method was used in the baseline model of the
ablation studies to encode the pitch spelling of the
\gls{bass19} and \gls{chroma19} input representations. The
pitch-spelling method is a replacement of a 35-dimensional
vector encoding used by a previous model
\parencite{micchi2021deep}. The new encoding method, among
other things, reduces the number of trainable parameters.
The current ablation study proposes to use the
35-dimensional representation in \textcite{micchi2021deep}
instead of the method chosen for the baseline. The
hypothesis of the experiment is that there will be no
noticeable gains in performance by using the 35-dimensional
representation, and that the 19-dimensional one is an
adequate replacement. The modifications proposed are shown
in \reffig{ablation1}.

\phdfigure[Modification proposed in the ablation study,
where the method in \textcite{micchi2020not} is used for
encoding pitch spellings, instead of the one in the
baseline. See \refsubsubsec{encodingnoteswithspelling} for
further details on the difference between the two
methods]{ablation1}
