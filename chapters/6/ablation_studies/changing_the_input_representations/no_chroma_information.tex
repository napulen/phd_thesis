% Copyright 2022 Néstor Nápoles López

An analog of the previous experiment would be to guess the
chord knowing only the lowest-sounding note at a given time,
but none of the other notes in the staff. The reminder
``upper'' notes above the lowest-sounding one are provided
in the \gls{chroma19} representation. An ablation study is
also proposed to demonstrate the effects of removing this
input representation. The hypothesis is that it should be
detrimental to the tasks related to the chord configuration:
\gls{pcset121}, \gls{rn31}, as well as the three upper
\gls{satb35} tasks: \gls{soprano35}, \gls{alto35}, and
\gls{tenor35}. The modifications proposed are shown in
\reffig{ablation3}.

\phdfigure[Modifications proposed in the third ablation
study]{ablation3}

