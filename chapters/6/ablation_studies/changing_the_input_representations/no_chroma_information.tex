% Copyright 2022 Néstor Nápoles López

An analog of the previous experiment would be to guess the
chord knowing only the lowest-sounding note at a given time,
but none of the other notes in the staff. The \gls{chroma19}
input representation provides all the note information at a
given time, without it, only the lowest-sounding one is
known. An ablation study is also proposed to demonstrate the
effects of removing the \gls{chroma19} input representation.
The hypothesis is that it should be detrimental to tasks
related to chords: \gls{pcset121}, \gls{rn31}, as well as
the three upper \gls{satb35} tasks: \gls{soprano35},
\gls{alto35}, and \gls{tenor35}. The modifications proposed
are shown in \reffig{ablation3}.

\phdfigure[Modifications proposed in the third ablation
study]{ablation3}

