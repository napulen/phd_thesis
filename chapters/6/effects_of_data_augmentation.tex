% Copyright 2022 Néstor Nápoles López

The effects of two data augmentation techniques were
explored on the baseline model: \emph{transposition} and
\emph{synthesis}. Transposition refers to the transposition
of musical examples to a different key. Synthesis refers to
a new data-augmentation technique where artificial training
examples are synthesized from the \gls{rna} annotations.
Both data-augmentation techniques are described in
\refsec{dataaugmentation}. The baseline model refers to the
same baseline model used in the ablation studies (see
\refsubsec{baselinemodel}). First, the effect of
transposition and synthetic files individually. As these two
methods are not mutually exclusive and generally benefit
from each other, the next experiments measure the effects of
both combined. 
% In the combined data augmentation experiments, the
% parameters of texturization for the synthetic data
% examples were modified.
