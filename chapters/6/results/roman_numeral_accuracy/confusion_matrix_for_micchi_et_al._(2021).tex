% Copyright 2022 Néstor Nápoles López


\reffig{confusion_matrix_micchi2021} shows the confusion
matrix for the numerators ($\elnum$) obtained by the
predictions of the \textcite{micchi2021deep} model, compared
against the ones in the ground truth \gls{romantext} file.
Among all the models, the \textcite{micchi2021deep} has the
best recognition of \gls{neapolitan} chords, predicting
64.4\% of them correctly. The vertical lines on the $\rnV$
and $\rnVsev$ are less prominent than in the
\textcite{chen2021attend} and \textcite{mcleod2021modular}
models, indicating perhaps that when the model mislabel a
chord, it tends to be a distinct chord, rather than the
often overused $\rnVsev$ chords, which most models emphasize
in their predictions.


model seems to have a noticeably larger vocabulary than
\gls{melisma}. Although the model does not explicitly
recognize \gls{neapolitan} chords, the confusion matrix
shows that 43\% of the time it is able to provide the
correct tonal context to decode a triad acting as a
\gls{neapolitan}. This is one of the strengths of the
standardization provided by \gls{natem}. The
\textcite{chen2021attend} also seems to have the highest
recognition of some chords, for example, $\rnivsev$, where
the accuracy is higher than any other model. Nevertheless,
the model seems to have a strong tendency to mislabel many
chords with $\rnVsev$.


\phdfigure[Confusion matrix of the Roman numeral numerators
($\elnum$) for model in
\textcite{micchi2021deep}]{confusion_matrix_micchi2021}

