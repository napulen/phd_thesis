% Copyright 2022 Néstor Nápoles López

Perhaps, one of the most important aspects of \gls{rna} is
that it provides an analytical framework for special chords,
which are usually not considered in chord labels. For
example, \gls{neapolitan} or \gls{augsix} chords.

These chords are extremely difficult to recognize because,
in some instances, they occur in less than 1\% of the
annotations. \reftab{rare_chords} complements the results of
the confusion matrices by comparing the accuracy of each
model on each chord class, when the chord classes are sorted
by least occurrence in the test set.

The results of this table are interesting, because it seems
that different model succeed and fail in different sets of
chords. For example, one of the most limited models,
\gls{melisma}, was nevertheless capable of recognizing 19\%
of the $\rnisev$ chords, whereas all the deep learning
models except for \textcite{mcleod2021modular} struggled to
recognize such chords. Out of all the models,
\gls{augmentednet} was the only one able to recognize $\rnFr$
chords, which occur in only 0.03\% of the annotations of the
test set. Perhaps the reason why the model was fairly
successful in this chord class is because $\rnFr$ chords
have a very specific intervallic structure. They are not
enharmonics to any other chord class, except for another
$\rnFr$ chord in a different key. Thus, the \gls{pcset121}
classification task in \gls{augmentednet} was particularly
helpful in recognizing the unique \gls{pcset} configuration
of these chords.


\phdtable[Accuracy performance of the models in all 31
numerator classes, sorted by least occurrence in the test
set]{rare_chords}
