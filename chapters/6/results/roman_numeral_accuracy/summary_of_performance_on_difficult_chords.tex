% Copyright 2022 Néstor Nápoles López

Arguably, one of the most important aspects of \gls{rna} is
that it provides an analytical framework for special chords,
which are usually not considered in chord labels. For
example, \gls{neapolitan} or \gls{augsix} chords.

These chords are extremely difficult to recognize because,
in some instances, they occur in less than 1\% of the
annotations. \reftab{rare_chords} complements the results of
the confusion matrices by comparing the accuracy of each
model on each chord class, when the chord classes are sorted
by least occurrence in the test set.

\phdtable[Accuracy performance of the models in all 31
numerator classes, sorted by least occurrence in the test
set]{rare_chords}
