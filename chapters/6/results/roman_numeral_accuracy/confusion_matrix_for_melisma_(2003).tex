% Copyright 2022 Néstor Nápoles López


\reffig{confusion_matrix_melisma} shows the confusion matrix
for the numerators ($\elnum$) obtained by the predictions of
the \gls{melisma} model, compared against the ones obtained
for the ground-truth. Compared to other models,
\gls{melisma} has a very limited chord vocabulary. The model
does not explicitly indicate what is the theoretical
vocabulary it supports, however, an inspection of the
predictions it generated revealed that it does not support
as many chords as the recent models do (see
\reftab{chord_vocabularies}). This is confirmed in the
confusion matrix with the thick vertical lines across the
$\rnI$, $\rnV$, and $\rnVsev$ columns, indicating that the
model mostly predicts tonic and dominant harmonies in most
instances. This is often misleading when seen from the point
of view of accuracy on chord recognition, as chords are
heavily skewed towards tonic and dominant harmonies. A model
that mostly predicts tonic chords may receive a good global
accuracy score, whereas the confusion matrix is better at
demonstrating the real limitations of the model.
Nevertheless, the model seemed to provide some of the
difficult annotations correct, such as the $\rnivsev$.


\phdfigure[Confusion matrix of the Roman numeral numerators
($\elnum$) for the \gls{melisma} model. Rows represent the
target class and columns the predicted class. Values
reported in percentage, rounded to the closest integer. The
color intensity is mapped to the percentage
value.]{confusion_matrix_melisma}

