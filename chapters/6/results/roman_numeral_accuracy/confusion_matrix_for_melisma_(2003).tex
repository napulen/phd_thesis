% Copyright 2022 Néstor Nápoles López


\reffig{confusion_matrix_melisma} shows the confusion matrix
for the numerators ($\elnum$) obtained by the predictions of
the \gls{melisma} model, compared against the ones in the
ground truth \gls{romantext} file. Compared to other models,
\gls{melisma} has a very limited chord vocabulary. The model
does not explicitly indicate what is the chord vocabulary it
supports, however, an inspection of all the predictions the
model was done to compile the vocabulary in
\reftab{chord_vocabularies}. In that table it revealed
already that \gls{melisma} does not support as many chords
as the recent models do. This is confirmed here in the
confusion matrix with the thick vertical lines across the
$\rnI$, $\rnV$, and $\rnVsev$ columns, indicating that the
model mostly predicts tonic and dominant harmonies in most
instances. This is often misleading when seen from the point
of view of accuracy on chord recognition, as chords are
heavily skewed towards tonic and dominant harmonies. The
confusion matrix is clearer introducing the limitations of
the model. Nevertheless, the model seems to provide some of
the difficult annotations correct, such as the $\rnivsev$.


\phdfigure[Confusion matrix of the Roman numeral numerators
($\elnum$) for the \gls{melisma}
model]{confusion_matrix_melisma}

