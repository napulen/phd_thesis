% Copyright 2022 Néstor Nápoles López


\reffig{confusion_matrix_napoleslopez2022} shows the
confusion matrix for the \gls{augmentednet} model. In this
evaluation, the performance of \gls{augmentednet} seemed to
be above all other models. This is an interesting finding,
because the results on the individual tasks shown in
\reftab{comparison} seemed to present a very similar
performance between \gls{augmentednet} and
\textcite{micchi2021deep}. However, in the confusion matrix
of numerator predictions, it seems that
\textcite{mcleod2021modular} provided a better performance,
in general, than \textcite{micchi2021deep}, and also that
the gap between \gls{augmentednet} and the other models
seems larger. This may be an indication of the misleading
nature of chords. Being heavily skewed towards certain
``common'' classes (e.g., $\rnI$, $\rnV$, $\rnVsev$) in the
musical practice, often the task of evaluating global
accuracy does not show the full picture of which chords are
overused by a model.

\phdfigure[Confusion matrix of the Roman numeral numerators
($\elnum$) for \gls{augmentednet}. Rows represent the target
class and columns the predicted class. Values reported in
percentage, rounded to the closest integer. The color
intensity is mapped to the percentage
value]{confusion_matrix_napoleslopez2022}
