% Copyright 2022 Néstor Nápoles López

In the third experiement, the accuracy of the models on the
numerator $\elnum$ was evaluated. One of the purposes of the
\gls{natem} algorithm was to standardize the conditions in
which a tonicization would be invoked. As long as the key
context and \gls{pcset} are the same between the
ground-truth and the prediction, the Roman numeral numerator
will be the same. This might also be useful to help the
models recognize chords that do not exist in the vocabulary
of their annotations, such as \gls{neapolitan} and
\gls{augsix} chords. For each model, a confusion matrix is
shown with the target chord classes in each row, and the
predicted chord classes in each column.
