% Copyright 2022 Néstor Nápoles López

In order to perform the evaluation, all the annotations in
the \gls{romantext} files, including the ground truth, were
decomposed into their \gls{pcset} $\elpcset$, key $\elkey$,
and inversion $\elinv$ components (middle columns). The
components $\elpcset$ and $\elkey$ were processed by
\gls{natem} to retrieve a numerator $\elnum$ and
tonicization $\elden$. This process is illustrated for a
ground-truth example of the test set in
\reftab{natem_translation}. The process performed on the
predictions of each model is identical, using their
\gls{romantext} files.

Three evaluation experiments were performed with this
workflow:
\begin{enumerate}
    \item First, the models were evaluated in their
    inference time obtaining the predictions from the test
    set \gls{musicxml} (or \gls{humkern}) files.
    \item Second, the models were evaluated in the accuracy
    of their $\elpcset$, $\elkey$ and $\elinv$ components
    \item Third, using the $\elnum$ retrieved from
    \gls{natem}, the accuracy of the $\elnum$ numerators was
    evaluated, compared to the $\elnum$ obtained for the
    ground truth
\end{enumerate}

\phdtable[Translation process of the ground-truth
annotations in one of the test files of the \gls{kmt}
dataset. The original annotations (leftmost column) are
parsed from a \gls{romantext} file, and decomposed into a
\gls{pcset} $\elpcset$, key $\elkey$, and inversion $\elinv$
components (middle columns). Then, a numerator $\elnum$ and
tonicization $\elden$ were  retrieved from \gls{natem}
(rightmost columns)]{natem_translation}

% In the evaluation of the models for accuracy on individual
% tasks, the components $\elpcset$, $\elkey$, and $\elinv$
% are compared to the ground truth. In the Roman numeral
% accuracy, the $\elnum$ numerator of the ground truth is
% compared against the $\elnum$ of the models. This second
% evaluation is helpful to find how well the models are able
% to retrieve the different tonal contexts of the chords in
% the ground truth. These results are shown in the form of
% confusion matrices of the target numerator $\elnum$
% against the predicted numerator $\elnum$.


