% Copyright 2022 Néstor Nápoles López

In order to perform the evaluation, all the annotations in
the \gls{romantext} files, including the ground truth, are
decomposed into \gls{pcset} $\elpcset$, key $\elkey$, and
inversion $\elinv$ components. The components $\elpcset$ and
$\elkey$ are sent to the \gls{natem} algorithm to retrieve a
numerator $\elnum$ and optional tonicization $\elden$.

In the evaluation of the models for accuracy on individual
tasks, the components $\elpcset$, $\elkey$, and $\elinv$ are
compared to the ground truth. In the Roman numeral accuracy,
the $\elnum$ numerator of the ground truth is compared
against the $\elnum$ of the models. This second evaluation
is helpful to find how well the models are able to retrieve
the different tonal contexts of the chords in the ground
truth. These results are shown in the form of confusion
matrices of the target numerator $\elnum$ against the
predicted numerator $\elnum$.

\phdtable[Translation process of the ground-truth
annotations in one of the test files from the \gls{kmt}
dataset]{natem_translation}

