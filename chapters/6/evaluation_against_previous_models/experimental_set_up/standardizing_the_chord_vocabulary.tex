% Copyright 2022 Néstor Nápoles López

The vocabulary of each model is different, as shown in
\reftab{chord_vocabularies}. In addition to the different
qualities of chords supported by each model, the \gls{rna}
syntax is also often distinct (see
\refsec{standardizationofromannumeralannotations}). Some of
the models provide two layers of key analysis, such as
\gls{augmentednet}, whereas others provide one key
prediction, such the model in \textcite{mcleod2021modular}.
This makes it difficult to have a fair comparison across all
of them. Furthermore, the models might be in fact able to
recognize chords that are not explicitly considered in their
vocabulary. For example, neither the \gls{melisma} nor the
\textcite{mcleod2021modular} models explicitly recognize
\gls{augsix} chords. However, a label of $\rnVsev$ may be
interpreted as an \gls{augsix} chord in a certain key
context, but the predictions of the model will not indicate
it as such. This is true for other types of chords too, such
as \gls{neapolitan} chords. This was a motivation to propose
the \gls{natem} algorithm (see
\refsec{thenumeratorandtonicizationestimationmethod}), which
makes it possible to use a common representation for all the
annotations of the five models, regardless of the source
format. The inputs to the \gls{natem} algorithm are a
\gls{pcset} $\elpcset \in \setpcset$ (see
\refsec{thevocabularyofpitch-classsets}) and a key $\elkey
\in \setkey$ (see \refsec{thevocabularyofmusicalkeys}). The
evaluation across all models considers 

\phdtable[The chord vocabularies of the compared
models]{chord_vocabularies}
