% Copyright 2022 Néstor Nápoles López

The vocabulary of each model is different, as shown in
\reftab{chord_vocabularies}. In addition to the different
qualities of chords supported by each model, the \gls{rna}
syntax is also often distinct (see
\refsec{standardizationofromannumeralannotations}). 

\phdtable[The chord vocabularies of the compared models.
C\&S21 refers to \textcite{chen2021attend}, Mi21 to
\textcite{micchi2021deep}, and M\&R21 to
\textcite{mcleod2021modular}]{chord_vocabularies}

Some of the models provide two layers of key analysis
(\gls{augmentednet}, \cite{micchi2021deep, chen2021attend}),
whereas others provide only one key prediction
(\gls{melisma}, \cite{mcleod2021modular}). This difficults a
fair comparison between the models. Furthermore, the models
might be in fact able to recognize chords that are not
explicitly considered in their vocabulary. For example,
neither the \gls{melisma} nor the
\textcite{mcleod2021modular} models explicitly recognize
\gls{augsix} chords. However, a label of $\rnVsev$ may be
interpreted as an \gls{augsix} chord in a certain key
context, even if the predictions of the model do not
indicate it as such.\footnote{See the examples in
\refsec{thenumeratorandtonicizationestimationmethod} where a
$\rnVsev$ chord in one key is the same \gls{pcset} as a
$\rnGer$ chord in a different key} This is true for other
types of chords too, such as \gls{neapolitan} chords. All of
these issues combined were a motivation to propose the
\gls{natem} algorithm (see
\refsec{thenumeratorandtonicizationestimationmethod}), which
makes it possible to use a common representation for all the
annotations of the five models, regardless of their original
syntax. The inputs to the \gls{natem} algorithm are a
\gls{pcset} $\elpcset \in \setpcset$ (see
\refsec{thevocabularyofpitch-classsets}) and a key $\elkey
\in \setkey$ (see \refsec{thevocabularyofmusicalkeys}).
Given those inputs, the algorithm provides one of the 31
numerators $\elnum \in \setnum$ (see
\refsec{thevocabularyofromannumeralnumerators}), and a
tonicization $\elden \in \setden$, if necessary. This
standardization is also applied to the ground truth
annotations, providing the same \gls{rna} vocabulary for all
models and ground truth for direct comparison. This process
of running the annotations in the ground truth through the
\gls{natem} algorithm is explained below.

