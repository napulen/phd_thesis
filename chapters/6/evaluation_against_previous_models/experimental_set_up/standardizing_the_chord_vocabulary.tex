% Copyright 2022 Néstor Nápoles López

The vocabulary of each model is different, as shown in
\reftab{chord_vocabularies}. In addition to the different
qualities of chords supported by each model, the \gls{rna}
syntax is also often distinct (see
\refsec{standardizationofromannumeralannotations}). This
makes it difficult for a fair comparison across all of them.
Furthermore, the models might be in fact able to recognize
chords that are not explicitly mentioned. For example,
neither the \gls{melisma} nor the
\textcite{mcleod2021modular} models explicitly recognize
\gls{augsix} chords. However, a label of $\rnVsev$ in the
correct key context may be interpreted as an \gls{augsix}
chord and recognized regardless of how it is annotated. This
is true for other types of chords too, such as
\gls{neapolitan}. This was a motivation to propose the
\algorithmrn{} (see
\refsec{thenumeratorandtonicizationestimationmethod})
algorithm for the retrieval of \gls{rna} numerators from
\gls{pcset} and keys. 

\phdtable[The chord vocabularies of the compared
models]{chord_vocabularies}
