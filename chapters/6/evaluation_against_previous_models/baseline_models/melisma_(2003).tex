% Copyright 2022 Néstor Nápoles López

A comparison against the \gls{melisma} model is presented
for historical reference. \gls{melisma} was arguably the
first end-to-end \gls{arna} system that could annotate an
arbitrary musical score. After the development of the
\emph{tsroot} program \parencite{sapp2009tsroot}, the
\gls{melisma} system was capable of processing music scores
in the \gls{humkern} representation. This model, to the best
of my knowledge, has never been evaluated against other
methods. Thus, an evaluation is presented here. The
evaluated model is the version capable of annotating
\gls{humkern} inputs, which is used in the \emph{KernScores}
library \parencite{sapp2005online}. Thus, all examples in
the test set of the comparison were converted to
\gls{humkern}. \reftab{haydnsample_melisma} shows the
representation of the input (left) and output annotations
(right) provided by \gls{melisma}.

\phdtablefitv[Example input (left \gls{spine} in
\gls{humkern}) and output (right \gls{spine} in
\gls{humharm}) from the \gls{melisma} model. The input
corresponds to the music excerpt in
\reffig{haydnsample}][0.9]{haydnsample_melisma}

Each of the \gls{humharm} outputs generated by \gls{melisma}
is translated to \gls{romantext} for direct comparison with
the ground truth of the test set. The translation of
\reftab{haydnsample_melisma} is shown below:

\begin{verbatim}
    Composer: Haydn, Franz Joseph
    Title: Op. 20 No. 3 - IV
    Model: Melisma (2003), translated by Néstor Nápoles López
    
    Time Signature: 4/4
    m1 b2 C: ii b2.5 i6 b3 V2
    m2 i b3 V43 b3.5 i b4 V6
    m3 V b1.5 V7 b2.5 i b3 V6 b3.5 i b4 ii
    m4 V b1.5 V2 b2 ii
\end{verbatim}

In this excerpt, the system unfortunately fails to provide
the annotation of the key at the beginning of the piece.
Although some of the annotations are correct, they are
interpreted in the default key context of $\keyC$. This is
not always the case. Oftentimes, the model is able to
provide key annotations at the beginning of the piece and in
modulations within the piece. The reasons for these failures
were not explored further and are left for future work.

In total, out of 94 files in the test set, the \gls{melisma}
is able to provide valid \gls{rna} outputs in the
\gls{humharm} format for \missing{XX} of them.
