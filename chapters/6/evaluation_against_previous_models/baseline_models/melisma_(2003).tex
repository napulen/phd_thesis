% Copyright 2022 Néstor Nápoles López

A comparison against the \gls{melisma} model is presented
for historical reference. \gls{melisma} was arguably the
first end-to-end \gls{arna} system that could annotate an
arbitrary musical score. After the development of the
\emph{tsroot} program in \textcite{sapp2009tsroot}, the
\gls{melisma} system was capable of processing music scores
in the \gls{humkern} representation. This model, to the best
of my knowledge, has never been evaluated against other
methods. Thus, an evaluation is presented here. The
evaluated model is the version capable of annotating
\gls{humkern} inputs, which is used in the \emph{KernScores}
library \parencite{sapp2005online}. Thus, all examples in
the test set of the comparison were converted to
\gls{humkern}. \reftab{haydnsample_melisma} shows the
representation of the input (left) and output annotations
(right) provided by \gls{melisma}. In this excerpt, the
system unfortunately fails to provide the annotation of the
key at the beginning of the piece. Although some of the
annotations are correct, they are interpreted in the default
key context of $\keyC$. This is one among several cases
where the inputs cannot be correctly processed by
\gls{melisma}.

\phdtablefitv[Example input (left \gls{spine} in
\gls{humkern}) and output (right \gls{spine} in
\gls{humharm}) from the \gls{melisma} model. The input
corresponds to the music excerpt in
\reffig{haydnsample}][0.9]{haydnsample_melisma}

Each of the \gls{humharm} outputs generated by \gls{melisma}
is translated to \gls{romantext} for direct comparison with
the ground truth of the test set. 

\begin{verbatim}
    Composer: Haydn, Joseph
    Title: Op.20 No.3 - IV
    
    Time Signature: 4/4
    m1 b2 C: ii b2.5 i6 b3 V2
    m2 i b3 V43 b3.5 i b4 V6
    m3 V b1.5 V7 b2.5 i b3 V6 b3.5 i b4 ii
    m4 V b1.5 V2 b2 ii
\end{verbatim}
