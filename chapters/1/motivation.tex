% Taken verbatim from thesis proposal

\guide{Introduction.}
In this scientific and cultural age, it is not difficult to
defend the use of computers for studying any field,
including music. We expect computers to automate much of our
``chores'', annotate large volumes of data that we can not
possibly annotate, and democratize the access to resources
beyond the wealthy societies to create more fair
opportunities for every human being to receive education and
pursue their passions, including music.

Yet, I would not like to start this thesis by making an
argument about how the development of an automated tonal
analysis machine is going to annotate large volumes of music
for us, as important as that application may be.

What I want to suggest, at first and overall, is that music
is a very complex, highly-dimensional phenomenon. As such,
our efforts to explain or simplify its underlying logic will
break at one point or another. Particularly, I see Western
classical music as an example of those efforts for
simplification and collective intelligibility; the evolution
of the Common Western Music Notation system, and the
evolution of the harmonic theories that have facilitated the
explanation of what composers of the canon have done. Giving
names to the conventions, patterns, and structures that
repeat and become familiar to the listener and the music
composer.

Music theory fails, in the sense that it is only as good as
it is able to explain conventions, patterns, and structures
of all existing music.

\dots something here

% Bring example of "augmented unison" in the Huron rules and
% a chromatic passage where it doesn't work

\textcite{huron2016voice} summarized voice leading
rules.\footnote{See Chapter~2.} Among these rules, there is
one on the forbidden augmented and diminished melodic
intervals. Although this rule seems consistent with the
expected practice of voice leading, it demonstrates to break
in passages with chromatic basslines. That is, a chord
progression with a chromatic line \textbf{requires} an
augmented unison to achieve the best voice leading possible.
Coming up with rules that work for every scenario is nearly
impossible. Testing them in a computer algorithm is
feasible, however.

\dots something here

\guide{Why do we study these things computationally.}
The reason why I think it is important to study tonality
computationally is because music easily evokes human bias.
It is easy to summon the personal bias when listening to a
piece of music. Likewise, when analyzing a piece of music. I
like to think of computer algorithms as unbiased judges of
the theory. Emotionless, deterministic. If the theories
generalize well to the musical repertoire, it will show in
the output of an algorithm. If not, it is likely that the
output will reveal where the theory fails to explain the
musical phenomenon. In this thesis, it is my intention not
only to present my experiments designing a neural network
for tonal music analysis, but also the interpretation of the
results. For example, when a state-of-the-art neural network
is trained to recognize modulations, what is it that the
internal representations of the network (i.e., the hidden
layers) are learning? To my knowledge, this aspect is
generally less explored in machine learning literature,
where the focus is on improving the evaluation metrics of
new algorithms versus old algorithms. As a theorist, I want
to know what is it that the ruthless optimization method is
learning when it provides meaningful outputs of tonal music
analysis. The multitask tonal analysis done here, although
it consists of multiple tasks, is focused around two
specific problems: chord analysis and key analysis. These
problems have been studied thoroughly in isolation and,
sometimes, in conjunction. I summarize existing methods for
chord estimation and key estimation.


I suggest that to be the reason why computers are and should
be the companion of old and new music theories. They help us
to test our assumptions in practical scenarios, in unbiased
ways and faithful to the formulations in the theory.

\guide{Problem.}
One of the most common ways to analyze a piece of tonal
music is through Roman numeral analysis. This requires the
inspection of several attributes related to chords and keys.
Chords can be inspected in terms of their properties: root,
quality, inversion, and function. Keys can be inspected in
terms of their temporal scope as modulations or
tonicizations~\parencite{napoles_lopez2020local}. Each of
these attributes (or tasks) of Roman numeral analysis can be
modeled in isolation. As a result, many models for automatic
key and chord analysis exist in the Music Information
Retrieval (MIR) literature. Recent research has found that
analyzing several tonal tasks simultaneously leads to more
robust MIR models. This happens via multitask learning, a
technique where a machine learning model solves several
problems at once~\parencite{ruder2017overview}. This has
motivated the research of multitask models for Roman numeral
analysis. However, even the best of the models has
significant limitations. For example, recent models output
the correct Roman numeral annotation $\sim$42\% of the
times~\parencite{chen2021attend, micchi2020not}.

\guide{Proposal.}
Here, I propose to address existing limitations in multitask
Roman numeral analysis models. In particular, I will focus
on four improvements: (1) to standardize the syntax and
quality across various existing Roman numeral analysis
datasets; (2) to investigate new data-augmentation
techniques that overcome the scarcity of expert-annotated
data; (3) to enhance the design of neural network
architectures for Roman numeral analysis; and (4) to explore
different combinations of tonal tasks in multitask learning
configurations.

Combining these ideas, I trained a new Roman numeral
analysis model. The resulting model is capable of annotating
large amounts of symbolic music files with Roman numeral
labels. Among other applications, this facilitates searching
in music collections. For example, searching by chord
progressions or by modulation trajectories.
