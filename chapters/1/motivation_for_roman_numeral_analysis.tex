% Copyright 2022 Néstor Nápoles López

One of the most common ways to analyze a piece of tonal
music is through \gls{rna}. This requires the inspection of
several attributes related to chords and keys. For example,
the inversion of a chord, its root, and whether it tonicizes
a momentary musical key. Clearly, if a \gls{rna} is computed
automatically, these tonal attributes become automatically
available too. However, the Roman numeral notation itself
also describes complex---sometimes exotic---chords,
fluctuations of musical key, and other tonal situations in a
very compact syntax. This is maybe the reason why a number
of musicians have preferred it throughout the years. It is a
notation that is compact enough to be annotated within a
regular musical score, yet encodes explanations to advanced
musical concepts that would require much more words and
effort to be described otherwise.

In order to illustrate this more visually, consider the
musical example in \reffig{op28no20_intro_analysis}. In this
example, which features the entirety of Chopin's Prelude Op.
28 No. 20, two successions of chords can be seen in measures
2 and 12. One interesting aspect about these two successions
of chords is that they start with two identical chords.
Namely, an $\pitchAb$ major triad, and a $\pitchDb$ major
triad. However, in the first occurrence (i.e., m.2), 

\phdfigurefullpage[Musical analysis of two successions of
chords within Chopin's Op.~28
No.~20.]{op28no20_intro_analysis}