% Copyright 2022 Néstor Nápoles López

One of the most common ways to analyze a piece of tonal
music is through \gls{rna}. This is a notation that is
compact enough to be annotated within a regular musical
score, yet encodes explanations to advanced musical concepts
that would require much more words and effort to be
described otherwise. In order to illustrate this more
visually, consider the musical example in
\reffig{op28no20_intro_analysis}. In this example, which
features the entirety of Chopin's Prelude Op. 28 No. 20, two
successions of chords can be seen in measures 2 and 12.
Although the note arrangement and chord labels of the first
two chords ($\pitchAb$ major and $\pitchDb$ major,
highlighted in green in the figure) are identical, their
\gls{rna} labels (annotated below the notes in the figure)
are not. This is because in \gls{rna}, not only the chords
are important, but also the context of the musical key.

\phdfigure[Musical analysis of two successions of chords
within Chopin's Op.~28 No.~20.]{op28no20_intro_analysis}

This requires the inspection of several attributes related
to chords and keys. For example, the inversion of a chord,
its root, and whether it tonicizes a momentary musical key.
Clearly, if a \gls{rna} is computed automatically, these
tonal attributes become automatically available too.
However, the Roman numeral notation itself also describes
complex---sometimes exotic---chords, fluctuations of musical
key, and other tonal situations in a very compact syntax.
This is maybe the reason why a number of musicians have
preferred it throughout the years. 


% Something interesting about these two successions is that
% they start with the same two chords. Namely, an $\pitchAb$
% major triad, and a $\pitchDb$ major triad. However, the
% musical key in both instances can be interpreted
% differently. In \gls{rna}, this differing key context is
% important, and part of the information that the analyst
% captures in the annotations.

% In the first instance (i.e., m.~2), the chords can be
% analyzed in the context of the $\keyAb$ major key. The
% first and second chord, $\pitchAb$ major and $\pitchDb$
% major, are a tonic and subdominant chords, respectively.
% These are followed by the third chord, a dominant seventh
% chord ($\pitchEb\rnseven$ dominant seventh), and the
% fourth chord, a tonic triad again ($\pitchAb$ major). In
% \gls{rna}, we express this as the sequence of Roman
% numeral labels in \refeq{succession1}.

% \begin{equation} \label{eq:succession1}
%     \rnwkey{\keyAb}{\rnI} \quad \rnIV \quad \rnV\rnseven
%     \quad \rnI \end{equation}

% In the second succession of chords (i.e., m.~12), these
% can be analyzed in the context of the $\keyc$ minor key,
% which is the global key of the piece. Here, the first
% chord ($\pitchAb$ major) is a submediant chord and the
% second chord ($\pitchDb$ major) is a flattened second
% degree, commonly known as a
% \gls{neapolitan}.\footnote{Although this chord is
% expressed here as $\musFlat{}\rn{II}$, it is more commonly
% represented as $\rnN$ in \gls{rna}.} The third chord is a
% dominant seventh chord ($\pitchG\rnseven$ dominant
% seventh), and the last chord is the tonic of the piece,
% $\pitchC$ minor. In \gls{rna}, this succession is
% expressed as in \refeq{succession2}.

% \begin{equation} \label{eq:succession2}
%     \rnwkey{\keyc}{\rnVI} \quad \musFlat{}\rn{II} \quad
%     \rnV\rnseven \quad \rni \end{equation}

