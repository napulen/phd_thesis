% Taken verbatim from thesis proposal

\guide{Problem}
One of the most common ways to analyze a piece of tonal music is through Roman numeral analysis.
This requires the inspection of several attributes related to chords and keys.
Chords can be inspected in terms of their properties: root, quality, inversion, and function.
Keys can be inspected in terms of their temporal scope as modulations or tonicizations~\cite{napoles_lopez_local_2020}.
Each of these attributes (or tasks) of Roman numeral analysis can be modeled in isolation.
As a result, many models for automatic key and chord analysis exist in the Music Information Retrieval (MIR) literature.
Recent research has found that analyzing several tonal tasks simultaneously leads to more robust MIR models.
This happens via multitask learning, a technique where a machine learning model solves several problems at once~\cite{ruder_overview_2017}.
This has motivated the research of multitask models for Roman numeral analysis.
However, even the best of the models has significant limitations.
For example, recent models output the correct Roman numeral annotation $\sim$42\% of the times~\cite{chen_attend_2021, micchi_not_2020}.

\guide{Proposal}
In this dissertation, I propose to address existing limitations in multitask Roman numeral analysis models.
% I consider that enhancements are possible all along the process.
In particular, I will focus on four improvements:
(1) to standardize the syntax and quality across various existing Roman numeral analysis datasets;
(2) to investigate new data-augmentation techniques that overcome the scarcity of expert-annotated data;
(3) to enhance the design of neural network architectures for Roman numeral analysis;
and (4) to explore different combinations of tonal tasks in multitask learning configurations.

Combining these ideas, I will train a new Roman numeral analysis model.
The resulting model will be capable of annotating large amounts of symbolic music files with Roman numeral labels.
Among other applications, this will facilitate advanced searching in music collections.
For example, searching by chord progressions or by modulation trajectories.
