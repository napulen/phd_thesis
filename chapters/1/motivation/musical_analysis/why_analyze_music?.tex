% Copyright 2021 Néstor Nápoles López

% This is \refsubsubsec{whyanalyzemusic?},
% which introduces the why analyze music?.

I like to think about musical analysis as a form of lossy compression of a musical performance, or a musical composition.

There are different types of analysis. For example, \emph{motivic} analysis or \emph{structural} analysis. The annotations and considerations in one type of analysis may have little to do with a different type of analysis. What I think is important is that in either case, the analyst decides to pay attention to a specific dimension of the analyzed material. That is, the analysis highlights something ``important'' in the music.

\phdparagraph{highlights ``important'' information}

If an analyst has decided that something is ``important'', I think it is likely that other analysts or musicians will also find value in the highlighted information.

\phdparagraph{lossy compression of the music around that information}

Assuming some information is important or ``salient'' in the musical performance (or composition), I consider one of the interesting outcomes of an analysis is the kind of compression process that is performed over the music. This compression is almost always lossy, highlighting the dimension in question and ignoring other dimensions.
