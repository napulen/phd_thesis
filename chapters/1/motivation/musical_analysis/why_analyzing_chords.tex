% Copyright 2021 Néstor Nápoles López

% This is \refsubsubsec{whyanalyzingchords},
% which introduces the why analyzing chords.

Arguably, the idea of a ``chord'' provides a framework for analyzing vertical sonorities in music. What is an extremely large combinatorial space (e.g., all possible pitches that may sound together), becomes a category (i.e., a chord). We can then speak in terms of these categories, and the relationships they have with each other. It is not surprising then, that chords have become a very useful analytical dimension in music of the common-practice period and beyond. It is a lossy compression that facilitates the intelligibility of simultaneous sounds.

% \phdparagraph{intelligibility of simultaneous sounds}
