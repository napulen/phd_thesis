% Copyright 2021 Néstor Nápoles López

This dissertation is organized in seven chapters and an
appendix. \refchap{introduction} described the motivations,
challenges, and existing progress toward \gls{arna}.
\refchap{introductiontoromannumeralanalysis} introduces
\gls{rna} from a musical perspective, its history and
digitization. \refchap{background} introduces the relevant
research around music representation, deep learning, and
\gls{mir} for tonal music analysis.
\refchap{dataacquisitionandpreparation} introduces the
publicly available datasets, the data aggregation process,
and the data-augmentation technique based on synthesis of
training examples. \refchap{modeldesign} presents the design
choices of the convolutional recurrent neural network:
number of layers, multitask configuration, input and output
representations, etc. \refchap{experimentalevaluation}
presents the evaluation of the model against the
state-of-the-art in \gls{arna}. \refchap{conclusions}
summarizes the main findings and presents closing remarks on
the current state of automatic tonal analysis and future
directions in the field.
