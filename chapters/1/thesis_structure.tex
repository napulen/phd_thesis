% Copyright 2021 Néstor Nápoles López

This dissertation is organized in seven chapters and an
appendix. \refchap{introduction} described the motivations,
challenges, and existing progress toward \gls{arna}.
\refchap{introductiontoromannumeralanalysis} introduces
\gls{rna} from a musical perspective, its history and
digitization. \refchap{background} introduces the relevant
research around music representation, deep learning, and
\gls{mir} for tonal music analysis.
\refchap{dataacquisitionandpreparation} introduces the
publicly available datasets, the data aggregation process,
and the data-augmentation technique based on synthesis of
training examples. \refchap{modeldesign} presents the design
choices of the \gls{crnn} presented in this dissertation,
\gls{augmentednet}: its input and output representations,
convolutional layers, recurrent layers, \gls{mtl}
configuration, and methods to generate \gls{rna} labels from
the predictions of the model.
\refchap{experimentalevaluation} presents the evaluation of
the model. This includes ablation studies over the design
choices of the network, an exploration of the effects of
data augmentation, a full evaluation on the aggregated
dataset and test sets of each individual dataset, and a
comparison against four existing \gls{arna} methods.
\refchap{conclusions} summarizes the main findings and
presents closing remarks on the current state of automatic
tonal analysis and future directions in the field. In
addition to this, this chapter introduces the resources for
reproducing, improving, and sharing the results of this
research. Lastly,
\refappendix{amethodforsystematicromannumeralanalysis} is a
special annex chapter that introduces several formal methods
for systematic Roman numeral analysis, which are referenced
throughout the remaining chapters.
