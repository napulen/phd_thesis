I have studied Western tonal music theory in different
levels with different instructors throughout my life. It is
unclear to me, \emph{why} is it that Harmony and
Counterpoint theories exist and have prevailed to this date.
Is it that these theories were aimed at facilitating the
training of new music composers? To understand how previous
composers thought about music? To disentangle the patterns
in existing music? To understand how music ``works''? To
simplify the perceptual organization of simultaneous notes
into a set of rules that can be more easily understood? In
any case, many theories have emerged over the years. One of
the meta-theories we use nowadays, in Western tonal music,
is Roman numeral analysis. An analytical tool, in principle,
Roman numeral analysis is useful as a framework for a
theorist or musician to describe the harmonic context of the
music. This is necessary as more complex tonal relationships
(e.g., modulations) are introduced in the music composition
process. When moving to the digital domain, we can still
think of this framework as an analytical tool that allows us
to understand music in a simpler way. A summary of the
harmonic context. My motivation with this dissertation is to
connect the theories that have evolved over the years, with
the ruthless and methodological outputs of computer
algorithms. Needless to say, I am not the first to do this,
and as the saying goes, I stand in the shoulders of giants
to see any further.
