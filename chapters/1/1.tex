\phdchapter{introduction}

% \begin{quote}
%     A quote
% \end{quote}
% \clearpage

\phdsection{motivation}
    \phdinput{introduction}
\phdsection{thesis structure}
    Chapter~\ref{chap:introductiontoromannumeralanalysis} introduces Roman numeral analysis, its history and digitization.
    Chapter~\ref{chap:background} introduces the relevant research around music information retrieval for tonal music analysis.
    Chapter~\ref{chap:dataacquisitionandpreparation} introduces the datasets and data operations (mlops) employed to train the Roman numeral analysis model presented in this thesis.
    Chapter~\ref{chap:modeldesign} presents the design choices of the convolutional recurrent neural network: number of layers, multitask configuration, input and output representations, etc.
    Chapter~\ref{chap:experimentalevaluation} presents the evaluation of the model against the state-of-the-art in automatic Roman numeral analysis
    Chapter~\ref{chap:conclusions} summarizes the main findings and presents closing remarks on the current state of automatic tonal analysis and future directions in the field.
