% Copyright 2021 Néstor Nápoles López

% This is \refsubsubsec{ambiguousannotations}, which
% introduces the ambiguous annotations.

In his dissertation, \textcite{ju2021addressing} describes
ambiguity in music analysis and, particularly, chord labels.
Ju argues that one possible reason for the ambiguity in
musical analysis is \emph{under-specification}. That is, the
fact that based on the information provided, multiple
answers can respond the same question. A typical example in
chord labels would be whether one analyst considers a note
as a non-chord tone, or a chord tone. If the note in
question follows the ``rules'' for non-chord tones, then it
may be considered as such, but a different analyst might
decide to consider it a chord tone. Both answers are
correct.

\phdfigure[Example of ambiguity in chord analysis. Two analyses are offered. In \textbf{[A]}, the second beat is considered an Emin7 chord. In \textbf{[B]}, the note in the second beat is considered a passing tone (i.e., there is no chord in the second beat). Reasonable arguments can be made for either analysis. Thus, both are correct.][1.0]{ambiguity}
