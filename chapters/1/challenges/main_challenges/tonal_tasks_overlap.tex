% Copyright 2021 Néstor Nápoles López

% This is \refsubsubsec{tonaltasksoverlap}, which introduces
% the tonal tasks overlap.

Tonal music analysis is abstract and often not well defined,
in terms of what constitutes one task or the other.

\phdparagraph{the boundary between chords and keys}

In his work with \emph{keyscapes},
\textcite{sapp2011computational} presented examples of key
analyses with different window lengths. In some of these
analyses, with sufficiently short windows, an overlap
between key analysis and chord analysis occurred. For
example, a short window captured a change in harmony instead
of a change of key. In our work designing a \gls{lke}
algorithm \parencite{napoleslopez2019keyfinding}, we found
something similar with key-profiles and sequences of pitch
classes analyzed with a \gls{hmm}. A \gls{keyprofile} that
heavily penalizes non-diatonic degrees finds a ``change of
key'' in Chopin's Op.28 No.20, which corresponds to a
\gls{neapolitan} chord, as shown in \reffig{op28no20}. It is
unclear whether this type of chromatic chord could be
considered a deviation of the musical key. Arguments can be
made for both positions on this topic. Thus, the line
between \emph{key analysis} and \emph{chord analysis} can be
blurry.

\phdfigure[An example of a \gls{lke} algorithm (bottom),
showing a change of key to $\keyf$ in measure 12. The
corresponding score, Chopin Op. 28 No. 20 (above), indicates
this part of the piece as a \gls{neapolitan} chord: measure
12, beat 2]{op28no20}

\phdparagraph{pitch spelling}

In their work on a pitch spelling algorithm,
\textcite{teodoru2007pitch} described the dependency of
``pitch spelling'' task to \gls{lke} and \gls{acr}, noting
that 

\begin{italicquotes}
[some tonal] situations require a deeper notion of the
harmonic state than provided by the local key, as in the
German \gls{augsix} chord, which seems nearly impossible to
spell correctly without recognizing it as such.
\end{italicquotes}

This poses the question of whether it is possible to design
an algorithm to spell the pitches in a \gls{midi} file,
without also developing a key estimation and chord
recognition algorithms. \gls{arna}, naturally sitting at the
intersection of various tonal tasks, is constantly involved
in similar ``chicken and egg'' problems. This may be also
one of the motivations for \gls{mtl} approaches, trying to
leverage the need to tackle several tonal problems at once.
