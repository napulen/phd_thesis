% Copyright 2021 Néstor Nápoles López

% This is \refsubsubsec{tonaltasksoverlap}, which introduces
% the tonal tasks overlap.

Unlike other perceptual problems, tonal music analysis is
often not well defined, in terms of what constitutes one
task or the other.

\phdparagraph{the boundary between chords and keys}

In his work with ``keyscapes'',
\textcite{sapp2011computational} presents examples of key
analysis with different window lengths. In some of these
analysis, with sufficiently short windows, an overlap
between key analysis and chord analysis can be inferred. In
our work designing a local-key-estimation algorithm
\parencite{napoleslopez2019keyfinding}, we found something
similar with key-profiles and sequences of pitch classes
analyzed with a \gls{hmm}. A key profile that heavily
penalizes non-diatonic degrees finds a ``change of key'' in
Chopin's Op.28 No.20, which corresponds to a Neapolitan
chord. It is unclear whether this type of chromatic chord
could be considered a deviation of the musical key.
Arguments can be made for both positions on this topic.
Thus, the line between \emph{key analysis} and \emph{chord
analysis} can be blurry.

\phdparagraph{pitch spelling}

In their work on a pitch spelling algorithm,
\textcite{teodoru2007pitch} describe the dependency of this
task to local-key estimation and chord recognition, noting
that ``[some tonal] situations require a deeper notion of
the harmonic state than provided by the local key, as in the
German augmented sixth chord, which seems nearly impossible
to spell correctly without recognizing it as such.'' This
poses the question of whether it is possible to design an
algorithm to spell the pitches in a \gls{midi} file, without
also developing a key estimation and chord recognition
algorithms. Similar ``chicken and egg'' problems can be
observed in several tasks related to perceived tonal
features, which make the definition and evaluation of the
problems quite challenging.
