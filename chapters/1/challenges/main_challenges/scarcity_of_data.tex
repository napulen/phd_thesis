% Copyright 2021 Néstor Nápoles López

% This is \refsubsubsec{scarcityofdata}, which introduces
% the scarcity of data.

Compared to other machine learning tasks, there is not as
much high-quality data available for training Roman numeral
analysis models. In order to illustrate some of the reasons
why this is the case, I contrast the process of annotating
data for the well-known MNIST dataset against Roman numeral
analysis. \reffig{mnist} shows a grid of nine handwritten
digits, which could be labeled to create a training
examples. \reffig{chords} shows a chord progression with
nine chords. Each of the chords can be labeled with a Roman
numeral annotation to create training examples. It can be
noted that:

\begin{enumerate}
    \item The handwritten digits can be labeled by a wider
    range of annotators, whereas the chords require an
    expert annotator of harmony
    \item The handwritten digits take less time to annotate
    than the chords
\end{enumerate}

\phdtwofigures[MNIST annotations][Roman numeral annotations][0.3][0.7]{mnist}{chords}

% \phdparagraph{comparison against mnist annotations}
% \phdparagraph{expert-annotated data} \phdparagraph{more
% time consuming}
