% Copyright 2021 Néstor Nápoles López

% This is \refsubsubsec{scarcityofdata}, which introduces
% the scarcity of data.

Compared to other machine learning tasks, there is not as
much high-quality data available for training \gls{arna}
models. In order to illustrate some of the reasons why this
is the case, I contrast the process of annotating data for
the well-known \gls{mnist} dataset
\parencite{lecun1989handwritten} against \gls{rna}.

\phdfigure[Handwritten digits to be annotated, as in the \gls{mnist} dataset][0.5]{mnist}
\phdfigure[Succession of chords to be annotated with Roman numerals, as in a \glspl{rna} dataset]{chords}

\reffig{mnist} shows a grid of nine handwritten digits,
which could be labeled to create training examples for the
\gls{mnist} dataset. \reffig{chords} shows a chord
progression with nine chords. Each of the chords can be
labeled with a Roman numeral annotation to create training
examples for a \gls{rna} dataset. It can be noted that:

\begin{enumerate}
    \item The handwritten digits can be labeled by a wider
    range of annotators, whereas the chords require an
    annotator with expertise on tonal harmony.
    \item Even for an annotator with expertise on tonal
    harmony, the handwritten digits take much less time to
    annotate than the chords.\footnote{As an informal test,
    I timed myself annotating the nine labels of each figure
    the first time I saw them. I spent, roughly, 6 seconds
    annotating the handwritten digits and 75 seconds
    annotating the Roman numeral labels. That is, annotating
    the chords was 12.5x more time consuming for me than
    annotating the handwritten digits.}
\end{enumerate}

In conjunction, these problems, make high-quality \gls{rna}
data expensive and scarce.


% \phdparagraph{comparison against mnist annotations}
% \phdparagraph{expert-annotated data} \phdparagraph{more
% time consuming}
