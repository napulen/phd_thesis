% Copyright 2021 Néstor Nápoles López

% This is \refsubsec{long-termgoal:bettermusictheories},
% which introduces the long-term goal: better music
% theories.

Perhaps a more interesting question is, what do we do if the
models become extremely accurate? One important aspect that
may be useful in music theory is that computers evaluate
rules that would take a long time for a human being to
internalize. A simple example would be voice-leading rules,
which take many hours for humans to develop expertise in,
but can be implemented and fine-tuned for a computer within
hours. \reffig{rimsky_voiceleading} shows an example of a
rule-based voice-leading algorithm, which I implemented
based on the rules in
\textcite[p.~10]{huron2016voice}.\footnotelink{https://github.com/napulen/romanyh}
The model arranges each chord based on the input \gls{rna}
annotations provided, respecting the encoded voice-leading
rules while connecting each pair of chords. 

\phdfigure[A voice-leading exercise that was automatically
generated from the \gls{rna} annotations in a modulation
example from Rimsky-Korsakov's Practical Manual of
Harmony]{rimsky_voiceleading}

One challenge in the implementation of this algorithm was
that some rules seemed to be detrimental to good voice
leading when they were implemented textually. For example,
the rule about augmented intervals:

\begin{italicquotes}
    Rule 16. Augmented intervals rule: Avoid augmented
    melodic intervals.
\end{italicquotes}

Implementing this rule unchanged results in a model avoiding
at all costs augmented unisons. One could argue that
computational models provide a rigorous empirical platform
to ``test'' the robustness of a music theory, a rigorous
platform that is difficult to come by in traditional human
analysis. Although the ``rules'' and assumptions in a deep
learning model are not as easy to interpret as the ones in
this voice-leading example, it is still possible to review
the features learned by a deep learning model in its
internal representations. This process has been studied in
computer vision, and it would be worth to be studied in
music as models keep getting increasingly accurate. I
consider this goal, interpretability, should be a long-term
goal of researchers developing these systems.
