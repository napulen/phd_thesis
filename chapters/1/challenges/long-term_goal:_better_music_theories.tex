% Copyright 2021 Néstor Nápoles López

% This is \refsubsec{long-termgoal:bettermusictheories},
% which introduces the long-term goal: better music
% theories.

A more interesting question is, what do we do when the
models become extremely accurate? I think this is the point
where the value of this research will become evident. In my
experience teaching computers to analyze musical
information, I have noticed that once set with the proper
methodologies, computers achieve results that would take a
long time for a human to develop from scratch. A simple
example would be voice-leading rules, which take many hours
for a human to internalize, but can be implemented and
fine-tuned for a computer within hours.
\reffig{rimsky_voiceleading} shows an example of a
rule-based voice-leading algorithm, which I implemented
based on the rules in
\textcite{huron2016voice}.\footnotelink{https://github.com/napulen/romanyh}
The model arranges each chord based on the input \gls{rna}
annotations provided, respecting voice-leading rules while
connecting each pair of chords. 

\phdfigure[A voice-leading exercise that was automatically
generated from the \gls{rna} annotations in a modulation
example from Rimsky-Korsakov's Practical Manual of
Harmony]{rimsky_voiceleading}

What I imagine a successful model can do is capture notions
of tonality that we have neither yet understood nor
materialized into a music theory. I imagine that
investigating the learned representations of such
hypothetical model could potentially accelerate the way we
create theories that explain the crafts of musical
composition, both in the \gls{commonpractice} as in modern
musical practice. I consider this should be a long-term goal
of these systems.
