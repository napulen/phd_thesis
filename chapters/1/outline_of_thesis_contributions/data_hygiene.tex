% Copyright 2021 Néstor Nápoles López

% This is
% \refsubsec{improveddataoperations(mlops)anddatahygiene},
% which introduces the improved data operations (mlops) and
% data hygiene.

In recent years, machine learning researchers have become
aware of the critical role that data plays in any
experiment. For example, making sure that the quality or
``hygiene'' of the data employed in an experiment meets a
certain quality. This effort frequently takes a considerable
amount of time in a machine learning project. The idea of
improving the data workflows has received the name of
\gls{mlops}. A description of some \gls{mlops} principles is
provided in \textcite{renggli2021data}.

There are at least four different digital standards of Roman
numeral annotations. The available data is encoded with
these different formats. Thus, it is challenging to
aggregate the data into a unified training dataset, often
leading to noisy and inaccurate training examples.
Throughout this \thesisdiss{}, I make an effort to describe
how to operate with the existing data. For example,
describing the process of detecting errors in the
annotations in \refsubsec{detectingerrorsintheannotations},
processing the outputs of different models in the evaluation
(see \refsubsec{experimentalsetup}), and, maybe more
importantly, releasing all the curated data used to train
the model presented here (see \refsubsec{preprocesseddata}).
This abstracts the data curation process for researchers
interested in re-training this or another model, or wanting
to reproduce the results.