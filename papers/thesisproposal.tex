\documentclass[12pt]{article}
\usepackage[utf8]{inputenc}
\usepackage{hyperref}

\title{Multitask tonal analysis in symbolic music representations}
\author{N\'estor N\'apoles L\'opez, PhD Candidate}
\date{}

\usepackage{natbib}
\usepackage{graphicx}
\usepackage[margin=2cm]{geometry}
\usepackage{enumitem}

% \newcommand{\guide}[1]{\textbf{#1}.}
\newcommand{\guide}[1]{}

\makeatletter
\renewcommand\@dotsep{200}   % default value 4.5
\makeatother

\begin{document}

\maketitle

\section*{Introduction / Motivation}

\guide{Problem}
One of the most common ways to analyze a piece of tonal music is through Roman numeral analysis.
This requires the inspection of several attributes related to chords and keys.
Chords can be inspected in terms of their properties: root, quality, inversion, and function.
Keys can be inspected in terms of their temporal scope as modulations or tonicizations~\cite{napoles_lopez_local_2020}.
Each of these attributes (or tasks) of Roman numeral analysis can be modeled in isolation.
As a result, many models for automatic key and chord analysis exist in the Music Information Retrieval (MIR) literature.
Recent research has found that analyzing several tonal tasks simultaneously leads to more robust MIR models.
This happens via multitask learning, a technique where a machine learning model solves several problems at once~\cite{ruder_overview_2017}.
This has motivated the research of multitask models for Roman numeral analysis.
However, even the best of the models has significant limitations.
For example, recent models output the correct Roman numeral annotation $\sim$42\% of the times~\cite{chen_attend_2021, micchi_not_2020}.

\guide{Proposal}
In this dissertation, I propose to address existing limitations in multitask Roman numeral analysis models.
% I consider that enhancements are possible all along the process.
In particular, I will focus on four improvements:
(1) to standardize the syntax and quality across various existing Roman numeral analysis datasets;
(2) to investigate new data-augmentation techniques that overcome the scarcity of expert-annotated data;
(3) to enhance the design of neural network architectures for Roman numeral analysis;
and (4) to explore different combinations of tonal tasks in multitask learning configurations.

Combining these ideas, I will train a new Roman numeral analysis model.
The resulting model will be capable of annotating large amounts of symbolic music files with Roman numeral labels.
Among other applications, this will facilitate advanced searching in music collections.
For example, searching by chord progressions or by modulation trajectories.


\section*{Previous Work}

\guide{Historical}
The first computational works on Roman numeral analysis were by Winograd~\cite{winograd_linguistics_1968} and Maxwell~\cite{maxwell_expert_1992}.
Later, the independent contributions by Temperley, Sleator, and Sapp led to the first end-to-end automatic Roman numeral analysis system~\cite{temperley_cognition_2004, sleator_melisma_2003, sapp_tsroot_2009}.
Subsequent studies include those of Raphael and Stoddard~\cite{raphael_functional_2004}, Illescas et al.~\cite{illescas_harmonic_2007}, and Magalh\~aes and de Haas~\cite{magalhaes_functional_2011}, who proposed Hidden Markov Models (HMMs), dynamic programming, and grammar-based approaches, respectively.

\guide{Nowadays}
More recently, deep neural networks have been the preferred method for approaching this problem.
Chen and Su~\cite{chen_functional_2018} were the first to introduce \emph{multitask learning}~\cite{ruder_overview_2017}.
With it, a neural network shares hidden representations across related tasks.
In their experiments, multitask learning outperforms task-specific models for key and chord analysis.
Chen and Su have recently continued this line of research using Transformer-based deep neural networks~\cite{chen_attend_2021}.
Meanwhile, Micchi et al.~\cite{micchi_not_2020} have also favored multitask learning, adapting it for convolutional recurrent neural networks.


\section*{Proposed Research / Methodology}

The methods of this dissertation follow the conventions of supervised machine learning projects.
In supervised learning, a model is trained with expert-annotated data to solve a task.
Here, several tonal tasks are solved simultaneously via multitask learning.
The model is then tested on unseen examples to evaluate its generalizability.
The evaluation metric used here consists of assigning the correct Roman numeral label.
The research process can be roughly divided in three steps:
(i) data acquisition and preparation;
(ii) model design;
and (iii) an experimental evaluation of the model.

\emph{Data acquisition and preparation.}
The training examples given to the neural network are $(score, annotation)$ input-output pairs.
The inputs are symbolic music scores in MusicXML~\cite{good2001musicxml}.
The outputs are digital Roman numeral annotations in RomanText~\cite{gotham_romantext_2019}.
There are currently seven datasets of Roman numeral analysis data, comprising $\sim$400 pieces of music with $\sim$100,000 Roman numeral annotations.
One caveat is that these seven datasets vary in their file formats, notational conventions, and the quality of their contents.
% Training a model requires them with a unified format and syntax.
% This process was started by Gotham et al.~\cite{gotham_romantext_2019} but misses two newer datasets~\cite{hentschel2021annotated, weiss2021schubert}. I will complete this work.
% Additionally,
I will develop automatic tools to clean the datasets.
This includes:
disambiguating synonyms (e.g., $\flat$\texttt{II6} and \texttt{N6});
detecting mislabeled data (e.g., wrong inversion);
and standardizing the format of the annotations (e.g., writing scripts to convert among different file formats~\cite{gotham_romantext_2019, napoles_lopez_harmalysis_2020}).
The datasets will then be homogenized to consistently train the model.
However, the resulting data will likely remain insufficient for training a deep learning model.
Thus, in addition to cleaning the data, I will develop a new data-augmentation technique.
The new data-augmentation technique will extend an existing approach based on dynamic programming and voice-leading rules~\cite{napoles_lopez_harmonic_2020}.
This will allow me to artificially augment the number of training examples.

\emph{Model design.}
Currently, there are two preferred architectures for multitask tonal analysis:
convolutional recurrent neural networks~\cite{micchi_not_2020} and Transformer-based networks~\cite{chen_attend_2021}.
% I will develop over convolutional neural networks, among other reasons, because it is easier to visualize what these networks learn~\cite{lin2016visualizing}.
% A goal of the new model is then to visualize the musical patterns it learns.
I will focus my research on convolutional recurrent neural networks because, among other reasons, it is easier to visualize what these networks learn~\cite{lin2016visualizing}.
The aspects of model design that I will investigate include:
new input representations to the neural network;
new ways to encode pitch spelling, which has benefitted previous models~\cite{micchi_not_2020};
and the extraction of additional tasks from the Roman numeral annotations (e.g., harmonic rhythm), as multitask learning often improves with more tasks.

\emph{Experimental evaluation of the model.}
It is conventional for machine learning projects to start with a \emph{baseline} model.
This is an initial model that has to be improved by the newer one.
In this dissertation, I will consider a previous convolutional recurrent neural network~\cite{micchi_not_2020} and a Transformer~\cite{chen_attend_2021} model as my baselines.
These models have been evaluated on Beethoven Piano Sonatas~\cite{chen_functional_2018} and the Well-Tempered Clavier~\cite{gotham_romantext_2019} datasets.
I will replicate these experiments for direct comparison.
Additionally, I will report the generalization of my model in five other datasets.

\section*{Summary and Contributions}
In recent literature dealing with chord and key analysis, multitask models are preferred over task-specific ones.
This has musical relevance to Roman numeral analysis, where keys and chords are analyzed together.
Thus, there has been a recent emergence of deep neural networks for Roman numeral analysis.
In this dissertation, I intend to extend this line of research by:
(1) improving the data-curation process for existing datasets;
(2) developing a new data-augmentation technique for Roman numeral analysis models;
(3) improving the design of existing convolutional recurrent neural networks;
and (4) extracting more tonal tasks from the Roman numeral annotations.
This will improve the ways in which we annotate and search over large collections of digital music scores.
% This will be useful for annotating and searching over large collections of digital music scores.

\bibliographystyle{plain}
{\footnotesize
\bibliography{references}}

\clearpage

\tableofcontents
\addtocontents{toc}{\protect\thispagestyle{empty}}
\pagenumbering{gobble}

\section*{Table of contents}
\section{Introduction}
\subsection{Motivation}
% \subsection{Research objectives}
\subsection{Thesis structure}

\section{Background}
\subsection{Music theory}
\subsubsection{A brief history of Roman numeral analysis}
\subsubsection{Digitization of Roman numeral annotations}
% \subsubsection{Ambiguity and inter-annotator agreement in Roman numeral analysis}

\subsection{Music Information Retrieval}
\subsubsection{Automatic chord and key estimation}
\subsubsection{Multitask tonal analysis}

\section{Data acquisition and preparation}
\subsection{Available datasets}
\subsubsection{Annotated Beethoven Corpus (ABC)}
\subsubsection{Beethoven Piano Sonatas (BPS)}
\subsubsection{Haydn ``Sun'' String Quartets (HaydnSun)}
\subsubsection{Mozart Piano Sonatas (MPS)}
\subsubsection{Theme and Variation Encodings with Roman Numerals (TAVERN)}
\subsubsection{Schubert Winterreise (SW)}
\subsubsection{When-in-Rome (WiR)}
\subsection{Data preparation}
\subsubsection{Aligning a score and an annotation file}
\subsubsection{Standardizing the notation between datasets}
\subsection{Data augmentation}
% \subsubsection{Synthesizing examples with voice-leading rules}
\subsubsection{Synthesis and artificial texturization of scores}

\section{Model design}
\subsection{Inputs}
\subsubsection{Encoding pitch spelling}
\subsection{The bass and chroma convolutional blocks}
\subsection{Dense and recurrent layers}
\subsection{Multitask learning configurations}
\subsubsection{Six conventional tonal tasks}
\subsubsection{Five additional tonal tasks}

\section{Experimental evaluation}
\subsection{Training procedure}
\subsection{Hyperparameters}
\subsubsection{Training epochs and weight selection}
\subsubsection{Learning schedule and learning rate}
\subsection{Evaluation procedure}
\subsection{Results}
\subsection{Discussion}
% \subsection{Visualization of learned convolutional filters}

% \section{Applications}
% \subsection{Music generation conditioned on tonal context}
% \subsection{Visualization of learned tonal features}

\section{Conclusions}

\clearpage

\end{document}
