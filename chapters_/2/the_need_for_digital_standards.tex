\guide{On the need for standardization.}
In the classroom setting, the flexibility of the notation is
arguably a desirable goal. Students may be encouraged to
develop their own ``style'' of tonal analysis, incorporating
aspects of voice-leading, motivic, and key analysis, as they
see fit. This is useful to extend the intrinsic limitations
of Roman numeral analysis to summarize tonal music.

Excessive flexibility may be an issue, however, in
computational work. Idiosyncratic and undocumented methods
of analysis lead to incompatible annotations. It is
unfeasible to assume that a single person will be able to
annotate a sufficiently large number of Roman numeral
analyses to be used for computational models. Thus,
compatibility between annotations and cooperation of
different analysts is necessary. This can only be achieved
by standardizing the notation, and the practice of Roman
numeral analysis. Several notational systems have emerged
over the years to attempt to solve this problem

In this section, I discuss the efforts that have been done
regarding the standardization of Roman numeral analysis into
a digital format.
