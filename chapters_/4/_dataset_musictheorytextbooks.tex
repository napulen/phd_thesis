\guide{Dataset}~\label{sec:dataset} All the labels in the
dataset have been obtained from the modulation excerpts of
five music theory textbooks, written by different music
theorists and/or composers (whom we simply refer to as
``theorists'' for the rest of this paper).

The dataset contains, in total, 201 excerpts of music with
annotated modulations and tonicizations. The annotations are
encoded in the form of roman numerals of all the chords in
the dataset, which can be helpful for utilizing the dataset
for other purposes beside the one presented here (e.g.,
chord labeling or cadence detection). Each file has been
encoded in Humdrum (**kern) symbolic music representations
~\parencite{huron02humdrum}. As mentioned, the roman numeral
annotations have been digitally encoded
\parencite{napoleslopez20harmalysis} within the scores.

When the theorists provided roman numeral annotations, those
have been preserved in our digital transcriptions.
Otherwise, we furnished them. All the annotations related to
modulations have been obtained exclusively from the
textbooks. Tonicizations rely on the roman numeral
annotations of the chords and these were not always provided
in the textbooks, therefore, we supplied some of them.

The issues of ambiguity discussed in Section
\ref{sec:ambiguity} might have implications mostly for
tonicizations (the ones we sometimes contributed ourselves).
However, modulations have always been provided by the
theorists, and we expect this to reduce the impact of these
issues. For some onsets, multiple key annotations were
provided by the theorists. For these excerpts, we decided to
encode the keys in chronological order. An example is shown
in Figure \ref{fig:multiplekeys}.\footnote{Whenever a new
key was established according to the theorists' annotations,
we considered that to be the only key in the modulation
column, until a new one appeared to replace it. We applied
this process systematically throughout the dataset.
%Despite the loss of informations, this unique annotation line is more easily processed by local key estimation models.
}

% \begin{figure}[h]
%   \centering
%   \includegraphics[width=\linewidth]{figs/multiplekey2}
%   \caption{Example 18-3 in Kostka-Payne's Tonal
%   Harmony~\parencite{kostka2008tonal}. Two concurrent keys (G
%   Major and F Major) are annotated in the fourth beat of
%   measure 1. We considered the key label of this onset to be
%   ``F Major'' for the modulation column.}
%   \Description{Example 18-3 in Kostka-Payne's Tonal Harmony.
%   Two concurrent keys appear in the same onset.}
%   \label{fig:multiplekeys}
% \end{figure}

We describe the five textbooks and their abbreviations,
which we use throughout the experiment section.

\guide{Sources of the Annotations}

	\guide{Aldwell, Schachter, and Cadwallader~(ASC) [USA, contemporary]}
	The modulation excerpts are taken from chapter 27
	\emph{Diatonic Modulations} of the \emph{Harmony and
	Voice Leading} ~\parencite{aldwell2018harmony}. This textbook
	provided seven excerpts to the dataset, the smallest
	amount among all textbooks. These excerpts are extracts
	from Bach Chorales (4), Mozart's Trio for Clarinet (1),
	and two original examples.

	\guide{Kostka and Payne~(KP) [USA, contemporary]}
	The modulation excerpts are taken from the 18th and 19th
	chapters of \emph{Tonal Harmony}~\parencite{kostka2008tonal}.
	We took fifteen excerpts from this book, which are
	fragments of pieces written by Classical and Romantic
	composers. Previously, the annotated audio excerpts of
	the accompanying workbook were used for another
	local-key estimation study \parencite{izmirli2007localized},
	however, we encoded the score excerpts of the main
	textbook.

	\guide{Reger~(Reg) [Germany, 1904]}
	A hundred modulation excerpts are taken from \emph{On
	the Theory of Modulation}\footnote{The book we used is
	the republication by \emph{Dover} with the title
	\emph{Modulation} (2007).}~\parencite{regermodulation}. The
	excerpts are very short and they are all written by
	Reger himself. Reger's goal was to provide cadence-like
	examples of modulation from two keys (C major and A
	minor) to almost every other possible key.

	Seventeen of the examples had two terminations: one in a
	major key and one in a minor key.\footnote{Usually
	parallel keys that shared a closing dominant harmony and
	resolved to either the major or minor mode, with the
	same tonic.} We separated these examples into two files,
	one for each of the terminations. This increased the
	total number of examples from 100 to 117.

	\guide{Rimsky-Korsakov~(Rim) [Russia, 1886]}
	The modulation excerpts are taken from the third section
	of the \emph{Practical Manual of
	Harmony}~\parencite{rimskitonality}. As with Reger, all the
	thirty-seven examples in this textbook are written by
	the author himself. Some of the examples, however, are
	more detailed and longer in duration than the ones by
	Reger.

	\guide{Tchaikovsky~(Tch) [Russia, 1872]}
	The modulation examples considered are taken from the
	third section of the \emph{Guide to the Practical Study
	of Harmony}~\parencite{tchaikovsky1872guide}. All twenty-five
	examples were written by Tchaikovsky himself.

\guide{Statistics About the Dataset}\label{ssec:stats}

Some statistics about the dataset are presented in
Table~\ref{tab:dataset}. We report, for each of the
textbooks: the number of files (excerpts), the number of
modulations, the number of tonicizations, and the number of
labels (which is equivalent to the number of onsets, as we
supplied one label per onset).

% \begin{table}
%   \caption{Summary of the dataset. Each value indicates the number of occurrences in the corresponding textbook.}
% % The column ``Tonicized'' denotes the number of labels with
% % a tonicization label that differs from the key of the
% % modulation (e.g., Offset 2 in Table
% % \ref{tab:annotations}).
%   \label{tab:dataset}
%   \begin{tabular}{l|cccc}
%     \toprule
%     Sample & Files & Modulations & Tonicizations & Labels\\
%     \midrule
%     ASC & 7 & 8 & 7 & 185\\
%     KP & 15 & 21 & 11 & 554\\
%     Reg & 117 & 220 & 40 & 768\\
%     Rim & 37 & 44 & 107 & 257\\
%     Tch & 25 & 60 & 38 & 238\\
%     Total & 201 & 555 & 203 & 2002 \\
%   \bottomrule
% \end{tabular}
% \end{table}

The \emph{Reg} textbook is by far the one that contributed
the largest number of excerpts. However, the ones providing
a higher ratio of labels per number of files are \emph{ASC}
($26.42$) and \emph{KP} ($36.93$). This may be due to the
use of musical examples taken from the literature, where
modulations occur within a musical context and, therefore,
span longer regions.

\emph{Rim} and \emph{Tch} are the textbooks that provided
the highest number of tonicizations. They show tonicizations
in $41.63\%$ and $15.97\%$ of the onsets, respectively. In
terms of investigating the relationship between predicted
local keys and modulations/tonicizations, these textbooks
contributed the most interesting examples.

\emph{Rim} and \emph{Tch} also tend to set the annotations
of the \emph{destination} key of a modulation in the tonic
degree, considering any preceding dominant chords as
secondary dominants and, consequently, part of the
\emph{departure} key (as shown on Figure~\ref{fig:example}).
Other theorists, on the other hand, often set the
\emph{destination} key already in the dominant chords that
precede the tonic. Therefore, they do not annotate (or
imply)\footnote{For the cases in which we provided the roman
numeral annotations because they were not provided by the
theorists.} a tonicization for the preceding dominant
chords.
