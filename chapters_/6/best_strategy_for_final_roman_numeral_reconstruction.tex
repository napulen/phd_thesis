% Copyright 2022 Néstor Nápoles López

Some of the predictions from the neural network provide
redundant information. For example, the \gls{pcset121} can
be derived from the \gls{bass35}, \gls{tenor35},
\gls{alto35}, and \gls{soprano35} or from the \gls{rn31} and
\gls{tonicization35}. There are advantages and disadvantages
of each of those classifiers. For instance, the
\gls{pcset121} is bounded to the chord vocabulary. That is,
the prediction from the highest probability at each timestep
will always result in a ``valid'' chord, whereas \gls{rn31}
and \gls{tonicization35} might provide a nonsensical
interpretation.

Another possibility, which is the one proposed here, is in
fact to use information from all the classifiers to
reconstruct the \gls{rna} label. In this section, the
results of using individual predictors is compared against
the combination of predictions.
