% Copyright 2021 Néstor Nápoles López

% This is \refsubsubsec{whyanalyzingkeychanges}, which
% introduces the why analyzing key changes.

In music of the common-practice period, fluctuations of key
throughout a piece are the norm, rather than the exception.
It is not atypical to find deviations of key, even in
beginner-level pieces meant to teach students the basics of
a musical instrument. Nevertheless, changes of key are
implicit in the music notation. Beside key signatures (and
therefore, accidentals), there is no indication in the
notation about when the musical key is changing. For these
two reasons: (1) how likely it is that a piece will have key
changes within, (2) how implicit key changes are to the bare
eye, I think key changes are an interesting analytical
dimension. That is, there is value in highlighting the key
changes in the piece, making them explicit and available for
other human beings to understand.

% \phdparagraph{because they often occur in common-practice
% music} \phdparagraph{and those changes are important}
