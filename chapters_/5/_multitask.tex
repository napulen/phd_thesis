% Taken verbatim from AugmentedNet

\guide{Multi-outputs}

The output of the network follows a MTL approach with hard
parameter sharing, similar to that proposed by Chen and Su
\parencite{chen2018functional}. For each of the output tasks, a
time-distributed dense layer is attached to the second GRU,
and used to predict its corresponding task. Six conventional
tasks are learned (similar to Micchi et al. and Chen and Su
\parencite{micchi2020not, chen2021attend}), plus five additional
ones. All the tasks (eleven in total) and their number of
output classes are shown on the right side of Figure
\ref{fig:network}.

\guide{Six conventional functional harmony tasks}
All the conventional tasks, except for the local key, have
the same number of output classes described by Micchi et al.
\parencite{micchi2020not}. The local key includes four additional
keys: $\{F\flat, G\sharp, d\flat, e\sharp\}$. These were
included because the data exploration process revealed
modulations reaching $G\sharp$ major in the dataset. Thus,
the number of allowed key signatures was extended by one
sharp and one flat, in both modes.
% This also introduced more training examples during the
% data augmentation process via key transposition.

\guide{Five Additional
tasks.}\label{sec:additionaltasks}

It is argued that MTL helps improving the generalization of
a model by preferring representations that are useful to
related tasks, acting as an implicit form of data
augmentation and regularization method
\parencite{ruder2017overview}. Roman numeral labels can be
decomposed into multiple different features, of which the
six conventional tasks are known examples. Motivated by the
possibility of improving the performance of our network, we
included five additional tasks with hard parameter sharing
in our MTL approach. We describe the five additional tasks,
which have relevance to harmonic analysis. One of them,
RomanNumeral75, was also used as an alternative task when
reconstructing the final Roman numeral label, a process we
explain below.

\begin{table}[]
\begin{tabular}{l|l|l|l|l}
1--15    & 16--30    & 31--45    & 46--60     & 61--75    \\
\hline
I       & V/V      & Ger     & viio7/v   & V+       \\
V7      & v        & N        & viio7/iii & viio/vi  \\
V       & V7/ii    & viio7/vi & IV/V      & III+     \\
i       & III      & V/ii     & I+        & V/iii    \\
IV      & iiø7     & viiø7    & I7        & ii/V     \\
ii      & iii      & V9       & viio/IV   & I/bVI    \\
vi      & iio      & viio/ii  & V/III     & viio7/IV \\
iv      & viio/V   & V/iv     & V7/iii    & V7/v     \\
viio7   & V7/vi    & Cad/V    & viio/iv   & i7       \\
viio    & VII      & iv7      & iio7      & iii7     \\
V7/V    & viio7/ii & viio7/iv & VI7       & Fr      \\
V7/IV   & I/V      & IV7      & I/III     & V/IV     \\
viio7/V & V7/iv    & V7/III   & V7/VI     & vii      \\
VI      & V/vi     & viiø7/V  & bVII      & V/v      \\
ii7     & vi7      & It       & bVI       & II
\end{tabular}
\caption{The 75 most-common Roman numeral strings, where the inversion has been removed and learned independently. During data exploration, these classes spanned 98\% of the Roman numeral annotations across all the annotated data. Predicting these classes is an alternative to predicting the chord root, chord quality, primary degree, and secondary degree simultaneously.}
\label{tab:top75rn}
\end{table}

\textbf{RomanNumeral75}. During data exploration, we found
that, when inversions were removed and synonyms (e.g.,
\texttt{bII6} and \texttt{N6}) were standardized, a set of
75 Roman numeral strings spanned approximately $98\%$ of all
the annotations across all datasets. This was a motivation
to predict the Roman numeral string itself. The correct
prediction of this task is equivalent to predicting the
chord root, chord quality, primary degree, and secondary
degree simultaneously. As an additional experiment in our
results section, we substituted the final reconstruction of
the Roman numeral label in this fashion, using the local
key, the inversion, and the RomanNumeral75 outputs. The
performance with this method is always better than through
the four conventional tasks. The 75 classes of common Roman
numeral strings are shown in Table \ref{tab:top75rn}.


\textbf{Harmonic Rhythm.} Whether a Roman numeral annotation
label starts at the given timestep. A binary classification
task that may be relevant for chord segmentation.

\textbf{Bass}. The bass note implied by the Roman numeral
label. This is predicted with 35 output classes, similarly
to the chord root. The 35 classes represent a spelled pitch.
This task is related to predicting the chord inversion.

\textbf{Tonicization}. The tonicization is predicted from
the key implied by a secondary degree (if any).
% When not present, the tonicization encodes the local key
% instead, to reduce the sparsity of classes. This idea is
% adapted from N\'apoles L\'opez et al.
% \parencite{napoleslopez2020local}.
The output classes are identical to the ones of the local
key and is an alternative approach to learn the secondary
degrees.

\textbf{Pitch Class Sets}. The set of pitch classes implied
by the chord. The number of classes (93) results from
computing all pitch class sets in all diatonic triad and
seventh chords, plus all augmented sixth chords in all keys.
This task is related to the chord quality, primary degree,
and to non-chord tones \parencite{ju2017nonchord}. All additional
tasks except for the RomanNumeral75 are computed solely for
their contributions to the shared representation of the MTL
approach.
